\section*{Voorwoord}
\label{sec:Voorwoord}
  Deze uitgave is geen offici�le uitgave van de Vrije Universiteit Brussel, slechts een formularium gemaakt door een student. 
  Mogelijk staan er hier of daar nog fouten in, indien u er tegenkomt, 
  stuur gerust een mailtje naar \hreftt{mailto:egon.geerardyn@vub.ac.be}{egon.geerardyn@vub.ac.be}.\par
  
  \begin{quote}
    Copyright \copyright{}  Egon Geerardyn.\par
    Permission is granted to copy, distribute and/or modify this document
    under the terms of the GNU Free Documentation License, Version 1.2
    or any later version published by the Free Software Foundation;
    with no Invariant Sections, no Front-Cover Texts, and no Back-Cover Texts.
    A copy of the license is included in the section entitled ``GNU
    Free Documentation License'' in the source code and available on:
    \linktt{http://www.gnu.org/copyleft/fdl.html}.
  \end{quote}
  \noindent 
  De \LaTeX -broncode is vrij beschikbaar onder GNU Free Document License. \par
  
  
  Mogelijk is er reeds een nieuwe versie beschikbaar op\par
  \hreftt{http://students.vub.ac.be/~egeerard/projects.html}{http://students.vub.ac.be/\tildefix egeerard/projects.html}
\section*{Referenties}
\begin{enumerate}
	\item \textsc{prof. R. Willem}, \textit{Chemie: Structuur en Transformaties van de materie}, Dienst Uitgaven VUB 2006.\par
  \item \textsc{prof. M. Biesemans} en \textsc{prof. R. Willem} , \textit{Chemie: Oefeningen}, Dienst Uitgaven VUB 2006.\par
  \item \textsc{Brown, LeMay} en \textsc{Bursten} , \textit{Chemistry: the Central Science (10th Edition)}, Pearson Education 2006.
\end{enumerate}

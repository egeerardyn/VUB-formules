\onecolumn
\newpage
\section{Chemische Thermodynamica}
\label{sec:H:Thermodynamica}

\paragraph{Tekenconventie} voor energie-overdracht
\label{sec:Tekenconventie}
\[
  \mbox{systeem} \stackrel{-}{\rightarrow} \mbox{buitenwereld}
\]
\[
  \mbox{systeem} \stackrel{+}{\leftarrow} \mbox{buitenwereld}
\]
\paragraph{Standaardomstandigheden}
\label{sec:Standaardomstandigheden}
\begin{tabular}{||ll||}
	\hline
	           & $T = 298 \eenheid{K} = 25 \degree \eenh{C}$\\
	gas        & $p = 1\eenh{atm}$\\
	vloeistof  & $c = 1 \eenh{M}$\\
	vaste stof & $n = 1 \eenh{mol}$\\
	vloeistof/%
	vloeistof  & zuivere toestand\\
	\hline
	elementen  & $\Delta G^0_{\textsf{vorming}} = 0$\\
	           & $\Delta H^0_{\textsf{vorming}} = 0$\\
	\hline
\end{tabular}

\paragraph{Inwendige Energie}
\label{sec:InwEnergie}
\[
  \Delta U = U_{eind} - U_{begin} = q + w
\]

\paragraph{Arbeid} door samendrukking/uitzetting
\[
  w = p \Delta V
\]

\paragraph{Warmte}
\[
  q = \int^{T_{eind}}_{T_{begin}} C \d{T}
\]

\subsection{Enthalpie}
\label{sec:HH:Enthalpie}
\[
  \Delta H = \Delta U + \Delta \left(pV \right)
\]
\paragraph{Bij constant volume:}
\[
  \Delta U = q_V
\]
\paragraph{Bij constante druk:}
\[
  \Delta H = q + w + \Delta \left(pV \right) = q_p
\]
\paragraph{Vrije enthalpie/Enthalpie van Gibbs $G$}
Enthalpie die bruikbaar is om arbeid te verrichten
\[
  \Delta G = \Delta H - \Delta \left( TS \right)
\]
\subsubsection{Wet van Hess}
\label{sec:HHH:WetHess}
\definitie{Wet van Hess}{
 De reactie-enthalpie is de som van de vormingsenthalpi�n van de reactieproducten min de vormingsenthalpi�n van de uitgangsstoffen. De reactie-enthalpie is onafhankelijk van de gevolgde reactieweg.
}
\[
  \Delta H_{reactie}^0 = \sum \alpha \Delta H_{vorming}^0 \mbox{(reactieproducten)}
                       - \sum \alpha \Delta H_{vorming}^0 \mbox{(uitgangsstoffen)}
\]

\subsection{Entropie}
\label{sec:HH:Entropie}
\[
  \Delta S = \frac{q}{T}
\]


% Table generated by Excel2LaTeX from sheet 'Sheet1'
\begin{tabular}{|rll|}
  \hline
  \multicolumn{ 1}{|r}{ge�soleerd systeem}    & $\Delta S_{\textsf{systeem}} > 0 $&onomkeerbaar\\
  \multicolumn{ 1}{|r}{}                      & $\Delta S_{\textsf{systeem}} = 0 $&omkeerbaar\\
  \multicolumn{ 1}{|r}{open/gesloten systeem} & $\Delta S_{\textsf{heelal}} =  \Delta S_{\textsf{systeem}} + \Delta S_{\textsf{buitenwereld}} > 0 $&onomkeerbaar\\
  \multicolumn{ 1}{|r}{}                      & $\Delta S_{\textsf{heelal}} =  \Delta S_{\textsf{systeem}} + \Delta S_{\textsf{buitenwereld}} = 0 $&omkeerbaar\\
\hline
\end{tabular}  


\subsection{Chemische Evenwichten}
\label{sec:HH:ChemEvenwicht}
\[
  \Delta G = \Delta G^0 + RT \ln \Q_p
\]
\[
  \Q_p = \prod^N \left( \frac{p_i}{ p_i^{std}} \right)^{\nu_i} \qquad \mbox{met $\nu_i > 0$ stoichiometrische co�ffici�nten voor de reactieproducten}
\]
\[
  \Q_p^e = \K_p
\]
\[
  \Q_c = \prod^N \left( \frac{c_i}{ c_i^{std}} \right)^{\nu_i} \qquad \mbox{met $\nu_i > 0$ stoichiometrische co�ffici�nten voor de reactieproducten}
\]
\[
  \Q_c^e = \K_c
\]

\subsubsection{Wet van Le Ch�telier}
\label{sec:HH:WetLeChatelier}
\definitie{Wet van Le Ch�telier}{Wanneer een sotring wordt aangebracht op een chemisch evenwicht, zal dit evenwicht zo verschuiven worden dat de verstoring tegengewerkt wordt}

\onecolumn
\newpage
\section{Fasenleer}
\label{sec:H:Fasenleer}

\paragraph{Faseregel van Gibbs}
\label{sec:FaseregelVanGibbs}
\[
  \Vr = c + 2 - f
\]
\begin{itemize}
	\item $\Vr$: het aantal vrijheidsgraden
	\item $c$: het aantal onafhankelijke componenten in het systeem
	\item $2$: de vrijheidsgraden: druk en temperatuur
	\item $f$: het aantal fasen in evenwicht in het systeem 
\end{itemize}

\paragraph{Clapeyron-Clausius: temperatuursafhankelijkheid van de dampdruk}
\[
  p_i^0 \left(T\right) = B e^{-\frac{\Delta H_{verd}}{RT}}
\]
\textbf{Gebruiksvoorwaarden en symbolen}
\begin{itemize}
	\item Enkel toepasbaar bij ideale gassen, dus niet rond de kritische temperatuur bruikbaar
	\item $\Delta H_{\textsf{verd}}$ wordt vaak vervangen door $\Delta H^0_{\textsf{verd}}$, 
	      dit is enkel toepasbaar als de temperatuur niet veel afwijkt van de kamertemperatuur (tot $50 \degree \eenh{C}$)!
	\item $B$ is een constante die karakteristiek is voor de stof $i$
\end{itemize}

\paragraph{Molaliteit}
\[
  \molaliteit_B = \frac{\eenheid{mol} \textsf{opgeloste stof } B}{1000 \eenheid{g} \textsf{oplosmiddel } A}
\]

\paragraph{Wet van Raoult}
\[
  p_A = p_A^0 x_A
\]

\paragraph{Dampdrukswijziging}
\[
  \Delta p_A = p_A^0x_B = \frac{p_A^0 M_A}{1000} \molaliteit_b = \mathbf{K}_A \molaliteit_B
\]

\paragraph{Kooktemperatuurswijziging}
\[
  \Delta T^{K}_A = \mathbf{K}^K_A \molaliteit_B
\]

\paragraph{Stoltemperatuurswijziging}
\[
  \Delta T^{S}_A = \mathbf{K}^S_A \molaliteit_B
\]

\paragraph{Osmotische druk}
\[
  p_{\textsf{osm}} = \pi = \conc{B}RT = \frac{m_B}{M_B}RT 
\]
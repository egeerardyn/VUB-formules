\subsection{Redoxhalfreacties opstellen}  
\label{sec:app:Halfreacties}
  \footnotesize{naar \textsc{Stefan Sandker}}
  \begin{enumerate}
    \item Begin met het redoxkoppel. 
	        Dit koppel moet bekend zijn, dus het moet expliciet in de opgave vermeld staan. 
	        Soms zul je uit de gegevens in de opgave moeten afleiden wat het koppel is, 
	        dus dan is het koppel impliciet gegeven.
	  \item \textbf{Massabalans:}
           \begin{enumerate}
	           \item Balanceer alle elementen behalve $O$ en $H$
	           \item Balanceer $O$ door het juiste aantal $H_2O$ aan te vullen.
	           \item Balanceer $H$ door het juiste aantal $H^+$ in te vullen. 
	                 Eigenlijk zou je $H_3O^+$ moet invullen, 
	                 maar in de praktijk is het gebruikelijk in redoxreacties $H^+$ in te vullen.
	         \end{enumerate}
	  \item \textbf{Ladingenbalans:}
	        Balanceer de lading door het juiste aantal $e^-$ in te vullen. 
	        Dit is overigens ook in ``gewone'' reacties belangrijk: 
	        bij het kloppend maken van een reactievergelijking moet je niet alleen de elementen kloppend maken, 
	        maar er ook altijd voor zorgen dat de lading kloppend is! In een reactievergelijking 
	        kan dat natuurlijk niet door elektronen in te vullen.
	  \item Als je bij het kloppend maken links $H^+$-ionen hebt ingevuld, maar uit de opgave blijkt dat de oplossing niet zuur is, 
	        dan moeten deze ionen verdwijnen. Dat doe je door het juiste aantal $OH^-$ toe te voegen: links ontstaat dan $H_2O$, 
	        rechts komen $OH^-$-ionen te staan. Het is ook mogelijk dat je bij het kloppend maken rechts $H^+$-ionen hebt ingevuld, 
	        terwijl uit de opgave blijkt dat dit niet correct is. Ook dan laat je de $H^+$-ionen ``verdwijnen'' door links en rechts 
	        het juiste aantal $OH^-$-ionen toe te voegen.
	  \item Controleer of tussen de deeltjes een zuur-base reactie of een neerslagreactie ontstaat.
	  \item \textbf{Vereenvoudig de halfreactie}: als je een deeltje links �n rechts ziet staan, moet je het wegstrepen.
  \end{enumerate}
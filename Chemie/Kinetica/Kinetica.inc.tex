\onecolumn
\newpage
\section{Chemische Kinetica}
\label{sec:H:Kinetica}

\paragraph{Snelheid voor een enkelvoudig reactiepad of samengesteld rectiepad met stationariteit}
\[
  s = \frac{1}{\nu_i}\frac{\d{\conc{X_i}}}{\d{t}} \qquad \mbox{met $\nu_i$ de stoichiometrische co�ffici�nt ($< 0$ voor de uitgangsstoffen)}
\]
\paragraph{Monomoleculair reactiepad}
\[
  A \reactie B + C + \ldots
\]
\[
  s = \frac{-\d{\conc{A}}}{\d{t}} = k \conc{A}
\]
\paragraph{Bimoleculair reactiepad}
\[
  A + B \reactie C + D + \ldots
\]
\[
  s = \frac{-\d{\conc{A}}}{\d{t}} = \frac{-\d{\conc{B}}}{\d{t}} = k \conc{A}\conc{B}
\]

\paragraph{Halveringstijd}
\[
  \conc{A}_{t=t_{1/2}} = \frac{\conc{A}_0}{2}
\]
\[
  \conc{A}_{t=t_{1/f}} = \frac{\conc{A}_0}{f}
\]
Bij een eerste-orde-reactie:
\[
  t_{1/f} = \frac{\ln f}{k}
\]
\[
  \frac{t_{1/f}}{t_{1/g}} = \frac{\ln f}{\ln g}
\]
Bij een $n$-de-orde-reactie:
\[
  t_{1/f} = \frac{f^{n-1}-1}{\conc{A}_0^{n-k} \left(n-1\right) k}
\]
\[
  \frac{t_{1/f}}{t_{1/g}} = \frac{f^{n-1}-1}{g^{n-1}-1}
\]
De verhouding tussen twee delingstijden is typerend voor de reactieorde.

Eerste-orde-reactie:
\[
  \conc{A}_t = \conc{A}_0 e^{-kt}
\]

Tweede-orde-reactie:
\[
  \frac{1}{\conc{A}_t} = \frac{1}{\conc{A}_0}+kt
\]

\paragraph{Stationariteit}
\label{sec:Stationariteit}
\begin{eqnarray*}
  A  & \stackrel{k_1}{\reactie} & B\\
  B  & \stackrel{k_1}{\reactie} & C\\
  \\
  A & \stackrel{k_?}{\reactie} & C  
\end{eqnarray*}
\[
  \frac{\d{\conc{B}}}{\d{t}} \approx 0 \Leftrightarrow \conc{B} <<< \conc{A} \mbox{ �n } \conc{B} <<< \conc{C}
\]

\paragraph{Activeringsenergie van Arrhenius}
\[
  k = A e^{\frac{-E_A}{RT}}
\]

\paragraph{Activeringsenthalpie van Eyring}
\[
  k = \frac{\kappa T}{h} e^{\frac{-\Delta G^{\ddag}}{R}}
\]
\[
  \Delta G^\ddag = \Delta H^\ddag - T\Delta S^\ddag
\]
\documentclass[pdftex,fleqn,a4paper]{article}
  
  \let\oldAuthor\author
  \renewcommand{\author}[1]{\newcommand{\theAuthor}{#1}\oldAuthor{#1}} 
  \let\oldTitle\title
  \renewcommand{\title}[1]{\newcommand{\theTitle}{#1}\oldTitle{#1}}
  \newcommand{\subtitle}[1]{\newcommand{\theSubtitle}{#1}}
  \let\oldDate\date
  \renewcommand{\date}[1]{\newcommand{\theDate}{#1}\oldDate{#1}}
  
  \usepackage{../huisstijl/vubtitlepage}
  \faculty{Faculteit Ingenieurswetenschappen}
  
  \author{\href{mailto:Egon.Geerardyn@vub.ac.be}{Egon Geerardyn}}
  \title{Formules Wiskunde}
  \newcommand{\revisie}{revisie 3.6}
  \date{\revisie\ (\today)}
 
 
 
 
  %\usepackage{graphicx}
  \usepackage{amsfonts}
  \usepackage{mathrsfs}
  \usepackage{fancyhdr}
  \usepackage[dutch]{babel}
  \usepackage[utf8]{inputenc}
  \usepackage[T1]{fontenc}
  \usepackage[pdftex]{hyperref}
  \usepackage{flow}
  %%%%
  
  
    
  \renewcommand{\rmdefault}{cmss}
  \renewcommand{\sfdefault}{cmr}
  \setlength\parindent{0cm}
  \addtolength{\textheight}{2cm}
  \addtolength{\textwidth}{2cm}
  \addtolength{\hoffset}{-1.5cm} 
  \pagestyle{fancy}
  \rhead{pagina \thepage}
  \chead{\footnotesize \revisie}
  \lhead{\textbf{Formules wiskunde}}
  \cfoot{}   
  
   % Egon Geerardyn common latex Commands
%
% 2007 01 03 : Version 0.5
%
%
%
% dependencies
\usepackage[usenames]{color}
\usepackage{amsfonts}
\usepackage{mathrsfs}
% hyperlinks (pdfLaTeX)
    \newcommand{\tildefix}{\textasciitilde}
    \newcommand{\hreftt}[2]{\href{#1}{\texttt{#2}}} %teletype set link
    \newcommand{\link}[1]{\href{#1}{#1}}
    \newcommand{\linktt}[1]{\hreftt{#1}{#1}}


% symbolen
  %   \usepackage{manfnt}
  % \newcommand{\vb}{\mbox{\manstar}}
   \newcommand{\qed}{\mbox{$\Box$}}
   \newcommand{\QED}{\begin{flushright}\qed\end{flushright}}
   \newcommand{\GO}{\mbox{$\{\aleph\}$}}
   \newcommand{\n}{\mbox{$^{\mbox{n}}$}}
   \newcommand{\e}{\mbox{$^{\mbox{e}}$}}
   \newcommand{\s}{\mbox{$^{\mbox{s}}$}}
     \usepackage{wasysym}
   \newcommand{\ctr}{\mbox{\lightning}}
   \newcommand{\cte}{\mathrm{ct{^{\underline{\mathrm{e}}}}}}
   \newcommand{\vgl}{\mbox{vgl}}
   \newcommand{\hn}{\mbox{$\overline{\mbox{h}}$}}
   \newcommand{\wn}{\mbox{$\overline{\mbox{w}}$}}
   \newcommand{\zn}{\mbox{$\overline{\mbox{z}}$}}
   \newcommand{\RA}{\mbox{$\Longrightarrow$}}
   \newcommand{\LA}{\mbox{$\Longleftarrow$}}
   \newcommand{\LRA}{\mbox{$\Longleftrightarrow$}}
   \newcommand{\rer}{\mbox{$\sim$}}
   \newcommand{\isdef}{\mbox{$\stackrel{\Delta}{=}$}}
   \newcommand{\elek}{\mbox{$e^{-}$}}
   \newcommand{\prot}{\mbox{$p^{+}$}}
   \newcommand{\neut}{\mbox{$n^{0}$}}
   \newcommand{\angstrom}{\eenh{\AA}}
     \let\SavedRightarrow=\Rightarrow %fix for Marvosym rightarrow
       \usepackage{marvosym}
     \let\Rightarrow=\SavedRightarrow %fix for Marvosym rightarrow
   \newcommand{\wwwsym}{\Ecommerce}
   \newcommand{\www}[1]{\wwwsym\ #1}
   \newcommand{\pompern}{\mbox{\Gentsroom}}
   \newcommand{\busbitch}{\Ladiesroom}
   \newcommand{\msun}{m_{\odot}}
   \newcommand{\mearth}{m_{\earth}}
   \newcommand{\rearth}{r_{\earth}}
   \newcommand{\GOis}{\stackrel{$\GO$}{\approx}}

%invultemplates
   \newcommand{\invulW}{\qquad \qquad \qquad \quad}
   \newcommand{\invulWs}{\qquad \qquad \qquad}
   \newcommand{\invulF}{\pm \qquad \qquad \quad}
   \newcommand{\invulM}{\qquad \quad}
   \newcommand{\invulWF}{\invulW \invulF}
   \newcommand{\invulWFM}{\left(\invulWF\right) \cdot \invulM}
   \newcommand{\invulT}[1]{\vspace{#1}}
   \newcommand{\foutenentry}[2]{ #1      & $\invulW$ & $#2$ \\ \hline}
   \newcommand{\tablesizer}[1]{\renewcommand\arraystretch{#1}}
   \newcommand{\bigtables}{\tablesizer{2.0}}
   \newcommand{\normtables}{\tablesizer{1.0}}
   \newcommand{\foutenbespreking}[2]{\bigtables
                                      \begin{tabular}{|l|rl|}
                                          \hline
                                        \foutenentry{\textbf{Waarde} $#2$}{#1}
                                          \hline
                                        \foutenentry{Afleesfout}{#1}
                                        \foutenentry{Instelfout}{#1}
                                        \foutenentry{Instrumentfout}{#1}
                                        \foutenentry{Nulpuntsfout}{#1}
                                          \hline
                                        \foutenentry{\textbf{Totale absolute fout}}{#1}
                                          \hline
                                        \foutenentry{\textbf{Totale relatieve fout}}{}
                                      \end{tabular}
                                     \normtables
                                    }
   \newcommand{\foutenbesprekingkort}[2]{\bigtables
                                      \begin{tabular}{|l|rl|}
                                          \hline
                                        \foutenentry{\textbf{Waarde} $#2$}{#1}
                                          \hline
                                        \foutenentry{\textbf{Totale relatieve fout}}{}
                                        \foutenentry{\textbf{Totale absolute fout}}{#1}
                                      \end{tabular}
                                     \normtables
                                    }

% dutch arc-goniometric functions
    \newcommand{\bgsin}{\mbox{\textsf{Bgsin}}\,}
    \newcommand{\bgcos}{\mbox{\textsf{Bgcos}}\,}
    \newcommand{\bgtan}{\mbox{\textsf{Bgtan}}\,}
    \newcommand{\bgcot}{\mbox{\textsf{Bgcot}}\,}

%alternative notation for arc-goniometric functions
    \newcommand{\argcos}{\mbox{\textsf{argcos}}\,}
    \newcommand{\argsin}{\mbox{\textsf{argsin}}\,}
    \newcommand{\argtan}{\mbox{\textsf{argtan}}\,}
    \newcommand{\argcot}{\mbox{\textsf{argcot}}\,}

%alternative notation for arc-hyperbolic functions
    \newcommand{\argcosh}{\mbox{\textsf{argcosh}}\,}
    \newcommand{\argsinh}{\mbox{\textsf{argsinh}}\,}
    \newcommand{\argtanh}{\mbox{\textsf{argtanh}}\,}
    \newcommand{\argcoth}{\mbox{\textsf{argcoth}}\,}

% coordinaat
    \newcommand{\co}{\mbox{ \textsf{co}}\,}
% absolute waarde
    \newcommand{\abs}[1]{\left| #1 \right|}
% degree symbol
    \newcommand{\degree}[0]{^\circ}
    \newcommand{\degC}[0]{\; \degree \eenh{C}}
% ronde B voor bol
    \newcommand{\bol}[0]{\mathscr{B}}

% ronde K voor kwadriek
    \newcommand{\kwadriek}[0]{\mathscr{K}}

%differentials
    \renewcommand{\d}[1]{\;\textsf{d}#1}
    \newcommand{\pd}[1]{\partial #1}
    \newcommand{\D}{\;\textsf{D}}
    \newcommand{\pdiff}[3][]{\frac{\pd^{#1}{#2}}{\pd{#3}^{#1}}}
    \newcommand{\diff}[3][]{\frac{\d^{#1}{#2}}{\d{#3}^{#1}}}

%dot and double dot for D_t en D_t^2
    \newcommand{\dt}[1]{\dot{#1}} %dot notation for d/dt
    \newcommand{\dtt}[1]{\ddot{#1}} % double dot notation for d^2 / dt^2

%accent (acute) for D_s and D_s^2
    \newcommand{\ds}[1]{#1 \acute{}\,} % accent notation for d/ds
    \newcommand{\dss}[1]{#1 \acute{}\, \acute{}\,} % double accent notation for d^2 / ds^2

%accent notation for arbitrrary derivative of order 1 or 2
    \newcommand{\dx}[1]{#1 \grave{}\,} % accent notation for arbitrary d / dx
    \newcommand{\dxx}[1]{#1 \grave{} \, \grave{}\,} % double accent notation for d^2 / dx^2

%differential operators
    \newcommand{\vnabla}{\vec{\nabla}}
    \newcommand{\Nabla}{\vnabla}
    % nabla notated
    \newcommand{\vgradN}[1]{\Nabla #1\;}
    \newcommand{\vrotN}[1]{\Nabla \times #1\;}
    \newcommand{\vdivN}[1]{\Nabla \cdot #1\;}
    % standard (Dutch) notated
    \newcommand{\vgradT}[1]{\;\vec{\textsf{grad}}\,#1\;}
    \newcommand{\vrotT}[1]{\;\vec{\textsf{rot}}\,#1\;}
    \newcommand{\vdivT}[1]{\;\textsf{div}\,#1\;}
    % wrapper for easy switching
    \newcommand{\vgrad}[1]{\vgradT{#1}}
    \newcommand{\vrot}[1]{\vrotT{#1}}
    \newcommand{\vdiv}[1]{\vdivT{#1}}

%norm of a vector
    \newcommand{\norm}[1]{\left\| #1 \right\|}

%infinity redeclariation for use with WikiPedia LaTeX notation
    \newcommand{\infin}{\infty}

%Probability notation
    \newcommand{\prob}[1]{P\left(#1\right)}

%Combination
    \newcommand{\combination}[2]{\left( \begin{array}{c} #1 \\ #2 \end{array} \right)}

% E and Var
    \newcommand{\E}[1]{\!\mathrm{E}\left[ #1 \right]}
    \newcommand{\Var}[1]{\!\mathrm{Var}\left[ #1 \right]}

%identieke matrix
    \newcommand{\idmatrix}{\textsf{I}}
    \newcommand{\spoor}[1]{\mbox{\textsf{sp}}\left( #1 \right)\,}
%signumfunctie
    \newcommand{\sign}[1]{\textsf{sign}\left( #1 \right)}
%regel van de l'hopital
    \newcommand{\hopital}{\stackrel{\textsf{H}}{=}}
%vector functions
    % vector notation
    \newcommand{\vect}[1]{\overline{#1}} %large notation
    %(scalar product, <>-notation
    \newcommand{\scalprod}[2]{\left\langle #1,#2 \right\rangle}
    \newcommand{\scalprodv}[2]{\scalprod{\vec{#1}}{\vec{#2}}} %includes vector arrows
    \newcommand{\scalprodV}[2]{\scalprod{\vect{#1}}{\vect{#2}}}
    %vectorr product
    \newcommand{\vectprod}[2]{\left( #1 \times #2 \right)}
    \newcommand{\vectprodv}[2]{\vectprod{\vec{#1}}{\vec{#2}}} % includes vector arrows
    \newcommand{\vectprodV}[2]{\vectprod{\vect{#1}}{\vect{#2}}}
    %gradient, rotatie, divergentie
    \newcommand{\Dgrad}[1]{\textsf{grad}\,#1\;}
    \newcommand{\Ddiv}[1]{\textsf{div}\,#1\;}
    \newcommand{\Drot}[1]{\textsf{rot}\,#1\;}

%chemistry
    %concentration
    \newcommand{\conc}[1]{\left[ #1 \right]}
    %equilibrum arrows
    \newcommand{\evenwicht}{\rightleftharpoons}
    \newcommand{\reactie}{\rightarrow}
    %reactieconstante
    \newcommand{\K}{\,\textsf{K}}
    \newcommand{\Q}{\,\textsf{Q}}
    %p-notations
    \newcommand{\pH}{\,\textsf{pH}}
    \newcommand{\pOH}{\,\textsf{pOH}}
    \newcommand{\pK}{\,\textsf{pK}}
    \newcommand{\pKa}{\pK_A}
    \newcommand{\pKb}{\pK_B}
    \newcommand{\pKw}{\pK_W}
    %eenheden
    \newcommand{\eenheid}[1]{\,\textsf{#1}\,}
    \newcommand{\eenh}[1]{\eenheid{#1}}
    \newcommand{\Vr}{\,\textsf{Vr}\,}
    \newcommand{\molaliteit}{\mathbf{m}}
    \newcommand{\molar}[1]{\overline{#1}}
    \newcommand{\standard}[1]{#1^{\circ}}

% tango colors
   \definecolor{TangoButter1}{rgb}{0.9882, 0.9137, 0.3098}
   \definecolor{TangoButter2}{rgb}{0.9294, 0.8313, 0.0000}
   \definecolor{TangoButter3}{rgb}{0.7686, 0.6274, 0.0000}

   \definecolor{TangoOrange1}{rgb}{0.9882, 0.6863, 0.2431}
   \definecolor{TangoOrange2}{rgb}{0.9608, 0.4745, 0.0000}
   \definecolor{TangoOrange3}{rgb}{0.8078, 0.3608, 0.0000}

   \definecolor{TangoChocolate1}{rgb}{0.9137, 0.7255, 0.4314}
   \definecolor{TangoChocolate2}{rgb}{0.7569, 0.4902, 0.0667}
   \definecolor{TangoChocolate3}{rgb}{0.5608, 0.3490, 0.0078}

   \definecolor{TangoChameleon1}{rgb}{0.5412, 0.8863, 0.2039}
   \definecolor{TangoChameleon2}{rgb}{0.4510, 0.8235, 0.0863}
   \definecolor{TangoChameleon3}{rgb}{0.3059, 0.6039, 0.0235}

   \definecolor{TangoSkyBlue1}{rgb}{0.4471, 0.6235, 0.8118}
   \definecolor{TangoSkyBlue2}{rgb}{0.2039, 0.3961, 0.6431}
   \definecolor{TangoSkyBlue3}{rgb}{0.1255, 0.2902, 0.5294}

   \definecolor{TangoPlum1}{rgb}{0.6784, 0.4980, 0.6588}
   \definecolor{TangoPlum2}{rgb}{0.4588, 0.3137, 0.4824}
   \definecolor{TangoPlum3}{rgb}{0.3608, 0.2078, 0.4000}

   \definecolor{TangoScarletRed1}{rgb}{0.9373, 0.1608, 0.1608}
   \definecolor{TangoScarletRed2}{rgb}{0.8000, 0.0000, 0.0000}
   \definecolor{TangoScarletRed3}{rgb}{0.6431, 0.0000, 0.0000}

   \definecolor{TangoScarletRed1}{rgb}{0.9373, 0.1608, 0.1608}
   \definecolor{TangoScarletRed2}{rgb}{0.8000, 0.0000, 0.0000}
   \definecolor{TangoScarletRed3}{rgb}{0.6431, 0.0000, 0.0000}

   \definecolor{TangoAluminium1}{rgb}{0.9333, 0.9333, 0.9255}
   \definecolor{TangoAluminium2}{rgb}{0.8275, 0.8431, 0.8118}
   \definecolor{TangoAluminium3}{rgb}{0.7294, 0.7412, 0.8392}
   \definecolor{TangoAluminium4}{rgb}{0.5333, 0.5412, 0.5216}
   \definecolor{TangoAluminium5}{rgb}{0.3333, 0.3412, 0.3255}
   \definecolor{TangoAluminium6}{rgb}{0.1804, 0.2039, 0.2118}

% opmaak
    \newcommand{\opmerking}{\par\textbf{\color{TangoScarletRed3}{Opmerking: }}}
    \newcommand{\pro}{$\Box\!\!\!\!$\color{TangoChameleon3}{\ding{52}}}
    \newcommand{\con}{$\Box\!\!\!\!$\color{TangoScarletRed2}{\ding{56}}$\;$}
    \newcommand{\warn}{$\Box\!\!\!\!\!$\color{TangoSkyBlue2}{\ding{72}}$\;$}

% average over time
    \newcommand{\average}[1]{\left\langle #1\right\rangle}

  
  
  
 %lay-out
  \hypersetup{colorlinks,%
            citecolor=black,%
            filecolor=black,%
            linkcolor=black,%
            urlcolor=black,%
            pdfauthor={Egon Geerardyn},%
            pdftitle={Formules Wiskunde},%
            plainpages=false}%,%
            %pdfpagelabels}
  \pdfpagewidth=\paperwidth
  \pdfpageheight=\paperheight
 %margins
\begin{document}
  \addtolength{\textwidth}{3cm}
  \maketitlepage
%\thispagestyle{empty}%
%  \null
%  \vfill
%  \begin{center}\leavevmode
%    \normalfont
%    %{\Large\raggedleft schooljaren 2004 -- 2005 -- 2006\par}%
%    {\Large\raggedleft Ingenieurswetenschappen\par}%
%    \hrulefill\par
%    {\Huge\raggedright Formularium Wiskunde \large \revisie\   \footnotesize (\today)\par }%
%  \end{center}%
%  \vfill
%  \null
%  \newpage
  \newpage 
  \setlength{\voffset}{-2cm}

  \section*{Voorwoord}
\label{sec:Voorwoord}
  Deze uitgave is geen officiële uitgave van de Vrije Universiteit Brussel, slechts een formularium gemaakt door een student.
  Mogelijk staan er hier of daar nog fouten in, indien u er tegenkomt,
  stuur gerust een mailtje naar \hreftt{mailto:egon.geerardyn@vub.ac.be}{egon.geerardyn@vub.ac.be}.\par

  \begin{quote}
    Copyright \copyright{}  Egon Geerardyn.\par
    Permission is granted to copy, distribute and/or modify this document
    under the terms of the GNU Free Documentation License, Version 1.2
    or any later version published by the Free Software Foundation;
    with no Invariant Sections, no Front-Cover Texts, and no Back-Cover Texts.
    A copy of the license is included in the section entitled ``GNU
    Free Documentation License'' in the source code and available on:
    \linktt{http://www.gnu.org/copyleft/fdl.html}.
  \end{quote}
  \noindent
  De \LaTeX -broncode is vrij beschikbaar onder GNU Free Document License. \par


  Mogelijk is er reeds een nieuwe versie beschikbaar op\par
  \hreftt{http://students.vub.ac.be/~egeerard/projects.html}{http://students.vub.ac.be/\tildefix egeerard/projects.html}
\section*{Referenties}
\begin{enumerate}
	\item \textsc{D. Lefeber}, \textit{Mechanica: Deel I}, Dienst Uitgaven VUB 2006.
	\item \textsc{D. Lefeber}, \textit{Mechanica: Deel II}, Dienst Uitgaven VUB 2006.
        \item \textsc{D. Lefeber}, \textit{Mechanica met ontwerpproject}, Polytechnische Kring 2007.
        \item \textsc{D. Van Hemelrijck}, \textit{Mechanica van materialen, mechanismen en vloeistoffen}, Pointcarré 2008.
	\item \textsc{D. Vandepitte}, \linktt{http://www.berekeningvanconstructies.be}, 2006.
\end{enumerate}



   \newpage
   \tableofcontents
   \newpage
  
 %Goniometrie
 \twocolumn
 \section{Goniometrie}
 \label{sec:Goniometrie}
   
    \subsection{Formules}
\label{sec:GonioFormules}

\paragraph{Hoofdeigenschap en afleidingen}
\label{sec:HoofdeigenschapGonio}

   \[
	   \sin^2 \alpha + \cos^2 \alpha = 1
   \]
   \[
	   1 + \tan^2 \alpha = \sec^2 \alpha 
   \]
   \[
	   1 + \cot^2 \alpha = \csc^2 \alpha 
   \]
  
\paragraph{Som-- en verschilformules}
\label{sec:SomVerschilfGonio}

   \[
	   \sin( \alpha \pm \beta )  = \sin \alpha \cos \beta \pm \cos \alpha \sin \beta 
   \]
   \[
	   \cos( \alpha \pm \beta )  = \cos \alpha \cos \beta \mp \sin \alpha \sin \beta 
   \]
   \[
	   \tan( \alpha \pm \beta )  = \frac{\tan\alpha \pm \tan \beta}{1 \mp \tan\alpha\tan\beta} 
   \]
  
\paragraph{Verdubbelingsformules}
\label{sec:VerdubbelingsformulesGonio}
   \begin{eqnarray*}
	   \sin 2\alpha & = & 2 \sin \alpha \cdot \cos \alpha\\
	                & = & \frac{2 \tan \alpha }{1 + \tan^2 \alpha}
   \end{eqnarray*}
   \begin{eqnarray*}
	   \cos 2\alpha & = & \cos^2 \alpha - \sin^2 \alpha\\
	                & = & 1 - 2 \sin^2 \alpha\\
	                & = & 2 \cos^2 \alpha - 1\\
	                & = & \frac{1-\tan^2 \alpha}{1+\tan^2 \alpha}
   \end{eqnarray*}
   \begin{eqnarray*}
	   \tan 2\alpha & = & \frac{2 \tan \alpha}{1- \tan^2 \alpha}
   \end{eqnarray*}  
  
\paragraph{t-formules}
\label{sec:tFormulesGonio}

 ( $t = \tan \frac{\alpha}{2} $ )
   \[
	   \sin \alpha = \frac{2t}{1+t^2} 
   \]
   \[
	   \cos \alpha = \frac{1-t^2}{1+t^2} 
   \]
   \[
	   \tan \alpha = \frac{2t}{1-t^2} 
   \]
  
\paragraph{Productformules}
\label{sec:ProductformulesGonio}

   \[
	   \sin 3\alpha = 3 \sin \alpha - 4 \sin^3 \alpha
   \]
   \[
	   \cos 3\alpha = 4 \cos^3 \alpha - 3\cos\alpha 
   \]
   \[
	   \sin^2 \alpha = \frac{1-\cos 2\alpha}{2}
   \]
   \[
	   \cos^2 \alpha = \frac{1+\cos 2\alpha}{2} 
   \]
  \newpage
  
\paragraph{Formules van Simpson (product naar som)}
\label{sec:FormulesVanSimpsonPS}
   \[
	   2\sin\alpha \cdot \cos\beta = \sin ( \alpha + \beta ) + \sin ( \alpha - \beta )
   \]
   \[
	   2\cos\alpha \cdot \cos\beta = \cos ( \alpha + \beta ) + \cos ( \alpha - \beta ) 
   \]
   \[
	   2\sin\alpha \cdot \sin\beta = \cos ( \alpha - \beta ) - \cos ( \alpha + \beta ) 
   \]
   \[
	   2\cos\alpha \cdot \sin\beta = \sin ( \alpha + \beta ) - \sin ( \alpha - \beta ) 
   \]
   
\paragraph{Formules van Simpson (som naar product)}
\label{sec:FormulesVanSimpsonSP}

   \[
	   \sin p + \sin q = 2\sin\frac{p+q}{2}\cos\frac{p-q}{2} 
   \]
   \[
	   \sin p - \sin q = 2\cos\frac{p+q}{2}\sin\frac{p-q}{2} 
   \]
   \[
	   \cos p + \cos q = 2\cos\frac{p+q}{2}\cos\frac{p-q}{2} 
   \]
   \[
	   \cos p - \cos q = -2\sin\frac{p+q}{2}\sin\frac{p-q}{2} 
   \]
    \onecolumn
     \subsection{Vergelijkingen}
 \label{sec:goniovgl}
  \[
	   \sin x = \sin \alpha
	   \Leftrightarrow 
	   x = \alpha + k \cdot 2\pi \vee x = \pi - \alpha + k \cdot 2\pi  
	   \qquad
	   \textrm{ met } k \in \mathbb{Z}
  \]
  \[
	   \cos x = \cos \alpha 
	   \Leftrightarrow
	   x = \alpha + k \cdot 2\pi \vee x = - \alpha + k \cdot 2\pi
	   \qquad
	   \textrm{ met } k \in \mathbb{Z}
  \]
  \[
	   \tan x = \tan \alpha 
	   \Leftrightarrow
	   x = \alpha + k \cdot \pi 
	   \qquad
	   \textrm{ met } k \in \mathbb{Z}
  \]
  \[
	   \cot x = \cot \alpha 
	   \Leftrightarrow
	   x = \alpha + k \cdot \pi 
	   \qquad
	   \textrm{ met } k \in \mathbb{Z}
  \]

     \subsection{Cyclometrie}
 \label{sec:cyclo}
 \[
	   \argsin x = \alpha \Leftrightarrow \sin \alpha = x \wedge \alpha \in \left[ -\frac{\pi}{2},\frac{\pi}{2} \right]
  \]
  \[
	   \argcos x = \alpha \Leftrightarrow \cos \alpha = x \wedge \alpha \in \left[ 0,\pi \right]
  \]
  \[
	   \argtan x = \alpha \Leftrightarrow \tan \alpha = x \wedge \alpha \in \left] -\frac{\pi}{2},\frac{\pi}{2} \right[
  \]
  \[
	   \argcot x = \alpha \Leftrightarrow \cot \alpha = x \wedge \alpha \in \left] 0,\pi \right[
  \]
  \begin{center}
	\begin{tabular}{|c||c|c|c|c|}
	  \hline
	  $\swarrow$  & $\sin$ 										& $\cos$ 										& $\tan$ & $\cot$\\
	  \hline \hline
	  $\argsin x$ & $x$ 											& $\sqrt{1-x^2}$ 						&	$\frac{x}{\sqrt{1-x^2}}$ & $\frac{\sqrt{1-x^2}}{x}$ \\
	  \hline
	  $\argcos x$ & $\sqrt{1-x^2}$ 						& $x$ 											& $\frac{\sqrt{1-x^2}}{x}$ & $\frac{x}{\sqrt{1-x^2}}$ \\
	  \hline
	  $\argtan x$ & $\frac{x}{\sqrt{1+x^2}}$ 	& $\frac{1}{\sqrt{1+x^2}}$ 	& $x$ & $\frac{1}{x}$ \\
	  \hline
	  $\argcot x$ & $\frac{1}{\sqrt{1+x^2}}$ 	& $\frac{x}{\sqrt{1+x^2}}$ 	& $\frac{1}{x}$ & $x$ \\
	  \hline 
	\end{tabular}
\end{center}

    \subsection{Hyperbolische functies}
\label{sec:HyperbolischeFuncties}
  \begin{eqnarray*}
    \cosh x = \frac{e^x+e^{-x}}{2} & \qquad \qquad \qquad &%
    \tanh x = \frac{\sinh x}{\cosh x} = \frac{e^x-e^{-x}}{e^x+e^{-x}} \\
    %
    \sinh x = \frac{e^x-e^{-x}}{2} & \qquad \qquad \qquad &%
    \coth x = \frac{\cosh x}{\sinh x} = \frac{e^x+e^{-x}}{e^x-e^{-x}}
  \end{eqnarray*}
  
\paragraph{Hoofdeigenschap}
\label{sec:HoofdeigenschapHyperbolisch}
   \[
	   \cosh^2 \alpha - \sinh^2 \alpha = 1
   \]
  
\paragraph{Som- en verschilformules}
\label{sec:SomVerschilHyperbolisch}
   \[
	   \sinh\left( x \pm y \right)  = \sinh x \cosh y \pm \cosh x \sinh y 
   \]
   \[
	   \cosh\left( x \pm y \right)  = \cosh x \cosh y \mp \sinh x \sinh y 
   \]

\paragraph{Symmetrie van Hyperbolische functies}
\label{sec:SymmetrieHyperbolisch}

\begin{center}
	\begin{tabular}{|c||c|c|c|c||l|}
	  \hline
	  $\swarrow$ & $\sinh$ & $\cosh$ & $\tanh$ & $\coth$ & beschrijving\\
	  \hline \hline
	  $- x$ & $-\sinh x$ & $\cosh x$ & $-\tanh x$ & $-\coth x$ & tegengesteld\\
	  \hline
	\end{tabular}
\end{center}

\paragraph{Inverse functies}
\label{sec:invHyperbFunc}
\[
  \argsinh \left( x \right) = \ln \left( x + \sqrt{x^2+1} \right)
\]
\[
  \argcosh \left( x \right) = \ln \left( x + \sqrt{x^2-1} \right)
\]
\[
  \argtanh \left( x \right) = \frac{1}{2} \ln \frac{1+x}{1-x}
\]

\paragraph{Omzettingstabel}
\label{sec:OmzettingHyperbolisch}

\newcommand{\FSH}[1]{\sinh^{#1}\left(x\right)}
\newcommand{\FCH}[1]{\cosh^{#1}\left(x\right)}
\newcommand{\FTH}[1]{\tanh^{#1}\left(x\right)}
\newcommand{\FCT}[1]{\coth^{#1}\left(x\right)}
\begin{center}
\[
  \begin{array}{|l|c|c|c|c|}
  \hline    & \FSH{} & \FCH{} & \FTH{} & \FCT{} \\ \hline \hline
                                                             
  \FSH{} =  & \FSH{}%                        
            & \pm \sqrt{\FCH{2}-1}% 
            & \frac{\FTH{}}{\sqrt{1-\FTH{2}}}%
            & \pm \frac{1}{\sqrt{\FCT{2}-1}}%
            \\ \hline %
  \FCH{} =  & \sqrt{1+\FSH{2}}%
            & \FCH{}%
            & \frac{1}{\sqrt{1-\FTH{2}}}%
            & \frac{\abs{\FCT{}}}{\sqrt{\FCT{2}-1}}%
            \\ \hline %
  \FTH{} =  & \frac{\FSH{}}{\sqrt{1+\FSH{2}}}%
            & \pm \frac{\sqrt{\FCH{2}-1}}{\FCH{}}%
            & \FTH{}%
            & \frac{1}{\FCT{}}%
            \\ \hline %
  \FCT{} =  & \frac{\sqrt{1+\FSH{2}}}{\FSH{}}%
            & \pm \frac{\FCH{}}{\sqrt{\FCH{2}-1}}%
            & \frac{1}{\FTH{}}
            & \FCT{}%
            \\ \hline
  \end{array}
\]
waarin $\pm$ overeenkomt met het teken van $x$
\end{center}
    
 \subsection{Tabellen}
 \label{sec:tabel}
  
\paragraph{Frequente waarden}
\label{sec:FrequenteWaarden}

   \begin{center}
	  \begin{tabular}{|r|r||c|c|c|c|}
	    \hline
	    radialen & graden & $\sin \alpha$ & $\cos \alpha$ & $\tan \alpha$ & $\cot \alpha$ \\
	    \hline \hline
	    $0$ & $0\degree$ & $0$ & $1$ & $0$ & $^{+\infty} | _{-\infty}$ \\ 
	    \hline
	    $\frac{\pi}{6}$ & $30\degree$ & $\frac{1}{2}$ & $\frac{\sqrt{3}}{2}$ & $\frac{\sqrt{3}}{3}$ & $\sqrt{3}$ \\
	    \hline
	    $\frac{\pi}{4}$ & $45\degree$ & $\frac{\sqrt{2}}{2}$ & $\frac{\sqrt{2}}{2}$ & $1$ & $1$ \\
	    \hline
	    $\frac{\pi}{3}$ & $60\degree$ & $\frac{\sqrt{3}}{2}$ & $\frac{1}{2}$ & $\sqrt{3}$ & $\frac{\sqrt{3}}{3}$ \\
	    \hline
	    $\frac{\pi}{2}$ & $90\degree$ & $1$ & $0$ & $^{+\infty} | _{-\infty}$ & $0$ \\
	    \hline
	    $\frac{2\pi}{3}$ & $120\degree$ & $\frac{\sqrt{3}}{2}$ & $\frac{-1}{2}$ & $-\sqrt{3}$ & $\frac{-\sqrt{3}}{3}$ \\
	    \hline
	    $\frac{3\pi}{4}$ & $135\degree$ & $\frac{\sqrt{2}}{2}$ & $\frac{-\sqrt{2}}{2}$ & $-1$ & $-1$ \\
	    \hline
	    $\frac{5\pi}{6}$ & $150\degree$ & $\frac{1}{2}$ & $\frac{-\sqrt{3}}{2}$ & $\frac{-\sqrt{3}}{3}$ & $-\sqrt{3}$ \\
	    \hline
	    $\pi$ & $180\degree$ & $0$ & $-1$ & $0$ & $_{-\infty} | ^{+\infty}$ \\
	    \hline  
	    $\frac{7\pi}{6}$ & $210\degree$ & $\frac{-1}{2}$ & $\frac{-\sqrt{3}}{2}$ & $\frac{\sqrt{3}}{3}$ & $\sqrt{3}$ \\
	    \hline
	    $\frac{5\pi}{4}$ & $225\degree$ & $\frac{-\sqrt{2}}{2}$ & $\frac{-\sqrt{2}}{2}$ & $1$ & $1$ \\
	    \hline
	    $\frac{4\pi}{3}$ & $240\degree$ & $\frac{-\sqrt{3}}{3}$ & $\frac{-1}{2}$ & $\sqrt{3}$ & $\frac{\sqrt{3}}{3}$ \\
	    \hline
	    $\frac{3\pi}{2}$ & $270\degree$ & $\frac{1}{2}$ & $\frac{\sqrt{3}}{2}$ & $^{+\infty} | _{-\infty}$ & $0$ \\
	    \hline
	    $\frac{5\pi}{3}$ & $300\degree$ & $\frac{-\sqrt{3}}{2}$ & $\frac{1}{2}$ & $-\sqrt{3}$ & $\frac{-\sqrt{3}}{3}$ \\
	    \hline
	    $\frac{7\pi}{4}$ & $315\degree$ & $\frac{-\sqrt{2}}{2}$ & $\frac{\sqrt{2}}{2}$ & $-1$ & $-1$ \\
	    \hline
	    $\frac{11\pi}{6}$ & $330\degree$ & $\frac{-1}{2}$ & $\frac{\sqrt{3}}{2}$ & $\frac{-\sqrt{3}}{3}$ & $-\sqrt{3}$ \\
	    \hline
	    $2\pi$ & $360\degree$ & $0$ & $1$ & $0$ & $^{+\infty} | _{-\infty}$ \\
	    \hline
	  \end{tabular}	
   \end{center}
   
\paragraph{Verwante hoeken}
\label{sec:VerwanteHoeken}

\begin{center}
	\begin{tabular}{|c||c|c|c|c||l|}
	  \hline
	  $\swarrow$ & $\sin$ & $\cos$ & $\tan$ & $\cot$ & beschrijving\\
	  \hline \hline
	  $\alpha + k\cdot 2\pi $ & $\sin \alpha$ & $\cos \alpha$ & $\tan \alpha$ & $\cot \alpha$ & gelijk\\
	  \hline
	  $- \alpha$ & $-\sin \alpha$ & $\cos \alpha$ & $-\tan \alpha$ & $-\cot \alpha$ & tegengesteld\\
	  \hline
	  $\pi - \alpha$ & $\sin \alpha$ & $-\cos \alpha$ & $-\tan \alpha$ & $-\cot \alpha$ & supplementair\\
	  \hline
	  $\pi + \alpha$ & $-\sin \alpha$ & $-\cos \alpha$ & $\tan \alpha$ & $\cot \alpha$ & anti-supplementair\\
	  \hline
	  $\frac{\pi}{2} - \alpha$ & $\cos \alpha$ & $\sin \alpha$ & $\cot \alpha$ & $\tan \alpha$ & complementair\\
	  \hline
	  $\frac{\pi}{2} + \alpha$ & $\cos \alpha$ & $-\sin \alpha$ & $-\cot \alpha$ & $-\tan \alpha$ & anti-complementair\\
	  \hline 
	\end{tabular}
\end{center}



    
 \newpage
 \section{Vlakke meetkunde}
 \label{sec:VlakkeMeetkunde}
     \subsection{Rechthoekige driehoeken}
 \label{sec:rh_drhk}
   In een rechthoekige driehoek $\triangle ABC$ met $\widehat{A} = 90\degree$ geldt:
   \[
	   \sin \widehat{B} = \frac{\textrm{overstaande zijde}}{\textrm{schuine zijde}} = \frac{b}{a}
   \]
   \[
	   \cos \widehat{B} = \frac{\textrm{aanliggende zijde}}{\textrm{schuine zijde}} = \frac{c}{a}
   \]  
   \[
	   \tan \widehat{B} = \frac{\sin \widehat{B}}{\cos \widehat{B}} = \frac{\textrm{overstaande zijde}}{\textrm{aanliggende zijde}} = \frac{b}{c}
   \]
   \[
	   \cot \widehat{B} = \frac{1}{\tan \widehat{B}} = \frac{\textrm{aanliggende zijde}}{\textrm{overstaande zijde}} = \frac{c}{b}
   \]
   \[
	   \sec \widehat{B} = \frac{1}{\cos \widehat{B}} = \frac{\textrm{schuine zijde}}{\textrm{aanliggende zijde}} = \frac{a}{c}
   \]
   \[
	   \csc \widehat{B} = \frac{1}{\sin \widehat{B}} = \frac{\textrm{schuine zijde}}{\textrm{overstaande zijde}} = \frac{a}{b}
   \]
   \[
	   a^{2} = b^{2} + c^{2} \qquad \textrm{Stelling van Pythagoras}
   \]
     \subsection{Willekeurige driehoeken}
 \label{sec:wk_drhk}
  
\paragraph{Sinusregel}
\label{sec:Sinusregel}
 (waarbij $r$ de straal van de omgeschreven cirkel is)
   \[
	   \frac{a}{\sin \widehat{A}} = \frac{b}{\sin \widehat{B}} = \frac{c}{\sin \widehat{C}} = 2 \cdot r
   \]
  
\paragraph{Cosinusregel}
\label{sec:Cosinusregel}
   \[
	   a^{2}=b^{2}+c^{2}-2bc\cdot\cos\widehat{A}
   \]
%   \[
%	   b^{2}=a^{2}+c^{2}-2ac\cdot\cos\widehat{B}
%   \]
%   \[
%	   c^{2}=a^{2}+b^{2}-2ab\cdot\cos\widehat{C}
%   \]
  
  
\paragraph{Tangensregel}
\label{sec:Tangensregel}
   \[
	   \frac{a+b}{a-b}=\frac{\tan\frac{A+B}{2}}{\tan\frac{A-B}{2}}
   \]
%   \[
%	   \frac{b+c}{b-c}=\frac{\tan\frac{B+C}{2}}{\tan\frac{B-C}{2}}
%   \]
%   \[
%	   \frac{a+c}{a-c}=\frac{\tan\frac{A+C}{2}}{\tan\frac{A-C}{2}}
%   \]
   
  
\paragraph{Algemene formules in een willekeurige driehoek $\Delta ABC$ :}
\label{sec:AlgemeneFormulesWKdriehoek}
   \[
	   a + b + c = 2p \Leftrightarrow a + b - c = 2(p - c)
   \]
%   \[
%	   a + b - c = 2(p - c)
%   \]   
%   \[
%	   a + c - b = 2(p - b)
%   \] 
%   \[
%	   b + c - a = 2(p - a)
%   \]
   
  
\paragraph{Cosinus van de halve hoek}
\label{sec:CosinusHalveHoek}
   \[
	    \cos\frac{\widehat{A}}{2}=\sqrt{\frac{p\cdot\left(p-a\right)}{b\cdot c}}
   \]  
%   \[
%	    \cos\frac{\widehat{B}}{2}=\sqrt{\frac{p\cdot\left(p-b\right)}{a\cdot c}}
%   \]
%   \[
%	    \cos\frac{\widehat{C}}{2}=\sqrt{\frac{p\cdot\left(p-c\right)}{a\cdot b}}
%   \]
  
\paragraph{Sinus van de halve hoek}
\label{sec:SinusHalveHoek}
   \[
	    \sin\frac{\widehat{A}}{2}=\sqrt{\frac{\left(p-b\right) \cdot \left(p-c\right)}{b \cdot c}}
   \]
%   \[
%	    \sin\frac{\widehat{B}}{2}=\sqrt{\frac{\left(p-a\right) \cdot \left(p-c\right)}{a \cdot c}}
%   \]
%   \[
%	    \sin\frac{\widehat{C}}{2}=\sqrt{\frac{\left(p-a\right) \cdot \left(p-b\right)}{a \cdot b}}
%   \]
  
\paragraph{Tangens van de halve hoek}
\label{sec:TangensVanDeHalveHoek}
   \[
	    \tan\frac{\widehat{A}}{2}=\sqrt{\frac{\left(p-b\right) \cdot \left(p-c\right)}{p\cdot\left(p-a\right)}}
   \]
%   \[
%	    \tan\frac{\widehat{B}}{2}=\sqrt{\frac{\left(p-a\right) \cdot \left(p-c\right)}{p\cdot\left(p-b\right)}}
%   \]
%   \[
%	    \tan\frac{\widehat{C}}{2}=\sqrt{\frac{\left(p-a\right) \cdot \left(p-b\right)}{p\cdot\left(p-c\right)}}
%   \]

\paragraph{Lengte van de hoogtelijnen $h$ in een driehoek}
\label{sec:LengteVanDeHoogtelijnen}
   \[
	   h_{A}= b \cdot \sin \widehat{C} = c \cdot \sin \widehat{B}
   \]
%   \[
%	   h_{B}= a \cdot \sin \widehat{C} = c \cdot \sin \widehat{A}
%   \]
%   \[
%	   h_{C}= a \cdot \sin \widehat{B} = b \cdot \sin \widehat{A}
%   \] 
  
\paragraph{Lengte van de bissectrices $d$ in een driehoek}
\label{sec:LengteVanDeBissectrices}
   \[
	   d_{A}=\frac{2 \cdot bc \cdot \cos \frac{\widehat{A}}{2}}{b + c}
   \]
%   \[
%	   d_{B}=\frac{2 \cdot ac \cdot \cos \frac{\widehat{B}}{2}}{a + c}
%   \]
%   \[
%	   d_{C}=\frac{2 \cdot ab \cdot \cos \frac{\widehat{C}}{2}}{a + b}
%   \]
  
\paragraph{Oppervlakteformules}
\label{sec:OppervlakteformulesWKDriehoek}
 ($S$)
   \[
	    S= \frac{1}{2}\cdot bc \cdot \sin \widehat{A} 
%	     = \frac{1}{2}\cdot ac \cdot \sin \widehat{B}
%	     = \frac{1}{2}\cdot ab \cdot \sin \widehat{C}
   \]
   \[
	    S=\sqrt{p \cdot \left(p-a\right) \cdot \left(p-b\right) \cdot \left(p-c\right)} 
	    \qquad \textrm{(Formule van \textsl{Heroon})}
   \]
  
     \subsection{Regelmatige Veelhoeken}
 \label{sec:reg_veelh}
  In een regelmatige $n$--hoek met straal $r$ geldt
   
\subparagraph{de middelpuntshoek $2\alpha_n$}
\label{sec:MiddelpuntshoekRegVeelh}
   \[
	    2\alpha_n = \frac{2\pi}{n} \Leftrightarrow \alpha_n= \frac{\pi}{n}
   \]
  
\subparagraph{de lengte van de apothema $a_n$}
\label{sec:LengteApothemaRegVeelh}
   \[
	   a_n= r \cdot \cos \frac{\pi}{n}
   \]   
   
\subparagraph{de lengte van de zijde $z_n$}
\label{sec:LengteZijdeRegVeelh}
   \[
	   z_n= 2 \cdot r \cdot \sin \frac{\pi}{n}
   \]   
   
\subparagraph{de omtrek $O$}
\label{sec:OmtrekRegVeelh}
   \[
     O = n \cdot z_n = 2 \cdot n \cdot r \cdot \sin \frac{\pi}{n}
   \]
   
\subparagraph{de oppervlakte $S$}
\label{sec:OppervlakteRegVeelh}
   \[
     S = \frac{n \cdot z_n \cdot a_n}{2} = n \cdot r^2 \cdot \sin\frac{\pi}{n}\cos\frac{\pi}{n}
   \]
     \subsection{Cirkel}
 \label{sec:cirkel}
  In een cirkel met straal $r$ en een gegeven middelpuntshoek $2\alpha$ geldt voor\par
  
\paragraph{de lengte $d$ van de bijhorende koorde}
\label{sec:LengteKoordeCirkel}
   \[
	   d = 2 \cdot r \cdot \sin \alpha
   \]
  
\paragraph{de lengte $d$ van het middelpunt tot de bijhorende koorde}
\label{sec:AfstandMiddelpuntKoordeCirkel}
   \[
	   d = r \cdot \cos \alpha
   \]
      
 
   \newpage
 \section{Analyse}
 \label{sec:Analyse}
   
\subsection{Logaritmen}
\label{sec:Logaritmen}
%\[ \forall a,b \in \mathbb{R}_0^+ \ \left{1\right} : \forall x,y \in \mathbb{R}^+_0 : \forall n \in \mathbb{R}: \]
\[
  ^a\log x = y \Leftrightarrow x = a^y
\]
%\[
% ^a\log a^y = y
%\]
\[
  x = a^{^a\log x}
\]
\[
 ^a\log \left( \prod x_i^{n_i} \right) = \sum n_i\;^a\log x_i
\]
%\[
% ^a\log\frac{x_1}{x_2} = ^a \log x_1 - ^a \log x_2
%\]
%\[
% ^a\log\frac{1}{x} = - ^a \log x
%\]
%\[
% ^a\log x^n = n \cdot ^a \log x
%\]
\[
 ^b\log x = \frac{^a\log x}{^a\log b}
\]
\[
 ^b\log a = \frac{1}{^a \log b}
\]

   \subsection{Taylorreeksen}
\label{sec:Taylorreeksen}
  \subsubsection{Algemene benaderende reeksen}
  \label{sec:algbBenadering}
    \paragraph{Taylor-reeks} benadering van $f\left(x\right)$ in $x=a$ van graad $n$
      \[
        P_n\left(x\right) = \sum_{n=0}^n \frac{f^{\left(n\right)}\left(a\right)}{n!}\left(x-a\right)^n + r_n\left(x\right)
      \]
    \paragraph{McLaurin-reeks} benadering van $f\left(x\right)$ in $x=0$ van graad $n$ 
      \[
        P_n\left(x\right) = \sum_{n=0}^n \frac{f^{\left(n\right)}\left(0\right)}{n!}x^n + r_n\left(x\right)
      \]
    \paragraph{Restterm van Lagrange van orde $n$}
      \[
        r_n\left(x\right) = \frac{f^{\left(n+1\right)}\left( \xi \right)}{\left(n+1\right)!}\left(x-a\right)^{n+1}
        \quad \mbox{met} \;
        \xi \in/; ]a,x[
      \]
    \paragraph{Restterm van de Liouville van orde $n$}
      \[
        r_n\left(x\right) = \frac{\left(x-a\right)^n}{n!}\lambda \left( x \right)
        \quad \mbox{met} \;
        \lim_{x \to a} \lambda (x) = 0
      \]
    \paragraph{Restterm van Lagrange van orde $n$ voor een McLaurin-reeks}
      \[
        r_n\left(x\right) = \frac{f^{\left(n+1\right)}\left( \theta x \right)}{\left(n+1\right)!}x^{n+1}
        \quad \mbox{met} \;
        \theta \in/; ]0,1[ \;\mbox{of}\; \theta x \in/; ]0,x[
      \]
  \subsubsection{Specifieke Taylor/McLaurin-reeksen}
  \label{sec:FreqBenadReeks}
    \[
      \forall x :
      e^{x} = \sum^{\infin}_{n=0} \frac{x^n}{n!}
      = x + \frac{x^2}{2} + \frac{x^3}{6} + \frac{x^4}{24} + \frac{x^5}{120} + \frac{x^6}{720} + \ldots
    \]
    \[
      \forall \left| x \right| < 1 :
      \ln(1+x) = \sum^{\infin}_{n=0} \frac{(-1)^n}{n+1} x^{n+1}
      = x-\frac{x^2}{2}+\frac{x^3}{3}-\frac{x^4}{4}+ \ldots
    \]
    \[
      \forall x :
      \sin x = \sum^{\infty}_{n=0} \frac{(-1)^n}{(2n+1)!} x^{2n+1} 
      = x - \frac{x^3}{6} + \frac{x^5}{120} - \frac{x^7}{5040} + \ldots
    \]
    \[
      \forall x :
      \cos x = \sum^{\infty}_{n=0} \frac{(-1)^n}{(2n)!} x^{2n} 
      = x - \frac{x^2}{2} + \frac{x^4}{24} - \frac{x^6}{720} + \ldots
    \]
    \[
      \forall \left| x \right| < \frac{\pi}{2} :
      \sec x = \sum^{\infty}_{n=0} \frac{(-1)^n E_{2n}}{(2n)!} x^{2n}
      = 1 + \frac{x^2}{2}+\frac{5x^4}{24}+\frac{61x^6}{720}+\frac{277x^8}{8064} + \ldots
    \]
    \[
      \forall \left| x \right| < 1 :
      \argsin x = \sum^{\infty}_{n=0} \frac{(2n)!}{4^n (n!)^2 (2n+1)} x^{2n+1}
      = x + \frac{x^3}{6} + \frac{3x^5}{40} + \frac{5x^7}{112} + \frac{35x^9}{1152} + \ldots
    \]
    \[
      \forall \left| x \right| \leq 1 :
      \argtan x = \sum^{\infty}_{n=0} \frac{(-1)^n}{2n+1} x^{2n+1}
      = x - \frac{x^3}{3} + \frac{x^5}{5} - \frac{x^7}{7} + \frac{x^9}{9} + \ldots
    \]
    \[
      \forall x :
      \sinh \left(x\right) = \sum^{\infty}_{n=0} \frac{1}{(2n+1)!} x^{2n+1}
      = x + \frac{x^3}{6} + \frac{x^5}{120} + \frac{x^7}{5040} + \frac{x^9}{362880} + \ldots
    \]
    \[
      \forall x :
      \cosh \left(x\right) = \sum^{\infty}_{n=0} \frac{1}{(2n)!} x^{2n}
      = 1 + \frac{x^2}{2} + \frac{x^4}{24} + \frac{x^6}{720} + \frac{x^8}{40320} + \ldots
    \]
    \[
      \forall \left|x\right| < \frac{\pi}{2} :
      \tanh\left(x\right) = \sum^{\infty}_{n=1} \frac{B_{2n} 4^n (4^n-1)}{(2n)!} x^{2n-1}
      = x - \frac{x^3}{3} + \frac{2x^5}{15}- \frac{17x^7}{315}+ \frac{62x^9}{2835} + \ldots
    \]
   \subsection{Limieten}
\label{sec:Limieten}

\paragraph{Limiet van een rij}
\[
  \lim_{n \to \infty} = l 
  \quad \Leftrightarrow \quad
  \forall \varepsilon > 0 : \exists\; n > N_\varepsilon : l - \varepsilon< u_n < l + \varepsilon
\]

\paragraph{Minimum en Maximum}
\[
  \min A \in A : \forall a \in A : \min A \leq a
\]
\[
  \max A \in A : \forall a \in A : a \leq \max A
\]

\paragraph{Infimum en Supremum}
\[
  \forall \varepsilon > o : \exists\; a \in A :  \inf A \leq a < \inf A + \varepsilon
\]
\[
  \forall \varepsilon > o : \exists\; a \in A :  \sup A  - \varepsilon < a \leq \sup A
\]

\paragraph{Bewerkingen met limieten}
\[
  \lim_{n \to \infty} u_n + v_n = \lim_{n \to \infty} u_n + \lim_{n \to \infty} v_n
\]
\[
  \lim_{n \to \infty} u_n \cdot v_n = \lim_{n \to \infty} u_n \cdot \lim_{n \to \infty} v_n
\]
\[
  \lim_{n \to \infty} \abs{u_n} = \abs{\lim_{n \to \infty} u_n}
\]
\[
  \lim_{n \to \infty} \frac{v_n}{u_n} = \frac{\lim_{n \to \infty} v_n}{\lim_{n \to \infty} u_n}
\]

\paragraph{Cauchyrij}
\[
  \forall \varepsilon > 0 : \exists\; N > 0 : n,m > N \Rightarrow \abs{u_n-u_m} < \varepsilon
\]

\paragraph{Limiet van een functie}
\[
  \lim_{x \to a} f\left(x\right) = b 
  \quad \Leftrightarrow \quad
  \forall \varepsilon > 0 : \exists\; \delta > 0 : 0 < \abs{x-a} < \delta 
                            \Rightarrow \abs{f(x) - b} < \varepsilon
\]
\[
  \lim_{\vec{x} \to \vec{a}} \vec{f}\left(\vec{x}\right) = \vec{b} 
  \quad \Leftrightarrow \quad
  \forall \varepsilon > 0 : \exists\; \delta > 0 : 0 < \norm{\vec{x} - \vec{a}} < \delta 
                            \Rightarrow \norm{f(x) - b} < \varepsilon
\]

\paragraph{Oneigenlijke limieten}
\[
  \lim_{x \to + \infty} f\left(x\right) = b 
  \quad \Leftrightarrow \quad
  \forall \varepsilon > 0 : \exists\; \alpha \in \mathbb{R} : x > \alpha
                            \Rightarrow \abs{f(x) - b} < \varepsilon
\]
\[
  \lim_{x \to - \infty} f\left(x\right) = b 
  \quad \Leftrightarrow \quad
  \forall \varepsilon > 0 : \exists\; \alpha \in \mathbb{R} : x < \alpha
                            \Rightarrow \abs{f(x) - b} < \varepsilon
\]
\[
  \lim_{x \to a} f\left(x\right) = + \infty 
  \quad \Leftrightarrow \quad
  \forall \alpha \in \mathbb{R} : \exists\; \delta > 0 : 0 < \abs{x-a} < \delta
                                  \Rightarrow f(x) > \alpha
\]
\[
  \lim_{x \to a} f\left(x\right) = - \infty 
  \quad \Leftrightarrow \quad
  \forall \alpha \in \mathbb{R} : \exists\; \delta > 0 : 0 < \abs{x-a} < \delta
                                  \Rightarrow f(x) < \alpha
\]
\paragraph{Rekenregels voor oneindig}
\[
  \forall a \in \mathbb{R}^0\; : \; 
  a \cdot \left(\pm \infty\right) = \sign{a} \left(\pm \infty \right)
\]
\[
  \forall a \in \mathbb{R} \; : \; 
  a + \left(\pm \infty\right) = \pm \infty
\]
\paragraph{Onbepaalde vormen}
\[
  0 \cdot \left( \pm \infty \right) \;,\;
  \frac{0}{0} \;,\;
  \pm \frac{\infty}{\infty} \;,\;
  \pm \infty \mp \infty \;,\;
  0^0 \;,\;
  \left( \pm \infty \right)^0 \;,\;
  1^{\pm \infty}
\]

\paragraph{Merkwaardige limieten}
\[
  \lim_{x \to +\infty} \left( 1 + \frac{1}{x} \right)^x = e
\]

\paragraph{Regel van de l'Hopital}
\[
  \mbox{geldig voor: } \quad \lim_{x \to a} f(x) = \lim_{x \to a} g(x) = 0 \quad \mbox{of} \quad  \lim_{x \to a} f(x) = \lim_{x \to a} g(x) = \pm \infty
\]
\[
  \lim_{x \to a} \frac{f(x)}{g(x)}
  \hopital
  \lim_{x \to a} \frac{f(x)}{g(x)}
\]

\paragraph{Omvormen van limieten} bij onbepaalde vormen om de regel van de l'Hopital toe te kunnen passen:
\[
  \lim_{x \to a} f(x) - g(x) = \lim_{x \to a} f(x)g(x)\left(\frac{1}{g(x)} - \frac{1}{f(x)}\right) 
\]
\[
  \lim_{x \to a} f(x)g(x) = \lim_{x \to a} \frac{f(x)}{\frac{1}{g(x)}}
                          = \lim_{x \to a} \frac{g(x)}{\frac{1}{f(x)}}
\]
\[
  \lim_{x \to a} f(x)^{g(x)} = \exp \lim_{x \to a} g(x) \ln f(x)
\]
   \twocolumn
\subsection{Afgeleiden}
\label{sec:Afgeleiden}

\paragraph{Definitie}
\[
 \D_x f \left( a \right) = \lim_{x \to a} \frac{ f\left(x\right) - f\left(a\right)}{x-a}
                         = \frac{ \d{f}}{\d{x}}
\]
\paragraph{Richtingsafgeleide} volgens $\vec{u}$
\[
 \D_{\vec{u}} f \left( \vec{a} \right) = \lim_{h \to 0} \frac{ f\left(\vec{a} + h\vec{u}\right) - f\left(\vec{a}\right)}{h}
\]
\paragraph{Rekenregels}
\[
  \D \alpha = 0
\]
\[
  \D x = 1
\]
\[
  \D\left(\alpha f \pm \beta g\right) = \alpha \D f \pm \beta \D g
\]
\[
  \D\left(f \cdot g\right) = f\D g + g\D f
\]
\[
  \D\left(\frac{f}{g}\right) = \frac{g\D f - f\D g}{g^2}
\]
\[
  \D x^n = nx^{n-1}
\]
\[
  \D f^m = m\cdot f^{m-1} \cdot \D f \qquad \textrm{met $m \in \mathbb{Q}_0$}
\]
\paragraph{Kettingregel}
\[
  \D \left[\left(g \circ f\right)\left(x\right)\right] = \D g\left[f\left(x\right)\right] \cdot \D f\left(x\right)
\]
\[
  \frac{dy}{dx} = \frac{dy}{du} \cdot \frac{du}{dx} \qquad \textrm{met $u = f\left(x\right)$}
\]
\paragraph{Afgeleide van de inverse functie}
\[
  \D f^{-1} = \frac{1}{\D f \left( f^{-1} \left( x \right) \right) }
\]
\[
  \frac{\d{x}}{\d{y}} = \left(\frac{\d{y}}{\d{x}} \right)^{-1}
\]
\paragraph{Frequente vormen}
\[
  \D e^x = e^x
\]
\[
  \D a^x = a^x \ln a
\]
\[
  \D\left(\ln x\right) = \frac{1}{x}
\]
\[
  \D\left(^a \log x\right)= \frac{1}{x \cdot \ln a}
\]
\[
  \D\left(f^g\right) = f^g \left( Dg \ln f + \frac{f}{g} Df \right)
\]
\newpage
\[
  \D\left(\sin x\right) = \cos x
\]
\[
  \D\left(\cos x\right) = -\sin x
\]
\[
  \D\left(\tan x\right) = \frac{1}{\cos^2 x} = \sec^2 x
\]
\[
  \D\left(\cot x\right) = \frac{ -1}{\sin^2 x} = - \csc^2 x
\]
\[
  \D\left(\argsin x\right) = \frac{1}{\sqrt{1-x^2}}
\]
\[
  \D\left(\argcos x\right) = \frac{-1}{\sqrt{1-x^2}}
\]
\[
  \D\left(\argtan x\right) = \frac{1}{1+x^2}
\]
\[
  \D\left(\argcot x\right) = \frac{-1}{1+x^2}
\]
\[
  \D\left(\sinh x\right) = \cosh x 
\]
\[
  \D\left(\cosh x\right) = \sinh x
\]
\[
  \D\left(\tanh x\right) = \frac{1}{\cosh^2 x}
\]
\[
  \D\left(\coth x\right) = \frac{-1}{\sinh^2 x}
\]
\[
  \D\left(\argsinh x\right) = \frac{1}{\sqrt{1 + x^2}}
\]
\[
  \D\left(\argcosh x\right) = \frac{1}{\sqrt{x^2 - 1}}
\]
\[
  \D\left(\argtanh x\right) = \frac{1}{1-x^2}
\]
\[
  \D\left(\argtanh x\right) = \frac{1}{1-x^2}
\]

\onecolumn
   \subsection{Differentialen}
\label{sec:Differentialen}
\[
  \d{f\left(x\right)} = \frac{\d{f}}{\d{x}} \d{x}
                      = \D_x\left(f\right) \d{x}
\]
\paragraph{Totale differentiaal van een functie}
\[
  f \left( x_1 , \ldots , x_N \right) 
  \Rightarrow 
  \d{f} = \sum_{i=1}^N \frac{\pd{f}}{\pd{x_i}}\d{x_i} 
\]
Differentiaal van hogere orde (voor een functie van 2 veranderlijken)
\[
  f \left( x,y \right) 
  \Rightarrow
  \d^n{f} = \sum_{i=0}^{n} \combination{n}{k} \frac{\pd^{n}{f}}{\pd{x}^{k}\pd{y}^{n-k}} \d{x}^{k} \d{y}^{n-k}
\]
Differentiaal van hogere orde (de macht $.^{\left[n\right]}$ is slechts formeel!)
\[
  f \left( x_1 , \ldots , x_N \right)
  \Rightarrow
  \d^n{f} = \left(\sum_{k=1}^{N} \frac{\pd{}}{\pd{x_i}}\d{x_i}  \right)^{\left[n\right]}f
\]

\paragraph{Kettingregel}
\[
  \frac{\d{f}}{\d{t}} = \sum^m_{i=1} \frac{\pd{f}}{\pd{x_i}} \frac{\d{x_i}}{\d{t}}
\]
   \twocolumn
\subsection{Integratie}
\label{sec:Integralen}

\subsubsection{Frequente integralen}
\label{sec:FundamenteleIntegralen}

\[
 \int \d{x} = x + c
\]

\[
 \int x^n \d{x} = \frac{x^{n+1}}{n+1} + c \quad \mbox{ met } n \in \mathbb{R} \not \; \left\{-1\right\}
\]

\[
 \int \frac{1}{x} \d{x} = \int x^{-1}\d{x} = \ln \abs{x} + c
\]

\[
 \int e^x \d{x} = e^x + c
\]

\[
 \int\ln (ax + b) \d{x} = x\ln (ax +b) - x + \frac{b}{a}\ln (ax + b) + c
\]

\[
 \int a^x \cdot \d{x} = \frac{a^x}{\ln a}+ c
\]

\[
 \int \sin x \cdot \d{x} = - \cos x + c
\]

\[
 \int \cos x \cdot \d{x} = \sin x + c
\]

\[
 \int \sec^2 x \cdot \d{x} = \int \frac{1}{\cos^2 x} = \tan x + c
\]

\[
 \int \csc^2 x \cdot \d{x} = \int \frac{1}{\sin^2 x} = - \cot x + c
\]

\[
 \int \cos^2 ax \cdot \d{x} = \frac{x + \sin(2ax)}{2} + c
\]

\[
 \int \frac{\d{x}}{1+x^2} = \argtan x + c = - \argcot x + c
\]

%\[
% \int \frac{1}{\sqrt{1-x^2}} \cdot \d{x} = \argsin x + c = - \argcos x + c
%\]

\[
 \int \frac{\d{x}}{\sqrt{1-x^2}} \cdot  = \argtanh x + c = \frac{1}{2} \ln \frac{1+x}{1-x}
\]

\[
 \int \sinh x \cdot \d{x} = \cosh x + c
\]

\[
 \int \cosh x \cdot \d{x} = \sinh x + c
\]

\[
 \int \cos^2 x \d{x} = \frac{x + \sin x \cdot \cos x}{2} + c 
\]

\[
 \int \sin^2 x \d{x} = \frac{x - \sin x \cdot \cos x}{2} + c 
\]

\[
 \int \tan x \d{x} = -\ln\left|\cos x \right| + c = \ln\left|\sec x \right| + c
\]

\[
 \int \cot x \d{x} = \ln\left|\sin x \right| + c
\]

\[
  \int \frac{\d{x}}{a^2+x^2} = \frac{\argtan \frac{x}{a}}{a} + c
\]

\[
  \int \frac{\d{x}}{\sqrt{a^2+x^2}} = \argsin \frac{x}{a} + c
\]

\[
 \int f^n (x) \cdot f' (x) \d{x} = \frac{f^{n+1}(x)}{n+1} + c
\]

\[
 \int \frac{f'(x)}{f(x)} \d{x} = \ln\left|f(x)\right|+c
\]

\[
 \int f^n(x)f'(x)\d{x} = \frac{f^{n+1}(x)}{n+1} + c \qquad \mbox{ met } n \in \mathbb{R} \not \; \left\{-1\right\}
\]

\[
 \int f(x)\d{g(x)} = f(x)g(x) - \int g(x) \d{f(x)} + c
\]

\onecolumn

\subsubsection{Integraaltheorema's }
\label{sec:Integraaltheoremas}

\paragraph{Green-Riemann}
\[
  \oint_{C+} P(x,y)\d{x} + Q(x,y)\d{y} = \iint_G \left(\frac{\pd{Q}}{\pd{x}} - \frac{\pd{P}}{\pd{y}}\right)\d{S}
\]

\paragraph{Oppervlakteintegraal}
\[
  S = \iint_G \left\| \frac{\pd{\vec{r}}}{\pd{u}} \times \frac{\pd{\vec{r}}}{\pd{v}} \right\| \d{u}\d{v}
\]

\paragraph{Stokes}
\[
  \iint \left(\vrot{\vec{f}}\right) \vec{n}_u = \oint_C \vec{f}(\vec{r}) \cdot \d{\vec{r}}
\]

\paragraph{Ostrogradsky}
\[
  \iint_S \left( \vec{f} \cdot \vec{n}_u\right) \d{S} = \iiint_V \vdiv{\vec{f}} \d{V}
\]



\subsubsection{Toepassingen}
\label{sec:Toepassingen}
\paragraph{Inhoud van een omwentelingslichaam}
\label{sec:InhoudOmwentelingslichaam}
bepaald door wenteling van $f(x)$ om de $x$-as begrensd door $a$ en $b$
\[
 V = \pi\int_a^b \left[f(x)\right]^2 \d{x}
\]

\paragraph{Booglengte van een kromme}
\label{sec:BooglengteKrommeAlg}
\[
  L = \int_a^b \left\| \frac{\d{\vec{r}}}{\d{t}}\right\| \d{t}
\]
\[
 L = \int_a^b \d{s} \qquad \mbox{ met } \d{s} = \sqrt{\d{x}^2 + \d{y}^2 + \d{z}^2}
\]

\paragraph{Booglengte van een kromme}
\label{sec:BooglengteKromme}
bepaald door $f(x)$ tussen $a$ en $b$
\[
 L = \int_a^b \sqrt{1+\left(\frac{\d{f}}{\d{x}}\right)^2} \d{x}
\]


\paragraph{Booglengte van een parameterkromme}
\label{sec:BooglengteParameterKromme}
bepaald door $x= f(t)$ en $y=g(t)$ tussen $a$ en $b$
\[
 L = \int_a^b \sqrt{\left(\frac{\d{x}}{\d{t}}\right)^2 + \left(\frac{\d{y}}{\d{t}}\right)^2} \d{t}
\]


\paragraph{Manteloppervlakte van een omwentellingslichaam}
\label{sec:ManteloppervlakteOmwentellingslichaam}
bepaald door wenteling van $f(x)$ om de $x$-as begrensd door $a$ en $b$
\[
 S = 2\pi\int_a^b \left|f(x)\right| \sqrt{1+\left[f'(x)\right]^2} \d{x}
\]
\[
 S = 2 \pi \int_a^b f(x) \d{s} \qquad \mbox{ met } \d{s} = \sqrt{\d{x}^2 + \d{y}^2 + \d{z}^2}
\]

\paragraph{Flux door een oppervlakte}
\label{sec:FluxOpp}
\[
  \Phi_F = \int\int_G \left( \vec{F} \cdot \vec{n}_u \right) \d{S}
\]



   
  % \section{Numerieke Integratie}
\label{sec:NumInt}
\paragraph{Intervalmiddens}
\label{sec:NumIntMiddens}
\begin{equation}
 \int_a^b f(x)\d{x} \approx \frac{b-a}{n} \sum_{k=1}^{n} f\left(a + k\frac{b-a}{2n} \right)
\end{equation}
Linkergrenzen
\begin{equation}
 \int_a^b f(x)\d{x} \approx \frac{b-a}{n} \sum_{k=0}^{n-1} f\left(a + k\frac{b-a}{n} \right)
\end{equation}
Rechtergrenzen
\begin{equation}
 \int_a^b f(x)\d{x} \approx \frac{b-a}{n} \sum_{k=1}^{n} f\left(a + k\frac{b-a}{n} \right)
\end{equation}

\paragraph{Trapeziumregel}
\label{sec:NumIntTrapezium}
\begin{equation}
 \int_a^b f(x)\d{x} \approx \frac{b-a}{2n} \left[f(a) + f(b) + 2\sum_{k=1}^{n-1} f\left(a + k\frac{b-a}{n} \right)\right]
\end{equation}

\paragraph{Paraboolregel}
\label{sec:NumIntParabool}
($n \in 2\mathbb{N}$)
\begin{equation}
 \int_a^b f(x)\d{x} \approx \frac{b-a}{3n} \left[f(a) + f(b) + 2\sum_{k=1}^{\frac{n-2}{2}} f\left(a + 2k\frac{b-a}{n} \right)
 																														+ 4\sum_{k=1}^{\frac{n}{2}} f\left(a + (2k-1)\frac{b-a}{n} \right)\right]
\end{equation}
   
 
 \section{Lineaire Algebra}
 \label{sec:LineaireAlgebra}  
   
\subsection{Vectorruimte}
\label{sec:Vectorruimte}


\paragraph{Euclidische Ruimte} is een reële vectorruimte uitgerust met
\label{sec:EuclidischeRuimte}
\begin{itemize}
	\item Scalair Product: $\scalprod{\vec{a}}{\vec{b}}$
	\begin{itemize}
	  \item Symmetrisch
	  \item Bilineair
	  \item Positief Definiet: $\forall \vec{x} \neq \vec{0} : \scalprodv{x}{x} > 0 $
  \end{itemize}
\end{itemize}

\paragraph{Pre-Hilbert-ruimte} is een complexe vectorruimte uitgerust met
\label{sec:preHilbertRuimte}
\begin{itemize}
	\item Scalair Product: $\scalprod{\vec{a}}{\vec{b}}$
	\begin{itemize}
	  \item Symmetrisch
	  \item Sequilineair
	  \item Positief Definiet: $\forall \vec{x} \neq \vec{0} : \scalprodv{x}{x} > 0 $
  \end{itemize}\end{itemize}

\subsection{Lineaire Afbeeldingen}
\label{sec:LineaireAfbeeldingen}

\subsubsection{Eigenwaarden en eigenvectoren}
\label{sec:EigenVectWaarden}
Karakteristieke veelterm
\[
  \det{\left(A-\lambda\idmatrix_n\right)} = 0
\]
   %complexe getallen
\subsection{Complexe Getallen}
\label{sec:H:ComplexeGetallen}
\[
  z = a + bi
\]
\[
  r = \sqrt{a^2+b^2}
\]
\[
  \theta = \argtan \frac{b}{a}
\]
\[
  z = r \left( \cos \theta + i\sin \theta \right)
\]

\[
  z = r \cdot e^{i \theta}
\]   
 \section{Vectoren}
 \label{sec:Vectoren}
   \[
  \overrightarrow{AB} = \vec{B} - \vec{A}
\]

\paragraph{Rekenregels}
\label{sec:RekenregelsVectoren}
     \[
       \forall \vec a:
          \vec a \cdot \vec o = 0 = \vec o \cdot \vec a
     \]
   \[
       \forall \vec a \neq \vec o:
          \vec a \cdot \vec a = ||\vec a||^2
     \]
     \[
       \forall \vec a, \vec b :
          \vec a \cdot \vec b = \vec b \cdot \vec a
           \Leftrightarrow \scalprodv{a}{b} = \scalprodv{b}{a}
     \]
     \[
       \forall \vec a, \vec b \neq \vec o:
          \vec a \bot \vec b \Leftrightarrow \vec a \cdot \vec b = 0
     \]

\paragraph{Scalair product}
\label{sec:scalairProduct}
     \[
       \forall \vec u, \vec v \neq \vec o :
          \vec u \cdot \vec v = ||\vec u|| \cdot ||\vec v|| \cdot \cos\left(\widehat{\vec u, \vec v}\right)
                              = \sum_{i=1}^n u_iv_i
     \]
\paragraph{Vectoriëel Product}
\label{sec:VectorieelProduct}
  \[
    \vec a \times \vec b =
    \left|
     \begin{array}{ccc}
       \vec u_x  &  \vec u_y  &  \vec u_z \\
      a_x & a_y & a_z\\
      b_x & b_y & b_z
     \end{array}
   \right|
  \]
\paragraph{Gemengd product}
\label{sec:GemengdProduct}
  \[
    \left(\vec a \times \vec b \right) \cdot \vec c=
    \left|
     \begin{array}{ccc}
      c_x & c_y & c_z \\
      a_x & a_y & a_z\\
      b_x & b_y & b_z
     \end{array}
   \right|
  \]

\paragraph{Lengtes/Normen}
\label{sec:Lengtes}
\[
  \left\| \vec{x} \right\| =
  \left\| \vec{x} \right\|_1 =
  \sqrt{\sum_i^n x_i^2}
  \qquad
  \mbox{Euclidische norm}
\]
\[
  \left\| \vec{x} \right\|_2 =
  \max \left\{ x_1,  \ldots, x_n \right\}
  \qquad
  \mbox{maximumnorm}
\]
\[
  \left\| \vec{x} \right\|_3 =
  \sum_i^n \left| x_i \right|
  \qquad
  \mbox{Chebyshevnorm}
\]

\paragraph{Differentiaaloperatoren}
\label{sec:Differentiaaloperatoren}

\subparagraph{Differentiaaloperator van de Eerste Orde}
\label{sec:Differentiaaloperator1}
  \[
    \vec \nabla = \left(\frac{\partial}{\partial x},\frac{\partial}{\partial y},\frac{\partial}{\partial z}\right)
  \]

\subparagraph{Gradiënt}
\label{sec:Gradient}
  \[
    \mbox{grad}\left(f\right) = \vec \nabla f =  \left(\frac{\partial f}{\partial x},\frac{\partial f}{\partial y},\frac{\partial f}{\partial z}\right)
  \]

\subparagraph{Divergentie}
\label{sec:Divergentie}
  \[
    \mbox{div}\left(\vec v\right) = \vec \nabla \cdot \vec v = \frac{\partial v_x}{\partial x} + \frac{\partial v_y}{\partial y} + \frac{\partial v_z}{\partial z}
  \]

\subparagraph{Rotatie}
\label{sec:Rotatie}
  \[
    \mbox{rot}\left(\vec v\right) = \vec \nabla \times \vec v =
    \left|
     \begin{array}{ccc}
       \vec u_x  &  \vec u_y  &  \vec u_z \\
       \frac{\partial}{\partial x} & \frac{\partial}{\partial y} & \frac{\partial}{\partial z}\\
      v_x & v_y & v_z
     \end{array}
   \right|
  \]

\subparagraph{Differentiaaloperator van de Tweede Orde (Laplaciaan)}
\label{sec:Differentiaaloperator2}
  \[
    \Delta\left(f\right)=\mbox{div}\left(\mbox{grad}\left(f\right)\right)=
    \vec \nabla \cdot\left(\vec \nabla f\right) = \vec \nabla^2 f =
    \frac{\partial^2 f}{\partial x^2} + \frac{\partial^2 f}{\partial y^2} + \frac{\partial^2 f}{\partial z^2}
  \]

\paragraph{Coördinatenstelsels}
\label{sec:CoordStelsels}

\subparagraph{Cartesisch}
\[
  \left\{
    \begin{array}{l}
      x = x\\
      y = y\\
      z = z\\
    \end{array}
  \right.
\]
\[
  \left| J \right| = 1
\]

\subparagraph{Poolcoördinaten/Cilindercoordinaten}
\[
  \left\{
    \begin{array}{l}
      x = r \cos \theta\\
      y = r \sin \theta\\
      z = z\\
    \end{array}
  \right.
\]
\[
  \left| J \right| =
  \left|
    \begin{array}{ccc}
      \cos \theta & - r \sin \theta & 0 \\
      \sin \theta &   r \cos \theta & 0 \\
      0           &   0             & 1 \\
    \end{array}
  \right| = r
\]

\subparagraph{Bolcoördinaten}
\[
  \left\{
    \begin{array}{l}
      x = r \sin \theta \cos \phi\\
      y = r \sin \theta \sin \phi\\
      z = r \cos \theta \\
    \end{array}
  \right.
\]
\[
  \left| J \right| =
  \left|
    \begin{array}{ccc}
      \sin \theta \cos \phi & r \cos \theta \cos \phi & - r \sin \theta \sin \phi \\
      \sin \theta \sin \phi & r \cos \theta \sin \phi &   r \sin \theta \cos \phi \\
      \cos \theta           & - r \sin \theta         & 0 \\
    \end{array}
  \right| = r^2 \sin \theta
\]
   
 %Ruimtemeetkunde
 \section{Ruimtemeetkunde}
 \label{sec:Ruimtemeetkunde}
   
%\subsection{Vectoriële basis}
%\label{sec:VectorieleBasis}

%\paragraph{Midden van $\left[AB\right]$}
% \[
%   \vec{M}=\frac{\vec A + \vec B}{2}
% \]
%\paragraph{Zwaartepunt van $\Delta ABC$}
%\[
% \vec{M}=\frac{\vec A + \vec B + \vec C}{3}
%\] 
%\paragraph{Zwaartepunt van viervlak $ABCD$}
%\[
%  \vec{M}=\frac{\vec A + \vec B + \vec C + \vec D}{4}
%\]

%\parapgraph{Puntvector van punt $P$ met deelverhouding $k \neq 0$ t.o.v. koppel $\left(P_1,P_2\right)$}
%\begin{eqnarray*}
% &\vec P = \frac{\vec P_1 - k \vec P2}{1-k}&\\
% &P\left(\frac{x_1-kx_2}{1-k},\frac{y_1-ky_2}{1-k},\frac{y_1-ky_2}{1-k}\right)&\\
% &k=\left(P_1 P_2 P\right) \Leftrightarrow \overrightarrow{PP_1} = k\overrightarrow{PP_2}
%\end{eqnarray*}

\subsection{Vergelijkingen van rechten}
\label{sec:VergelijkingenVanRechten}

\paragraph{Rechte bepaald door een punt $P_1(x_1,y_1,z_1)$ en richtingsvector $\vec R(a,b,c)$}
\label{sec:RechtePuntRichting}
  \subparagraph{vectoriële vergelijking}
  \label{sec:RechtePuntRichtingVECTOR}
    \[
      \vec P = \vec P_1 + k \cdot \vec R \qquad \textrm{met $k \in \mathbb{R}$}
    \]
  \subparagraph{parametervoorstelling}
  \label{sec:RechtePuntRichtingPARAM}
    \[
      \begin{array}{lcr}
        \left\{
          \begin{array}{rcl}
             x &=& x_1 + k \cdot a\\
             y &=& y_1 + k \cdot b\\
             z &=& z_1 + k \cdot c
          \end{array}%
        \right. &%
       \textrm{of} &%
        \left[
          \begin{array}{c}
            x\\ y\\  z
          \end{array}
         \right]
        =
         \left[
           \begin{array}{c}
             x_1\\ y_1\\ z_1
           \end{array}
         \right]
       + k \cdot
         \left[
           \begin{array}{c}
             a\\ b\\ c
           \end{array}
         \right]
     \end{array}
   \]
  \subparagraph{cartesische vergelijking}
  \label{sec:RechtePuntRichtingCART}
    \[
      \frac{x-x_1}{a} = \frac{y-y_1}{b} = \frac{z-z_1}{c} \qquad \textrm{met $a,b,c \in \mathbb{R}_0$} 
    \]

\paragraph{Rechte bepaald door twee punten $P_1(x_1,y_1,z_1)$ en $P_2(x_2,y_2,z_2)$}
\label{sec:RechtePuntPunt}
  \subparagraph{vectoriële vergelijking}
  \label{sec:RechtePuntPuntVECTOR}
    \[
      \vec P = \vec P_1 + k \cdot \left( \vec P_2 - \vec P_1 \right)  \qquad \textrm{met $k \in \mathbb{R}$}
    \]
  \subparagraph{parametervoorstelling}
  \label{sec:RechtePuntPuntPARAM}
    \[
      \begin{array}{lcr}
        \left\{
          \begin{array}{rcl}
            x &=& x_1 + k \cdot \left(x_2-x_1\right)\\
            y &=& y_1 + k \cdot \left(y_2-y_1\right)\\
            z &=& z_1 + k \cdot \left(z_2-z_1\right)
          \end{array}%
        \right. &%
      \textrm{of} &%
        \left[
          \begin{array}{c}
            x\\ y\\ z
          \end{array}
        \right]
       =
        \left[
          \begin{array}{c}
            x_1\\ y_1\\ z_1
          \end{array}
        \right]
      + k \cdot
        \left[
          \begin{array}{c}
            x_2-x_1\\ y_2-y_1\\ z_2-z_1
          \end{array}
        \right]
     \end{array}
   \]
  \subparagraph{cartesische vergelijking}
  \label{sec:RechtePuntPuntCART}
    \[
      \frac{x-x_1}{x_2-x_1} = \frac{y-y_1}{y_2-y_1} = \frac{z-z_1}{z_2-z_1} 
    \]
 
\paragraph{Richtingsgetallen $(a,b,c,)$ van de snijlijn van twee vlakken$\alpha$ en $\beta$}
\label{sec:SnijlijnTweeVlakken}
 \[
   d \leftrightarrow 
   \left\{
     \begin{array}{rcl}
       u_1 x + v_1 y + w_1 z + t_1 & = & 0\\
       u_2 x + v_2 y + w_2 z + t_2 & = & 0
     \end{array}
   \right.
   \ 
   \textrm{waarbij}
   \  
   r\left( \left[
     \begin{array}{ccc}
       u_1 & v_1 & w_1\\
       u_2 & v_2 & w_2\\
     \end{array}
     \right] \right)
   = 2  
 \]
 \[
     a = k \cdot
       \left|
         \begin{array}{cc}
          v_1 & w_1 \\
          v_2 & w_2
         \end{array}
       \right|
     \ \textrm{en} \  
     b = -k \cdot
       \left|
         \begin{array}{cc}
          u_1 & w_1 \\
          u_2 & w_2
         \end{array}
       \right|
     \ \textrm{en} \ 
     c = k \cdot
       \left|
         \begin{array}{cc}
          u_1 & v_1 \\
          u_2 & v_2
         \end{array}
       \right|
 \]
 
\subsection{Vergelijkingen van vlakken}
\label{sec:VergelijkingenVlakken}

\paragraph{Algemene vergelijking van een vlak}
\label{sec:AlgVglVlak}
 \[
   ux + vy + wz + t = 0 \qquad \textrm{met \ } \neg\left(u=v=w=0\right)
 \]

\paragraph{Vergelijking van basisvlakken}
\label{sec:VergelijkingVanBasisvlakken}
 \[  
     \begin{array}{rl}
       \textbf{vlak yz:} & x=0\\
       \textbf{vlak xz:} & y=0\\
       \textbf{vlak xy:} & z=0\\
     \end{array}
 \]

\paragraph{Vlak bepaald door ëen punt $P_1(x_1,y_1,z_1)$ en twee richtingsvectoren $\vec R_1(a_1,b_1,c_1)$ en $\vec R_2(a_2,b_2,c_2)$}
\label{sec:VlakPuntRichtingRichting}
 \subparagraph{vectoriële vergelijking}
  \label{sec:VlakPuntRichtingRichtingVECTOR}
 \[
 \vec P = \vec P_1 + k \cdot \vec R_1  + m \cdot \vec R_2
 \]
 \subparagraph{parametervoorstelling}
  \label{sec:VlakPuntRichtingRichtingPARAM}
 \[
   \begin{array}{lr}
     \left\{
       \begin{array}{rcl}
         x &=& x_1 + k \cdot a_1 + m \cdot a_2\\
         y &=& y_1 + k \cdot b_1 + m \cdot b_2\\
         z &=& z_1 + k \cdot c_1 + m \cdot c_2
       \end{array}%
     \right.
     \  \textrm{of} \  &%
     \left[
       \begin{array}{c}
         x\\ y\\ z
       \end{array}
     \right]
     =
     \left[
       \begin{array}{c}
         x_1\\ y_1\\ z_1
       \end{array}
     \right]
     + k \cdot
     \left[
       \begin{array}{c}
         a_1\\ b_1\\ c_1
       \end{array}
     \right]
     + m \cdot
     \left[
       \begin{array}{c}
         a_2\\ b_2\\ c_2
       \end{array}
     \right]
   \end{array}
 \]
 \subparagraph{cartesische vergelijking}
  \label{sec:VlakPuntRichtingRichtingCART}
 \[
   \left|
     \begin{array}{cccc}
       x  &  y  &  z  & 1\\
      x_1 & y_1 & z_1 & 1\\
      a_1 & b_1 & c_1 & 0\\
      a_2 & b_2 & c_2 & 0
     \end{array}
   \right|
    = 0%
    \quad \textrm{of} \quad
   \left|
     \begin{array}{ccc}
       x - x_1 & y - y_1 & z - z_1 \\
       a_1 & b_1 & c_1\\
       a_2 & b_2 & c_2
     \end{array}
   \right| 
    = 0
 \]
%\newpage \noindent
\paragraph{Vlak bepaald door drie niet-collineaire punten $P_1$,$P_2$ en $P_3$}
\label{sec:VlakPuntPuntPunt}
 \subparagraph{vectoriële vergelijking}
  \label{sec:VlakPuntPuntPuntVECTOR}
 \[
   \vec P = \vec P_1 + k \cdot \left(\vec P_2 - \vec P_1\right)  + m \cdot \left(\vec P_3 - \vec P_1\right)
 \]
 \subparagraph{parametervoorstelling}
  \label{sec:VlakPuntPuntPuntPARAM}
 \[
       \left\{
       \begin{array}{rcl}
         x &=& x_1 + k \cdot \left(x_2 - x_1\right) + m \cdot \left(x_3 - x_1\right)\\
         y &=& y_1 + k \cdot \left(y_2 - y_1\right) + m \cdot \left(y_3 - y_1\right)\\
         z &=& z_1 + k \cdot \left(z_2 - z_1\right) + m \cdot \left(z_3 - z_1\right)
       \end{array}%
     \right. 
 \]
 \[ 
     \left[
       \begin{array}{c}
         x\\ y\\ z
       \end{array}
     \right]
     =
     \left[
       \begin{array}{c}
         x_1\\ y_1\\ z_1
       \end{array}
     \right]
     + k \cdot
     \left[
       \begin{array}{c}
         x_2 - x_1\\
         y_2 - y_1\\
         z_2 - z_1
       \end{array}
     \right]
     + m \cdot
     \left[
       \begin{array}{c}
         x_3 - x_1\\
         y_3 - y_1\\
         z_3 - z_1
       \end{array}
     \right]
 \]
 \subparagraph{cartesische vergelijking}
  \label{sec:VlakPuntPuntPuntCART}
 \[
   \left|
     \begin{array}{cccc}
       x  &  y  &  z  & 1\\
      x_1 & y_1 & z_1 & 1\\
      x_2 & y_2 & z_2 & 1\\
      x_3 & y_3 & z_3 & 1\\
     \end{array}
   \right|
    = 0  
 \]
 
\paragraph{Vergelijking van een vlak op de assegmenten}
\label{sec:VlakAssegmenten}
 vlak $\alpha$ snijdt $x$, $y$, $z$  in $P_1\left(a,0,0\right)$, $P_2\left(0,b,0\right)$, $P_3\left(0,0,c\right)$
 \[
  \alpha \leftrightarrow \frac{x}{a} + \frac{y}{b} + \frac{z}{c} = 1
 \]

\paragraph{Vergelijking van de vlakkenwaaier door $d$}
\label{sec:Vlakkenwaaier}
   \[
     d \leftrightarrow
     \left\{
       \begin{array}{rcl}
         u_1 x + v_1 y + w_1 z + t_1 & = & 0\\
         u_2 x + v_2 y + w_2 z + t_2 & = & 0
       \end{array}
     \right.
   \]
   \[
     k\left(u_1 x + v_1 y + w_1 z + t_1\right) + m\left(u_2 x + v_2 y + w_2 z + t_2\right) = 0
     \ \textrm{met } k,m \in \mathbb{R} \ \textrm{en } \ \neg\left(k=m=0\right)
   \]

\subsection{Loodrechte en evenwijdige stand}
\label{sec:LoodrechteEnEvenwijdigeStand}

\paragraph{Evenwijdigheid rechten $e$ en $f$ met richtingsgetallen $\left(a_1,b_1,c_1\right)$ en $\left(a_2,b_2,c_2\right)$}
\label{sec:EvenwijdigheidRechten}
   \[
     e // f \Leftrightarrow  \exists k \in \mathbb{R}_0 : a_2 = ka_1 \wedge b_2=kb_1 \wedge c_2=kc_1
   \]
 \textbf{Evenwijdigheid vlakken $\alpha \leftrightarrow u_1 x + v_1 y + w_1 z + t_1 = 0$ 
                          en $\beta  \leftrightarrow u_2 x + v_2 y + w_2 z + t_2 = 0$}
   \[
     \alpha // \beta \Leftrightarrow  \exists k \in \mathbb{R}_0 : u_2 = ku_1 \wedge v_2=kv_1 \wedge w_2=kw_1
   \]
   
\paragraph{Evenwijdigheid rechte $d$ met richtingsgetallen $\left(a,b,c\right)$ en het vlak $\alpha \leftrightarrow u_1 x + v_1 y + w_1 z + t_1 = 0$}
\label{sec:EvenwijdigheidRechteVlak}
   \[
     d // \alpha \Leftrightarrow ua + vb + wc = 0
   \]
   
\paragraph{Loodrechte stand van rechten $e$ en $f$ met richtingsgetallen $\left(a_1,b_1,c_1\right)$ en $\left(a_2,b_2,c_2\right)$}
\label{sec:LoodrechteStandRechten}
   \[
     e \bot f \Leftrightarrow a_1 a_2 + b_1 b_2 + c_1 c_2 = 0
   \]  
   
\paragraph{Loodrechte stand van een rechte $e$ met richtingsgetallen $\left(a,b,c\right)$\\ en een vlak $\alpha \leftrightarrow ux + vy + wz + t = 0$}
\label{sec:LoodrechteStandVanEenRechteEMetRichtingsgetallenLeftABCRightEnEenVlakAlphaLeftrightarrowUxVyWzT0}
   \[
     e \bot \alpha \Leftrightarrow 
                    a = k \cdot u \wedge b = k \cdot v \wedge c = k \cdot w 
                    \qquad \textrm{met } k \in \mathbb{R}_0
   \] 
   
\paragraph{Loodrechte stand van twee vlakken $\alpha \leftrightarrow u_1 x + v_1 y + w_1 z + t_1 = 0$ 
                                           en $\beta \leftrightarrow u_2 x + v_2 y + w_2 z + t_2 = 0$}
\label{sec:LoodrechteStandVlakken}
   \[
     \alpha \bot \beta \Leftrightarrow 
                    u_1 u_2 + v_1 v_2 + w_1 w_2 = 0
   \]
   
\paragraph{Normaalvector van een vlak}
\label{sec:NormaalvectorVlak}
   \[
    \vec N\left(u,v,w\right) \ \textrm{is een normaalvector van het vlak } \alpha \leftrightarrow ux + vy + wz + t = 0
   \]  
    
\paragraph{Stelsel vergelijkingen van de loodlijn $m$ uit $P\left(x_1,y_1,z_1\right)$ op het vlak $\alpha \leftrightarrow ux + vy + wz + t = 0$}
\label{sec:LoodlijnPuntVlak}
   \[
     m \leftrightarrow \frac{x-x_1}{u} = \frac{y-y_1}{v} = \frac{z-z_1}{w} \qquad \textrm{met } u, v,w \in \mathbb{R}_0
   \]
   
\paragraph{Vergelijking van het loodvlak $\alpha$ uit $P\left(x_1,y_1,z_1\right)$ op een rechte $e$ met richtingsgetallen $\left(a,b,c\right)$}
\label{sec:LoodvlakRechte}
  \[
    \alpha \leftrightarrow a\left(x-x_1\right)+b\left(y-y_1\right)+c\left(z-z_1\right)=0
  \]


\subsection{Afstanden en hoeken}
\label{sec:AfstandenEnHoeken}
      
\paragraph{Afstand tussen de punten $A\left(x_1,y_1,z_1\right)$ en $B\left(x_2,y_2,z_2\right)$}
\label{sec:AfstandPunten}
   \[
    d\left(A,B\right) = |AB| = \sqrt{\left(x_2-x_1\right)^2+\left(y_2-y_1\right)^2+\left(z_2-z_1\right)^2}
   \]
 
\paragraph{Afstand van een punt $A\left(x_1,y_1,z_1\right)$ tot het vlak $\alpha \leftrightarrow ux + vy + wz + t = 0$}
\label{sec:AfstandPuntVlak}
    \[
      d\left(A,a\right) = \frac{\left| ux_1 + vy_1 + wz_1 + t\right|}{\sqrt{u^2 + v^2 + w^2}}
    \]
   
\paragraph{Hoek van twee rechten met richtingsgetallen $\left(a_1,b_1,c_1\right)$ en $\left(a_2,b_2,c_2\right)$}
\label{sec:HoekRechten}
    \[
      \cos\left(\widehat{ab} \right) = \frac {\left| a_1 a_2 + b_1 b_2 + c_1 c_2\right|}%
                                             {\sqrt{a_1^2 + b_1^2 + c_1^2} \cdot \sqrt{a_2^2 + b_2^2 + c_2^2}}
    \]

\paragraph{De hoek van een rechte $a$ en een vlak $\alpha$ is het complement van de hoek gevormd door de rechte $a$ en de loodlijn $m$ op dat vlak.}
\label{sec:HoekRechteVlak}
    $a$ met stel richtingsgetallen $\left(a,b,c\right)$ en $\alpha \leftrightarrow ux + vy + wz + t = 0$
    \[
      \cos\left(\widehat{am} \right) = \frac {\left| au + vb + wc\right|}%
                                             {\sqrt{a^2 + b^2 + c^2} \cdot \sqrt{u^2 + v^2 + w^2}}
                                             \qquad \textrm{ en } \widehat{a \alpha} = 90 \degree - \widehat{am}
    \]
    
\paragraph{Hoek van van twee vlakken $\alpha \leftrightarrow u_1 x + v_1 y + w_1 z + t_1 = 0$ en $\beta  \leftrightarrow u_2 x + v_2 y + w_2 z + t_2 = 0$ \\ is de hoek van twee loodlijnen op de respectievelijke vlakken}
\label{sec:HoekVlakken}
    \[
      \cos\left(\widehat{\alpha\beta} \right) = \frac {\left| u_1 u_2 + v_1 v_2 + w_1 w_2\right|}%
                                             {\sqrt{u_1^2 + v_1^2 + w_1^2} \cdot \sqrt{u_2^2 + v_2^2 + w_2^2}}
    \]
    
    
      
\subsection{Bollen}
\label{sec:Bollen}

\paragraph{Middelpuntsvergelijking van een bol}
\label{sec:MiddelpuntsvergelijkingBol}
   \[
     \bol\left(M\left(x_1,y_1,z_1\right),r\right)
     \leftrightarrow
     \left(x-x_1\right)^2+\left(y-y_1\right)^2+\left(z-z_1\right)^2 = r^2
   \]
   
\paragraph{Algemene vergelijking van een bol}
\label{sec:BolVergelijking}
   \[
     x^2 + y^2 + z^2 + 2ax + 2by + 2 cz + d = 0 \qquad \textrm{als } a^2 + b^2 + c^2 - d \geq 0
   \]
   \[
     \textrm{ met middelpunt } M\left(-a,-b,-c\right) \textrm{ en straal } r = \sqrt{a^2 + b^2 + c^2 - d}
   \]
   
%\paragraph{Onderlinge ligging van een twee bollen $\bol_1\left(M_1,r_1\right)$ en $\bol_2\left(M_2,r_2\right)$ }
%\label{sec:LiggingBollen}
%    \begin{eqnarray*}
%      \bol_1 \cap \bol_2 = \left\{ \right\} \Leftrightarrow%
%        &             & \left| M_1 M_2\right| > r_1 + r_2 \qquad \textrm{liggen volledig buiten elkaar}\\
%        & \textrm{of} & \left| M_1 M_2\right| < \left|r_1 - r_2\right| \qquad \textrm{liggen volledig in elkaar}\\
%      \bol_1 \cap \bol_2 = \left\{ S\right\} \Leftrightarrow%
%        &             & \left| M_1 M_2\right| = r_1 + r_2 \qquad \textrm{raken elkaar uitwendig}\\
%        & \textrm{of} & \left| M_1 M_2\right| = \left|r_1 - r_2\right| \qquad \textrm{raken elkaar inwendig}\\
%      \bol_1 \cap \bol_2 = c \Leftrightarrow%
%        &             &  \left|r_1 - r_2 \right| < \left| M_1 M_2\right| < r_1 + r_2 
%    \end{eqnarray*} 
 
  
%\paragraph{Onderlinge ligging van een bol $\bol\left(M,r\right)$ en een vlak $\alpha$ }
%\label{sec:LiggingBolVlak}
%    \begin{eqnarray*}
%      \bol\left(M,r\right) \cap \alpha = \left\{ \right\} & \Leftrightarrow & d\left(M,\alpha\right) > r\\
%      \bol\left(M,r\right) \cap \alpha = \left\{ S\right\} & \Leftrightarrow & d\left(M,\alpha\right) = r\\
%      \bol\left(M,r\right) \cap \alpha = c & \Leftrightarrow & d\left(M,\alpha\right) < r\\
%    \end{eqnarray*}
%    
 
   
   
\subsection{Combinatieleer}
\label{sec:Combinatieleer}

\[
   V^p_n = \frac{n!}{\left(n-p\right)!}
\]
\[
  \overline{V}^p_n = n!
\]
\[
  P_n = V^n_n = n!
\]
\[
  \overline{P}^{a_1,a_2,\ldots,a_i}_n = \frac{n!}{a_1!a_2!\ldots a_i!}
\]
\[
  C^p_n =  \combination{n}{p} = \frac{V^p_n}{P_p} = \frac{n!}{p!\left(n-p\right)!}
\]
\[
  \overline{C}_n = C^p_{n+p-1} = \frac{V^p_{n+p-1}}{P_p} = \frac{\left(n+p-1\right)!}{p!\left(n-1\right)!}
\]
\paragraph{Formule van Stifel-Pascal}
\[
  C^p_{n+1} = C^p_n + C_n^{p-1} \qquad \mbox{of} \qquad \combination{n+1}{p} = \combination{n}{p} + \combination{n}{p-1}
\]
\paragraph{Binomium van Newton}
\[
  \left(a+b\right)^n = \sum_{k=0}^n \combination{n}{k} a^{n-k}b^k
\]

\scriptsize
\setiftext{ja}{nee}
\STRUCT{opgave}{}{%
  \IF{volgorde van belang?}%
    \THEN{%
      \IF{Herhaling mogelijk?}%
        \THEN{%
          \IF{Aantal herhalingen bepaald?}%
            \THEN{%
              \ACTION{$\overline{P}^{a_1,a_2,\ldots,a_i}_n$}%
            }%
            \ELSE{%
              \ACTION{$\overline{V}^p_n$}%
            }%
            \ENDIF%
         }%
         \ELSE{%
           \IF{$p=n$}%
             \THEN{%
               \ACTION{$P_n$}%
             }%
             \ELSE{%
               \ACTION{$V^p_n$}%
             }%
             \ENDIF%
         }%
         \ENDIF%
    }%
    \ELSE{%
      \IF{Herhaling mogelijk?}%
        \THEN{%
          \ACTION{$\overline{C}^p_n$}%
        }%
        \ELSE{%
          \ACTION{$C^p_n$}
        }%
        \ENDIF%
    }%
    \ENDIF%
}%
\normalsize

\subsection{Rekenregels voor kansen}
\label{sec:RekenregelsVoorKansen}


\paragraph{Somregel}
\[
  \prob{A \vee B} = \prob{A \cup B} = \prob{A} + \prob{B}
  \Leftrightarrow
  \mbox{ $A$ en $B$ zijn onafhankelijke gebeurtenissen}
\]
\[
  \prob{A \vee B} = \prob{A \cup B} = \prob{A} + \prob{B} - \prob{A \cap B}
\]

\paragraph{Verschilregel}
\label{sec:VerschilregelKangsrekening}
\[
  \prob{A \backslash B} = \prob{A} - \prob{A \cap B}
\]

\paragraph{Stelling van Boole}
\label{sec:StellingBooleKangsrekening}
\[
  \prob{A \cup B} = \prob{A} + \prob{B} - \prob{A \cap B}
\]

\paragraph{Productregel}
\label{sec:ProductregelKangsrekening}
\[
  \prob{A \wedge B} = \prob{A \cap B} = \prob{A} \cdot \prob{B}
  \Leftrightarrow
  \mbox{ $A$ en $B$ zijn onafhankelijke gebeurtenissen}
\]

\paragraph{Voorwaardelijke kansen}
\[
  \prob{A | B} = \frac{\prob{A \cap B}}{\prob{B}}
\]

\paragraph{Regel van Bayes}
\[
  \prob{A|B}\prob{B} = \prob{B|A}\prob{A}
\]
\[
  \prob{A|B} = \frac{\prob{B|A}\prob{A}}{\prob{B}}
\]
\[
  \prob{A|B} = \frac{\prob{B|A}\prob{A}}{\prob{B|A}\prob{A} + \prob{B|A^c}\prob{A^c}}
\]
 %\newpage
 %\section{Statistiek}
 %\label{sec:Statistiek}  
 %  
\subsection{Standaardbegrippen}
\paragraph{Gemiddelde}
\[
	\overline{x} = \frac{1}{n} \sum^n_{i=1} x_i
\]

\paragraph{Variantie}
\[
	s^2 = \frac{1}{n-1} \sum^n_{i=1} \left(x_i - \overline{x} \right)^2
\]

\paragraph{Standaardafwijking}
\[
  s = \sqrt{s^2}
\]

MED
MOD
MAD


\subsection{Steekproef}
\label{sec:Steekproef}

  \paragraph{Gemiddelde}
  \label{sec:GemiddeldeSteekproef}
  \[
      \overline{x} = \frac{1}{n}\sum^n_{i=1} x_1
                   = \frac{1}{n}\sum^n_{i=1} n_i x_1 \mbox{ voor gegroepeerde gegevens}
  \]

  \paragraph{Variantie}
  \label{sec:VariantieSteekproef}
  \[
      s^2 = \frac{1}{n-1} \sum^n_{i=1} \left(x_i-\overline{x}\right)^2
          = \frac{1}{n-1} \sum^n_{i=1} n_i\left(x_i-\overline{x}\right)^2 \mbox{ voor gegroepeerde gegevens}
  \]

  \paragraph{Standaardafwijking}
  \label{sec:StandaardafwijkingSteekproef}
  \[
      s = \sqrt{\frac{1}{n-1} \sum^n_{i=1} \left(x_i-\overline{x}\right)^2}
        = \sqrt{\frac{1}{n-1} \sum^n_{i=1} n_i\left(x_i-\overline{x}\right)^2} \mbox{ voor gegroepeerde gegevens}
  \]

\subsection{Populatie}
\label{sec:Populatie}
  \paragraph{Gemiddelde}
  \label{sec:GemiddeldePopulatie}
  \[
      \mu = \frac{1}{N}\sum^N_{i=1} x_1
          = \frac{1}{N}\sum^N_{i=1} n_i x_1 \mbox{ voor gegroepeerde gegevens}
  \]

  \paragraph{Variantie}
  \label{sec:VariantiePopulatie}
  \[
      \sigma^2 = \frac{1}{N} \sum^N_{i=1} \left(x_i-\mu\right)^2
               = \frac{1}{N} \sum^N_{i=1} n_i\left(x_i-\mu\right)^2 \mbox{ voor gegroepeerde gegevens}
  \]

  \paragraph{Standaardafwijking}
  \label{sec:StandaardafwijkingPopulatie}
  \[
      \sigma = \sqrt{\frac{1}{N} \sum^N_{i=1} \left(x_i-\mu\right)^2}
             = \sqrt{\frac{1}{N} \sum^N_{i=1} n_i\left(x_i-\mu\right)^2} \mbox{ voor gegroepeerde gegevens}
  \]



\subsection{Kansverdelingen}
\label{sec:Kansverdelingen}

  \subsubsection{Verwachtingswaarde}
  \[
    \E{X} = \int_{-\infty}^{+\infty} x \d{F_X}
  \]
  \paragraph{Eigenschappen}
  \[
    \E{aX+b} = a \cdot \E{X} + b
  \]
  \[
    \E{X + Y} = \E{X} + \E{Y}
  \]
  \[
    \left| \E{X} \right| \geq
  \]

  \subsubsection{Variantie}
  \[
    \Var{X} = \E{\left(X - \E{X} \right)^2}
  \]


  \subsubsection{\texorpdfstring{$\sqrt{n}$-wet}{Wortel-n-wet}}
  \label{sec:SqrtNWet}
    \[
      S = \sum^n_{i=1} X_i
    \]
    \[
      \mu_S = E\left[S\right] = n \cdot E\left[X\right] = n\mu_X
    \]
    \[
      \sigma^2_S = n \sigma^2_X \Leftrightarrow \sigma_S = \sqrt{n}\sigma_X
    \]
    \[
      \overline{X} = \frac{1}{n}S = \frac{1}{n}\sum_{i=1}^n X_i
    \]
    \[
      \mu_{\overline{X}} = \mu_X
    \]
    \[
      \sigma^2_{\overline{X}}= \frac{1}{n^2}n\sigma^2_X = \frac{1}{n}\sigma_X^2
      \leftrightarrow \sigma_{\overline{X}}=\frac{\sigma_X}{\sqrt{n}}
    \]

\subsection{Discrete Verdelingen}
\label{sec:DiscreteVerdelingen}
  \[
    \E{X} = \sum^n_{i=1} x_i \prob{X=x_i} = \sum^n_{i=1} x_i p_i
  \]
  \[
    \Var{X} = \sum^n_{i=1}\left(x_i-\E{X}\right)^2 \prob{X=x_i}
                      = \sum^n_{i=1}\left(x_i-\E{X}\right)^2 p_i
  \]

  \subsubsection{Uniforme discrete verdeling}
  \label{sec:UniformeDiscreteVerdeling}
    \[
      E\left[X\right] = \mu_x = \frac{n+1}{2}
    \]
    \[
      \prob{X=x_i} = \frac{1}{n}
    \]
    \[
      \Var{X} = \sigma^2_x = \frac{n^2-1}{12}
    \]


    \subsubsection{Bernouilli-verdeling}
    Een experiment waarbij de mogelijke uitkomsten $1$ (waar) en $0$ (niet-waar) zijn.
    \[
      X \sim \mathrm{Bernouilli}\left(p\right)
    \]
    \[
      \begin{array}{l}
        \prob{X = 1} = p\\
        \prob{X = 0} = q = 1 - p
      \end{array}
    \]
    \[
      \E{X} = p
    \]
    \[
      \Var{X} = pq = p(1-p)
    \]

  \subsubsection{Binomiale verdeling}
  \label{sec:BinomialeVerdeling}
  Een herhaling van $n$ Bernouilli-experimenten met kans $p$.
    \[
       X \sim \mathrm{Binom}\left(n,p\right)
    \]
    \[
      p + q = 1
    \]
    \[
      \prob{X=i} = \binom{n}{k} p^iq^{n-i}
    \]
    \[
      \E{X} = np
    \]
    \[
      \Var{X} = npq
    \]
    \[
     \gamma_1 = \frac{q-p}{\sqrt{npq}}
    \]
    \[
      \gamma_2 = \frac{1-6pq}{npq}
    \]

  \subsubsection{Hypergeometrische verdeling}
  Een populatie met $N$ elementen waarvan er $P$ elementen voldoen ($1$) en de andere niet. De kans op $k$ succesvolle experimenten bij trekken zonder teruglegging. Een som van $N$ Bernouilli-experimenten, waarbij de kans steeds afneemt. $p=\frac{P}{N}$
  \[
    X \sim \mathrm{hypergeom}\left(N,p,n\right) \quad \mbox{met } p = \frac{P}{N}
  \]
  \[
    \prob{X = j} = \frac{\binom{P}{j} \binom{N-P}{n-j}}{ \binom{N}{n}}
  \]
  \[
    \E{X} = np
  \]
  \[
    \Var{X} = npq\frac{N-n}{N-1}
  \]

  \subsubsection{Geometrische verdeling}
  De kans dat de eerste $k$ trekkingen falen uit een experiment zoals de hypergeometrische.
  \[
    \prob{X_i = 0 \forall i < k; X_k = 1} = pq^k
  \]
  \[
    \E{X} = \frac{q}{p}
  \]
  \[
    \Var{X} = \frac{q}{p^2}
  \]

  \subsubsection{Poissonverdeling}
  \[
    f_X = \frac{\left(\lambda t\right)^n}{n!}e^{-\lambda t}
  \]
  \[
    \E{X} = \lambda
  \]
  \[
    \Var{X} = \lambda
  \]

\subsection{Continue Verdelingen}
\label{sec:ContinueVerdelingen}
  \[
    E\left[X\right]   =\mu_X
                      =\int_a^b xf\left(x\right) \d{x}
  \]
  \[
    Var\left[X\right] = \sigma^2_X
                      = \int_a^b \left(x-\mu_x\right)^2 f\left(x\right) \d{x}
                      = \int_a^b x^2 f\left(x\right) \d{x} - \mu_x^2
  \]


\subsubsection{Uniforme continue verdeling}
\label{sec:UniformeContinueVerdeling}
  \[
    f\left(x\right)
     = \left\{
          \begin{array}{ll}
             \forall x \in \left[a,b\right] : & f\left(x\right)= \frac{1}{b-a}\\
             \forall x \not\in \left[a,b\right] : & f\left(x\right)= 0\\
          \end{array}%
        \right.
  \]
  \[
    \mu_X = \frac{a+b}{2}
  \]
  \[
    \sigma_X = \frac{b-a}{\sqrt{12}} = \frac{b-a}{2\sqrt{3}}
  \]

\subsubsection{Normaalverdeling}
\label{sec:Normaalverdeling}
  \[
    X \sim N\left(\mu,\sigma\right) = \frac{1}{\sigma\sqrt{2\pi}}e^{-\frac{1}{2}\left(\frac{x-\mu}{\sigma}\right)^2}
  \]
  \paragraph{Standaardnormaalverdeling}
  \[
    z = \frac{x-\mu}{\sigma}
  \]
  \[
    X \sim Z = N\left(0,1\right) = \frac{1}{\sqrt{2\pi}}e^{-\frac{1}{2}\left(z\right)^2}
  \]

  \subsubsection{Exponentiële verdeling}
  \[
    f_X(t) =
    \left\{
      \begin{array}{ll}
        \forall t > 0 & \lambda e^{- \lambda t}\\
        \forall t \leq 0 & 0
      \end{array}
    \right.
  \]
  \[
    F_X(t) =
    \left\{
      \begin{array}{ll}
        \forall t > 0 & 1 - e^{- \lambda t}\\
        \forall t \leq 0 & 0
      \end{array}
    \right.
  \]
  \[
    \E{X} = \frac{1}{\lambda}
  \]
  \[
    \Var{X} = \frac{1}{\lambda^2}
  \]

\subsection{Betrouwbaarheidsintervallen}
\label{sec:Betrouwbaarheidsintervallen}
  \[
    \mu_{\hat{p}} = \frac{1}{n}np = p
  \]
  \[
    \sigma_{\hat{p}} = \frac{1}{n}\sqrt{npq} = \sqrt{\frac{p\left(1-p\right)}{n}}
  \]

  \[
    \mu_{\hat{\mu}} = \mu_x
  \]
  \[
    \sigma_{\hat{\mu}} = \frac{\sigma_x}{\sqrt{n}}
  \]

  \paragraph{a\%-betrouwbaarheidsinterval voor $p$}
  \label{sec:aBetrouwbaarheidsintervalVoorP}
  \[
    \left[\hat{p}-z_a\cdot\sqrt{\frac{\hat{p}\left(1-\hat{p}\right)}{n}};
      \hat{p}+z_a\cdot\sqrt{\frac{\hat{p}\left(1-\hat{p}\right)}{n}}\right]
  \]

  \paragraph{a\%-betrouwbaarheidsinterval voor $\hat{\mu}$}
  \label{sec:aBetrouwbaarheidsintervalVoorMu}
  \[
    \left[\mu-z_a\cdot\frac{\sigma}{\sqrt{n}};
          \mu+z_a\cdot\frac{\sigma}{\sqrt{n}}\right]
  \]


\subsection{Lineaire enkelvoudige regressie en correlatie}
\label{sec:LineaireEnkelvoudigeRegressieEnCorrelatie}
  \paragraph{Regressielijn Kleinste kwadratenmethode}
  \label{sec:KleinsteKwadratenmethode}
    \[
     y = ax + b
    \]
    \[
     a = \frac{\sum_{i=1}^{n}
               \left(x_i-\overline{x}\right)\left(y_i-\overline{y}\right)
              }{\sum_{i=1}^{n}\left(x_i-\overline{x}\right)^2}
    \]
    \[
     b = \overline{y} - a\overline{x}
    \]


\paragraph{Correlatiecoëfficiënt}
\label{sec:Correlatiecoefficient}
  \[
     r = \frac{1}{n-1}\sum_{i=1}^{n}z_x \cdot z_y
       = \frac{1}{n-1}\sum_{i=1}^{n}\left(\frac{x_i-\overline{x}}{s_x}\right)
                             \cdot \left(\frac{y_i-\overline{y}}{s_y}\right)
       = \frac{\sum_{i=1}^{n}\left(x_i-\overline{x}\right)\left(y_i-\overline{y}\right)}
              {\left(n-1\right)s_x s_y}
    \]


 
 \ttfamily  

\end{document}
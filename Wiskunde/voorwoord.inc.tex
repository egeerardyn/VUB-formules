\section*{Voorwoord}
\label{sec:Voorwoord}
  Deze uitgave is geen officiële uitgave van de Vrije Universiteit Brussel, slechts een formularium gemaakt door een student.
  Mogelijk staan er hier of daar nog fouten in, indien u er tegenkomt,
  stuur gerust een mailtje naar \hreftt{mailto:egon.geerardyn@vub.ac.be}{egon.geerardyn@vub.ac.be}.\par

  \begin{quote}
    Copyright \copyright{}  Egon Geerardyn.\par
    Permission is granted to copy, distribute and/or modify this document
    under the terms of the GNU Free Documentation License, Version 1.2
    or any later version published by the Free Software Foundation;
    with no Invariant Sections, no Front-Cover Texts, and no Back-Cover Texts.
    A copy of the license is included in the section entitled ``GNU
    Free Documentation License'' in the source code and available on:
    \linktt{http://www.gnu.org/copyleft/fdl.html}.
  \end{quote}
  \noindent
  De \LaTeX -broncode is vrij beschikbaar onder GNU Free Document License. \par


  Mogelijk is er reeds een nieuwe versie beschikbaar op\par
  \hreftt{http://students.vub.ac.be/~egeerard/projects.html}{http://students.vub.ac.be/\tildefix egeerard/projects.html}


\section*{Referenties}
\label{sec:Referenties}
\begin{enumerate}
  \item[0.] e.a., \textit{Van Basis Tot Limiet 5/6 (\ldots)}, Die Keure 2004.
  \item \textsc{S. Caenepeel}, \textit{Analyse I}, Dienst uitgaven Vrije Universiteit Brussel 2006.
  \item \textsc{S. Caenepeel}, \textit{Analyse II}, Dienst uitgaven Vrije Universiteit Brussel 2006.
  \item \textsc{S. Caenepeel}, \textit{Complexe Analyse}, Dienst uitgaven Vrije Universiteit Brussel 2007.
  \item \textsc{S. Caenepeel}, \textit{Basistechnieken voor Computersimulaties}, Dienst uitgaven Vrije Universiteit Brussel 2001.
  \item \textsc{Ph. Cara}, \textit{Lineaire Algebra Volume I}, Dienst uitgaven Vrije Universiteit Brussel 2006.
  \item \textsc{Ph. Cara}, \textit{Lineaire Algebra Volume II}, Dienst uitgaven Vrije Universiteit Brussel 2006.
  \item \textsc{J.C.A. Wevers}, \textit{Wiskundig Formularium},
  \hreftt{http://www.xs4all.nl/~johanw/}{http://www.xs4all.nl/\tildefix johanw/} 2005.
  \item \textsc{J. Claeys}, \textit{MATH-abundance},
  \hreftt{http://home.scarlet.be/~ping1339/}{http://home.scarlet.be/\tildefix ping1339/} 2004.
\end{enumerate}

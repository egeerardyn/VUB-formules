\subsection{Hyperbolische functies}
\label{sec:HyperbolischeFuncties}
  \begin{eqnarray*}
    \cosh x = \frac{e^x+e^{-x}}{2} & \qquad \qquad \qquad &%
    \tanh x = \frac{\sinh x}{\cosh x} = \frac{e^x-e^{-x}}{e^x+e^{-x}} \\
    %
    \sinh x = \frac{e^x-e^{-x}}{2} & \qquad \qquad \qquad &%
    \coth x = \frac{\cosh x}{\sinh x} = \frac{e^x+e^{-x}}{e^x-e^{-x}}
  \end{eqnarray*}
  
\paragraph{Hoofdeigenschap}
\label{sec:HoofdeigenschapHyperbolisch}
   \[
	   \cosh^2 \alpha - \sinh^2 \alpha = 1
   \]
  
\paragraph{Som- en verschilformules}
\label{sec:SomVerschilHyperbolisch}
   \[
	   \sinh\left( x \pm y \right)  = \sinh x \cosh y \pm \cosh x \sinh y 
   \]
   \[
	   \cosh\left( x \pm y \right)  = \cosh x \cosh y \mp \sinh x \sinh y 
   \]

\paragraph{Symmetrie van Hyperbolische functies}
\label{sec:SymmetrieHyperbolisch}

\begin{center}
	\begin{tabular}{|c||c|c|c|c||l|}
	  \hline
	  $\swarrow$ & $\sinh$ & $\cosh$ & $\tanh$ & $\coth$ & beschrijving\\
	  \hline \hline
	  $- x$ & $-\sinh x$ & $\cosh x$ & $-\tanh x$ & $-\coth x$ & tegengesteld\\
	  \hline
	\end{tabular}
\end{center}

\paragraph{Inverse functies}
\label{sec:invHyperbFunc}
\[
  \argsinh \left( x \right) = \ln \left( x + \sqrt{x^2+1} \right)
\]
\[
  \argcosh \left( x \right) = \ln \left( x + \sqrt{x^2-1} \right)
\]
\[
  \argtanh \left( x \right) = \frac{1}{2} \ln \frac{1+x}{1-x}
\]

\paragraph{Omzettingstabel}
\label{sec:OmzettingHyperbolisch}

\newcommand{\FSH}[1]{\sinh^{#1}\left(x\right)}
\newcommand{\FCH}[1]{\cosh^{#1}\left(x\right)}
\newcommand{\FTH}[1]{\tanh^{#1}\left(x\right)}
\newcommand{\FCT}[1]{\coth^{#1}\left(x\right)}
\begin{center}
\[
  \begin{array}{|l|c|c|c|c|}
  \hline    & \FSH{} & \FCH{} & \FTH{} & \FCT{} \\ \hline \hline
                                                             
  \FSH{} =  & \FSH{}%                        
            & \pm \sqrt{\FCH{2}-1}% 
            & \frac{\FTH{}}{\sqrt{1-\FTH{2}}}%
            & \pm \frac{1}{\sqrt{\FCT{2}-1}}%
            \\ \hline %
  \FCH{} =  & \sqrt{1+\FSH{2}}%
            & \FCH{}%
            & \frac{1}{\sqrt{1-\FTH{2}}}%
            & \frac{\abs{\FCT{}}}{\sqrt{\FCT{2}-1}}%
            \\ \hline %
  \FTH{} =  & \frac{\FSH{}}{\sqrt{1+\FSH{2}}}%
            & \pm \frac{\sqrt{\FCH{2}-1}}{\FCH{}}%
            & \FTH{}%
            & \frac{1}{\FCT{}}%
            \\ \hline %
  \FCT{} =  & \frac{\sqrt{1+\FSH{2}}}{\FSH{}}%
            & \pm \frac{\FCH{}}{\sqrt{\FCH{2}-1}}%
            & \frac{1}{\FTH{}}
            & \FCT{}%
            \\ \hline
  \end{array}
\]
waarin $\pm$ overeenkomt met het teken van $x$
\end{center}

\subsection{Vergelijkingen oplossen}
\label{sec:NAVgln}
\[
 f(x) = 0 \Rightarrow x = ?
\]
Conventies
\begin{eqnarray*}
  f_n &=& f(x_n)\\
  f'_n &=& f'(x_n)
\end{eqnarray*}
Algemeen algoritme:
\[
  x_{n+1} = x_n + k(x)f(x)
\]

\paragraph{Newton} (tweede orde)
\[
  x_{n+1} = x_n - \frac{f_n}{f'_n}
\]
\begin{itemize}
  \item[\pro] Vrij snelle convergentie
  \item[\con] Afgeleide berekenen is moeilijk
\end{itemize}


\paragraph{Vaste richting}
\[
 x_{n+1} = x_n - \frac{f_n}{f'_0}
\]
\begin{itemize}
 \item[\pro] Slechts 1 afgeleide te berekenen
 \item[\con] Tragere convergentie dan Newton
\end{itemize}

\paragraph{Koorde}
\[
 x_{n+1} = \frac{x_nf_{n-1} - x_{n-1}f_n}{f_{n-1} - f_n}
\]
\begin{itemize}
  \item[\warn] $x_0$ en $x_1$ nodig bij het begin
  \item[\pro] Geen afgeleide nodig
\end{itemize}


\paragraph{Regula Falsi}
\begin{itemize}
  \item[\warn] Zelfde als methode van de koorde, met de extra voorwaarde dat $x_{k}$ en $x_{k+1}$ steeds een verschillend teken hebben op elke iteratiestap $k$.
\end{itemize}


\paragraph{Dichotomie}
\begin{itemize}
  \item[\warn] Analoog aan regula falsi, maar het interval wordt steeds in twee exact even grote delen gesplitst.
  \item[\con] Tragere convergentie dan regula falsi
\end{itemize}

\paragraph{Newton voor veelterm}
\[
 f(x) = \sum_{i = 0}^N a_i x^{N-i} = a_0x^N + a_1x6 {N-1} + \cdots + a_N
\]
\begin{eqnarray*}
  b_0     & = & a_0\\
  c_0     & = & a_0 + a_k \\
  b_k     & = & x_n b_{k-1} + a_k\qquad \forall k = 1 \to N\\
  c_k     & = & x_n c_{k-1} + b_k\qquad \forall k = 1 \to N-1\\
  x_{n+1} & = & x_n - \frac{b_N}{c_{N-1}}
\end{eqnarray*}


\subsection{Stelsels oplossen}
\label{sec:NAStelsels}
\[
  \vec{F}(\vec{x}) = 0
  \Leftrightarrow
  \left\{
  \begin{array}{l}
    f_1(x_1, \ldots, x_N) = 0 \\
    \vdots \\
    f_N(x_1, \ldots, x_N) = 0
  \end{array}
  \right.
\]
Algemeen algoritme:
\[
  \vec{x}_{m+1} = \vec{x}_{m} + K(\vec{x}_m)\vec{F}(\vec{x}_m) \qquad \mbox{met $K$ een $n\times n$-matrix}
\]
\paragraph{Newton-Raphson} (cfr. Newton)
\[
  \pdiff{(f_1, \ldots, f_n)}{(x_1, \ldots, x_n)}(\vec{x}) =
  J(\vec{x}) =
  \left[
  \begin{array}{ccc}
    \pdiff{f_1}{x_1} (\vec{x}) & \cdots & \pdiff{f_1}{x_n}(\vec{x})\\
    \vdots & \ddots & \vdots\\
    \pdiff{f_n}{x_1} (\vec{x})& \cdots & \pdiff{f_n}{x_n} (\vec{x})\\
  \end{array}
  \right]
  \qquad \qquad
  \mbox{Jacobiaan}
\]
\[
 \vec{x}_{m+1} = \vec{x}_m - J^{-1}(\vec{x}_m) \vec{F}(\vec{x}_m)
\]
Per iteratie een lineair stelsel oplossen.

\paragraph{Morrey} (cfr. vaste richting)
\[
 \vec{x}_{m+1} = \vec{x}_m - J^{-1}(\vec{x}_0) \vec{F}(\vec{x}_m)
\]

\paragraph{Diagonaalterm}
\[
  D(\vec{x}) =
  \left[
  \begin{array}{ccc}
    \pdiff{f_1}{x_1} (\vec{x}) & 0 & 0\\
    0 & \ddots & 0\\
    0 & 0 & \pdiff{f_n}{x_n} (\vec{x})\\
  \end{array}
  \right]
  \qquad \qquad
  \mbox{diagonaalmatrix van de Jacobiaan}
\]
\[
 \vec{x}_{m+1} = \vec{x}_m - D^{-1}(\vec{x}_m) \vec{F}(\vec{x}_m)
\]
\begin{itemize}
  \item[\con] Enkel gebruiken als de niet-diagonaal elementen van de Jacobiaan verwaarloosbaar klein zijn.
\end{itemize}


\subsection{Lineaire stelsels oplossen}
\label{sec:NALinStelsels}

\subsubsection{Iteratieve methodes}
\label{sec:NALinSIteratief}

\paragraph{Versnelling van de convergentie: overrelaxatie}
\[
  \omega \geq 1
\]

\paragraph{Stabilisatie van de convergentie: onderrelaxatie}
\[
  \omega \leq 1
\]

\paragraph{Jacobi}
\[
  G =
  \qquad \mbox{iteratiematrix}
\]


\paragraph{Gauss-Seidel}


\subsubsection{Directe methodes}
\label{sec:NALinSDirect}



\subsection{Eigenwaardes en eigenvectoren}

\subsection{Interpolatie}
Vertrekkende van een functietabel $f_i = f(x_i)$ voor waarden van $i = 0 \to n$,
de functie benaderen in de punten daartussen of daarbuiten. Eventueel zijn ook $f_i' = f'(x_i)$ gekend.

\subsubsection{Lagrangeveeltermen}
\[
  L_j(x) = \prod_{\stackrel{i=0}{i\neq j}}^n \frac{x-x_i}{x_j-x_i}
\]
\[
  p_n(x) = \sum_{i=0}{n} f_i L_i(x)
\]

\subsubsection{Hermiteveeltermen}
\[
  H(x) = \sum_{i=0}^n f_i H_{i} + \sum_{i=0}^n f_i' \hat{H}_{i}
\]
\[
  H_{i}(x) = \left[  1 - 2(x-x_i) L_i'(x_i)\right]L_i^2(x)
\]
\[
  \hat{H}_i(x) = (x-x_i)L_i^2(x)
\]

\subsubsection{Cubic Spline}
\[
  \forall x \in [x_j,x_{j+1}]:\quad S_j(x) = a_j + b_j(x-x_j) + c_j(x-x_j)^2 + d_j(x-x_j)^3
\]
\paragraph{Vrije / natuurlijke randvoorwaarden}
\[
  \begin{array}{l}
    S''(x_0) = S''0(x_0) = 0\\
    S''(x_n) = S''_{n-1} = 0
  \end{array}
\]

\paragraph{Ingeklemde randvoorwaarden}
\[
  \begin{array}{l}
    S'(x_0) = S'0(x_0) = f'(x_0)\\
    S'(x_n) = S'_{n-1} = f'(x_n)
  \end{array}
\]

\paragraph{Uitwerking voor vrije randvoorwaarden}
\[
  h_j = x_{j=1} - x_j
\]
\[
  \forall j \in [0,n]:\quad a_j = f_j
\]
\[
  \forall j \in [0,n-1]:\quad b_j = \frac{a_{j+1}-a_j}{h_j} - \left.\left.\frac{h_j}{3}\right( 2*c_j + c_{j+1}\right)
\]
\[
  \forall j \in [0,n-1]:\quad d_j = \frac{c_{j+1} - c_j}{3h_j}
\]







\subsection{Least-squares benadering}

\subsection{Numerieke differentiatie}

\subsection{Numerieke integratie}
\[
  \int_a^b f(x) \d{x} \quad \mbox{berekenen}
\]

\subsubsection{Newton-Cotes-integratie}
Deze formules worden ook vaak gebruikt door per deelinterval te integreren om zo hun nauwkeurigheid op te drijven. Per conventie wordt de integratie uitgevoerd over het interval $[a,b]$, waarbij de functiewaardes $f_0 = f(x_0)$ tot $f_n = f(x_n)$ gebruikt worden. $\mathrm{L}_i(x)$ is de interpolatieveelterm van Lagrange. We werken hier steeds met equidistante spilpunten.

\[
  \int_a^b f(x) \d{x} = \sum_{i=0}^n c_i f_i + E(f)
\]
\[
  c_i = \int_a^b \mathrm{L}_i(x)\d{x}
  = \int_a^b  \prod_{\stackrel{j=0}{j\neq i}}^n \frac{x-x_j}{x_i-x_j}\d{x}
  = \frac{(-1)^n}{h^n i! (n-i)!} \int_a^b  \prod_{\stackrel{j=0}{j\neq i}}^n (x - x_j)\d{x}
\]

\paragraph{Open formules} ($a$ en $b$ worden niet als spilpunt gebruikt)
\[
  \begin{array}{l}
    a = x_{-1}\\
    b = x_{n+1}\\
    h = \frac{b-a}{n+2}\\
    x_k= x_{-1} + (k+1)h\\
  \end{array}
\]
\begin{eqnarray*}
  \int_a^b f(x)\d{x} & = & \ldots\\
  \int_{x_{-1}}^{x_1} f(x)\d{x} & = & 2hf_0 + \frac{h^3}{3}f^{(2)}(\xi) \qquad \mbox{centrale Riemannsom}\\
  \int_{x_{-1}}^{x_2} f(x)\d{x} & = & \left.\left. \frac{3h}{2}\right(f_0 + f_1\right) + \frac{3h^3}{4} f^{(2)}(\xi)\\
  \int_{x_{-1}}^{x_3} f(x)\d{x} & = & \left.\left. \frac{4h}{3}\right(2f_0 -f_1 +2f_2\right) + \frac{28h^5}{90} f^{(4)}(\xi) \qquad \mbox{Milne}\\
  \int_{x_{-1}}^{x_4} f(x)\d{x} & = & \left.\left. \frac{5h}{24}\right(11f_0 +f_1 +f_2 + 11f_3\right) + \frac{95h^5}{144} f^{(4)}(\xi)
\end{eqnarray*}


\paragraph{Gesloten formules} ($a$ en $b$ worden wel als spilpunt gebruikt)
\[
  \begin{array}{l}
    a = x_0\\
    b = x_n\\
    h = \frac{b-a}{n}\\
    x_k= x_0 +kh\\
  \end{array}
\]
\begin{eqnarray*}
  \int_a^b f(x)\d{x} & = & \ldots\\
  \int_{x_0}^{x_ 1} f(x)\d{x} & = & f\left( \frac{a+b}{2} \right)\\
  \int_{x_0}^{x_2} = \left. \left. \frac{h}{3} \right( f_0 + 4f_1 + f_2 \right) - \frac{h^5}{90}f^{(4)}(\xi) \qquad \mbox{Simpson}\\
\end{eqnarray*}


\paragraph{Trapeziumregel} (gesloten met 2 spilpunten)
\[
  \int_{x_0}^{x_1} f(x)\d{x} = \left. \left. \frac{h}{2} \right(f_0 + f_1\right)
\]


\paragraph{Simpson} (gesloten met 3 spilpunten)
\[
  x
\]


\paragraph{Milne} (open met 3 spilpunten)
\[
x
\]

\subsubsection{Gauss-integratie}


\subsection{Differentiaalvergelijkingen met beginvoorwaarden}
\label{sec:DiffVglnNum}

\paragraph{Probleemstelling} Oplossen van een differentiaalvergelijking,
of een stelsel differentiaalvergelijkingen. Voor hogere ordes herleiden tot $N \times N$-stelsel van eerste orde.
\[
  \frac{\d{y}}{\d{x}} = f(x, y)
  \qquad
  \mbox{of}
  \qquad
  \frac{\d{\vec{y}}}{\d{x}} = f(x, \vec{y})
  \qquad
  \mbox{met }
  \vec{y}(x_0) = \vec{y}_0
\]
Hierbij maakt men gebruik van een stapgrootte $h$.

\subsubsection{Expliciete eenstapsmethodes}
Predictors worden steeds als $\hat{y}$ weergegeven, $f_i$ is een verkorte notatie voor $f(x_i,y_i)$.

\paragraph{Euler} (orde 1)
\[
  y_{n+1} = y_n + h \cdot f_n
\]

\paragraph{Heun} (orde 2).
\[
  \hat{y}_{n+1} = y_n + h \cdot f_n
\]
\[
  y_{n+1} = y_n + \left.\left. \frac{h}{2}\right(f(x_n,y_n) + f(x_{n+1},\hat{y}_{n+1}) \right)
\]

\paragraph{Midpoint method} (orde 2)
\[
  \hat{y}_{n+1} = y_n + h \cdot f_n
\]
\[
  y_{n+1} = y_n + h \cdot f\left(x_n + \frac{h}{2}, y_n + \frac{h}{2} f(x_n,y_n) \right)
\]

\paragraph{Runge-Kutta I} (orde 4)
\begin{eqnarray*}
  k_1 & = & h\cdot f_n\\
  k_2 & = & h\cdot f\left(x_n+\frac{h}{2},y_n+\frac{k_1}{2}\right)\\
  k_3 & = & h\cdot f\left(x_n+\frac{h}{2},y_n+\frac{k_1+k_2}{4}\right)\\
  k_4 & = & h\cdot f\left(x_n+h,y_n-k_2+2k_3\right)
\end{eqnarray*}
\[
  y_{n+1} = y_n + \frac{k_1 + 4k_3 + k_4}{6}
\]

\paragraph{Runge-Kutta II} (orde 4)
\begin{eqnarray*}
  k_1 & = & h\cdot f_n\\
  k_2 & = & h\cdot f\left(x_n+\frac{h}{2},y_n+\frac{k_1}{2}\right)\\
  k_3 & = & h\cdot f\left(x_n+\frac{h}{2},y_n+\frac{k_2}{2}\right)\\
  k_4 & = & h\cdot f\left(x_n+h,y_n+k_3\right)
\end{eqnarray*}
\[
  y_{n+1} = y_n + \frac{k_1 + 2k_2 + 2k_3 + k_4}{6}
\]

\subsubsection{Expliciete meerstapsmethodes}
Om te starten, maakt men voor de eerste $k$ spilpunten gebruik van een eenstapsmethode van dezelfde of hogere orde, voordat men deze formules kan toepassen.

\paragraph{Adams-Bashforth} (2 spilpunten, orde 2)
\[
  y_{n+1} = y_n + \left.\left. \frac{h}{2} \right( 3f_n - f_{n-1}\right)
\]

\paragraph{Adams-Bashforth} (3 spilpunten, orde 3)
\[
  y_{n+ 1} = y_n + \left.\left. \frac{h}{12} \right( 23f_n - 16f_{n-1} + f_{n-2}\right)
\]

\paragraph{Adams-Bashforth} (4 spilpunten, orde 4)
\[
  y_{n+1} = y_n + \left.\left. \frac{h}{24} \right( 55f_n - 59f_{n-1} + 37f_{n-2} - 9f_{n-3}\right)
\]

\subsubsection{Impliciete meerstapsmethodes}
Praktisch gezien zal men voor $f_{n+1}$ een predictor $\hat{f}_{n+1}$ gebruiken, zoals berekend met de expliciete meerstapsmethodes. Hierbij maakt men steeds gebruik van een methode van gelijke of hogere orde. Dit is voor Adams-Moluton en Adams-Bashforth gelijk aan het aantal spilpunten.

\paragraph{Adams-Moulton / Crank-Nicholson / trapeziumschema} (2 spilpunten, orde 2)
\[
  y_{n+1} = y_n + \left.\left. \frac{h}{2} \right( \hat{f}_{n+1} + f_n\right)
\]

\paragraph{Adams-Moulton} (3 spilpunten, orde 3)
\[
  y_{n+1} = y_n + \left.\left. \frac{h}{12} \right( 5 \hat{f}_{n+1} + 8f_n -f_{n-1}\right)
\]

\paragraph{Adams-Moulton} (4 spilpunten, orde 4)
\[
  y_{n+1} = y_n + \left.\left. \frac{h}{24} \right( 9 \hat{f}_{n+1} + 19f_n -5f_{n-1} + f_{n-2}\right)
\]

\subsection{Differentiaalvergelijkingen met randvoorwaarden}
\paragraph{Probleemstelling}
\[
  \frac{\d^2{y}}{\d{x}} = f\left(x,y, \frac{\d{y}}{\d{x}} \right)
  \qquad
  \mbox{met randvoorwaarden $y(a) = y_a$ en $y(b) = y_b$}
\]

\paragraph{Shooting method]}Door $y'(a) = \lambda_a$ te gokken en de zo bekomen functie op te lossen met een methode voor beginvoorwaarden. Nadien kan gecontroleerd worden of de geschatte waarde plausibel is, door de randvoorwaarden na te gaan.

\paragraph{Discretisatie} Men deelt het interval $[a,b]$ op in $n$ deelintervallen en stelt $h = \frac{b-a}{n}$ en $x_k = a + kh$. Men bepaalt de oplossing door in de differentiaalvergelijking alle afgeleides te vervangen door een numerieke afleiding en zodoende bekomt men per spilpunt een lineaire vergelijking met $y_k$ als onbekenden, ofwel in het totaal een lineair $n \times n$-stelsel.
\section{Elektromagnetisme}
\label{sec:EM}


\paragraph{Samenvatting: 4 Maxwelbetrekkingen}

  \begin{eqnarray*}
     \vdiv{\vec{E}} = \frac{\rho_{alle}}{\epsilon_0} &&
     \mbox{Gauss} \\
     c^2 \vrot{\vec{B}} = \frac{\vec{J}_{alle}}{\epsilon_0} + \frac{\pd{\vec{E}}}{\pd{t}} &&
     \mbox{Ampère-Maxwell}\\
     \vrot{\vec{E}} = - \frac{\pd{\vec{B}}}{\pd{t}} &&
     \mbox{Faraday-Lenz}\\
     \vdiv{\vec{B}} = 0 &&
  \end{eqnarray*} 




\paragraph{Coulomb}
\[
  \vec{F}_{q(tot)} = \frac{q}{4 \pi \epsilon_0} \sum_{i=1}^n \frac{q_i \left(\vec{r} - \vec{r}_i\right)}{\norm{\vec{r} - \vec{r}_i}^3}
\]
\paragraph{Gauss/Maxwell 1}
\[
  \Phi_E = \oiint_{\partial V}\vec{E}(\vec{r})\cdot \vec{n}_u \d{S} = \frac{Q_{in}}{\epsilon_0}
\]
\[
  \vdiv{\vec{E}} = \frac{\rho_{elek}}{\epsilon_0}
\]


\paragraph{Elektrostatische potentiaal}
\[
  U_{elek} = E_{p(elek)} = q V
\]
\[
  \vec{E} = -\vgrad{V}
\]
\[
  \Delta V = - \int_\ell \vec{E} \cdot \d{\vec{\ell}}
\]


\paragraph{Dipoolmoment}
\[
  \vec{\mu}_{el} = |q| \vec{\ell} \qquad \mbox{met $\vec{\ell}$ gericht van negatief naar positief}
\]
Bijhorende potentieële energie
\[
  U(\theta) = - \vec{\mu}_{el} \cdot \vec{E}
\]


\paragraph{Stromen}
\[
  I = \iint_S \vec{J}_{el} \cdot \d{\vec{S}} = \iint_S \vec{J}_{el} \cdot \vec{n}_+ \d{s}
\]


\paragraph{Lorentzkracht}
\[
  \vec{F}_{Lo} = q \left(\vec{E} + \vec{v} \times \vec{B} \right)
\]


\paragraph{Laplacekracht}
\[
  \vec{F}_{La} = I \int_{\ell} \d{\vec{\ell}} \times \vec{B}
\]

\paragraph{Definitie magnetisch inductieveld}
\[
  \vec{B} = \frac{\mu_0}{4\pi} \sum_i \frac{q_i \vec{v}_i \times \left( \vec{r} - \vec{r}_i \right) }{\norm{\vec{r} - \vec{r}_i}^3}
\]


\paragraph{Biot en Savart}
\[
  \d{\vec{B}}(\vec{r}) = \frac{\mu_0 I}{4\pi} \frac{\d{\vec{\ell}} \times \vec{r}_\Delta}{\norm{\vec{r}_\Delta}^3} 
  \qquad
  \mbox{met $\vec{r}_\Delta = \vec{r} - \vec{r}_i$ en $\ell$ de (booglengte van) geleider} 
\]
\[
  \vec{B}_{tot} = \frac{I \mu_0}{4\pi} \int_\ell \frac{\d{\vec{\ell}} \times \vec{r}_\Delta}{\norm{\vec{r}_\Delta}^3}
\]

\paragraph{Cyclotron}
\[
  f = \frac{qB}{2\pi m}
\]
\[
  \omega_c = \frac{m}{qB}
\]

Larmorstraal
\[
  r_L = \frac{v_{\perp}}{\omega_c} = \frac{mv_{\{perp}}{qB}
\]
Driftsnelheid
\[
  v_D = \frac{I}{S n_e \abs{q_e}} = \frac{I}{S \rho_{el}}
\]

\paragraph{Potentiële energie in een magnetisch veld}
\[
  U_{magn} = E_{p(magn)} = - \vec{\mu}_{magn} \cdot \vec{B}
\]


\paragraph{Ampère2} (enkel bij een constante stroom!)
\[
  c^2 \vrot{\vec{B}} = \frac{1}{\epsilon_0} \vec{J}_{alle}
\]
\[
  c^2 \iint_S \vrot{\vec{B}} \cdot \vec{n}_u \d{S} = c^2 \oint_{\partial S} \vec{B} \cdot \d{\vec{\ell}}
                                                   = \frac{1}{\epsilon_0} \iint_S \vec{J}_{alle} \cdot \vec{n}_u \d{S}
                                                   = \frac{I_{alle}}{\epsilon_0}
\]
\[
  \epsilon_0 \mu_0 c^2 = 1
\]

\paragraph{Ampère-Maxwell}
\[
  c^2 \vrot{\vec{B}} = \frac{1}{\epsilon_0} \vec{J}_{alle} + \frac{\pd{\vec{E}}}{\pd{t}}
\]

\paragraph{Faraday-Lenz}
\[
  \frac{\d{}}{\d{t}} \iint_S \vec{E} \cdot \vec{n}_u \d{S} = - \oint_{\partial S} \vec{E} \cdot \d{\vec{\ell}}
\]
\[
  \frac{\pd{\vec{B}}}{\pd{t}} = - \vrot{\vec{E}}
\]

\paragraph{Wet van Pouillet}
\[
  R = \frac{\rho_{R} \ell}{A} \qquad \mbox{met $\rho_R$ de resistiviteit}
\]


\paragraph{Poynting}
\[
  \vec{S} \isdef  \epsilon_0 c^2 \left( \vec{E} \times \vec{B} \right) 
\]

\[
  \rho_{EM} = \frac{\epsilon_0}{2} \norm{\vec{E}}^2 + \frac{\epsilon_0 c^2}{2} \norm{\vec{B}}^2
\]
Bilanvergelijking:
\[
  \frac{\pd{\rho_{EM}}}{\pd{t}} = - \vdiv{\vec{S}} - \vec{E} \cdot \vec{J}_{alle}
\]

\paragraph{Irradiantie}
\[
  I_{RR} = \norm{\vec{S}}
\]
\[
  \average{I_{RR}} = \frac{\epsilon_0 c}{2} \norm{\vec{E}}^2
\]
\[
  \average{\rho_{EM}} = \frac{\epsilon_0}{2} \norm{\vec{E}}^2
\]

\documentclass[pdftex,fleqn,a4paper]{article}
  
  \let\oldAuthor\author
  \renewcommand{\author}[1]{\newcommand{\theAuthor}{#1}\oldAuthor{#1}} 
  \let\oldTitle\title
  \renewcommand{\title}[1]{\newcommand{\theTitle}{#1}\oldTitle{#1}}
  \newcommand{\subtitle}[1]{\newcommand{\theSubtitle}{#1}}
  \let\oldDate\date
  \renewcommand{\date}[1]{\newcommand{\theDate}{#1}\oldDate{#1}}
  
  \usepackage{../huisstijl/vubtitlepage}
  \faculty{Faculteit Ingenieurswetenschappen}
  
  \author{\href{mailto:Egon.Geerardyn@vub.ac.be}{Egon Geerardyn}}
  \title{Formules Fysica}
  \newcommand{\revisie}{revisie 0.1}
  \date{\revisie\ (\today)}
 
 
 
 
  %\usepackage{graphicx}
  \usepackage{amsfonts}
  \usepackage{mathrsfs}
  \usepackage{fancyhdr}
  \usepackage[dutch]{babel}
  \usepackage[utf8]{inputenc}
  \usepackage[T1]{fontenc}
  \usepackage[pdftex]{hyperref}
 
  \usepackage{graphicx}
  \usepackage{wrapfig}
  \usepackage{rotating}
  %%%%

  \addtolength{\textheight}{2cm}
  \addtolength{\textwidth}{3cm}
  \addtolength{\hoffset}{-1.5cm}
  \renewcommand{\rmdefault}{cmss}
  \renewcommand{\sfdefault}{cmr}

  \newcommand{\oper}[1]{\widehat{\mathrm{#1}}}
  \newcommand{\bra}[1]{\left<#1\right|}
  \newcommand{\ket}[1]{\left|#1\right>}
  \newcommand{\braket}[2]
  
  \newenvironment{CNTABLE}{\begin{center}\begin{tabular}{|l|c|l|l|}
                           \hline
                           \textbf{Beschrijving} & \textbf{Eenheid} & \textbf{> omzetting >} & \textbf{Eenheid} \\
                           \hline}{\hline \end{tabular}\end{center}}
  \newcommand{\CNENTRY}[4]{#1 & $x = [#2]$ & $y = #3$ & $y = [#4]$\\}

  \newenvironment{CTTABLE}{\begin{center}\begin{tabular}{|l|c|r|l|}
                           \hline
                           \textbf{Beschrijving} & \textbf{Symbool} & \textbf{Waarde} & \textbf{Eenheid} \\
                           \hline}{\hline \end{tabular}\end{center}}
  \newcommand{\CTENTRY}[4]{#1 & $#2$ & $#3$ & $#4$\\}

  \newenvironment{GOTABLE}{\begin{center}\begin{tabular}{|l|r|l|}
                           \hline
                           \textbf{Beschrijving} & \textbf{Waarde} & \textbf{Eenheid} \\
                           \hline}{\hline \end{tabular}\end{center}}
  \newcommand{\GOENTRY}[3]{#1 & $#2$ & $#3$ \\}

     % Egon Geerardyn common latex Commands
%
% 2007 01 03 : Version 0.5
%
%
%
% dependencies
\usepackage[usenames]{color}
\usepackage{amsfonts}
\usepackage{mathrsfs}
% hyperlinks (pdfLaTeX)
    \newcommand{\tildefix}{\textasciitilde}
    \newcommand{\hreftt}[2]{\href{#1}{\texttt{#2}}} %teletype set link
    \newcommand{\link}[1]{\href{#1}{#1}}
    \newcommand{\linktt}[1]{\hreftt{#1}{#1}}


% symbolen
  %   \usepackage{manfnt}
  % \newcommand{\vb}{\mbox{\manstar}}
   \newcommand{\qed}{\mbox{$\Box$}}
   \newcommand{\QED}{\begin{flushright}\qed\end{flushright}}
   \newcommand{\GO}{\mbox{$\{\aleph\}$}}
   \newcommand{\n}{\mbox{$^{\mbox{n}}$}}
   \newcommand{\e}{\mbox{$^{\mbox{e}}$}}
   \newcommand{\s}{\mbox{$^{\mbox{s}}$}}
     \usepackage{wasysym}
   \newcommand{\ctr}{\mbox{\lightning}}
   \newcommand{\cte}{\mathrm{ct{^{\underline{\mathrm{e}}}}}}
   \newcommand{\vgl}{\mbox{vgl}}
   \newcommand{\hn}{\mbox{$\overline{\mbox{h}}$}}
   \newcommand{\wn}{\mbox{$\overline{\mbox{w}}$}}
   \newcommand{\zn}{\mbox{$\overline{\mbox{z}}$}}
   \newcommand{\RA}{\mbox{$\Longrightarrow$}}
   \newcommand{\LA}{\mbox{$\Longleftarrow$}}
   \newcommand{\LRA}{\mbox{$\Longleftrightarrow$}}
   \newcommand{\rer}{\mbox{$\sim$}}
   \newcommand{\isdef}{\mbox{$\stackrel{\Delta}{=}$}}
   \newcommand{\elek}{\mbox{$e^{-}$}}
   \newcommand{\prot}{\mbox{$p^{+}$}}
   \newcommand{\neut}{\mbox{$n^{0}$}}
   \newcommand{\angstrom}{\eenh{\AA}}
     \let\SavedRightarrow=\Rightarrow %fix for Marvosym rightarrow
       \usepackage{marvosym}
     \let\Rightarrow=\SavedRightarrow %fix for Marvosym rightarrow
   \newcommand{\wwwsym}{\Ecommerce}
   \newcommand{\www}[1]{\wwwsym\ #1}
   \newcommand{\pompern}{\mbox{\Gentsroom}}
   \newcommand{\busbitch}{\Ladiesroom}
   \newcommand{\msun}{m_{\odot}}
   \newcommand{\mearth}{m_{\earth}}
   \newcommand{\rearth}{r_{\earth}}
   \newcommand{\GOis}{\stackrel{$\GO$}{\approx}}

%invultemplates
   \newcommand{\invulW}{\qquad \qquad \qquad \quad}
   \newcommand{\invulWs}{\qquad \qquad \qquad}
   \newcommand{\invulF}{\pm \qquad \qquad \quad}
   \newcommand{\invulM}{\qquad \quad}
   \newcommand{\invulWF}{\invulW \invulF}
   \newcommand{\invulWFM}{\left(\invulWF\right) \cdot \invulM}
   \newcommand{\invulT}[1]{\vspace{#1}}
   \newcommand{\foutenentry}[2]{ #1      & $\invulW$ & $#2$ \\ \hline}
   \newcommand{\tablesizer}[1]{\renewcommand\arraystretch{#1}}
   \newcommand{\bigtables}{\tablesizer{2.0}}
   \newcommand{\normtables}{\tablesizer{1.0}}
   \newcommand{\foutenbespreking}[2]{\bigtables
                                      \begin{tabular}{|l|rl|}
                                          \hline
                                        \foutenentry{\textbf{Waarde} $#2$}{#1}
                                          \hline
                                        \foutenentry{Afleesfout}{#1}
                                        \foutenentry{Instelfout}{#1}
                                        \foutenentry{Instrumentfout}{#1}
                                        \foutenentry{Nulpuntsfout}{#1}
                                          \hline
                                        \foutenentry{\textbf{Totale absolute fout}}{#1}
                                          \hline
                                        \foutenentry{\textbf{Totale relatieve fout}}{}
                                      \end{tabular}
                                     \normtables
                                    }
   \newcommand{\foutenbesprekingkort}[2]{\bigtables
                                      \begin{tabular}{|l|rl|}
                                          \hline
                                        \foutenentry{\textbf{Waarde} $#2$}{#1}
                                          \hline
                                        \foutenentry{\textbf{Totale relatieve fout}}{}
                                        \foutenentry{\textbf{Totale absolute fout}}{#1}
                                      \end{tabular}
                                     \normtables
                                    }

% dutch arc-goniometric functions
    \newcommand{\bgsin}{\mbox{\textsf{Bgsin}}\,}
    \newcommand{\bgcos}{\mbox{\textsf{Bgcos}}\,}
    \newcommand{\bgtan}{\mbox{\textsf{Bgtan}}\,}
    \newcommand{\bgcot}{\mbox{\textsf{Bgcot}}\,}

%alternative notation for arc-goniometric functions
    \newcommand{\argcos}{\mbox{\textsf{argcos}}\,}
    \newcommand{\argsin}{\mbox{\textsf{argsin}}\,}
    \newcommand{\argtan}{\mbox{\textsf{argtan}}\,}
    \newcommand{\argcot}{\mbox{\textsf{argcot}}\,}

%alternative notation for arc-hyperbolic functions
    \newcommand{\argcosh}{\mbox{\textsf{argcosh}}\,}
    \newcommand{\argsinh}{\mbox{\textsf{argsinh}}\,}
    \newcommand{\argtanh}{\mbox{\textsf{argtanh}}\,}
    \newcommand{\argcoth}{\mbox{\textsf{argcoth}}\,}

% coordinaat
    \newcommand{\co}{\mbox{ \textsf{co}}\,}
% absolute waarde
    \newcommand{\abs}[1]{\left| #1 \right|}
% degree symbol
    \newcommand{\degree}[0]{^\circ}
    \newcommand{\degC}[0]{\; \degree \eenh{C}}
% ronde B voor bol
    \newcommand{\bol}[0]{\mathscr{B}}

% ronde K voor kwadriek
    \newcommand{\kwadriek}[0]{\mathscr{K}}

%differentials
    \renewcommand{\d}[1]{\;\textsf{d}#1}
    \newcommand{\pd}[1]{\partial #1}
    \newcommand{\D}{\;\textsf{D}}
    \newcommand{\pdiff}[3][]{\frac{\pd^{#1}{#2}}{\pd{#3}^{#1}}}
    \newcommand{\diff}[3][]{\frac{\d^{#1}{#2}}{\d{#3}^{#1}}}

%dot and double dot for D_t en D_t^2
    \newcommand{\dt}[1]{\dot{#1}} %dot notation for d/dt
    \newcommand{\dtt}[1]{\ddot{#1}} % double dot notation for d^2 / dt^2

%accent (acute) for D_s and D_s^2
    \newcommand{\ds}[1]{#1 \acute{}\,} % accent notation for d/ds
    \newcommand{\dss}[1]{#1 \acute{}\, \acute{}\,} % double accent notation for d^2 / ds^2

%accent notation for arbitrrary derivative of order 1 or 2
    \newcommand{\dx}[1]{#1 \grave{}\,} % accent notation for arbitrary d / dx
    \newcommand{\dxx}[1]{#1 \grave{} \, \grave{}\,} % double accent notation for d^2 / dx^2

%differential operators
    \newcommand{\vnabla}{\vec{\nabla}}
    \newcommand{\Nabla}{\vnabla}
    % nabla notated
    \newcommand{\vgradN}[1]{\Nabla #1\;}
    \newcommand{\vrotN}[1]{\Nabla \times #1\;}
    \newcommand{\vdivN}[1]{\Nabla \cdot #1\;}
    % standard (Dutch) notated
    \newcommand{\vgradT}[1]{\;\vec{\textsf{grad}}\,#1\;}
    \newcommand{\vrotT}[1]{\;\vec{\textsf{rot}}\,#1\;}
    \newcommand{\vdivT}[1]{\;\textsf{div}\,#1\;}
    % wrapper for easy switching
    \newcommand{\vgrad}[1]{\vgradT{#1}}
    \newcommand{\vrot}[1]{\vrotT{#1}}
    \newcommand{\vdiv}[1]{\vdivT{#1}}

%norm of a vector
    \newcommand{\norm}[1]{\left\| #1 \right\|}

%infinity redeclariation for use with WikiPedia LaTeX notation
    \newcommand{\infin}{\infty}

%Probability notation
    \newcommand{\prob}[1]{P\left(#1\right)}

%Combination
    \newcommand{\combination}[2]{\left( \begin{array}{c} #1 \\ #2 \end{array} \right)}

% E and Var
    \newcommand{\E}[1]{\!\mathrm{E}\left[ #1 \right]}
    \newcommand{\Var}[1]{\!\mathrm{Var}\left[ #1 \right]}

%identieke matrix
    \newcommand{\idmatrix}{\textsf{I}}
    \newcommand{\spoor}[1]{\mbox{\textsf{sp}}\left( #1 \right)\,}
%signumfunctie
    \newcommand{\sign}[1]{\textsf{sign}\left( #1 \right)}
%regel van de l'hopital
    \newcommand{\hopital}{\stackrel{\textsf{H}}{=}}
%vector functions
    % vector notation
    \newcommand{\vect}[1]{\overline{#1}} %large notation
    %(scalar product, <>-notation
    \newcommand{\scalprod}[2]{\left\langle #1,#2 \right\rangle}
    \newcommand{\scalprodv}[2]{\scalprod{\vec{#1}}{\vec{#2}}} %includes vector arrows
    \newcommand{\scalprodV}[2]{\scalprod{\vect{#1}}{\vect{#2}}}
    %vectorr product
    \newcommand{\vectprod}[2]{\left( #1 \times #2 \right)}
    \newcommand{\vectprodv}[2]{\vectprod{\vec{#1}}{\vec{#2}}} % includes vector arrows
    \newcommand{\vectprodV}[2]{\vectprod{\vect{#1}}{\vect{#2}}}
    %gradient, rotatie, divergentie
    \newcommand{\Dgrad}[1]{\textsf{grad}\,#1\;}
    \newcommand{\Ddiv}[1]{\textsf{div}\,#1\;}
    \newcommand{\Drot}[1]{\textsf{rot}\,#1\;}

%chemistry
    %concentration
    \newcommand{\conc}[1]{\left[ #1 \right]}
    %equilibrum arrows
    \newcommand{\evenwicht}{\rightleftharpoons}
    \newcommand{\reactie}{\rightarrow}
    %reactieconstante
    \newcommand{\K}{\,\textsf{K}}
    \newcommand{\Q}{\,\textsf{Q}}
    %p-notations
    \newcommand{\pH}{\,\textsf{pH}}
    \newcommand{\pOH}{\,\textsf{pOH}}
    \newcommand{\pK}{\,\textsf{pK}}
    \newcommand{\pKa}{\pK_A}
    \newcommand{\pKb}{\pK_B}
    \newcommand{\pKw}{\pK_W}
    %eenheden
    \newcommand{\eenheid}[1]{\,\textsf{#1}\,}
    \newcommand{\eenh}[1]{\eenheid{#1}}
    \newcommand{\Vr}{\,\textsf{Vr}\,}
    \newcommand{\molaliteit}{\mathbf{m}}
    \newcommand{\molar}[1]{\overline{#1}}
    \newcommand{\standard}[1]{#1^{\circ}}

% tango colors
   \definecolor{TangoButter1}{rgb}{0.9882, 0.9137, 0.3098}
   \definecolor{TangoButter2}{rgb}{0.9294, 0.8313, 0.0000}
   \definecolor{TangoButter3}{rgb}{0.7686, 0.6274, 0.0000}

   \definecolor{TangoOrange1}{rgb}{0.9882, 0.6863, 0.2431}
   \definecolor{TangoOrange2}{rgb}{0.9608, 0.4745, 0.0000}
   \definecolor{TangoOrange3}{rgb}{0.8078, 0.3608, 0.0000}

   \definecolor{TangoChocolate1}{rgb}{0.9137, 0.7255, 0.4314}
   \definecolor{TangoChocolate2}{rgb}{0.7569, 0.4902, 0.0667}
   \definecolor{TangoChocolate3}{rgb}{0.5608, 0.3490, 0.0078}

   \definecolor{TangoChameleon1}{rgb}{0.5412, 0.8863, 0.2039}
   \definecolor{TangoChameleon2}{rgb}{0.4510, 0.8235, 0.0863}
   \definecolor{TangoChameleon3}{rgb}{0.3059, 0.6039, 0.0235}

   \definecolor{TangoSkyBlue1}{rgb}{0.4471, 0.6235, 0.8118}
   \definecolor{TangoSkyBlue2}{rgb}{0.2039, 0.3961, 0.6431}
   \definecolor{TangoSkyBlue3}{rgb}{0.1255, 0.2902, 0.5294}

   \definecolor{TangoPlum1}{rgb}{0.6784, 0.4980, 0.6588}
   \definecolor{TangoPlum2}{rgb}{0.4588, 0.3137, 0.4824}
   \definecolor{TangoPlum3}{rgb}{0.3608, 0.2078, 0.4000}

   \definecolor{TangoScarletRed1}{rgb}{0.9373, 0.1608, 0.1608}
   \definecolor{TangoScarletRed2}{rgb}{0.8000, 0.0000, 0.0000}
   \definecolor{TangoScarletRed3}{rgb}{0.6431, 0.0000, 0.0000}

   \definecolor{TangoScarletRed1}{rgb}{0.9373, 0.1608, 0.1608}
   \definecolor{TangoScarletRed2}{rgb}{0.8000, 0.0000, 0.0000}
   \definecolor{TangoScarletRed3}{rgb}{0.6431, 0.0000, 0.0000}

   \definecolor{TangoAluminium1}{rgb}{0.9333, 0.9333, 0.9255}
   \definecolor{TangoAluminium2}{rgb}{0.8275, 0.8431, 0.8118}
   \definecolor{TangoAluminium3}{rgb}{0.7294, 0.7412, 0.8392}
   \definecolor{TangoAluminium4}{rgb}{0.5333, 0.5412, 0.5216}
   \definecolor{TangoAluminium5}{rgb}{0.3333, 0.3412, 0.3255}
   \definecolor{TangoAluminium6}{rgb}{0.1804, 0.2039, 0.2118}

% opmaak
    \newcommand{\opmerking}{\par\textbf{\color{TangoScarletRed3}{Opmerking: }}}
    \newcommand{\pro}{$\Box\!\!\!\!$\color{TangoChameleon3}{\ding{52}}}
    \newcommand{\con}{$\Box\!\!\!\!$\color{TangoScarletRed2}{\ding{56}}$\;$}
    \newcommand{\warn}{$\Box\!\!\!\!\!$\color{TangoSkyBlue2}{\ding{72}}$\;$}

% average over time
    \newcommand{\average}[1]{\left\langle #1\right\rangle}


    \newcommand{\definitie}[2]{\par \textbf{#1:}  #2\par}

  
 %lay-out
  \hypersetup{colorlinks,%
            citecolor=black,%
            filecolor=black,%
            linkcolor=black,%
            urlcolor=black,%
            pdfauthor={Egon Geerardyn},%
            pdftitle={Formules Fysica},%
            plainpages=false}%,%
            %pdfpagelabels}
  \pdfpagewidth=\paperwidth
  \pdfpageheight=\paperheight
 %margins
\begin{document}
  \maketitlepage
%  \titlepage
%  \thispagestyle{empty}%
%  \null
%  \vfill
%  \begin{center}\leavevmode
%    \normalfont
%    %{\Large\raggedleft schooljaren 2004 -- 2005 -- 2006\par}%
%    {\Large\raggedleft Ingenieurswetenschappen\par}%
%    \hrulefill\par
%    {\Huge\raggedright Formularium Chemie \large \revisie\   \footnotesize (\today)\par }%
%  \end{center}%
%  \vfill
%  \footnotesize
%  auteur: \textsc{E. Geerardyn}\par
%  bron: \textsc{prof. R. Willem}, \textit{Chemie: Structuur en Transformaties van de materie}, Dienst Uitgaven VUB 2006.\par
%  bron: \textsc{prof. M. Biesemans} en \textsc{prof. R. Willem} , \textit{Chemie: Oefeningen}, Dienst Uitgaven VUB 2006.\par
%  bron: \textsc{Brown, LeMay} en \textsc{Bursten} , \textit{Chemistry: the Central Science (10th Edition)}, Pearson Education 2006.
%  \normalsize
%  
%  \null
    \setlength{\voffset}{-2cm}
  
  \newpage

  \pagestyle{fancy}
  \rhead{pagina \thepage}
  \chead{\footnotesize \revisie}
  \lhead{\textbf{Formules Fysica}}
   \cfoot{}
     \section*{Voorwoord}
\label{sec:Voorwoord}
  Deze uitgave is geen officiële uitgave van de Vrije Universiteit Brussel, slechts een formularium gemaakt door een student.
  Mogelijk staan er hier of daar nog fouten in, indien u er tegenkomt,
  stuur gerust een mailtje naar \hreftt{mailto:egon.geerardyn@vub.ac.be}{egon.geerardyn@vub.ac.be}.\par

  \begin{quote}
    Copyright \copyright{}  Egon Geerardyn.\par
    Permission is granted to copy, distribute and/or modify this document
    under the terms of the GNU Free Documentation License, Version 1.2
    or any later version published by the Free Software Foundation;
    with no Invariant Sections, no Front-Cover Texts, and no Back-Cover Texts.
    A copy of the license is included in the section entitled ``GNU
    Free Documentation License'' in the source code and available on:
    \linktt{http://www.gnu.org/copyleft/fdl.html}.
  \end{quote}
  \noindent
  De \LaTeX -broncode is vrij beschikbaar onder GNU Free Document License. \par


  Mogelijk is er reeds een nieuwe versie beschikbaar op\par
  \hreftt{http://students.vub.ac.be/~egeerard/projects.html}{http://students.vub.ac.be/\tildefix egeerard/projects.html}
\section*{Referenties}
\begin{enumerate}
	\item \textsc{D. Lefeber}, \textit{Mechanica: Deel I}, Dienst Uitgaven VUB 2006.
	\item \textsc{D. Lefeber}, \textit{Mechanica: Deel II}, Dienst Uitgaven VUB 2006.
        \item \textsc{D. Lefeber}, \textit{Mechanica met ontwerpproject}, Polytechnische Kring 2007.
        \item \textsc{D. Van Hemelrijck}, \textit{Mechanica van materialen, mechanismen en vloeistoffen}, Pointcarré 2008.
	\item \textsc{D. Vandepitte}, \linktt{http://www.berekeningvanconstructies.be}, 2006.
\end{enumerate}



     \newpage
     \tableofcontents
   \newpage
   
   \newpage
\section{Golven}
\label{sec:Golven}

\paragraph{Golfbetrekking}

Vlakke monochromatisch golf
\[
  \frac{\pd^2{\Psi(x,t)}}{\pd{x}^2} - \frac{1}{v^2}\frac{\pd^2{\Psi(x,t)}}{\pd{t}^2} = 0
\]

Vectorvorm
\[
  \Delta \Psi(\vec{r},t) - \frac{1}{v^2} \frac{\pd^2{\Psi(\vec{r},t)}}{\pd{x}^2}
\]

Algemene vorm van een golf:
\[
  \Psi(x,t) = \Psi_1(x-vt) + \Psi_2(x+vt)
\]


\paragraph{Harmonische oscillator}
\[
  H(x,t) = A \sin(\kappa x- \omega t + \phi_0)
\]
\[
  \omega = \kappa \cdot v
\]



\[
  v \isdef \frac{\omega}{\kappa}
\]
Bij mechanische lin. HO:
\[
  \omega = \sqrt(\frac{\kappa}{m})
\]

\subsection{Staande Golven}
\[
  \begin{array}{ll}
    \mbox{Knoop} & x_K =  2n     \frac{\lambda}{4} \\
    \mbox{Buik } & x_B = (2n +1) \frac{\lambda}{4} 
  \end{array}
\]

Resonante staande golven (beide uiteinden vast)
\[
  \lambda_n = \frac{2 l}{n}
\]
Voor $n=1$ : grondtoon.

\subsection{Zwevingen}
\[
  \Psi_1 = A \sin(\kappa_1 x - \omega_1 t)
\]
\[
  \Psi_2 = A \sin(\kappa_2 x - \omega_2 t)
\]
\[
  \left\{
    \begin{array}{l}
      \kappa_1 = K + \delta\kappa \\
      \kappa_2 = K - \delta\kappa
    \end{array}
  \right.
  \qquad \qquad
  \left\{
    \begin{array}{l}
      \omega_1 = \Omega + \delta\omega \\
      \omega_2 = \Omega - \delta\omega
    \end{array}
  \right.
\]
\[
  \Psi = \Psi_1 + \Psi_2 = 2 A \cos \left( \delta \kappa x - \delta \omega t \right) 
                               \sin \left( K x - \Omega t \right)
\]

Bij een transversale mechnisch golf:
\[
  v = k \sqrt{\frac{T}{\mu}} = k \sqrt{\frac{\mbox{terugroepingskracht}}{\mbox{inertiefactor}}}
\]


   \newpage
\section{Geluid}
\label{sec:Geluid}

\[
  I = 
\]

\[
  \beta = 10\; \log_{10} \frac{I}{I_0} \qquad \mbox{ met } I_0 = 10^{-12} \eenh{W} \eenh{m}^{-2}
\]
   \section{Elektromagnetisme}
\label{sec:EM}


\paragraph{Samenvatting: 4 Maxwelbetrekkingen}

  \begin{eqnarray*}
     \vdiv{\vec{E}} = \frac{\rho_{alle}}{\epsilon_0} &&
     \mbox{Gauss} \\
     c^2 \vrot{\vec{B}} = \frac{\vec{J}_{alle}}{\epsilon_0} + \frac{\pd{\vec{E}}}{\pd{t}} &&
     \mbox{Ampère-Maxwell}\\
     \vrot{\vec{E}} = - \frac{\pd{\vec{B}}}{\pd{t}} &&
     \mbox{Faraday-Lenz}\\
     \vdiv{\vec{B}} = 0 &&
  \end{eqnarray*} 




\paragraph{Coulomb}
\[
  \vec{F}_{q(tot)} = \frac{q}{4 \pi \epsilon_0} \sum_{i=1}^n \frac{q_i \left(\vec{r} - \vec{r}_i\right)}{\norm{\vec{r} - \vec{r}_i}^3}
\]
\paragraph{Gauss/Maxwell 1}
\[
  \Phi_E = \oiint_{\partial V}\vec{E}(\vec{r})\cdot \vec{n}_u \d{S} = \frac{Q_{in}}{\epsilon_0}
\]
\[
  \vdiv{\vec{E}} = \frac{\rho_{elek}}{\epsilon_0}
\]


\paragraph{Elektrostatische potentiaal}
\[
  U_{elek} = E_{p(elek)} = q V
\]
\[
  \vec{E} = -\vgrad{V}
\]
\[
  \Delta V = - \int_\ell \vec{E} \cdot \d{\vec{\ell}}
\]


\paragraph{Dipoolmoment}
\[
  \vec{\mu}_{el} = |q| \vec{\ell} \qquad \mbox{met $\vec{\ell}$ gericht van negatief naar positief}
\]
Bijhorende potentieële energie
\[
  U(\theta) = - \vec{\mu}_{el} \cdot \vec{E}
\]


\paragraph{Stromen}
\[
  I = \iint_S \vec{J}_{el} \cdot \d{\vec{S}} = \iint_S \vec{J}_{el} \cdot \vec{n}_+ \d{s}
\]


\paragraph{Lorentzkracht}
\[
  \vec{F}_{Lo} = q \left(\vec{E} + \vec{v} \times \vec{B} \right)
\]


\paragraph{Laplacekracht}
\[
  \vec{F}_{La} = I \int_{\ell} \d{\vec{\ell}} \times \vec{B}
\]

\paragraph{Definitie magnetisch inductieveld}
\[
  \vec{B} = \frac{\mu_0}{4\pi} \sum_i \frac{q_i \vec{v}_i \times \left( \vec{r} - \vec{r}_i \right) }{\norm{\vec{r} - \vec{r}_i}^3}
\]


\paragraph{Biot en Savart}
\[
  \d{\vec{B}}(\vec{r}) = \frac{\mu_0 I}{4\pi} \frac{\d{\vec{\ell}} \times \vec{r}_\Delta}{\norm{\vec{r}_\Delta}^3} 
  \qquad
  \mbox{met $\vec{r}_\Delta = \vec{r} - \vec{r}_i$ en $\ell$ de (booglengte van) geleider} 
\]
\[
  \vec{B}_{tot} = \frac{I \mu_0}{4\pi} \int_\ell \frac{\d{\vec{\ell}} \times \vec{r}_\Delta}{\norm{\vec{r}_\Delta}^3}
\]

\paragraph{Cyclotron}
\[
  f = \frac{qB}{2\pi m}
\]
\[
  \omega_c = \frac{m}{qB}
\]

Larmorstraal
\[
  r_L = \frac{v_{\perp}}{\omega_c} = \frac{mv_{\{perp}}{qB}
\]
Driftsnelheid
\[
  v_D = \frac{I}{S n_e \abs{q_e}} = \frac{I}{S \rho_{el}}
\]

\paragraph{Potentiële energie in een magnetisch veld}
\[
  U_{magn} = E_{p(magn)} = - \vec{\mu}_{magn} \cdot \vec{B}
\]


\paragraph{Ampère2} (enkel bij een constante stroom!)
\[
  c^2 \vrot{\vec{B}} = \frac{1}{\epsilon_0} \vec{J}_{alle}
\]
\[
  c^2 \iint_S \vrot{\vec{B}} \cdot \vec{n}_u \d{S} = c^2 \oint_{\partial S} \vec{B} \cdot \d{\vec{\ell}}
                                                   = \frac{1}{\epsilon_0} \iint_S \vec{J}_{alle} \cdot \vec{n}_u \d{S}
                                                   = \frac{I_{alle}}{\epsilon_0}
\]
\[
  \epsilon_0 \mu_0 c^2 = 1
\]

\paragraph{Ampère-Maxwell}
\[
  c^2 \vrot{\vec{B}} = \frac{1}{\epsilon_0} \vec{J}_{alle} + \frac{\pd{\vec{E}}}{\pd{t}}
\]

\paragraph{Faraday-Lenz}
\[
  \frac{\d{}}{\d{t}} \iint_S \vec{E} \cdot \vec{n}_u \d{S} = - \oint_{\partial S} \vec{E} \cdot \d{\vec{\ell}}
\]
\[
  \frac{\pd{\vec{B}}}{\pd{t}} = - \vrot{\vec{E}}
\]

\paragraph{Wet van Pouillet}
\[
  R = \frac{\rho_{R} \ell}{A} \qquad \mbox{met $\rho_R$ de resistiviteit}
\]


\paragraph{Poynting}
\[
  \vec{S} \isdef  \epsilon_0 c^2 \left( \vec{E} \times \vec{B} \right) 
\]

\[
  \rho_{EM} = \frac{\epsilon_0}{2} \norm{\vec{E}}^2 + \frac{\epsilon_0 c^2}{2} \norm{\vec{B}}^2
\]
Bilanvergelijking:
\[
  \frac{\pd{\rho_{EM}}}{\pd{t}} = - \vdiv{\vec{S}} - \vec{E} \cdot \vec{J}_{alle}
\]

\paragraph{Irradiantie}
\[
  I_{RR} = \norm{\vec{S}}
\]
\[
  \average{I_{RR}} = \frac{\epsilon_0 c}{2} \norm{\vec{E}}^2
\]
\[
  \average{\rho_{EM}} = \frac{\epsilon_0}{2} \norm{\vec{E}}^2
\]

   \newpage
\section{Kernfysica}
\label{sec:kernfys}

\paragraph{notaties}
\[
  ^A_ZX 
\]
\[
  m\left( ^A_ZX \right) \quad \mbox{kernmassa}
\]
\[
  M\left( ^A_ZX \right) \quad \mbox{atoommassa}
\]

\paragraph{Massadefect}
\[
  m_\Delta = Z \cdot m_p + N \cdot m_n + Z \cdot m_e  \; - M\left( ^A_ZX \right)
\]

\paragraph{Bindingsenergie}
\[
  E_b = m_\Delta c^2
\]

\paragraph{Druppelmodel} Formule van vonc Weiszäcker
\[
  m\left( ^A_ZX \right) = \left( Z m_p + (A-Z)m_n \right) -a_v A + a_s A^{2/3} + a_c \frac{Z^2}{A^{1/3}} + a_A \frac{(A-2Z)^2}{A} + \Delta
\]
\begin{itemize}
 \item $\left( Z m_p + (A-Z)m_n \right)$: massaterm
 \item $-a_v A$: volumeterm, interacties binnenin de kern
 \item $a_s A^{2/3}$: oppervlakteterm, interacties aan de rand
 \item $a_c \frac{Z^2}{A^{1/3}}$: Coulombterm: afstoting tussen verschillende protonen
 \item $a_A \frac{(A-2Z)^2}{A}$: correctieterm, stabiliteit bij $N=Z$, mindere invloed bij hogere $A$
 \item $\Delta$: symmetrieterm: 
    \[
    \begin{array}{ccc}
           N | Z    & \mbox{even}                     & \mbox{oneven} \\
      \mbox{even}   & \Delta = - \frac{a_p}{\sqrt{A}} & \Delta = 0 \\
      \mbox{oneven} & \Delta = 0                      & \Delta = \frac{a_p}{\sqrt{A}} 
    \end{array}
    \]
\end{itemize}

\paragraph{Activiteit}
\[
  A(t) = \abs{\frac{\d{N}}{\d{t}}} = \lambda N(t) = \lambda N_0 e^{-\lambda t}
\]

\paragraph{$\alpha$-verval}
\[
  ^A_ZX  \longrightarrow ^{A-4}_{Z-2}Y + ^4_2He
\]

\paragraph{$\beta^-$-verval}
\[
  ^A_ZX \longrightarrow ^{A}_{Z+1}Y + e^- + \overline{\nu}
  \qquad
  \mbox{waarbij }
  n \longrightarrow p^+ + e^- + \overline{\nu}
\]


\paragraph{$\beta^+$-verval}
\[
  ^A_ZX \longrightarrow ^{A}_{Z-1}Y + e^+ + \nu
  \qquad
  \mbox{waarbij }
  n \longrightarrow p^+ + e^+ + \nu
\]

\paragraph{Elektronenvangst}
\[
  e^- + ^A_{Z+1}X \longrightarrow ^{A}_{Z}Y + + \nu
  \qquad
  \mbox{waarbij }
  e^- + n \longrightarrow p^+ + \nu
\]

\paragraph{$\gamma$-verval}

   \section{Quantumfysica}
\label{ref:QPhi}

\[
  \bra{X}\oper{H}\ket{Y}
\]


   
   \appendix
   
   
   \newpage\begin{enumerate}
         \item 
        \end{enumerate}

\section{Het SI-stelsel}
\label{app:SI}

\subsection{SI-prefixen}
\label{app:SIprefixes}

\begin{center}
% Table generated by Excel2LaTeX from sheet 'Sheet1'
\tablesizer{1.1}
\begin{tabular}{|r|c|c||r|c|c|}
\hline
\multicolumn{ 2}{|c|}{{\bf Decimale Prefix}} & {\bf Waarde} & \multicolumn{ 2}{|c|}{{\bf Binaire Prefix}} & {\bf Waarde} \\
\hline
\hline
     yotta & $\eenh{Y}$ & $10^{24^{\phantom n}}$ &       yobi & $\eenh{Yi}$ &  $2^{80^{\phantom n}}$ \\
\hline
     zetta & $\eenh{Z}$ & $10^{21^{\phantom n}}$ &       zebi & $\eenh{Zi}$ &  $2^{70^{\phantom n}}$ \\
\hline
       exa & $\eenh{E}$ & $10^{18^{\phantom n}}$ &       exbi & $\eenh{Ei}$ &  $2^{60^{\phantom n}}$ \\
\hline
      peta & $\eenh{P}$ & $10^{15^{\phantom n}}$ &       pebi & $\eenh{Pi}$ &  $2^{50^{\phantom n}}$ \\
\hline
      tera & $\eenh{T}$ & $10^{12^{\phantom n}}$ &       tebi & $\eenh{Ti}$ &  $2^{40^{\phantom n}}$ \\
\hline
      giga & $\eenh{G}$ &  $10^{9^{\phantom n}}$ &       gibi & $\eenh{Gi}$ &  $2^{30^{\phantom n}}$ \\
\hline
      mega & $\eenh{M}$ &  $10^{6^{\phantom n}}$ &       mebi & $\eenh{Mi}$ &  $2^{20^{\phantom n}}$ \\
\hline
      kilo & $\eenh{k}$ &  $10^{3^{\phantom n}}$ &       kibi & $\eenh{Ki}$ &  $2^{10^{\phantom n}}$ \\
\hline
     hecto & $\eenh{h}$ &  $10^{2^{\phantom n}}$ &            &            &            \\
\hline
      deca & $\eenh{da}$ &  $10^{1^{\phantom n}}$ &            &            &            \\
\hline
\hline
      deci & $\eenh{d}$ & $10^{-1^{\phantom n}}$ &            &            &            \\
\hline
     centi & $\eenh{c}$ & $10^{-2^{\phantom n}}$ &            &            &            \\
\hline
     milli & $\eenh{m}$ & $10^{-3^{\phantom n}}$ &            &            &            \\
\hline
     micro & $\mu$ & $10^{-6^{\phantom n}}$ &                 &            &            \\
\hline
      nano & $\eenh{n}$ & $10^{-9^{\phantom n}}$ &            &            &            \\
\hline
      pico & $\eenh{p}$ & $10^{-12^{\phantom n}}$ &            &            &            \\
\hline
     femto & $\eenh{f}$ & $10^{-15^{\phantom n}}$ &            &            &            \\
\hline
      atto & $\eenh{a}$ & $10^{-18^{\phantom n}}$ &            &            &            \\
\hline
     zepto & $\eenh{z}$ & $10^{-21^{\phantom n}}$ &            &            &            \\
\hline
     yocto & $\eenh{y}$ & $10^{-24^{\phantom n}}$ &            &            &            \\
\hline
\end{tabular}  
\tablesizer{1.0}
\end{center}

\begin{opmerking}
 Bij binaire eenheden worden vaak de decimale prefixen gebruikt met de waarde van hun binaire tegenhanger.
\end{opmerking}

   \newpage
\section{Conversies}
\label{app:Conversies}

\subsection*{Energie}
\begin{CNTABLE}
  \CNENTRY{Elektronvolt}{\eenh{eV}}{x \cdot \left|q_e\right|}{\eenh{J}}
\end{CNTABLE}

\subsection*{Temperatuur}
\begin{CNTABLE}
  \CNENTRY{Celsius}{\degree \eenh{C}}{x + 273,15}{\eenh{K}}
  \CNENTRY{Fahrenheit}{\degree \eenh{F}}{(x + 459,67) \cdot \frac{5}{9}}{\eenh{K}}
  \CNENTRY{Réaumur}{\degree \eenh{Ré}}{\frac{5}{4}x + 273,15}{\eenh{K}}
  \CNENTRY{Rømer}{\degree \eenh{Rø}}{(x-7,5)\frac{40}{21} + 273,15}{\eenh{K}}
  \CNENTRY{Rankine}{\degree \eenh{Ra}}{\frac{5}{9}x}{\eenh{K}}
\end{CNTABLE}
   \newpage
\section{Constantes}
\label{sec:app:ctes}

\begin{CTTABLE}
  \CTENTRY{Pi}{\pi}{3,141592}{}
  \CTENTRY{Lichtsnelheid in het vacuüm}{c}{299 792 458}{\eenh{m}\eenh{s}^{-1}}
  \CTENTRY{Permeabiliteit vacuüm}{\mu_0}{4\pi \cdot 10^{-7} \approx ...}{\eenh{N}\eenh{A}^{-2}}
  \CTENTRY{Permittiviteit vacuüm}{\epsilon_0}{8,85419 \cdot 10^{-12}}{\eenh{F}\eenh{m}^{-1}}
  \CTENTRY{Gravitatieconstante}{G}{6,67260 \cdot 10^{-11}}{\eenh{m}^3\eenh{kg}^{-1}\eenh{s}^{-2}}
  \CTENTRY{Valversnelling}{g}{9,81}{\eenh{m}\eenh{s}^{-2}}
  \CTENTRY{Gasconstante}{R}{8,31451}{\eenh{J}\eenh{mol}^{-1}\eenh{K}^{-1}}
  \CTENTRY{Constante van Avogadro}{N_A}{6,0228 \cdot 10^{-23}}{\eenh{mol}^{-1}}
  \CTENTRY{Constante van Boltzman}{k_B}{1,3 \cdot 10 ^{-23} }{\eenh{J}\eenh{K}^{-1}}
  \CTENTRY{Constante van Planck}{h}{6,6260693 \cdot 10^{-34} }{\eenh{J} \eenh{s}}
  \CTENTRY{Gereduceerde constante van Planck}{\hbar}{1,0545717 \cdot 10^{-34}}{\eenh{J} \eenh{s}}
  \CTENTRY{Elementaire lading/Lading van het elektron}{q_e}{-1,6021765 \cdot 10 ^{-19} }{\eenh{C}}
\end{CTTABLE}
  \[
    \epsilon_0 \mu_0 c^2 = 1
  \]
  \[
    k_B = \frac{R}{N_A}
  \]
  \[
    \hbar = \frac{h}{2\pi}
  \]


   \newpage
\section{Grootte-ordes en referentiewaardes}
\label{sec:app:GO}

\subsection*{Relatieve geluidsintensiteit}
\begin{GOTABLE}
  \GOENTRY{Onhoorbaar voor de mens}{0}{\eenh{dB}}
  \GOENTRY{Fluisteren, ritselende bladeren}{10 - 20}{\eenh{dB}}
  \GOENTRY{Gesprek}{60}{\eenh{dB}}
  \GOENTRY{Negende van Beethoven}{100}{\eenh{dB}}
  \GOENTRY{Pop/Rock-festival}{110}{\eenh{dB}}
  \GOENTRY{Pijngrens}{130}{\eenh{dB}}
\end{GOTABLE}

\subsection*{Voor de mens waarneembare spectra}
\begin{GOTABLE}
  \GOENTRY{Geluidsspectrum (frequentie)}{20 \; - \; 20 000}{\eenh{Hz}}
\end{GOTABLE}

\subsection*{Elektromagnetisch Spectrum}
\begin{GOTABLE}
  \GOENTRY{Zichtbaar licht}{350 \; - \; 750}{\eenh{nm}}
  \GOENTRY{Zichtbaar licht (frequentie)}{450 \; - \; 750}{\eenh{THz}}
  \GOENTRY{Violet}{380 \; - \; 450}{\eenh{nm}}
  \GOENTRY{Blauw}{450 \; - \; 495}{\eenh{nm}}
  \GOENTRY{Groen}{495 \; - \; 570}{\eenh{nm}}
  \GOENTRY{Geel}{570 \; - \; 590}{\eenh{nm}}
  \GOENTRY{Oranje}{590 \; - \; 620}{\eenh{nm}}
  \GOENTRY{Rood}{620 \; - \; 750}{\eenh{nm}}
\end{GOTABLE}

\subsection*{Bulkmodulus}
\begin{GOTABLE}
  \GOENTRY{Vaste stoffen en vloeistoffen}{\approx 10^{10}}{\eenh{N}{m}^{-2}}
  \GOENTRY{Water}{0,21 \cdot 10^{10}}{\eenh{N}{m}^{-2}}
  \GOENTRY{Lucht}{1,41 \cdot 10^5}{\eenh{N}{m}^{-2}}
\end{GOTABLE}

\subsection*{Massadichtheden}
\begin{GOTABLE}
  \GOENTRY{Water}{1000}{\eenh{kg}\eenh{m}^{-3}}
  \GOENTRY{Lucht}{1,21}{\eenh{kg}\eenh{m}^{-3}}
\end{GOTABLE}

\subsection*{Stroom}
\begin{GOTABLE}
  \GOENTRY{microchips}{10^{-12}\; - \;10^{-16}}{\eenh{A}}
  \GOENTRY{Cathod Ray Tube}{10^{-2}}{\eenh{A}}
  \GOENTRY{gloeilamp}{10^{0}}{\eenh{A}}
  \GOENTRY{dodelijk voor de mens}{100 \; - \; 200}{\eenh{mA}}
\end{GOTABLE}


\subsection*{Resistiviteit}
\begin{GOTABLE}
  \GOENTRY{metalen}{10^{-8}}{\Omega \eenh{m}}
  \GOENTRY{supergeleiders}{\approx 0}{\Omega \eenh{m}}
  \GOENTRY{halfgeleiders}{\approx 640}{\Omega \eenh{m}}
  \GOENTRY{isolatoren}{10^{10} \;-\; 10^{14}}{\Omega \eenh{m}}
\end{GOTABLE}

\subsection*{Geluidssnelheden}
\begin{GOTABLE}
  \GOENTRY{$v_s$ in lucht ($20 \degC$ op zeeniveau)}{\approx 344}{\eenh{m}\eenh{s}^{-1}}
  \GOENTRY{$v_s$ in lucht ($15 \degC$ op zeeniveau)}{\approx 340}{\eenh{m}\eenh{s}^{-1}}
  \GOENTRY{$v_s$ in lucht ($0 \degC$ op zeeniveau)}{\approx 331}{\eenh{m}\eenh{s}^{-1}}
  \GOENTRY{$v_s$ in water}{\approx 1551}{\eenh{m}\eenh{s}^{-1}}
  \GOENTRY{$v_s$ in staal}{\approx 5100}{\eenh{m}\eenh{s}^{-1}}
\end{GOTABLE}

\subsection*{Energie}
\begin{GOTABLE}
  \GOENTRY{bindingsenergie}{1}{\eenh{eV}}
  \GOENTRY{ionisatieenergie}{10}{\eenh{eV}}
  \GOENTRY{elektrostatische afstotingsenergie i.e. atoom}{10^{6}}{\eenh{eV}}
  \GOENTRY{$v_s$ in staal}{\approx 5100}{\eenh{m}\eenh{s}^{-1}}
\end{GOTABLE}


   
 
 

\end{document}
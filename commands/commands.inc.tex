 % Egon Geerardyn common latex Commands
%
% 2007 01 03 : Version 0.5
%
%
%
% dependencies
\usepackage[usenames]{color}
\usepackage{amsfonts}
\usepackage{mathrsfs}
% hyperlinks (pdfLaTeX)
    \newcommand{\tildefix}{\textasciitilde}
    \newcommand{\hreftt}[2]{\href{#1}{\texttt{#2}}} %teletype set link
    \newcommand{\link}[1]{\href{#1}{#1}}
    \newcommand{\linktt}[1]{\hreftt{#1}{#1}}


% symbolen
  %   \usepackage{manfnt}
  % \newcommand{\vb}{\mbox{\manstar}}
   \newcommand{\qed}{\mbox{$\Box$}}
   \newcommand{\QED}{\begin{flushright}\qed\end{flushright}}
   \newcommand{\GO}{\mbox{$\{\aleph\}$}}
   \newcommand{\n}{\mbox{$^{\mbox{n}}$}}
   \newcommand{\e}{\mbox{$^{\mbox{e}}$}}
   \newcommand{\s}{\mbox{$^{\mbox{s}}$}}
     \usepackage{wasysym}
   \newcommand{\ctr}{\mbox{\lightning}}
   \newcommand{\cte}{\mathrm{ct{^{\underline{\mathrm{e}}}}}}
   \newcommand{\vgl}{\mbox{vgl}}
   \newcommand{\hn}{\mbox{$\overline{\mbox{h}}$}}
   \newcommand{\wn}{\mbox{$\overline{\mbox{w}}$}}
   \newcommand{\zn}{\mbox{$\overline{\mbox{z}}$}}
   \newcommand{\RA}{\mbox{$\Longrightarrow$}}
   \newcommand{\LA}{\mbox{$\Longleftarrow$}}
   \newcommand{\LRA}{\mbox{$\Longleftrightarrow$}}
   \newcommand{\rer}{\mbox{$\sim$}}
   \newcommand{\isdef}{\mbox{$\stackrel{\Delta}{=}$}}
   \newcommand{\elek}{\mbox{$e^{-}$}}
   \newcommand{\prot}{\mbox{$p^{+}$}}
   \newcommand{\neut}{\mbox{$n^{0}$}}
   \newcommand{\angstrom}{\eenh{\AA}}
     \let\SavedRightarrow=\Rightarrow %fix for Marvosym rightarrow
       \usepackage{marvosym}
     \let\Rightarrow=\SavedRightarrow %fix for Marvosym rightarrow
   \newcommand{\wwwsym}{\Ecommerce}
   \newcommand{\www}[1]{\wwwsym\ #1}
   \newcommand{\pompern}{\mbox{\Gentsroom}}
   \newcommand{\busbitch}{\Ladiesroom}
   \newcommand{\msun}{m_{\odot}}
   \newcommand{\mearth}{m_{\earth}}
   \newcommand{\rearth}{r_{\earth}}
   \newcommand{\GOis}{\stackrel{$\GO$}{\approx}}

%invultemplates
   \newcommand{\invulW}{\qquad \qquad \qquad \quad}
   \newcommand{\invulWs}{\qquad \qquad \qquad}
   \newcommand{\invulF}{\pm \qquad \qquad \quad}
   \newcommand{\invulM}{\qquad \quad}
   \newcommand{\invulWF}{\invulW \invulF}
   \newcommand{\invulWFM}{\left(\invulWF\right) \cdot \invulM}
   \newcommand{\invulT}[1]{\vspace{#1}}
   \newcommand{\foutenentry}[2]{ #1      & $\invulW$ & $#2$ \\ \hline}
   \newcommand{\tablesizer}[1]{\renewcommand\arraystretch{#1}}
   \newcommand{\bigtables}{\tablesizer{2.0}}
   \newcommand{\normtables}{\tablesizer{1.0}}
   \newcommand{\foutenbespreking}[2]{\bigtables
                                      \begin{tabular}{|l|rl|}
                                          \hline
                                        \foutenentry{\textbf{Waarde} $#2$}{#1}
                                          \hline
                                        \foutenentry{Afleesfout}{#1}
                                        \foutenentry{Instelfout}{#1}
                                        \foutenentry{Instrumentfout}{#1}
                                        \foutenentry{Nulpuntsfout}{#1}
                                          \hline
                                        \foutenentry{\textbf{Totale absolute fout}}{#1}
                                          \hline
                                        \foutenentry{\textbf{Totale relatieve fout}}{}
                                      \end{tabular}
                                     \normtables
                                    }
   \newcommand{\foutenbesprekingkort}[2]{\bigtables
                                      \begin{tabular}{|l|rl|}
                                          \hline
                                        \foutenentry{\textbf{Waarde} $#2$}{#1}
                                          \hline
                                        \foutenentry{\textbf{Totale relatieve fout}}{}
                                        \foutenentry{\textbf{Totale absolute fout}}{#1}
                                      \end{tabular}
                                     \normtables
                                    }

% dutch arc-goniometric functions
    \newcommand{\bgsin}{\mbox{\textsf{Bgsin}}\,}
    \newcommand{\bgcos}{\mbox{\textsf{Bgcos}}\,}
    \newcommand{\bgtan}{\mbox{\textsf{Bgtan}}\,}
    \newcommand{\bgcot}{\mbox{\textsf{Bgcot}}\,}

%alternative notation for arc-goniometric functions
    \newcommand{\argcos}{\mbox{\textsf{argcos}}\,}
    \newcommand{\argsin}{\mbox{\textsf{argsin}}\,}
    \newcommand{\argtan}{\mbox{\textsf{argtan}}\,}
    \newcommand{\argcot}{\mbox{\textsf{argcot}}\,}

%alternative notation for arc-hyperbolic functions
    \newcommand{\argcosh}{\mbox{\textsf{argcosh}}\,}
    \newcommand{\argsinh}{\mbox{\textsf{argsinh}}\,}
    \newcommand{\argtanh}{\mbox{\textsf{argtanh}}\,}
    \newcommand{\argcoth}{\mbox{\textsf{argcoth}}\,}

% coordinaat
    \newcommand{\co}{\mbox{ \textsf{co}}\,}
% absolute waarde
    \newcommand{\abs}[1]{\left| #1 \right|}
% degree symbol
    \newcommand{\degree}[0]{^\circ}
    \newcommand{\degC}[0]{\; \degree \eenh{C}}
% ronde B voor bol
    \newcommand{\bol}[0]{\mathscr{B}}

% ronde K voor kwadriek
    \newcommand{\kwadriek}[0]{\mathscr{K}}

%differentials
    \renewcommand{\d}[1]{\;\textsf{d}#1}
    \newcommand{\pd}[1]{\partial #1}
    \newcommand{\D}{\;\textsf{D}}
    \newcommand{\pdiff}[3][]{\frac{\pd^{#1}{#2}}{\pd{#3}^{#1}}}
    \newcommand{\diff}[3][]{\frac{\d^{#1}{#2}}{\d{#3}^{#1}}}

%dot and double dot for D_t en D_t^2
    \newcommand{\dt}[1]{\dot{#1}} %dot notation for d/dt
    \newcommand{\dtt}[1]{\ddot{#1}} % double dot notation for d^2 / dt^2

%accent (acute) for D_s and D_s^2
    \newcommand{\ds}[1]{#1 \acute{}\,} % accent notation for d/ds
    \newcommand{\dss}[1]{#1 \acute{}\, \acute{}\,} % double accent notation for d^2 / ds^2

%accent notation for arbitrrary derivative of order 1 or 2
    \newcommand{\dx}[1]{#1 \grave{}\,} % accent notation for arbitrary d / dx
    \newcommand{\dxx}[1]{#1 \grave{} \, \grave{}\,} % double accent notation for d^2 / dx^2

%differential operators
    \newcommand{\vnabla}{\vec{\nabla}}
    \newcommand{\Nabla}{\vnabla}
    % nabla notated
    \newcommand{\vgradN}[1]{\Nabla #1\;}
    \newcommand{\vrotN}[1]{\Nabla \times #1\;}
    \newcommand{\vdivN}[1]{\Nabla \cdot #1\;}
    % standard (Dutch) notated
    \newcommand{\vgradT}[1]{\;\vec{\textsf{grad}}\,#1\;}
    \newcommand{\vrotT}[1]{\;\vec{\textsf{rot}}\,#1\;}
    \newcommand{\vdivT}[1]{\;\textsf{div}\,#1\;}
    % wrapper for easy switching
    \newcommand{\vgrad}[1]{\vgradT{#1}}
    \newcommand{\vrot}[1]{\vrotT{#1}}
    \newcommand{\vdiv}[1]{\vdivT{#1}}

%norm of a vector
    \newcommand{\norm}[1]{\left\| #1 \right\|}

%infinity redeclariation for use with WikiPedia LaTeX notation
    \newcommand{\infin}{\infty}

%Probability notation
    \newcommand{\prob}[1]{P\left(#1\right)}

%Combination
    \newcommand{\combination}[2]{\left( \begin{array}{c} #1 \\ #2 \end{array} \right)}

% E and Var
    \newcommand{\E}[1]{\!\mathrm{E}\left[ #1 \right]}
    \newcommand{\Var}[1]{\!\mathrm{Var}\left[ #1 \right]}

%identieke matrix
    \newcommand{\idmatrix}{\textsf{I}}
    \newcommand{\spoor}[1]{\mbox{\textsf{sp}}\left( #1 \right)\,}
%signumfunctie
    \newcommand{\sign}[1]{\textsf{sign}\left( #1 \right)}
%regel van de l'hopital
    \newcommand{\hopital}{\stackrel{\textsf{H}}{=}}
%vector functions
    % vector notation
    \newcommand{\vect}[1]{\overline{#1}} %large notation
    %(scalar product, <>-notation
    \newcommand{\scalprod}[2]{\left\langle #1,#2 \right\rangle}
    \newcommand{\scalprodv}[2]{\scalprod{\vec{#1}}{\vec{#2}}} %includes vector arrows
    \newcommand{\scalprodV}[2]{\scalprod{\vect{#1}}{\vect{#2}}}
    %vectorr product
    \newcommand{\vectprod}[2]{\left( #1 \times #2 \right)}
    \newcommand{\vectprodv}[2]{\vectprod{\vec{#1}}{\vec{#2}}} % includes vector arrows
    \newcommand{\vectprodV}[2]{\vectprod{\vect{#1}}{\vect{#2}}}
    %gradient, rotatie, divergentie
    \newcommand{\Dgrad}[1]{\textsf{grad}\,#1\;}
    \newcommand{\Ddiv}[1]{\textsf{div}\,#1\;}
    \newcommand{\Drot}[1]{\textsf{rot}\,#1\;}

%chemistry
    %concentration
    \newcommand{\conc}[1]{\left[ #1 \right]}
    %equilibrum arrows
    \newcommand{\evenwicht}{\rightleftharpoons}
    \newcommand{\reactie}{\rightarrow}
    %reactieconstante
    \newcommand{\K}{\,\textsf{K}}
    \newcommand{\Q}{\,\textsf{Q}}
    %p-notations
    \newcommand{\pH}{\,\textsf{pH}}
    \newcommand{\pOH}{\,\textsf{pOH}}
    \newcommand{\pK}{\,\textsf{pK}}
    \newcommand{\pKa}{\pK_A}
    \newcommand{\pKb}{\pK_B}
    \newcommand{\pKw}{\pK_W}
    %eenheden
    \newcommand{\eenheid}[1]{\,\textsf{#1}\,}
    \newcommand{\eenh}[1]{\eenheid{#1}}
    \newcommand{\Vr}{\,\textsf{Vr}\,}
    \newcommand{\molaliteit}{\mathbf{m}}
    \newcommand{\molar}[1]{\overline{#1}}
    \newcommand{\standard}[1]{#1^{\circ}}

% tango colors
   \definecolor{TangoButter1}{rgb}{0.9882, 0.9137, 0.3098}
   \definecolor{TangoButter2}{rgb}{0.9294, 0.8313, 0.0000}
   \definecolor{TangoButter3}{rgb}{0.7686, 0.6274, 0.0000}

   \definecolor{TangoOrange1}{rgb}{0.9882, 0.6863, 0.2431}
   \definecolor{TangoOrange2}{rgb}{0.9608, 0.4745, 0.0000}
   \definecolor{TangoOrange3}{rgb}{0.8078, 0.3608, 0.0000}

   \definecolor{TangoChocolate1}{rgb}{0.9137, 0.7255, 0.4314}
   \definecolor{TangoChocolate2}{rgb}{0.7569, 0.4902, 0.0667}
   \definecolor{TangoChocolate3}{rgb}{0.5608, 0.3490, 0.0078}

   \definecolor{TangoChameleon1}{rgb}{0.5412, 0.8863, 0.2039}
   \definecolor{TangoChameleon2}{rgb}{0.4510, 0.8235, 0.0863}
   \definecolor{TangoChameleon3}{rgb}{0.3059, 0.6039, 0.0235}

   \definecolor{TangoSkyBlue1}{rgb}{0.4471, 0.6235, 0.8118}
   \definecolor{TangoSkyBlue2}{rgb}{0.2039, 0.3961, 0.6431}
   \definecolor{TangoSkyBlue3}{rgb}{0.1255, 0.2902, 0.5294}

   \definecolor{TangoPlum1}{rgb}{0.6784, 0.4980, 0.6588}
   \definecolor{TangoPlum2}{rgb}{0.4588, 0.3137, 0.4824}
   \definecolor{TangoPlum3}{rgb}{0.3608, 0.2078, 0.4000}

   \definecolor{TangoScarletRed1}{rgb}{0.9373, 0.1608, 0.1608}
   \definecolor{TangoScarletRed2}{rgb}{0.8000, 0.0000, 0.0000}
   \definecolor{TangoScarletRed3}{rgb}{0.6431, 0.0000, 0.0000}

   \definecolor{TangoScarletRed1}{rgb}{0.9373, 0.1608, 0.1608}
   \definecolor{TangoScarletRed2}{rgb}{0.8000, 0.0000, 0.0000}
   \definecolor{TangoScarletRed3}{rgb}{0.6431, 0.0000, 0.0000}

   \definecolor{TangoAluminium1}{rgb}{0.9333, 0.9333, 0.9255}
   \definecolor{TangoAluminium2}{rgb}{0.8275, 0.8431, 0.8118}
   \definecolor{TangoAluminium3}{rgb}{0.7294, 0.7412, 0.8392}
   \definecolor{TangoAluminium4}{rgb}{0.5333, 0.5412, 0.5216}
   \definecolor{TangoAluminium5}{rgb}{0.3333, 0.3412, 0.3255}
   \definecolor{TangoAluminium6}{rgb}{0.1804, 0.2039, 0.2118}

% opmaak
    \newcommand{\opmerking}{\par\textbf{\color{TangoScarletRed3}{Opmerking: }}}
    \newcommand{\pro}{$\Box\!\!\!\!$\color{TangoChameleon3}{\ding{52}}}
    \newcommand{\con}{$\Box\!\!\!\!$\color{TangoScarletRed2}{\ding{56}}$\;$}
    \newcommand{\warn}{$\Box\!\!\!\!\!$\color{TangoSkyBlue2}{\ding{72}}$\;$}

% average over time
    \newcommand{\average}[1]{\left\langle #1\right\rangle}


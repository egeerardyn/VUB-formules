\section{Basisbegrippen}
\label{sec:Basisbegrippen}

  \subsection{Fouten}
  \label{sec:Fouten}
	  \begin{description}
	    \item[toevallig:] ten gevolge van fluctuerende invloeden
	    \item[systematisch:] ten gevolge van systematische afwijkingen
	      \begin{description}
	        \item[theoretisch:] gebruik van onjuiste (of te eenvoudige) theorie
	        \item[instrumentaal:] onjuiste ijking of calibratie van meettoestellen
        \end{description}
    \end{description}
    
  \subsection{Nauwkeurigheid}
  \label{sec:Nauwkeurigheid}
    Maat voor de herhaalbaarheid van een meetproces. 
    Hoe dichter de resultaten bij elkaar liggen, hoe nauwkeurig het meetproces.
    Wordt bepaald door de toevallige fouten.
    
  \subsection{Accuratie}
  \label{sec:Accuratie}    
    Hoe dichter de resultaten bij de referentiewaarde of echte waarden liggen, hoe hoger de accuratie.
    Wordt bepaald door de systematische fouten.
    
  
  \subsection{Foutopgave}
  \label{sec:Foutopgave}
  
    \subsubsection{Afronding}
    \label{sec:Afronding}
      \paragraph{Statistisch berekende fout (nauwkeurigheid))}
      2 beduidende cijfers
      \paragraph{Systematische fout, maximumfout (accuratie)}
      1 �f 2 beduidende cijfers
      \paragraph{Voorbeeld}
      \[
        \left( 12,834 \pm 0,018 \right ) \cdot 10^{-3} A
      \]

  \subsection{Eigenschappen meettoestellen}
  \label{sec:EigenschappenMeettoestellen}
    
    
\begin{description}
	\item[meetbereik:] Grootste aanduiding die het meetoestel kan aangeven
	\item[resolutie:] Kleinste verandering in de te meten grootheid
	\item[gevoeligheid:] Relatieve maat voor hoe veel een bepaalde verandering in de grootheid een afleesbare verandering op het meettoestel veroorzaakt (is afhankelijk van meetbereik en resolutie)
\end{description}

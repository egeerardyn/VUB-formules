\newpage
\section{Regressie-analyse}
\label{sec:RegressieAnalyse}
  \[
    \left(x , y\right) : \left(x_1 ,y_1 \right), \left(x_2 ,y_2 \right), \ldots, \left(x_n , y_n \right)
  \]
  \[
    y = \Psi\left(x\right)
  \]
  \[
    \sigma\left\{x_i\right\} = 0 \qquad \mbox{en} \qquad \sigma\left\{y_i\right\} = \sigma_i
  \]
  \paragraph{Parameters voor de regressiekromme}
    \[
      \theta_j : \theta_1, \theta_2, \ldots, \theta_k
    \]
  
  \subsection{Correlatieco�ffici�nt}
  \label{sec:correlatiecoefficient}
    \[
      R^2 = \frac{\mbox{verklaarde variantie van y}}{\mbox{totale variantie van y}}
          = 1 - \frac{\mbox{onverklaarde variantie van y}}{\mbox{totale variantie van y}}
    \]
    \[
      R^2 = 1 - \frac{\frac{1}{n}\chi^2}{\frac{1}{n}\sum_{i=1}^n \left(y_i-m_y\right)^2}
          = 1 - \frac{\chi^2}{\sum_{i=1}^n \left(y_i-m_y\right)^2}
    \]
    De correlatieco�ffici�nt is de vierkantswortel van $R^2$ met hetzelfde teken als de richtingsco�ffici�nt van de regressiekromme.
  
  \subsection{Gelijkwaardige y-waarden}
  \label{sec:RegGelijkwaardigeY}
    \[
      \chi^2 = \sum_{i=1}^n \left(y_i - \Psi\left(x_i ; \theta_j\right) \right)^2
    \]
    \paragraph{Stelsel der normaalvergelijkingen}
      \[
        \forall j \in \left[1, \ldots, k\right] : \frac{\partial \chi^2}{\partial \theta_j} 
                  = \frac{\partial}{\partial \theta_j}
                    \left[ \sum_{i=1}^n \left(y_i - \Psi\left(x_i ; \theta_j\right) \right)^2 \right]
                  = 0
      \]
    \paragraph{Standaardafwijking op y}
      \[
        \sigma^2 = \frac{1}{n} \sum_{i=1}^n \left(y_i - \Psi\left(x_i ; \theta_j\right) \right)^2 = \frac{\chi^2}{n}
      \]
      \[
        \sigma^2 \hat{=} \frac{1}{n-k} \sum_{i=1}^n \left(y_i - \Psi\left(x_i ; \hat{\theta}_j\right) \right)^2 = \frac{\chi^2}{n-k}
      \]
    \paragraph{Correlatieco�ffici�nt}
      \[
        R^2 = 1 - \frac{\sum_{i=1}^n \left(y_i - \Psi\left(x_i ; \theta_j\right) \right)^2}{\sum_{i=1}^n \left(y_i-m_y\right)^2}
      \]
      \[
        m_y = \frac{1}{n} \sum_{i=1}^n y_i
      \]
      
  \subsection{Ongelijkwaardige y-waarden met bekende nauwkeurigheid}
  \label{sec:RegOngelijkwaardigeYNauwkeurigheid}
    \[
      \chi^2 = \sum_{i=1}^n \frac{1}{\sigma_i} \left(y_i - \Psi\left(x_i ; \theta_j\right) \right)^2
    \]
  \subsection{Ongelijkwaardige y-waarden met bekend gewicht}
  \label{sec:RegOngelijkwaardigeYGewicht}
  
  \subsection{Standaardfout op de parameters}
  \label{sec:StandaardfoutOpDeParameters}
    \[
      \forall j \in \left[1,k\right]: \theta_j = \phi_j\left(x_1, \ldots, x_n ; y_1, \ldots, y_n\right\}
    \]
    \[
      \sigma^2\left\{\theta_j\right\} = \sigma^2\left\{x_i\right\} \sum_{i=1}^n \left(\frac{\partial \phi_j}{\partial x_i}\right)^2
                                      + \sigma^2\left\{y_i\right\} \sum_{i=1}^n \left(\frac{\partial \phi_j}{\partial y_i}\right)^2
                                      = \sigma^2_i \sum_{i=1}^n \left(\frac{\partial \phi_j}{\partial y_i}\right)^2
    \]
    \[
      \theta_j = \hat{\theta}_j \pm \sigma\left\{\theta_j\right\}
    \]
    
  
  
  \subsection{Praktische Toepassingen}
  \label{sec:RegToepassing}
    \subsubsection{Rechte door de oorsprong}
    \label{sec:RegAX}
  
    \subsubsection{Rechte niet door de oorsprong}
    \label{sec:RegAXplusB}
  
    \subsubsection{Veelterm}
    \label{sec:RegPolynoom}
  
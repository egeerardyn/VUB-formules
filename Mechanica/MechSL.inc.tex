\section{Mechanica van starre lichamen}

\subsection{Algemene begrippen}

\[
  \vec{r}_P = f\left(q_1, \ldots, q_n\right) \qquad \mbox{skleronoom}
\]
\[
  \vec{r}_P = f\left(q_1, \ldots, q_n, t\right) \qquad \mbox{rheonoom}
\]

\paragraph{Veralgemeende snelheden} $\dt{q}_i$
\[
  \vec{v}_P = \diff{\vec{r}_P}{t} = \pdiff{\vec{r}_P}{q_i} \dt{q}_i + \pdiff{\vec{r}_P}{t}
\]
\paragraph{Veralgemeende versnellingen} $\dtt{q}_i$
\[
  \vec{a}_P = \diff[2]{\vec{r}_P}{t} = \pdiff{\vec{r}_P}{q_i}\dtt{q}_i + \pdiff{^2\vec{r}_P}{q_i \pd{q_j}}\dt{q}_i \dt{q}_j + 2 \pdiff{^2 \vec{r}_P}{q_i \pd{t}} \dt{q}_i + \pdiff[2]{\vec{r}_P}{t}
\]

  \subsubsection{Eulerhoeken}
  \begin{tabular}{|c|c|l|}
    \hline
     \textbf{As} & \textbf{Hoek} & \textbf{Naam} \\
    \hline
    $Z_0$  & $\psi$   & Precessie\\
    $X_1$  & $\theta$ & Nutatie\\
    $Z_2$  & $\phi$   & Rotatie\\
    \hline
  \end{tabular}


  \subsubsection{Bryanthoeken}
  \begin{tabular}{|c|c|l|}
    \hline
     \textbf{As} & \textbf{Hoek} & \textbf{Naam} \\
    \hline
    $Z_0$  & $\psi$   & Precessie\\
    $X_1$  & $\theta$ & Nutatie\\
    $Y_2$  & $\phi$   & Rotatie\\
    \hline
  \end{tabular}


  \subsubsection{Roll, Pitch, Yaw / Rollen, Stampen, Gieren}
  \begin{tabular}{|c|c|l|}
    \hline
     \textbf{As} & \textbf{Hoek} & \textbf{Naam} \\
    \hline
    $X_0$   & $\gamma$ & Roll (Rollen)\\
    $Y_0$   & $\beta$  & Pitch (Stampen)\\
    $Z_0$   & $\alpha$ & Yaw (Gieren)\\
    \hline
  \end{tabular}

\subsection{Eindige verplaatsingen}

\subsubsection{Translaties}

\subsubsection{Rotaties}

\paragraph{Samenstelling rotaties}
\begin{itemize}
  \item 2 rotaties met evenwijdige assen geeft een rotatie met:
  \[
    \ldots
  \]

  \item 2 rotaties met evenwijdige assen en tegengestelde amplitude geeft een translatie met amplitude $d$ en richting $\theta$ t.o.v. $\vect{P_1 P_2}$
    \[
      d = 2 \left\| \vect{P_1 P_2} \right\| \sin{\frac{\phi}{2}}
      \qquad
      \mbox{en}
      \qquad
      \theta = \frac{\pi - \phi}{2}
    \]
\end{itemize}

\paragraph{Rotatie-elementen bepalen}
De rotatieas wordt bepaald door de eigenvector van de rotatiematrix $M$ met bijhorende eigenwaarde $\lambda = 1$.
De rotatiehoek wordt bepaald door:
  \[
    \theta = \bgcos{\frac{\spoor{M} - 1}{2}}
  \]

\paragraph{Formule van Rodriguez}
\[
  \vec{r}' = \left(\vec{1}_k \cdot \vec{r} \right)\left(1 - \cos{\phi}\right)\vec{1}_k + \cos{\phi}\; \vec{r} + \sin{\phi}\vectprod{\vec{1}_k}{\vec{r}}
\]

\subsection{Dynamica}

\paragraph{König}
\[
  E_{kin} = \frac{mv_G^2}{2} + E_{rot,G}
          = \frac{mv_G^2}{2} + \vec{v}_G \cdot \sum_h m_h \left.\vec{v}_h \right|_{rel}
\]

\subsubsection{Algemene stellingen (Newton-Euler)}
\paragraph{Kinetisch resultante}
\[
  \vect{\mathscr{R}} = \int_{(h)} \vect{v}_h \d{m}_h
    = m \vec{v}_G
\]
\[
  \diff{\vect{\mathscr{R}}}{t} = \vec{R}^u
\]

\paragraph{Kinetisch moment}
\[
  \vect{\mathscr{M}}_O = \int_{(h)} \vect{OP}_h \times \vect{v}_h \d{m}_h
    = m \vect{OG} \times \vec{v}_O + \vect{\vect{I}}_O \vec{\omega}
\]
\[
  \diff{\vect{\mathscr{M}}_Q}{t} = \vec{M}^u_Q + m \vec{v}_G \times \vec{v}_Q
\]
\[
  \vect{\mathscr{M}}_A = \vect{\mathscr{M}}_O + \vect{AO} \times \vect{\mathscr{R}}
\]


\paragraph{Euler-vergelijkingen}

In HTA:
\[
\left\{
  \begin{array}{l@{\;=\;}l}
    A \dt{p} - qr\left(B-C \right) & M_x\\
    B \dt{q} - pr\left(C-A \right) & M_y\\
    C \dt{r} - pq\left(A-B \right) & M_z
  \end{array}
\right.
\qquad \qquad \mbox{met }
 \vect{\vect{I}} =
  \left[
    \begin{array}{ccc}
      A & 0 & 0\\
      0 & B & 0\\
      0 & 0 & C
    \end{array}
  \right]
 \qquad
 \vec{\omega} =
  \left[
    \begin{array}{c}
      p \\
      q \\
      r
    \end{array}
  \right]
\]


\paragraph{Kinetische energie}
\[
  E_K = \frac{1}{2}\sum_{(h)} m_h v_h^2
      = \frac{mv_O^2}{2} + m\vec{v}_O \cdot \left( \vec{\omega} \times \vect{OG} \right) + \frac{\vec{\omega} \overline{\overline{I}}_O \vec{\omega}}{2}
\]
\[
  \diff{E_K}{t} = P^u + P^i \stackrel{\mbox{ \tiny SL}}{=} P^u
\]

\subsubsection{Gyrostaat}
\[
  \diff{\vect{\mathscr{M}}_{G0}}{t} = \vec{M}_G^u + \Gamma \vec{\Omega} \times \vec{\omega}
\]


\subsubsection{Virtuële arbeid}

\subsubsection{Lagrangiaanse aanpak}
\[
  Q_i = \sum_k \vec{F}_k^u \cdot \pdiff{\vec{r}_k}{q_i}
\]
\[
  \diff{}{t} \left( \pdiff{E_K}{\dt{q}_i} \right) - \pdiff{E_K}{q_i} = Q_i + \lambda_j \pdiff{\Psi_j}{q_i}
\]
\[
  \Psi_j = 0 \qquad \mbox{Bindingen tussen de coördinaten}
\]
\paragraph{Als alle krachten afgeleid zijn van een potentiaal}
\[
  \mathscr{L} = E_K - E_p
\]
\[
  \diff{}{t} \left( \pdiff{\mathscr{L}}{\dt{q}_i} \right) - \pdiff{\mathscr{L}}{q_i} = 0
\]




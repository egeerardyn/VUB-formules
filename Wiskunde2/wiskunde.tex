% $Id: wiskunde.tex,v 2.26 2005/05/24 14:39:25 johanw Exp johanw $ 
\mag=1000
\documentstyle[a4wide,twoside]{report}
\parindent 0pt

\voffset 0mm
\hoffset 0mm
\oddsidemargin 0mm
\evensidemargin 0mm
\textwidth 16.1925cm

%\setlength{\headrulewidth}{0.5pt}
%\setlength{\footrulewidth}{0.5pt}
\unitlength 1mm

\makeatletter

% changes to report.sty

\def\@makechapterhead#1{{\parindent 0pt\raggedright
\ifnum\c@secnumdepth >\m@ne\huge\bf\@chapapp{} \thechapter\par
\vskip 18pt\fi\Huge\bf #1\par
\nobreak\vskip 40pt}}

\def\@makeschapterhead#1{{\parindent 0pt\raggedright
\Huge\bf #1\par
\nobreak\vskip 40pt}}

\makeatother

\begin{document}
\typeout{Wiskundig formularium door J.C.A. Wevers <johanw@vulcan.xs4all.nl>}

\newfont{\sfd}{cmssdc10 scaled\magstep0}
\newfont{\cmu}{cmu10 scaled\magstep0}

\def\npar{\par\medskip}
%\def\vec#1{\mbox{\boldmath$#1$\unboldmath}} % Uncomment this when you don't like the arrows
\def\vvec#1{\vec{#1}\,}
\def\e#1{\vec{e}_{\rm #1}}
\def\ee#1{\vec{e}_{#1}}
\def\Q#1#2{\frac{\partial #1}{\partial #2}}
\def\QQ#1#2{\frac{\partial^2 #1}{\partial #2^2}}
\def\Qc#1#2#3{\left(\frac{\partial #1}{\partial #2}\right)_{#3}}
\def\RR{I\hspace{-1mm}R}
\def\NN{I\hspace{-1mm}N}
\def\II{I\hspace{-1mm}I}
\def\KK{I\hspace{-1mm}K}
\def\MM{I\hspace{-1mm}M}
\def\ZZ{Z\hspace{-1ex}Z}
\def\CC{C\hspace*{-1.2ex}\rule{0.4pt}{1.5ex}\hspace*{1.2ex}}
\def\dddot#1{\stackrel{...}{#1}}
\def\half{\mbox{$\frac{1}{2}$}}
\def\kwart{\mbox{$\frac{1}{4}$}}
\def\av#1{\left\langle #1 \right\rangle}
\def\oiint{\int\hspace{-2ex}\int\hspace{-3ex}\bigcirc~}
\def\iint{\int\hspace{-1.5ex}\int}
\def\iiint{\int\hspace{-1.5ex}\int\hspace{-1.5ex}\int}

\pagestyle{plain}

\thispagestyle{empty}
\hrule
\rule{.4pt}{22.35cm}\hspace*{161.62mm}\rule{.4pt}{22.35cm}
\vspace*{-17cm}
\begin{center}
\Huge
\fbox{\fbox{Wiskundig Formularium}}
\end{center}
\vspace{2cm}
\centerline{\Large\bf Door ir. J.C.A. Wevers}
\vfill
\hrule
\newpage
\thispagestyle{empty}
\copyright~1998, 2005~~J.C.A. Wevers\hfill Versie: 24 mei 2005
\npar
\hrule
\par
\bigskip
Geachte lezer,
\npar
Dit document is een 66 pagina's tellend wiskundig formuleboek gericht
op fysici en ingenieurs. Het is bedoeld als naslagwerk voor iedereen die
veel formules op moet zoeken.
\npar
Dit document is tevens te verkrijgen bij de auteur, Johan Wevers ({\tt johanw@vulcan.xs4all.nl}).
\npar
Het is tevens te vinden op het WWW op {\tt http://www.xs4all.nl/\~{}johanw/index.html}.
\npar
Het copyright van dit document berust bij J.C.A. Wevers. Alle rechten
voorbehouden. U heeft toestemming om dit ongemodificeerde document te
vermenigvuldigen op iedere daarvoor ge\"eigende manier en voor ieder mogelijk
doel {\it tenzij er sprake is van een winstoogmerk}, tenzij in overleg met de
auteur een andere regeling getroffen wordt.
\npar
De C code voor de nulpuntsberekening volgens Newton en de FFT in hoofdstuk
\ref{chap:num} komen uit ``{\it Numerical Recipes in C}~'', 2nd Edition,
ISBN 0-521-43108-5.
\npar
Het Wiskundig Formularium werd gemaakt met behulp van te\TeX\ en \LaTeX\ versie 2.09.
\npar
Indien U de layout waarin vectoren vet weergegeven worden prefereert kunt U
het commentaar voor het alternatieve {\tt $\backslash$vec} commando in deze
file weghalen en hercompileren.
\npar
Ik houd me altijd aanbevolen voor opmerkingen en suggesties over het
Wiskundig Formularium.
\npar
Johan Wevers
\vfill
\hrule

\newpage

\pagenumbering{Roman}
\addcontentsline{toc}{chapter}{Inhoudsopgave}
\tableofcontents
\cleardoublepage

\pagenumbering{arabic}

\renewcommand{\chaptermark}[1]{\markboth{#1}{#1}}

\chapter{Basiswiskunde}
\typeout{Basiswiskunde}
\section{Goniometrische functies}
Voor de goniometrische verhoudingen van een punt $p$ op de eenheidscirkel geldt:
\[
\cos(\phi)=x_p~~,~~\sin(\phi)=y_p~~,~~\tan(\phi)=\frac{y_p}{x_p}
\]
$\sin^2(x)+\cos^2(x)=1$ en $\cos^{-2}(x)=1+\tan^2(x)$.
\[
\cos(a\pm b)=\cos(a)\cos(b)\mp\sin(a)\sin(b)~~,~~
\sin(a\pm b)=\sin(a)\cos(b)\pm\cos(a)\sin(b)
\]
\[
\tan(a\pm b)=\frac{\tan(a)\pm\tan(b)}{1\mp\tan(a)\tan(b)}
\]
De {\bf somformules} zijn:
\begin{eqnarray*}
\sin(p)+\sin(q)&=&2\sin(\half(p+q))\cos(\half(p-q))\\
\sin(p)-\sin(q)&=&2\cos(\half(p+q))\sin(\half(p-q))\\
\cos(p)+\cos(q)&=&2\cos(\half(p+q))\cos(\half(p-q))\\
\cos(p)-\cos(q)&=&-2\sin(\half(p+q))\sin(\half(p-q))\\
\end{eqnarray*}
Hieruit volgt:
\begin{eqnarray*}
2\cos^2(x)=1+\cos(2x)   &~~,~~&2\sin^2(x)=1-\cos(2x)\\
\sin(\pi-x)=\sin(x)     &~~,~~&\cos(\pi-x)=-\cos(x)\\
\sin(\half\pi-x)=\cos(x)&~~,~~&\cos(\half\pi-x)=\sin(x)
\end{eqnarray*}
{\bf Conclusies uit Gelijkheden}:
\begin{eqnarray*}
\underline{\sin(x)=\sin(a)}&~~\Rightarrow~~&x=a\pm2k\pi\mbox{ of }x=(\pi-a)\pm2k\pi,~~k\in\NN\\
\underline{\cos(x)=\cos(a)}&~~\Rightarrow~~&x=a\pm2k\pi\mbox{ of }x=-a\pm2k\pi\\
\underline{\tan(x)=\tan(a)}&~~\Rightarrow~~&x=a\pm k\pi\mbox{ \'en }x\neq\frac{\pi}{2}\pm k\pi
\end{eqnarray*}
Tussen de inverse functies gelden de volgende relaties:
\[
\arctan(x)=\arcsin\left(\frac{x}{\sqrt{x^2+1}}\right)=\arccos\left(\frac{1}{\sqrt{x^2+1}}\right)~~,~~
\sin(\arccos(x))=\sqrt{1-x^2}
\]

\section{Hyperbolische functies}
De hyperbolische functies zijn gedefinieerd door:
\[
\sinh(x)=\frac{{\rm e}^x-{\rm e}^{-x}}{2}~~,~~~\cosh(x)=\frac{{\rm e}^x+{\rm e}^{-x}}{2}~~,~~~\tanh(x)=\frac{\sinh(x)}{\cosh(x)}
\]
Hieruit volgt dat $\cosh^2(x)-\sinh^2(x)=1$. Verder geldt:
\[
{\rm arsinh}(x)=\ln|x+\sqrt{x^2+1}|~~~,~~~{\rm arcosh}(x)={\rm arsinh}(\sqrt{x^2-1})
\]

\section{Infinitesimaalrekening}
De definitie van de afgeleide functie is:
\[
\frac{df}{dx}=\lim_{h\rightarrow0}\frac{f(x+h)-f(x)}{h}
\]
Rekenregels voor afgeleiden zijn:
\[
d(x\pm y)=dx\pm dy~~,~~d(xy)=xdy+ydx~~,~~d\left(\frac{x}{y}\right)=\frac{ydx-xdy}{y^2}
\]
Voor de afgeleide van de {\it inverse functie} $f^{\rm inv}(y)$, gedefinieerd
door $f^{\rm inv}(f(x))=x$, geldt in punt $P=(x,f(x))$:
\[
\left(\frac{df^{\rm inv}(y)}{dy}\right)_P\cdot\left(\frac{df(x)}{dx}\right)_P=1
\]
Kettingregel: als $f=f(g(x))$, dan geldt
\[
\frac{df}{dx}=\frac{df}{dg}\frac{dg}{dx}
\]
Verder geldt voor afgeleiden van producten van functies:
\[
(f\cdot g)^{(n)}=\sum\limits_{k=0}^n{n\choose k}f^{(n-k)}\cdot g^{(k)}
\]
Voor de {\it primitieve functie} $F(x)$ geldt: $F'(x)=f(x)$.
Een overzicht van afgeleiden en primitieven is:
\begin{center}
\begin{tabular}{||c|c|c||}
\hline
\boldmath$y=f(x)$\unboldmath&\boldmath$dy/dx=f'(x)$\unboldmath&\boldmath$\int f(x)dx$\rule{0pt}{12pt}\rule[-7pt]{0pt}{0pt}\unboldmath\\
\hline
\hline
$ax^n$&$anx^{n-1}$&$a(n+1)^{-1}x^{n+1}$\rule{0pt}{12pt}\\
$1/x$&$-x^{-2}$&$\ln|x|$\\
$a$&$0$&$ax$\\
\hline
$a^x$&$a^x\ln(a)$&$a^x/\ln(a)$\rule{0pt}{12pt}\\
${\rm e}^x$&${\rm e}^x$&${\rm e}^x$\\
$^a\log(x)$&$(x\ln(a))^{-1}$&$(x\ln(x)-x)/\ln(a)$\\
$\ln(x)$&$1/x$&$x\ln(x)-x$\\
\hline
$\sin(x)$&$\cos(x)$&$-\cos(x)$\\
$\cos(x)$&$-\sin(x)$&$\sin(x)$\\
$\tan(x)$&$\cos^{-2}(x)$&$-\ln|\cos(x)|$\\
$\sin^{-1}(x)$&$-\sin^{-2}(x)\cos(x)$&$\ln|\tan(\half x)|$\\
$\sinh(x)$&$\cosh(x)$&$\cosh(x)$\\
$\cosh(x)$&$\sinh(x)$&$\sinh(x)$\\
$\arcsin(x)$&$1/\sqrt{1-x^2}$&$x\arcsin(x)+\sqrt{1-x^2}$\\
$\arccos(x)$&$-1/\sqrt{1-x^2}$&$x\arccos(x)-\sqrt{1-x^2}$\\
$\arctan(x)$&$(1+x^2)^{-1}$&$x\arctan(x)-\half\ln(1+x^2)$\rule[-7pt]{0pt}{0pt}\\
\hline
$(a+x^2)^{-1/2}$&$-x(a+x^2)^{-3/2}$&$\ln|x+\sqrt{a+x^2}|$\rule{0pt}{12pt}\\
$(a^2-x^2)^{-1}$&$2x(a^2+x^2)^{-2}$&$\displaystyle\frac{1}{2a}\ln|(a+x)/(a-x)|$\rule[-10pt]{0pt}{23pt}\\
\hline
\end{tabular}
\end{center}
De {\it kromtestraal} $\rho$ van een kromme is gegeven door:
$\displaystyle\rho=\frac{(1+(y')^2)^{3/2}}{|y''|}$
\npar
Stelling van De 'l H\^opital: als $f(a)=0$ en $g(a)=0$, dan is
$\displaystyle\lim_{x\rightarrow a}\frac{f(x)}{g(x)}=\lim_{x\rightarrow a}\frac{f'(x)}{g'(x)}$

\section{Limieten}
\[
\lim_{x\rightarrow0}\frac{\sin(x)}{x}=1~~,~~
\lim_{x\rightarrow0}\frac{{\rm e}^x-1}{x}=1~~,~~
\lim_{x\rightarrow0}\frac{\tan(x)}{x}=1~~,~~
\lim_{k\rightarrow0}(1+k)^{1/k}={\rm e}~~,~~
\lim_{x\rightarrow\infty}\left(1+\frac{n}{x}\right)^x={\rm e}^n
\]
\[
\lim_{x\downarrow0}x^a\ln(x)=0~~,~~
\lim_{x\rightarrow\infty}\frac{\ln^p(x)}{x^a}=0~~,~~
\lim_{x\rightarrow0}\frac{\ln(x+a)}{x}=a~~,~~
\lim_{x\rightarrow\infty}\frac{x^p}{a^x}=0~~\mbox{als }|a|>1.
\]
\[
\lim_{x\rightarrow0}\left(a^{1/x}-1\right)=\ln(a)~~,~~
\lim_{x\rightarrow0}\frac{\arcsin(x)}{x}=1~~,~~
\lim_{x\rightarrow\infty}\sqrt[x]{x}=1
\]

\section{Complexe getallen en quaternionen}
\subsection{Complexe getallen}
Het complexe getal $z=a+bi$ met $a$ en $b\in\RR$. $a$ is het {\it re\"eele
deel}, $b$ het {\it imaginaire deel} van $z$. $|z|=\sqrt{a^2+b^2}$. Per
definitie geldt: $i^2=-1$. Elk complex getal is te schrijven als
$z=|z|\exp(i\varphi)$, met $\tan(\varphi)=b/a$. De {\it complex geconjugeerde}
van $z$ is gedefinieerd als $\overline{z}=z^*:=a-bi$. Er geldt verder:
\begin{eqnarray*}
(a+bi)(c+di)&=&(ac-bd)+i(ad+bc)\\
(a+bi)+(c+di)&=&a+c+i(b+d)\\
\frac{a+bi}{c+di}&=&\frac{(ac+bd)+i(bc-ad)}{c^2+d^2}
\end{eqnarray*}
Men kan de goniometrische functies schrijven als complexe e-machten:
\begin{eqnarray*}
\sin(x)&=&\frac{1}{2i}({\rm e}^{ix}-{\rm e}^{-ix})\\
\cos(x)&=&\frac{1}{2}({\rm e}^{ix}+{\rm e}^{-ix})
\end{eqnarray*}
Hieruit volgt $\cos(ix)=\cosh(x)$ en $\sin(ix)=i\sinh(x)$. Verder volgt
hieruit dat ${\rm e}^{\pm ix}=\cos(x)\pm i\sin(x)$, dus
${\rm e}^{iz}\neq0\forall z$. Ook volgt hieruit de stelling van De Moivre:
$(\cos(\varphi)+i\sin(\varphi))^n=\cos(n\varphi)+i\sin(n\varphi)$.
\npar
Producten en quoti\"enten van complexe getallen zijn te schrijven als:
\begin{eqnarray*}
z_1\cdot z_2&=&|z_1|\cdot|z_2|(\cos(\varphi_1+\varphi_2)+i\sin(\varphi_1+\varphi_2))\\
\frac{z_1}{z_2}&=&\frac{|z_1|}{|z_2|}(\cos(\varphi_1-\varphi_2)+i\sin(\varphi_1-\varphi_2))
\end{eqnarray*}
Er is af te leiden dat geldt:
\[
|z_1+z_2|\leq|z_1|+|z_2|~~,~~|z_1-z_2|\geq|~|z_1|-|z_2|~|
\]
En uit $z=r\exp(i\theta)$ volgt: $\ln(z)=\ln(r)+i\theta$, $\ln(z)=\ln(z)\pm2n\pi i$.

\subsection{Quaternionen}
Quaternionen zijn gedefinieerd als: $z=a+bi+cj+dk$, met $a,b,c,d\in\RR$.
Er geldt: $i^2=j^2=k^2=-1$. De producten van $i,j,k$ onderling zijn gegeven
door $ij=-ji=k$, $jk=-kj=i$ en $ki=-ik=j$.

\section{Meetkunde}
\subsection{Driehoeken}
Wanneer in een driehoek $\alpha$ de hoek is tegenover zijde $a$, $\beta$ de
hoek tegenover zijde $b$ en $\gamma$ de hoek is tegenover zijde $c$ luidt de
sinusregel:
\[
\frac{a}{\sin(\alpha)}=\frac{b}{\sin(\beta)}=\frac{c}{\sin(\gamma)}
\]
en de cosinusregel: $a^2=b^2+c^2-2bc\cos(\alpha)$. In elke driehoek geldt dat
$\alpha+\beta+\gamma=180^\circ$.
\npar
Verder geldt:
\[
\frac{\tan(\half(\alpha+\beta))}{\tan(\half(\alpha-\beta))}=\frac{a+b}{a-b}
\]
De oppervlakte van een driehoek is
$\half ab\sin(\gamma)=\half ah_a=\sqrt{s(s-a)(s-b)(s-c)}$ met $h_a$ de
hoogtelijn op zijde $a$ en $s=\half(a+b+c)$.

\subsection{Krommen}
{\bf Cyclo\"{\i}de}: indien een cirkel met straal $a$ rolt over een
lijn geldt voor de baan van een punt op de cirkel de parametervergelijking:
$x=a(t+\sin(t))$, $y=a(1+\cos(t))$.
\npar
{\bf Epicyclo\"{\i}de}: indien een kleine cirkel met straal $a$ rolt over een
grote cirkel met straal $R$ geldt voor de baan van een punt op de kleine
cirkel de parametervergelijking:
\[
x=a\sin\left(\frac{R+a}{a}t\right)+(R+a)\sin(t)~~,~~
y=a\cos\left(\frac{R+a}{a}t\right)+(R+a)\cos(t)
\]
{\bf Hypocyclo\"{\i}de}: indien een kleine cirkel met straal $a$ rolt
binnenin een grote cirkel met straal $R$ geldt voor de baan van een punt op
de kleine cirkel de parametervergelijking:
\[
x=a\sin\left(\frac{R-a}{a}t\right)+(R-a)\sin(t)~~,~~
y=-a\cos\left(\frac{R-a}{a}t\right)+(R-a)\cos(t)
\]
Een hypocyclo\"{\i}de met $a=R$ heet een {\bf cardio\"{\i}de}. In
poolco\"ordinaten is deze gegeven door de vergelijking:
$r(\varphi)=2a[1-\cos(\varphi)]$.

\section{Vectoren}
Het {\it inproduct} is gedefinieerd door:
$\displaystyle\vec{a}\cdot\vec{b}=\sum_i a_ib_i=|\vvec{a}|\cdot|\vvec{b}|\cos(\varphi)$
\npar
waarbij $\varphi$ de hoek tussen $\vec{a}$ en $\vec{b}$ is. Het
{\it uitproduct} is in $\RR^3$ gedefinieerd door:
\[
\vec{a}\times\vec{b}=\left(\begin{array}{c}
a_yb_z-a_zb_y\\
a_zb_x-a_xb_z\\
a_xb_y-a_yb_x\end{array}\right)=
\left|\begin{array}{ccc}
\vec{e}_x&\vec{e}_y&\vec{e}_z\\
a_x&a_y&a_z\\
b_x&b_y&b_z\end{array}\right|
\]
Verder geldt: $|\vec{a}\times\vvec{b}|=|\vvec{a}|\cdot|\vvec{b}|\sin(\varphi)$, en
$\vec{a}\times(\vec{b}\times\vvec{c})=(\vec{a}\cdot\vvec{c})\vec{b}-(\vec{a}\cdot\vvec{b})\vec{c}$.

\section{Reeksen}
\subsection{Ontwikkelingen}
Het Binomium van Newton is: $\displaystyle (a+b)^n=\sum_{k=0}^n{n\choose k}a^{n-k}b^k$,
waarin $\displaystyle{n\choose k}:=\frac{n!}{k!(n-k)!}$.
\npar
Door de reeksen $\sum\limits_{k=0}^n r^k$ en $r\sum\limits_{k=0}^n r^k$ van
elkaar af te trekken vindt men
\[
\sum_{k=0}^n r^k=\frac{1-r^{n+1}}{1-r}
\]
dit geeft voor $|r|<1$ de {\it geometrische reeks}:
$\displaystyle\sum_{k=0}^\infty r^k=\frac{1}{1-r}$.
\npar
De {\it rekenkundige reeks} is gegeven door:
$\displaystyle\sum_{n=0}^N(a+nV)=a(N+1)+\half N(N+1)V$.
\npar
De ontwikkeling van een functie rondom het punt $a$ is gegeven door de
{\it Taylorreeks}:
\[
f(x)=f(a)+(x-a)f'(a)+\frac{(x-a)^2}{2}f''(a)+\cdots+\frac{(x-a)^n}{n!}f^{(n)}(a)+R
\]
waarin de restterm gegeven is door:
\[
R_n(h)=(1-\theta)^n\frac{h^n}{n!}f^{(n+1)}(\theta h)
\]
en voldoet aan:
\[
\frac{mh^{n+1}}{(n+1)!}\leq R_n(h)\leq\frac{Mh^{n+1}}{(n+1)!}
\]
Hieruit kan men afleiden dat:
\[
(1-x)^\alpha=\sum_{n=0}^\infty{\alpha\choose n}x^n
\]
Er is af te leiden dat:
\[
\sum_{n=1}^\infty\frac{1}{n^2}=\frac{\pi^2}{6}~,~~
\sum_{n=1}^\infty\frac{1}{n^4}=\frac{\pi^4}{90}~,~~
\sum_{n=1}^\infty\frac{1}{n^6}=\frac{\pi^6}{945}
\]
\[
\sum_{k=1}^nk^2=\mbox{$\frac{1}{6}$}n(n+1)(2n+1)~,~~
\sum_{n=1}^\infty\frac{(-1)^{n+1}}{n^2}=\frac{\pi^2}{12}~,~~
\sum_{n=1}^\infty\frac{(-1)^{n+1}}{n}=\ln(2)
\]
\[
\sum_{n=1}^\infty\frac{1}{4n^2-1}=\mbox{$\frac{1}{2}$}~,~~
\sum_{n=1}^\infty\frac{1}{(2n-1)^2}=\frac{\pi^2}{8}~,~~
\sum_{n=1}^\infty\frac{1}{(2n-1)^4}=\frac{\pi^4}{96}~,~~
\sum_{n=1}^\infty\frac{(-1)^{n+1}}{(2n-1)^3}=\frac{\pi^3}{32}
\]

\subsection{Convergentie en divergentie van reeksen}
Als $\sum\limits_n|u_n|$ convergeert, convergeert $\sum\limits_n u_n$ ook.
\npar
Als $\lim\limits_{n\rightarrow\infty}u_n\neq0$ divergeert $\sum\limits_n u_n$.
\npar
Een alternerende reeks waarvan de termen in absolute waarde monotoon tot 0
naderen is convergent (Leibniz).
\npar
Als $\int_p^{\infty}f(x)dx<\infty$ dan convergeert $\sum\limits_n f_n$.
\npar
Als $u_n>0~\forall n$ dan convergeert $\sum\limits_n u_n$ \'als
$\sum\limits_n\ln(u_n+1)$ convergeert.
\npar
Als $u_n=c_nx^n$ geldt voor de convergentiestraal $\rho$ van
$\sum\limits_n u_n$:
$\displaystyle\frac{1}{\rho}=\lim_{n\rightarrow\infty}\sqrt[n]{|c_n|}=
\lim_{n\rightarrow\infty}\left|\frac{c_{n+1}}{c_n}\right|$.
\npar
De reeks $\displaystyle\sum_{n=1}^\infty \frac{1}{n^p}$ is convergent als
$p>1$ en divergent als $p\leq1$.
\npar
Als geldt:
$\displaystyle\lim_{n\rightarrow\infty}\frac{u_n}{v_n}=p$,
dan geldt: als $p>0$ zijn $\sum\limits_{n}u_n$ en $\sum\limits_{n}v_n$ beide
divergent of beide convergent, als $p=0$ geldt: als $\sum\limits_{n}v_n$
convergeert, d\'an convergeert $\sum\limits_{n}u_n$ ook.
\npar
Wanneer $L$ gedefinieerd wordt door:
$\displaystyle L=\lim_{n\rightarrow\infty}\sqrt[n]{|n_n|}$ \'of door:
$\displaystyle L=\lim_{n\rightarrow\infty}\left|\frac{u_{n+1}}{u_n}\right|$,
dan is $\sum\limits_{n}u_n$ divergent als $L>1$ en convergent als $L<1$.

\subsection{Convergentie en divergentie van functies}
\label{sec:convf}
$f(x)$ is continu in $x=a$ dan en slechts dan als de boven- en onderlimiet
gelijk zijn: $\lim\limits_{x\uparrow a}f(x)=\lim\limits_{x\downarrow a}f(x)$.
Dit wordt wel genoteerd als: $f(a^-)=f(a^+)$.
\npar
Indien $f(x)$ continu is in $a$ {\it en}:
$\lim\limits_{x\uparrow a}f'(x)=\lim\limits_{x\downarrow a}f'(x)$,
{\it dan} is $f(x)$ differentieerbaar in $x=a$.
\npar
We defini\"eren: $\|f\|_W:={\rm sup}(|f(x)|~|x\in W)$ en $\lim\limits_{x\rightarrow\infty}f_n(x)=f(x)$.
Dan geldt: $\{f_n\}$ is uniform convergent als $\lim\limits_{n\rightarrow\infty}\|f_n-f\|=0$,
of: $\forall(\varepsilon>0)\exists(N)\forall(n\geq N)\|f_n-f\|<\varepsilon$.
\npar
De toets van Weierstrass: als $\sum\|u_n\|_W$ convergent is, dan is $\sum u_n$
uniform convergent.
\npar
We defini\"eren $\displaystyle S(x)=\sum_{n=N}^\infty u_n(x)$ en
$\displaystyle F(y)=\int\limits_a^bf(x,y)dx:=F$. Dan is te bewijzen dat:
\begin{center}
\begin{tabular}{||l|l|p{5cm}|p{6cm}||}
\hline
\bf Stelling&\bf Voor&\bf Eisen op $W$&\bf Dan geldt op $W$\\
\hline
\hline
 &rijen    &$f_n$ continu,                              &$f$ is continu\\
 &         &$\{f_n\}$ uniform convergent                &\\
\cline{2-4}
C&reeksen  &$S(x)$ uniform convergent,                  &$S$ is continu\\
 &         &$u_n$ continu                               &\\
\cline{2-4}
 &integraal&$f$ is continu                              &$F$ is continu\\
\hline
 &rijen    &$f_n$ integreerbaar,                        &$f_n$ is integreerbaar,\\
 &         &$\{f_n\}$ uniform convergent                &$\int f(x)dx=\lim\limits_{n\rightarrow\infty}\int f_ndx$\rule[-10pt]{0pt}{0pt}\\
\cline{2-4}
I&reeksen  &$S(x)$ is uniform convergent,               &$S$ integreerbaar, $\int Sdx=\sum\int u_ndx$\rule{0pt}{13pt}\\
 &         &$u_n$ is integreerbaar                      &\\
\cline{2-4}
 &integraal&$f$ is continu                              &$\int Fdy=\int\hspace*{-1.5mm}\int f(x,y)dxdy$\rule{0pt}{13pt}\rule[-8pt]{0pt}{0pt}\\
\hline
 &rijen    &$\{f_n\}\in$C$^{-1}$; $\{f_n'\}$ unif. conv.$\rightarrow\phi$&$f'=\phi(x)$\rule{0pt}{13pt}\rule[-7pt]{0pt}{0pt}\\
\cline{2-4}
D&reeksen  &$u_n\in$C$^{-1}$; $\sum u_n$ conv; $\sum u_n'$ u.c.&$S'(x)=\sum u_n'(x)$\rule{0pt}{13pt}\rule[-7pt]{0pt}{0pt}\\
\cline{2-4}
 &integraal&$\partial f/\partial y$ continu             &$F_y=\int f_y(x,y)dx$\rule{0pt}{13pt}\rule[-7pt]{0pt}{0pt}\\
\hline
\end{tabular}
\end{center}

\section{Producten en quoti\"enten}
Voor $a,b,c,d\in\RR$ gelden:\\
De {\bf distributieve eigenschap}: $(a+b)(c+d)=ac+ad+bc+bd$\\
De {\bf associatieve eigenschap}: $a(bc)=b(ac)=c(ab)$ en $a(b+c)=ab+ac$\\
De {\bf commutatieve eigenschap}: $a+b=b+a$, $ab=ba$.
\npar
Verder geldt:
\[
\frac{a^{2n}-b^{2n}}{a\pm b}=a^{2n-1}\pm a^{2n-2}b+a^{2n-3}b^2\pm\cdots\pm b^{2n-1}~~~,~~~
\frac{a^{2n+1}-b^{2n+1}}{a+b}=\sum_{k=0}^n a^{2n-k}b^{2k}
\]
\[
(a\pm b)(a^2\pm ab+b^2)=a^3\pm b^3~,~~(a+b)(a-b)=a^2+b^2~,~~
\frac{a^3\pm b^3}{a+b}=a^2\mp ba+b^2
\]

\section{Logaritmen}
{\bf Definitie}: $^a\log(x)=b\Leftrightarrow a^b=x$. Voor logaritmen met
grondtal $e$ schrijft men $\ln(x)$.
\npar
{\bf Rekenregels}: $\log(x^n)=n\log(x)$, $\log(a)+\log(b)=\log(ab)$,
$\log(a)-\log(b)=\log(a/b)$.

\section{Polynomen}
Vergelijkingen van de vorm
\[
\sum_{k=0}^n a_kx^k=0
\]
hebben $n$ nulpunten die eventueel samenvallen. Ieder polynoom $p(z)$ met
graad $n\geq1$ heeft tenminste \'e\'en nulpunt in $\CC$.
Als alle $a_k\in\RR$, dan geldt: is $x=p$ met $p\in\CC$ een nulpunt, dan is
$p^*$ ook een nulpunt. Polynomen tot en met graad 4 zijn algemeen analytisch
oplosbaar, voor polynomen met graad $\geq5$ bestaat geen algemeen geldige
analytische oplossing.
\npar
Voor $a,b,c\in\RR$ en $a\neq0$ geldt:
de 2e graads vergelijking $ax^2+bx+c=0$ heeft als algemene oplossing:
\[
x=\frac{-b\pm\sqrt{b^2-4ac}}{2a}
\]
Voor $a,b,c,d\in\RR$ en $a\neq0$ geldt:
de 3e graads vergelijking $ax^3+bx^2+cx+d=0$ heeft als algemene analytische
oplossing:
\begin{eqnarray*}
x_1&=&~K-\frac{3ac-b^2}{9a^2K}-\frac{b}{3a}\\
x_2=x_3^*&=&-\frac{K}{2}+\frac{3ac-b^2}{18a^2K}-\frac{b}{3a}+i\frac{\sqrt{3}}{2}\left(K+\frac{3ac-b^2}{9a^2K}\right)\\
\end{eqnarray*}
met $\displaystyle K=\left(\frac{9abc-27da^2-2b^3}{54a^3}+
\frac{\sqrt{3}\,\sqrt{4ac^3-c^2b^2-18abcd+27a^2d^2+4db^3}}{18a^2}\right)^{1/3}$

\section{Priemgetallen}
Een {\it priemgetal} is een getal $\in\NN$ dat slechts deelbaar is door
zichzelf en door 1. Er zijn $\infty$ veel priemgetallen. Bewijs: stel dat
de verzameling priemgetallen $P$ eindig is, vorm dan het getal
$q=1+\prod\limits_{p\in P}p$, dan geldt $q=1(p)$ en $q$ kan dus niet als
product van priemgetallen uit $P$ geschreven worden. Dit is een tegenspraak.
\npar
Indien $\pi(x)$ het aantal priemgetallen $\leq x$ is, dan geldt:
\[
\lim_{x\rightarrow\infty}\frac{\pi(x)}{x/\ln(x)}=1~~~\mbox{en}~~~
\lim_{x\rightarrow\infty}\frac{\pi(x)}{\int\limits_2^x\frac{dt}{\ln(t)}}=1
\]
Voor iedere $N\geq2$ is er een priemgetal tussen $N$ en $2N$.
\npar
De getallen $F_k:=2^k+1$ met $k\in\NN$ heten de {\it getallen van Fermat}.
Hiertussen zitten vele priemgetallen.
\npar
De getallen $M_k:=2^k-1$ heten de {\it getallen van Mersenne}. Men stuit hierop
bij het opsporen van {\it perfecte getallen}, d.w.z. getallen $n\in\NN$ die
tevens de som van hun verschillende delers zijn, bv.\ $6=1+2+3$. Voor $k<12000$
zijn er 23 priemgetallen van Mersenne: voor $k\in\{2,3,5,7,13,17,19,31,61,89,
107,127,521,607,$ $1279,2203,2281,3217,4253,4423,9689,9941,11213\}$.
\npar
Om te bepalen of een gegeven getal $n$ een priemgetal is kan men met
zeefmethoden werken, waarvan die van Eratosthenes de eerst bekende is. Een
snellere manier voor grote getallen zijn de 4 testen van Fermat, die welliswaar
niet bewijzen dat een getal priem is maar wel een hoge waarschijnlijkheid geven:
\begin{enumerate}
\item Neem de eerste 4 priemgetallen: $b=\{2,3,5,7\}$,
\item Neem $w(b)=b^{n-1}~{\rm mod}~n$, voor iedere $b$,
\item Als $w=1$ voor elke $b$ is $n$ waarschijnlijk priem. Voor elke andere
      waarde van $w$ is $n$ zeker geen priemgetal.
\end{enumerate}


\chapter{Kansrekening en statistiek}
\typeout{Kansrekening en statistiek}
\section{Combinatoriek}
Het aantal mogelijke {\it combinaties} van $k$ elementen uit $n$ elementen is
gegeven door
\[
{n\choose k}=\frac{n!}{k!(n-k)!}
\]
Het aantal {\it permutaties} van $p$ uit $n$ is gegeven door
\[
\frac{n!}{(n-p)!}=p!{n\choose p}
\]
Het aantal verschillende manieren om $n_i$ elementen in $i$ groepen onder te
brengen bij een totaal van $N$ elementen is
\[
\frac{N!}{\prod\limits_i n_i!}
\]

\section{Kansrekening}
De kans $P(A)$ dat een gebeurtenis $A$ plaatsvindt is gedefinieerd door:
\[
P(A)=\frac{n(A)}{n(U)}
\]
waarin $n(A)$ het aantal keren is dat $A$ plaatsvindt en $n(U)$ het totale
aantal gebeurtenissen is.
\npar
De kans $P(\neg A)$ dat $A$ {\it niet} plaatsvindt is: $P(\neg A)=1-P(A)$.
De kans $P(A\cup B)$ dat $A$ en $B$ {\it beide} plaatsvinden is gegeven door:
$P(A\cup B)=P(A)+P(B)-P(A\cap B)$. Als $A$ en $B$ onafhankelijk zijn, geldt:
$P(A\cap B)=P(A)\cdot P(B)$.
\npar
De kans $P(A|B)$ dat $A$ plaatsvindt, gegeven dat $B$ plaatsvindt, is:
\[
P(A|B)=\frac{P(A\cap B)}{P(B)}
\]

\section{Statistiek}
\subsection{Algemeen}
Het {\it gemiddelde} $\av{x}$ van een verzameling waarden is: $\av{x}=\sum_i x_i/n$.
De {\it standaarddeviatie} $\sigma_x$ in de verdeling van $x$ is gegeven door:
\[
\sigma_x=\sqrt{\frac{\sum\limits_{i=1}^n(x_i-\av{x})^2}{n}}
\]
Wanneer met steekproeven gewerkt wordt is de variantie $s$ van de toevalsgrootte
gegeven door $\displaystyle s^2=\frac{n}{n-1}\sigma^2$.
\npar
De {\it covariantie} $\sigma_{xy}$ van $x$ en $y$ is gegeven door:
\[
\sigma_{xy}=\frac{\sum\limits_{i=1}^n(x_i-\av{x})(y_i-\av{y})}{n-1}
\]
De {\it correlatieco\"effici\"ent} $r_{xy}$ van $x$ en $y$ is dan:
$r_{xy}=\sigma_{xy}/\sigma_x\sigma_y$.
\npar
De standaarddeviatie in een variabele $f(x,y)$ ten gevolge van afwijkingen in
de bepaaldheid in $x$ en $y$ is:
\[
\sigma^2_{f(x,y)}=\left(\Q{f}{x}\sigma_x\right)^2+\left(\Q{f}{y}\sigma_y\right)^2+
\Q{f}{x}\Q{f}{y}\sigma_{xy}
\]

\subsection{Verdelingen}
\begin{enumerate}
\item {\bf De Binomiaalverdeling} is de verdeling die een trekking met
      teruglegging beschrijft. De kans op succes is $p$. De kans $P$ op
      $k$ successen uit $n$ trekkingen is dan gegeven door:
      \[
      P(x=k)={n\choose k}p^k(1-p)^{n-k}
      \]
      De standaarddeviatie is gegeven door $\sigma_x=\sqrt{np(1-p)}$ en de
      verwachtingswaarde is $\varepsilon=np$.
\item {\bf De Hypergeometrische verdeling} is de verdeling die een trekking
      zonder terugleggen beschrijft waarin de volgorde er niet toe doet. De
      kans op $k$ successen in een trekking uit $A$ mogelijk positieve en $B$
      mogelijk negatieve resultaten is dan gegeven door:
      \[
      P(x=k)=\frac{\displaystyle{A\choose k}{B\choose n-k}}{\displaystyle{A+B\choose n}}
      \]
      De verwachtingswaarde is gegeven door $\varepsilon=nA/(A+B)$.
\item {\bf De Poisson verdeling} ontstaat uit de binomiaalverdeling wanneer
      $p\rightarrow0$, $n\rightarrow\infty$, en tegelijkertijd het product
      $np=\lambda$ constant is.
      \[
      P(x)=\frac{\lambda^x {\rm e}^{-\lambda}}{x!}
      \]
      Deze verdeling is genormeerd op $\displaystyle\sum\limits_{x=0}^\infty P(x)=1$.
\item {\bf De Normaalverdeling} is een benadering van de binomiaalverdeling
      voor continue variabelen:
      \[
      P(x)=\frac{1}{\sigma\sqrt{2\pi}}\exp\left(-\frac{1}{2}\left(\frac{x-\av{x}}{\sigma}\right)^2\right)
      \]
\item {\bf De Uniforme verdeling} treedt op als aselect een getal $x$ genomen
      wordt uit de verzameling $a\leq x\leq b$ en is gegeven door:
      \[
      \left\{\begin{array}{l}\displaystyle
      P(x)=\frac{1}{b-a}~~~\mbox{als}~~~a\leq x\leq b\\
      \\
      P(x)=0~~~\mbox{in alle andere gevallen}
      \end{array}\right.
      \]
      $\av{x}=\half(b-a)$ en $\displaystyle\sigma^2=\frac{(b-a)^2}{12}$.
\item {\bf De Gamma verdeling} is gegeven door:
      \[
      \left\{\begin{array}{l}\displaystyle
      P(x)=\frac{x^{\alpha-1}{\rm e}^{-x/\beta}}{\beta^\alpha\Gamma(\alpha)}~~~\mbox{als}~~~0\leq y\leq\infty
      \end{array}\right.
      \]
      met $\alpha>0$ en $\beta>0$. De verdeling heeft
      de volgende eigenschappen: $\av{x}=\alpha\beta$, $\sigma^2=\alpha\beta^2$.
\item {\bf De Beta verdeling} is gegeven door:
      \[
      \left\{\begin{array}{l}\displaystyle
      P(x)=\frac{x^{\alpha-1}(1-x)^{\beta-1}}{\beta(\alpha,\beta)}~~~\mbox{als}~~~0\leq x\leq1\\
      \\
      P(x)=0~~~\mbox{elders}
      \end{array}\right.
      \]
      en heeft de volgende eigenschappen: $\displaystyle\av{x}=\frac{\alpha}{\alpha+\beta}$,
      $\displaystyle\sigma^2=\frac{\alpha\beta}{(\alpha+\beta)^2(\alpha+\beta+1)}$.
      \npar
      Voor $P(\chi^2)$ geldt: $\alpha=V/2$ en $\beta=2$.
\item {\bf De Weibull verdeling} is gegeven door:
      \[
      \left\{\begin{array}{l}\displaystyle
      P(x)=\frac{\alpha}{\beta}x^{\alpha-1}{\rm e}^{-x^\alpha}~~~\mbox{als}~~~0\leq x\leq\infty\wedge\alpha\wedge\beta>0\\
      \\
      P(x)=0~~~\mbox{in alle andere gevallen}
      \end{array}\right.
      \]
      Het gemiddelde is $\av{x}=\beta^{1/\alpha}\Gamma((\alpha+1)\alpha)$
\item Bij een {\bf tweeledige verdeling} geldt:
      \[
      P_1(x_1)=\int P(x_1,x_2)dx_2~~,~~~P_2(x_2)=\int P(x_1,x_2)dx_1
      \]
      met
      \[
      \varepsilon(g(x_1,x_2))=\iint g(x_1,x_2)P(x_1,x_2)dx_1dx_2=\sum_{x_1}\sum_{x_2}g\cdot P
      \]
\end{enumerate}

\section{Regressie analyse}
Wanneer er een verband tussen de grootheden $x$ en $y$ bestaat van de vorm
$y=ax+b$, en men een set $x_i$ en bijbehorende $y_i$ gemeten heeft, geldt het
volgende verband voor $a$ en $b$ ($\vec{x}=(x_1,x_2,...,x_n)$ en
$\vec{e}=(1,1,...,1)$):
\[
\vec{y}-a\vec{x}-b\vec{e}\in<\vec{x},\vec{e}>^\perp
\]
Hieruit volgt dat de inproducten 0 zijn:
\[
\left\{
\begin{array}{l}
(\vec{y},\vvec{x})-a(\vec{x},\vvec{x})-b(\vec{e},\vvec{x})=0\\
(\vec{y},\vvec{e})-a(\vec{x},\vvec{e})-b(\vec{e},\vvec{e})=0
\end{array}\right.
\]
met $(\vec{x},\vvec{x})=\sum\limits_i x_i^2$, $(\vec{x},\vvec{y})=\sum\limits_ix_iy_i$,
$(\vec{x},\vvec{e})=\sum\limits_ix_i$ en $(\vec{e},\vvec{e})=n$.
Hieruit volgen $a$ en $b$.
\npar
Dit geldt ook voor hogere orde polynoom fits: voor een 2e orde fit geldt:
\[
\vec{y}-a\vec{x^2}-b\vec{x}-c\vec{e}\in<\vec{x^2},\vec{x},\vec{e}>^\perp
\]
met $\vec{x^2}=(x_1^2,...,x_n^2)$.
\npar
De {\it correlatie co\"effici\"ent} $r$ is een maat voor hoe goed een fit is.
Bij lineaire regressie is ze gegeven door:
\[
r=\frac{n\sum xy-\sum x\sum y}{\sqrt{(n\sum x^2-(\sum x)^2)(n\sum y^2-(\sum y)^2)}}
\]


\chapter{Analyse}
\typeout{Analyse}
\section{Integraalrekening}
\subsection{Rekenregels}
De primitieve functie $F(x)$ van $f(x)$ voldoet aan: $F'(x)=f(x)$. Met
$F(x)$ de primitieve van $f(x)$ geldt voor de bepaalde integraal
\[
\int\limits_a^bf(x)dx=F(b)-F(a)
\]
Als $u=f(x)$ geldt:
\[
\int\limits_a^bg(f(x))df(x)=\int\limits_{f(a)}^{f(b)}g(u)du
\]
{\bf Parti\"eel integreren}: met $F$ en $G$ de primitieven van $f$ resp.\ $g$
geldt:
\[
\int f(x)\cdot g(x)dx=f(x)G(x)-\int G(x)\frac{df(x)}{dx}dx
\]
Afgeleide onder het integraalteken brengen (zie {\bf sectie \ref{sec:convf}}
voor de eisen wanneer dit kan):
\[
\frac{d}{dy}\left[\int\limits_{x=g(y)}^{x=h(y)}f(x,y)dx\right]=
\int\limits_{x=g(y)}^{x=h(y)}\Q{f(x,y)}{y}dx-f(g(y),y)\frac{dg(y)}{dy}+f(h(y),y)\frac{dh(y)}{dy}
\]

\subsection{Lengten, oppervlakten en volumen}
De lengte $\ell$ van een kromme $y(x)$ is gegeven door:
\[
\ell=\int\sqrt{1+\left(\frac{dy(x)}{dx}\right)^2}dx
\]
De lengte $\ell$ van een parameterkromme $F(\vec{x}(t))$ is:
\[
\ell=\int Fds=\int F(\vec{x}(t))|\dot{\vec{x}}(t)|dt
\]
met
\[
\vec{t}=\frac{d\vec{x}}{ds}=\frac{\dot{\vec{x}}(t)}{|\dot{\vec{x}}(t)|}~~~,~~|\vec{t}~|=1
\]
\[
\int(\vec{v},\vec{t})ds=\int(\vec{v},\dot{\vec{t}}(t))dt=\int(v_1dx+v_2dy+v_3dz)
\]
De oppervlakte $A$ van een omwentelingslichaam is:
\[
A=2\pi\int y\sqrt{1+\left(\frac{dy(x)}{dx}\right)^2}dx
\]
Het volume $V$ van een omwentelingslichaam is:
\[
V=\pi\int f^2(x)dx
\]

\subsection{Breuksplitsen}
\label{sec:breuksplits}
Iedere rationele functie $P(x)/Q(x)$ met $P$ en $Q$ polynomen is te schrijven
als een lineaire combinatie van functies van de gedaante $(x-a)^k$ met
$k\in\ZZ$, en van functies van de gedaante
\[
\frac{px+q}{((x-a)^2+b^2)^n}
\]
met $b>0$ en $n\in\NN$. Dus:
\[
\frac{p(x)}{(x-a)^n}=\sum_{k=1}^n\frac{A_k}{(x-a)^k}~~,~~~
\frac{p(x)}{((x-b)^2+c^2)^n}=\sum_{k=1}^n \frac{A_kx+B}{((x-b)^2+c^2)^k}
\]
Recurrente betrekking: voor $n\neq0$ geldt:
\[
\int\frac{dx}{(x^2+1)^{n+1}}=\frac{1}{2n}\frac{x}{(x^2+1)^n}+\frac{2n-1}{2n}\int\frac{dx}{(x^2+1)^n}
\]

\subsection{Speciale functies}
\subsubsection{Elliptische functies}
Elliptische functies zijn als volgt als reeks te schrijven:
\[
\sqrt{1-k^2\sin^2(x)}=1-\sum_{n=1}^\infty\frac{(2n-1)!!}{(2n)!!(2n-1)}k^{2n}\sin^{2n}(x)
\]
\[
\frac{1}{\sqrt{1-k^2\sin^2(x)}}=1+\sum_{n=1}^\infty\frac{(2n-1)!!}{(2n)!!}k^{2n}\sin^{2n}(x)
\]
met $n!!=n(n-2)!!$.

\subsubsection{De Gamma functie}
De gammafunctie $\Gamma(y)$ is gedefinieerd door:
\[
\Gamma(y)=\int\limits_0^\infty{\rm e}^{-x}x^{y-1}dx
\]
Er geldt: $\Gamma(y+1)=y\Gamma(y)=y!$. Op deze manier zijn faculteiten voor
niet-gehele getallen te defini\"eren. Verder is af te leiden dat
\[
\Gamma(n+\half)=\frac{\sqrt{\pi}}{2^n}(2n-1)!!~~\mbox{ en }~~
\Gamma^{(n)}(y)=\int\limits_0^\infty{\rm e}^{-x}x^{y-1}\ln^n(x)dx
\]

\subsubsection{De Beta functie}
De betafunctie $\beta(p,q)$ is gedefinieerd door:
\[
\beta(p,q)=\int\limits_0^1x^{p-1}(1-x)^{q-1}dx
\]
met $p$ en $q$ $>0$. Tussen de beta- en gammafunctie geldt het volgende
verband:
\[
\beta(p,q)=\frac{\Gamma(p)\Gamma(q)}{\Gamma(p+q)}
\]

\subsubsection{De Delta functie}
De delta functie $\delta(x)$ is een oneindig dunne piekfunctie met oppervlakte
1. Ze is te defini\"eren als:
\[
\delta(x)=\lim_{\varepsilon\rightarrow0}P(\varepsilon,x)~~\mbox{met}~~
P(\varepsilon,x)=\left\{
\begin{array}{l}
0~~~\mbox{voor}~|x|>\varepsilon\\
\displaystyle\frac{1}{2\varepsilon}~~~\mbox{voor}~|x|<\varepsilon
\end{array}\right.
\]
Enkele eigenschappen zijn:
\[
\int\limits_{-\infty}^\infty\delta(x)dx=1~~,~~~
\int\limits_{-\infty}^\infty F(x)\delta(x)dx=F(0)
\]

\subsection{Goniometrische integralen}
Voor het oplossen van goniometrische integralen is het vaak nuttig te wisselen
van variabele. Als men stelt dat $\tan(\frac{1}{2}x):=t$ geldt:
\[
dx=\frac{2dt}{1+t^2}~,~~\cos(x)=\frac{1-t^2}{1+t^2}~,~~\sin(x)=\frac{2t}{1+t^2}
\]
Iedere integraal van het type $\int R(x,\sqrt{ax^2+bx+c})dx$ is te herleiden
tot de types die in {\bf sectie \ref{sec:breuksplits}} behandeld zijn. Na
herleiding substitueert men in:
\begin{eqnarray*}
\int R(x,\sqrt{x^2+1})dx&~:~~&x=\tan(\varphi)          ~,dx=\frac{d\varphi}{\cos(\varphi)}               ~~\mbox{of}~~\sqrt{x^2+1}=t+x\\
\int R(x,\sqrt{1-x^2})dx&~:~~&x=\sin(\varphi)          ~,dx=\cos(\varphi)d\varphi                        ~~\mbox{of}~~\sqrt{1-x^2}=1-tx\\
\int R(x,\sqrt{x^2-1})dx&~:~~&x=\frac{1}{\cos(\varphi)}~,dx=\frac{\sin(\varphi)}{\cos^2(\varphi)}d\varphi~~\mbox{of}~~\sqrt{x^2-1}=x-t
\end{eqnarray*}
De volgende bepaalde integralen zijn eenvoudig op te lossen:
\[
\int\limits_0^{\pi/2}\cos^n(x)\sin^m(x)dx=\frac{(n-1)!!(m-1)!!}{(m+n)!!}\cdot
\left\{\begin{array}{l}
\pi/2\mbox{~~als $m$ en $n$ beide even zijn}\\
1\mbox{~~~~~~in alle andere gevallen}
\end{array}\right.
\]
Enkele belangrijke integralen zijn:
\[
\int\limits_0^\infty\frac{xdx}{{\rm e}^{ax}+1}=\frac{\pi^2}{12a^2}~~,~~
\int\limits_{-\infty}^\infty\frac{x^2dx}{({\rm e}^x+1)^2}=\frac{\pi^2}{3}~~,~~
\int\limits_0^\infty\frac{x^3dx}{{\rm e}^x+1}=\frac{\pi^4}{15}
\]

\section{Functies met meer variabelen}
\subsection{Afgeleiden}
De {\it parti\"ele afgeleide} naar $x$ van een functie $f(x,y)$ is gedefinieerd
door:
\[
\Qc{f}{x}{x_0}=\lim_{h\rightarrow0}\frac{f(x_0+h,y_0)-f(x_0,y_0)}{h}
\]
De {\it richtingsafgeleide} in richting $\alpha$ is gedefinieerd door:
\[
\Q{f}{\alpha}=\lim_{r\downarrow0}\frac{f(x_0+r\cos(\alpha),y_0+r\sin(\alpha))-f(x_0,y_0)}{r}=
(\vec{\nabla}f,(\sin\alpha,\cos\alpha))=\frac{\nabla f\cdot\vec{v}}{|\vec{v}|}
\]
Wanneer men overgaat op andere co\"ordinaten $f(x(u,v),y(u,v))$ geldt:
\[
\Q{f}{u}=\Q{f}{x}\Q{x}{u}+\Q{f}{y}\Q{y}{u}
\]
Als $x(t)$ en $y(t)$ van maar \'e\'en parameter $t$ afhangen geldt:
\[
\Q{f}{t}=\Q{f}{x}\frac{dx}{dt}+\Q{f}{y}\frac{dy}{dt}
\]
De {\it totale differentiaal} $df$ van een functie met 3 variabelen is
gegeven door:
\[
df=\Q{f}{x}dx+\Q{f}{y}dy+\Q{f}{z}dz
\]
Dus
\[
\frac{df}{dx}=\Q{f}{x}+\Q{f}{y}\frac{dy}{dx}+\Q{f}{z}\frac{dz}{dx}
\]
De {\it raaklijn} in het punt $\vec{x}_0$ aan het oppervlak $f(x,y)=0$ heeft
als vergelijking:\\ $f_x(\vec{x}_0)(x-x_0)+f_y(\vec{x}_0)(y-y_0)=0$.
\npar
Het {\it raakvlak} in $\vec{x}_0$ is gegeven door:
$f_x(\vec{x}_0)(x-x_0)+f_y(\vec{x}_0)(y-y_0)=z-f(\vec{x}_0)$.

\subsection{Taylorreeksen}
Men kan een functie van 2 variabelen als volgt in een Taylorreeks ontwikkelen:
\[
f(x_0+h,y_0+k)=\sum\limits_{p=0}^n \frac{1}{p!}
\left(h\Q{^p}{x^p}+k\Q{^p}{y^p}\right)f(x_0,y_0)+R(n)
\]
met $R(n)$ de restfout en
\[
\left(h\Q{^p}{x^p}+k\Q{^p}{y^p}\right)f(a,b)=\sum\limits_{m=0}^p{p\choose m}
h^mk^{p-m}\frac{\partial^pf(a,b)}{\partial x^m\partial y^{p-m}}
\]

\subsection{Extrema}
Indien $f$ continu is op een compacte rand $V$ heeft $f$ een globaal
maximum en een globaal minimum op deze rand. Een rand is compact als hij
begrensd en gesloten is.
\npar
Kandidaten voor extrema van $f(x,y)$ op een rand $V\in\RR^2$ zijn:
\begin{enumerate}
\item Punten op $V$ waar $f(x,y)$ niet differentieerbaar is,
\item Punten waar $\vec{\nabla}f=\vec{0}$,
\item Indien de rand $V$ gegeven is door $\varphi(x,y)=0$ zijn alle punten
      waar $\vec{\nabla}f(x,y)+\lambda\vec{\nabla}\varphi(x,y)=0$ kandidaten
      voor extrema. Dit is de multiplicatoren methode van Lagrange, $\lambda$
      is een multiplicator.
\end{enumerate}
In $\RR^3$ geldt dat, indien het te onderzoeken gebied begrensd is door
een compacte $V$, met $V$ gedefinieerd door $\varphi_1(x,y,z)=0$ en
$\varphi_2(x,y,z)=0$, voor extrema van $f(x,y,z)$ hetzelfde als op $\RR^2$
m.b.t.\ punt (1) en (2). Punt (3) wordt als volgt herschreven: potenti\"ele
extra zijn punten waar
$\vec{\nabla}f(x,y,z)+\lambda_1\vec{\nabla}\varphi_1(x,y,z)+\lambda_2\vec{\nabla}\varphi_2(x,y,z)=0$.

\subsection{De $\nabla$-operator}
In cartesische co\"ordinaten $(x,y,z)$ geldt:
\begin{eqnarray*}
\vec{\nabla}     &=&\Q{}{x}\ee{x}+\Q{}{y}\ee{y}+\Q{}{z}\ee{z}\\
{\rm grad}f      &=&\Q{f}{x}\ee{x}+\Q{f}{y}\ee{y}+\Q{f}{z}\ee{z}\\
{\rm div}~\vec{a}&=&\Q{a_x}{x}+\Q{a_y}{y}+\Q{a_z}{z}\\
{\rm rot}~\vec{a}&=&\left(\Q{a_z}{y}-\Q{a_y}{z}\right)\ee{x}+
                    \left(\Q{a_x}{z}-\Q{a_z}{x}\right)\ee{y}+
                    \left(\Q{a_y}{x}-\Q{a_x}{y}\right)\ee{z}\\
\nabla^2 f       &=&\QQ{f}{x}+\QQ{f}{y}+\QQ{f}{z}
\end{eqnarray*}

In cylinderco\"ordinaten $(r,\varphi,z)$ geldt:
\begin{eqnarray*}
\vec{\nabla}     &=&\Q{}{r}\ee{r}+\frac{1}{r}\Q{}{\varphi}\ee{\varphi}+\Q{}{z}\ee{z}\\
{\rm grad}f      &=&\Q{f}{r}\ee{r}+\frac{1}{r}\Q{f}{\varphi}\ee{\varphi}+\Q{f}{z}\ee{z}\\
{\rm div}~\vec{a}&=&\Q{a_r}{r}+\frac{a_r}{r}+\frac{1}{r}\Q{a_\varphi}{\varphi}+\Q{a_z}{z}\\
{\rm rot}~\vec{a}&=&\left(\frac{1}{r}\Q{a_z}{\varphi}-\Q{a_\varphi}{z}\right)\ee{r}+
                    \left(\Q{a_r}{z}-\Q{a_z}{r}\right)\ee{\varphi}+
                    \left(\Q{a_\varphi}{r}+\frac{a_\varphi}{r}-\frac{1}{r}\Q{a_r}{\varphi}\right)\ee{z}\\
\nabla^2 f       &=&\QQ{f}{r}+\frac{1}{r}\Q{f}{r}+\frac{1}{r^2}\QQ{f}{\varphi}+\QQ{f}{z}
\end{eqnarray*}

In bolco\"ordinaten $(r,\theta,\varphi)$ geldt:
\begin{eqnarray*}
\vec{\nabla}     &=&\Q{}{r}\ee{r}+\frac{1}{r}\Q{}{\theta}\ee{\theta}+\frac{1}{r\sin\theta}\Q{}{\varphi}\ee{\varphi}\\
{\rm grad}f      &=&\Q{f}{r}\ee{r}+\frac{1}{r}\Q{f}{\theta}\ee{\theta}+\frac{1}{r\sin\theta}\Q{f}{\varphi}\ee{\varphi}\\
{\rm div}~\vec{a}&=&\Q{a_r}{r}+\frac{2a_r}{r}+\frac{1}{r}\Q{a_\theta}{\theta}+\frac{a_\theta}{r\tan\theta}+\frac{1}{r\sin\theta}\Q{a_\varphi}{\varphi}\\
{\rm rot}~\vec{a}&=&\left(\frac{1}{r}\Q{a_\varphi}{\theta}+\frac{a_\theta}{r\tan\theta}-\frac{1}{r\sin\theta}\Q{a_\theta}{\varphi}\right)\ee{r}+
                    \left(\frac{1}{r\sin\theta}\Q{a_r}{\varphi}-\Q{a_\varphi}{r}-\frac{a_\varphi}{r}\right)\ee{\theta}+\\
                 &&\left(\Q{a_\theta}{r}+\frac{a_\theta}{r}-\frac{1}{r}\Q{a_r}{\theta}\right)\ee{\varphi}\\
\nabla^2 f       &=&\QQ{f}{r}+\frac{2}{r}\Q{f}{r}+\frac{1}{r^2}\QQ{f}{\theta}+\frac{1}{r^2\tan\theta}\Q{f}{\theta}+\frac{1}{r^2\sin^2\theta}\QQ{f}{\varphi}
\end{eqnarray*}

Algemene kromlijnige orthonormale co\"ordinaten $(u,v,w)$ kunnen uit
cartesische co\"ordinaten verkregen worden door de transformatie
$\vec{x}=\vec{x}(u,v,w)$. Dan worden de eenheidsvectoren gegeven door:
\[
\ee{u}=\frac{1}{h_1}\Q{\vec{x}}{u}~,~~ \ee{v}=\frac{1}{h_2}\Q{\vec{x}}{v}~,~~
\ee{w}=\frac{1}{h_3}\Q{\vec{x}}{w}
\]
waarin de factoren $h_i$ zorgen voor de normering op lengte 1. De
differentiaaloperatoren worden dan gegeven door:
\begin{eqnarray*}
{\rm grad}f      &=&\frac{1}{h_1}\Q{f}{u}\ee{u}+\frac{1}{h_2}\Q{f}{v}\ee{v}+\frac{1}{h_3}\Q{f}{w}\ee{w}\\
{\rm div}~\vec{a}&=&\frac{1}{h_1h_2h_3}\left(\Q{}{u}(h_2h_3a_u)+\Q{}{v}(h_3h_1a_v)+\Q{}{w}(h_1h_2a_w)\right)\\
{\rm rot}~\vec{a}&=&\frac{1}{h_2h_3}\left(\Q{(h_3a_w)}{v}-\Q{(h_2a_v)}{w}\right)\ee{u}+
                    \frac{1}{h_3h_1}\left(\Q{(h_1a_u)}{w}-\Q{(h_3a_w)}{u}\right)\ee{v}+\\
                  &&\frac{1}{h_1h_2}\left(\Q{(h_2a_v)}{u}-\Q{(h_1a_u)}{v}\right)\ee{w}\\
\nabla^2 f       &=&\frac{1}{h_1h_2h_3}\left[\Q{}{u}\left(\frac{h_2h_3}{h_1}\Q{f}{u}\right)+
                    \Q{}{v}\left(\frac{h_3h_1}{h_2}\Q{f}{v}\right)+
                    \Q{}{w}\left(\frac{h_1h_2}{h_3}\Q{f}{w}\right)\right]
\end{eqnarray*}

Enkele eigenschappen van de $\nabla$-operator zijn:
\[
\begin{array}{l@{~~~~~}l@{~~~~~}l}
{\rm div}(\phi\vec{v})=\phi{\rm div}\vec{v}+{\rm grad}\phi\cdot\vec{v}&
{\rm rot}(\phi\vec{v})=\phi{\rm rot}\vec{v}+({\rm grad}\phi)\times\vec{v}&{\rm rot~grad}\phi=\vec{0}\\
{\rm div}(\vec{u}\times\vec{v})=\vec{v}\cdot({\rm rot}\vec{u})-\vec{u}\cdot({\rm rot}\vec{v})&
{\rm rot~rot}\vec{v}={\rm grad~div}\vec{v}-\nabla^2\vec{v}&{\rm div~rot}\vec{v}=0\\
{\rm div~grad}\phi=\nabla^2\phi&\nabla^2\vec{v}\equiv(\nabla^2v_1,\nabla^2v_2,\nabla^2v_3)
\end{array}
\]
Hierin  is $\vec{v}$ een willekeurig vectorveld en $\phi$ een willekeurig
scalarveld.

\subsection{Integraalstellingen}
Enkele belangrijke integraalstellingen zijn:
\[
\begin{array}{l@{~~}l}
\mbox{Gauss:}&\displaystyle\oiint(\vec{v}\cdot\vec{n})d^2A=\iiint({\rm div}\vvec{v})d^3V\\[5mm]
\mbox{Stokes voor een scalarveld:}&\displaystyle\oint(\phi\cdot\e{t})ds=\iint(\vec{n}\times{\rm grad}\phi)d^2A\\[5mm]
\mbox{Stokes voor een vectorveld:}&\displaystyle\oint(\vec{v}\cdot\e{t})ds=\iint({\rm rot}\vec{v}\cdot\vec{n})d^2A\\[5mm]
\mbox{dit geeft:}&\displaystyle\oiint({\rm rot}\vec{v}\cdot\vec{n})d^2A=0\\[5mm]
\mbox{Ostrogradsky:}&\displaystyle\oiint(\vec{n}\times\vvec{v})d^2A=\iiint({\rm rot}\vvec{v})d^3A\\[5mm]
\mbox{}&\displaystyle\oiint(\phi\vvec{n})d^2A=\iiint({\rm grad}\phi)d^3V
\end{array}
\]
Hierin wordt het ori\"enteerbare oppervlak $\int\hspace{-1mm}\int d^2A$
begrensd door de Jordankromme $s(t)$.

\subsection{Meervoudige integralen}
Als $A$ een gesloten kromme is die gegeven wordt door $f(x,y)=0$ wordt
het oppervlak $A$ binnen de kromme in $\RR^2$ gegeven door
\[
A=\iint d^2A=\iint dxdy
\]
Aanname: het oppervlak $A$ wordt gedefinieerd door de functie $z=f(x,y)$.
Het volume $V$ tussen $A$ en het $xy$ vlak wordt dan gegeven door:
\[
V=\iint f(x,y)dxdy
\]
Het volume binnen een gesloten oppervlak dat gedefinieerd is door $z=f(x,y)$
is gegeven door:
\[
V=\iiint d^3V=\iint f(x,y)dxdy=\iiint dxdydz
\]

\subsection{Co\"ordinatentransformaties}
De uitdrukkingen $d^2A$ en $d^3V$ transformeren als volgt wanneer men
overgaat op co\"ordinaten $\vec{u}=(u,v,w)$ via de transformatie $x(u,v,w)$:
\[
V=\iiint f(x,y,z)dxdydz=\iiint f(\vec{x}(\vec{u}))\left|\Q{\vec{x}}{\vec{u}}\right|dudvdw
\]
In $\RR^2$ geldt:
\[
\Q{\vec{x}}{\vec{u}}=\left|\begin{array}{cc}x_u&x_v\\ y_u&y_v\end{array}\right|
\]
Als het oppervlak $A$ gegeven is door $z=F(x,y)=X(u,v)$ is het volume tussen
het $xy$ vlak en $F$ gegeven door:
\[
\iint\limits_Sf(\vec{x})d^2A=\iint\limits_Gf(\vec{x}(\vec{u}))
\left|\Q{X}{u}\times\Q{X}{v}\right|dudv=
\iint\limits_Gf(x,y,F(x,y))\sqrt{1+\partial_xF^2+\partial_yF^2}dxdy
\]

\section{Orthogonaliteit van functies}
Het inproduct van twee functies $f(x)$ en $g(x)$ op het interval $[a,b]$ is
gegeven door:
\[
(f,g)=\int\limits_a^bf(x)g(x)dx
\]
of, als men een gewichtsfunctie $p(x)$ hanteert door:
\[
(f,g)=\int\limits_a^bp(x)f(x)g(x)dx
\]
De {\it norm} $\|f\|$ volgt dan uit: $\|f\|^2=(f,f)$. Een set functies $f_i$
is {\it orthonormaal} als geldt: $(f_i,f_j)=\delta_{ij}$.
\npar
Elke functie $f(x)$ is te schrijven als een som van orthogonale functies:
\[
f(x)=\sum_{i=0}^\infty c_ig_i(x)
\]
en $\sum c_i^2\leq\|f\|^2$. Als de set $g_i$ orthogonaal is volgt:
\[
c_i=\frac{f,g_i}{(g_i,g_i)}
\]

\section{Fourier reeksen}
Elke functie is te schrijven als een som van onafhankelijke basisfuncties.
Indien men voor de orthogonale basis $(\cos(nx),\sin(nx))$ kiest spreekt men
van een Fourier reeks.
\npar
Een periodieke functie $f(x)$ met periode $2L$ is te schrijven als:
\[
f(x)=a_0+\sum_{n=1}^\infty\left[a_n\cos\left(\frac{n\pi x}{L}\right)+b_n\sin\left(\frac{n\pi x}{L}\right)\right]
\]
Voor de co\"effici\"enten volgt dan vanwege de orthogonaliteit:
\[
a_0=\frac{1}{2L}\int\limits_{-L}^Lf(t)dt~~,~~
a_n=\frac{1}{L}\int\limits_{-L}^Lf(t)\cos\left(\frac{n\pi t}{L}\right)dt~~,~~
b_n=\frac{1}{L}\int\limits_{-L}^Lf(t)\sin\left(\frac{n\pi t}{L}\right)dt
\]
Men kan een Fourier reeks ook schrijven als een som van complexe e-machten:
\[
f(x)=\sum_{n=-\infty}^\infty c_n{\rm e}^{inx}
\]
met
\[
c_n=\frac{1}{2\pi}\int\limits_{-\pi}^\pi f(x){\rm e}^{-inx}dx
\]
De {\it Fouriertransformatie} van een functie $f(x)$ levert een
getransformeerde functie $\hat{f}(\omega)$ op:
\[
\hat{f}(\omega)=\frac{1}{\sqrt{2\pi}}\int\limits_{-\infty}^\infty f(x){\rm e}^{-i\omega x}dx
\]
De inverse transformatie is gegeven door:
\[
\frac{1}{2}\left[f(x^+)+f(x^-)\right]=\frac{1}{\sqrt{2\pi}}\int\limits_{-\infty}^\infty\hat{f}(\omega){\rm e}^{i\omega x}d\omega
\]
waarin $f(x^+)$ en $f(x^-)$ gedefinieerd zijn door de onder- en bovenlimiet:
\[
f(a^-)=\lim_{x\uparrow a}f(x)~~,~~f(a^+)=\lim_{x\downarrow a}f(x)
\]
Voor continue functies geldt dat $\frac{1}{2}\left[f(x^+)+f(x^-)\right]=f(x)$.

\chapter{Differentiaalvergelijkingen}
\typeout{Differentiaalvergelijkingen}
\section{Lineaire differentiaalvergelijkingen}
\subsection{Eerste orde lineaire DV}
De algemene oplossing van een lineaire differentiaalvergelijking is
gegeven door $y_{\rm A}=y_{\rm H}+y_{\rm P}$, waarin $y_{\rm H}$ de oplossing
van de {\it homogene vergelijking} is en $y_{\rm P}$ een {\it particuliere
oplossing}.
\npar
Een differentiaalvergelijking van de eerste orde is gegeven door:
$y'(x)+a(x)y(x)=b(x)$. De bijbehorende homogene vergelijking is
$y'(x)+a(x)y(x)=0$.
\npar
De oplossing van de homogene vergelijking is gegeven door
\[
y_{\rm H}=k\exp\left(\int a(x)dx\right)
\]
Stel dat $a(x)=a=$constant.
\npar
Substitutie van $\exp(\lambda x)$ in de homogene vergelijking leidt tot de
{\it karakteristieke vergelijking} $\lambda+a=0\Rightarrow \lambda=-a$.
\npar
Stel dat $b(x)=\alpha\exp(\mu x)$. Dan zijn er twee gevallen:
\begin{enumerate}
\item $\lambda\neq\mu$: een particuliere oplossing is: $y_{\rm P}=\exp(\mu x)$
\item $\lambda=\mu$: een particuliere oplossing is: $y_{\rm P}=x\exp(\mu x)$
\end{enumerate}
\npar
Bij oplossing door {\it variatie van constante} stelt men:
$y_{\rm P}(x)=y_{\rm H}(x)f(x)$, en lost hieruit $f(x)$ op.

\subsection{Tweede orde lineaire DV}
Een differentiaalvergelijking van de tweede orde met constante
co\"effici\"enten is gegeven door: $y''(x)+ay'(x)+by(x)=c(x)$. Indien
$c(x)=c=$constant bestaat er een particuliere oplossing $y_{\rm P}=c/b$.
\npar
Substitutie van $y=\exp(\lambda x)$ leidt tot de karakteristieke vergelijking
$\lambda^2+a\lambda+b=0$.
\npar
Er zijn nu 2 mogelijkheden:
\begin{enumerate}
\item $\lambda_1\neq\lambda_2$: dan is $y_{\rm H}=\alpha\exp(\lambda_1 x)+\beta\exp(\lambda_2 x)$.
\item $\lambda_1=\lambda_2=\lambda$: dan is $y_{\rm H}=(\alpha +\beta x)\exp(\lambda x)$.
\end{enumerate}
\npar
Als $c(x)=p(x)\exp(\mu x)$ met $p(x)$ een polynoom is zijn er 3 mogelijkheden:
\begin{enumerate}
\item $\lambda_1,\lambda_2\neq\mu$: $y_{\rm P}=q(x)\exp(\mu x)$.
\item $\lambda_1=\mu,\lambda_2\neq\mu$: $y_{\rm P}=xq(x)\exp(\mu x)$.
\item $\lambda_1=\lambda_2=\mu$: $y_{\rm P}=x^2q(x)\exp(\mu x)$.
\end{enumerate}
waarin $q(x)$ een polynoom van dezelfde graad is als $p(x)$.
\npar
Als gegeven is: $y''(x)+\omega^2y(x)=\omega f(x)$ en $y(0)=y'(0)=0$ volgt:
$y(x)=\int\limits_0^xf(x)\sin(\omega(x-t))dt$.

\subsection{De Wronskiaan}
We gaan uit van de LDV $y''(x)+p(x)y'(x)+q(x)y(x)=0$ en de twee beginvoorwaarden
$y(x_0)=K_0$ en $y'(x_0)=K_1$. Als $p(x)$ en $q(x)$ continu zijn op het open
interval $I$ is er een unieke oplossing $y(x)$ op dit interval.
\npar
De algemene oplossing is dan te schrijven als $y(x)=c_1y_1(x)+c_2y_2(x)$ en
$y_1$ en $y_2$ zijn lineair onafhankelijk. Dit zijn tevens {\it alle}
oplossingen van de LDV.
\npar
De {\it Wronskiaan} is gedefinieerd als:
\[
W(y_1,y_2)=
\left|\begin{array}{cc}
y_1&y_2\\
y'_1&y'_2
\end{array}\right|=y_1y'_2-y_2y'_1
\]
$y_1$ en $y_2$ zijn lineair afhankelijk dan en slechts dan als op interval
$I$ $\exists x_0\in I$ zodanig dat geldt:\\
$W(y_1(x_0),y_2(x_0))=0$.

\subsection{Machtreekssubstitutie}
Wanneer in de LDV met constante co\"effici\"enten $y''(x)+py'(x)+qy(x)=0$ de
reeks $y=\sum a_nx^n$ ingevuld wordt geeft dit:
\[
\sum_n\left[n(n-1)a_nx^{n-2}+pna_nx^{n-1}+qa_nx^n\right]=0
\]
Gelijkstellen van gelijke machten van $x$ levert:
\[
(n+2)(n+1)a_{n+2}+p(n+1)a_{n+1}+qa_n=0
\]
Dit geeft een algemeen verband tussen de co\"effici\"enten. Speciale gevallen
zijn $n=0,1,2$.

\section{Enkele speciale gevallen}
\subsection{De methode van Frobenius}
Gegeven de LDV
\[
\frac{d^2y(x)}{dx^2}+\frac{b(x)}{x}\frac{dy(x)}{dx}+\frac{c(x)}{x^2}y(x)=0
\]
met $b(x)$ en $c(x)$ analytisch op $x=0$. Deze heeft ten minste \'e\'en
oplossing van de vorm:
\[
y_i(x)=x^{r_i}\sum_{n=0}^\infty a_nx^n~~~\mbox{met}~~i=1,2
\]
met $r$ re\"eel of complex en zodanig gekozen dat $a_0\neq0$. Waneer we $b(x)$
en $c(x)$ dan als volgt expanderen: $b(x)=b_0+b_1x+b_2x^2+...$ en
$c(x)=c_0+c_1x+c_2x^2+...$ volgt voor $r$:
\[
r^2+(b_0-1)r+c_0=0
\]
Er kunnen zich nu 3 gevallen voordoen:
\begin{enumerate}
\item $r_1=r_2$: dan is $y(x)=y_1(x)\ln|x|+y_2(x)$.
\item $r_1-r_2\in\NN$: dan is $y(x)=ky_1(x)\ln|x|+y_2(x)$.
\item $r_1-r_2\neq\ZZ$: dan is $y(x)=y_1(x)+y_2(x)$.
\end{enumerate}

\subsection{Euler}
Gegeven de LDV
\[
x^2\frac{d^2y(x)}{dx^2}+ax\frac{dy(x)}{dx}+by(x)=0
\]
Substitutie van $y(x)=x^r$ geeft een vergelijking voor $r$: $r^2+(a-1)r+b=0$.
Hieruit volgen twee oplossingen $r_1$ en $r_2$. Er zijn nu 2 mogelijkheden:
\begin{enumerate}
\item $r_1\neq r_2$: dan is $y(x)=C_1x^{r1}+C_2x^{r_2}$.
\item $r_1=r_2=r$: dan is $y(x)=(C_1\ln(x)+C_2)x^r$.
\end{enumerate}

\subsection{De DV van Legendre}
Gegeven de LDV 
\[
(1-x^2)\frac{d^2y(x)}{dx^2}-2x\frac{dy(x)}{dx}+n(n-1)y(x)=0
\]
De oplossingen van deze vergelijking worden gegeven door $y(x)=aP_n(x)+by_2(x)$
waarin de {\it Legendre polynomen} $P(x)$ gedefinieerd worden door:
\[
P_n(x)=\frac{d^n}{dx^n}\left(\frac{(1-x^2)^n}{2^n n!}\right)
\]
Hiervoor geldt: $\|P_n\|^2=2/(2n+1)$.

\subsection{De geassocieerde Legendre vergelijking}
Deze vergelijking volgt uit het $\theta$ afhankelijke deel van de
golfvergelijking $\nabla^2\Psi=0$ door te substitueren: $\xi=\cos(\theta)$.
Dan volgt:
\[
(1-\xi^2)\frac{d}{d\xi}\left((1-\xi^2)\frac{dP(\xi)}{d\xi}\right)+
[C(1-\xi^2)-m^2]P(\xi)=0
\]
Er bestaan alleen reguliere oplossingen als $C=l(l+1)$. Deze hebben de vorm:
\[
P_l^{|m|}(\xi)=(1-\xi^2)^{m/2}\frac{d^{|m|}P^0(\xi)}{d\xi^{|m|}}=
\frac{(1-\xi^2)^{|m|/2}}{2^ll!}\frac{d^{|m|+l}}{d\xi^{|m|+l}}(\xi^2-1)^l
\]
Voor $|m|>l$ is $P_l^{|m|}(\xi)=0$.
Enkele eigenschappen van $P_l^0(\xi)$ zijn:
\[
\int\limits_{-1}^1P_l^0(\xi)P_{l'}^0(\xi)d\xi=\frac{2}{2l+1}\delta_{ll'}~~~,~~~
\sum_{l=0}^\infty P_l^0(\xi)t^l=\frac{1}{\sqrt{1-2\xi t+t^2}}
\]
Dit polynoom is te schrijven als:
\[
P_l^0(\xi)=\frac{1}{\pi}\int\limits_0^\pi(\xi+\sqrt{\xi^2-1}\cos(\theta))^ld\theta
\]

\subsection{Oplossingen van de Besselvergelijking}
Gegeven de LDV
\[
x^2\frac{d^2y(x)}{dx^2}+x\frac{dy(x)}{dx}+(x^2-\nu^2)y(x)=0
\]
ook wel de {\it Besselvergelijking} genoemd, en de Besselfuncties van de eerste
soort
\[
J_\nu(x)=x^\nu\sum_{m=0}^\infty\frac{(-1)^mx^{2m}}{2^{2m+\nu}m!\Gamma(\nu+m+1)}
\]
voor $\nu:=n\in\NN$ wordt dit:
\[
J_n(x)=x^n\sum_{m=0}^\infty\frac{(-1)^mx^{2m}}{2^{2m+n}m!(n+m)!}
\]
Als $\nu\neq\ZZ$ is de oplossing gegeven door $y(x)=aJ_\nu(x)+bJ_{-\nu}(x)$.
Maar omdat voor $n\in\ZZ$ geldt: $J_{-n}(x)=(-1)^nJ_n(x)$, geldt dit niet voor
gehele getallen. De algemene oplossing van de Besselvergelijking is
$y(x)=aJ_\nu(x)+bY_\nu(x)$ waarin $Y_\nu$ {\it Besselfuncties van de tweede
soort} zijn:
\[
Y_\nu(x)=\frac{J_\nu(x)\cos(\nu\pi)-J_{-\nu}(x)}{\sin(\nu\pi)}~~~\mbox{en}~~~
Y_n(x)=\lim_{\nu\rightarrow n}Y_\nu(x)
\]
De vergelijking $x^2y''(x)+xy'(x)-(x^2+\nu^2)y(x)=0$ heeft als oplossingen
gemodificeerde Besselfuncties van de eerste soort $I_\nu(x)=i^{-\nu}J_\nu(ix)$,
en oplossingen $K_\nu=\pi[I_{-\nu}(x)-I_\nu(x)]/[2\sin(\nu\pi)]$.
\npar
Soms is het handig de oplossingen van de Besselvergelijking te schrijven in
termen van de Hankelfuncties
\[
H^{(1)}_n(x)=J_n(x)+iY_n(x)~~,~~H^{(2)}_n(x)=J_n(x)-iY_n(x)
\]

\subsection{Eigenschappen van Besselfuncties}
Besselfuncties zijn orthogonaal wanneer $p(x)=x$ als normeringsfunctie gebruikt
wordt.
\npar
$J_{-n}(x)=(-1)^nJ_n(x)$. De Neumann functies $N_m(x)$ zijn gedefinieerd als:
\[
N_m(x)=\frac{1}{2\pi}J_m(x)\ln(x)+\frac{1}{x^m}\sum_{n=0}^\infty \alpha_nx^{2n}
\]
Er geldt: $\lim\limits_{x\rightarrow0}J_m(x)=x^m$,
$\lim\limits_{x\rightarrow0}N_m(x)=x^{-m}$ voor $m\neq0$,
$\lim\limits_{x\rightarrow0}N_0(x)=\ln(x)$.
\[
\lim_{r\rightarrow\infty}H(r)=\frac{{\rm e}^{\pm ikr}{\rm e}^{i\omega t}}{\sqrt{r}}~~,~~
\lim_{x\rightarrow\infty}J_n(x)=\sqrt{\frac{2}{\pi x}}\cos(x-x_n)~~,~~
\lim_{x\rightarrow\infty}J_{-n}(x)=\sqrt{\frac{2}{\pi x}}\sin(x-x_n)
\]
met $x_n=\half\pi(n+\half)$.
\[
J_{n+1}(x)+J_{n-1}(x)=\frac{2n}{x}J_n(x)~~,~~J_{n+1}(x)-J_{n-1}(x)=-2\frac{dJ_n(x)}{dx}
\]
Verder gelden de integraalbetrekkingen
\[
J_m(x)=\frac{1}{2\pi}\int\limits_0^{2\pi}\exp[i(x\sin(\theta)-m\theta)]d\theta=
\frac{1}{\pi}\int\limits_0^\pi\cos(x\sin(\theta)-m\theta)d\theta
\]

\subsection{Laguerre}
Gegeven de LDV
\[
x\frac{d^2y(x)}{dx^2}+(1-x)\frac{dy(x)}{dx}+ny(x)=0
\]
Oplossingen hiervan zijn de Laguerre polynomen $L_n(x)$:
\[
L_n(x)=\frac{{\rm e}^x}{n!}\frac{d^n}{dx^n}\left(x^n{\rm e}^{-x}\right)=
\sum_{m=0}^\infty\frac{(-1)^m}{m!}{n\choose m}x^m
\]

\subsection{De geassocieerde Laguerre vergelijking}
Gegeven de LDV
\[
\frac{d^2y(x)}{dx^2}+\left(\frac{m+1}{x}-1\right)\frac{dy(x)}{dx}+\left(\frac{n+\half(m+1)}{x}\right)y(x)=0
\]
Oplossingen hiervan zijn de geassocieerde Laguerre polynomen $L_n^m(x)$:
\[
L_n^m(x)=\frac{(-1)^mn!}{(n-m)!}{\rm e}^{-x}x^{-m}\frac{d^{n-m}}{dx^{n-m}}\left({\rm e}^{-x}x^n\right)
\]

\subsection{Hermite}
\def\Hn{{\rm H}_n}
\def\Hen{{\rm He}_n}
De differentiaalvergelijkingen van Hermite zijn:
\[
\frac{d^2\Hn(x)}{dx^2}-2x\frac{d\Hn(x)}{dx}+2n\Hn(x)=0~~\mbox{en}~~
\frac{d^2\Hen(x)}{dx^2}-x\frac{d\Hen(x)}{dx}+n\Hen(x)=0
\]
De oplossingen hiervan zijn de Hermite polynomen, die worden gegeven door:
\[
\Hn(x)=(-1)^n\exp\left(\frac{1}{2}x^2\right)\frac{d^n(\exp(-\half x^2))}{dx^n}=2^{n/2}\Hen(x\sqrt{2})
\]
\[
\Hen(x)=(-1)^n(\exp\left(x^2\right)\frac{d^n(\exp(-x^2))}{dx^n}=2^{-n/2}\Hn(x/\sqrt{2})
\]

\subsection{Chebyshev}
De LDV
\[
(1-x^2)\frac{d^2U_n(x)}{dx^2}-3x\frac{dU_n(x)}{dx}+n(n+2)U_n(x)=0
\]
heeft oplossingen
\[
U_n(x)=\frac{\sin[(n+1)\arccos(x)]}{\sqrt{1-x^2}}
\]
De LDV
\[
(1-x^2)\frac{d^2T_n(x)}{dx^2}-x\frac{dT_n(x)}{dx}+n^2T_n(x)=0
\]
heeft oplossingen $T_n(x)=\cos(n\arccos(x))$.

\subsection{Weber}
De LDV $W''_n(x)+(n+\half-\kwart x^2)W_n(x)=0$ heeft oplossingen:
$W_n(x)={\rm He}_n(x)\exp(-\kwart x^2)$.

\section{Niet-lineaire differentiaalvergelijkingen}
Enkele niet-lineaire differentiaalvergelijkingen en een oplossing zijn:
\[
\begin{array}{lll}
y'=a\sqrt{y^2+b^2}&~~~~&y=b\sinh(a(x-x_0))\\
y'=a\sqrt{y^2-b^2}&~~~~&y=b\cosh(a(x-x_0))\\
y'=a\sqrt{b^2-y^2}&~~~~&y=b\cos(a(x-x_0))\\
y'=a(y^2+b^2)     &~~~~&y=b\tan(a(x-x_0))\\
y'=a(y^2-b^2)     &~~~~&y=b\coth(a(x-x_0))\\
y'=a(b^2-y^2)     &~~~~&y=b\tanh(a(x-x_0))\\
\displaystyle y'=ay\left(\frac{b-y}{b}\right)&~~~~&\displaystyle y=\frac{b}{1+Cb\exp(-ax)}
\end{array}
\]

\section{Sturm-Liouville vergelijkingen}
Sturm-Liouville vergelijkingen zijn 2e orde LDV's van de vorm:
\[
-\frac{d}{dx}\left(p(x)\frac{dy(x)}{dx}\right)+q(x)y(x)=\lambda m(x)y(x)
\]
De randvoorwaarden worden zodanig gekozen dat de operator
\[
L=-\frac{d}{dx}\left(p(x)\frac{d}{dx}\right)+q(x)
\]
Hermitisch is. Van de normeringsfunctie $m(x)$ wordt ge\"eist dat
\[
\int\limits_a^bm(x)y_i(x)y_j(x)dx=\delta_{ij}
\]
Als $y_1(x)$ en $y_2(x)$ twee lineair onafhankelijke oplossingen zijn kan men
de Wronskiaan in de volgende vorm schrijven:
\[
W(y_1,y_2)=\left|\begin{array}{cc}y_1&y_2\\ y_1'&y_2' \end{array}\right|=
\frac{C}{p(x)}
\]
met $C$ een constante. Door overgang naar een andere afhankelijke variabele
$u(x)$ die gegeven is door: $u(x)=y(x)\sqrt{p(x)}$ gaat de LDV over in de
{\it normaalvorm}:
\[
\frac{d^2u(x)}{dx^2}+I(x)u(x)=0~~~\mbox{met}~~~
I(x)=\frac{1}{4}\left(\frac{p'(x)}{p(x)}\right)^2-\frac{1}{2}\frac{p''(x)}{p(x)}-\frac{q(x)-\lambda m(x)}{p(x)}
\]
Als $I(x)>0$ is $y''/y<0$ en heeft de oplossing een oscillatoir gedrag, als
$I(x)<0$ is $y''/y>0$ en heeft de oplossing een exponentieel gedrag.

\section{Lineaire parti\"ele differentiaalvergelijkingen}
\subsection{Algemeen}
De {\it normale afgeleide} is gedefinieerd door:
\[
\Q{u}{n}=(\vec{\nabla}u,\vec{n})
\]
Een veel gebruikte oplossingsmethode voor PDV's is {\it scheiding van
variabelen}: men neemt aan dat de oplossing $u(x,t)$ geschreven kan worden
als $u(x,t)=X(x)T(t)$. Uit substitutie volgen dan gewone DV's voor $X(x)$ en
$T(t)$.

\subsection{Bijzondere gevallen}
\subsubsection{De golfvergelijking}
De {\it golfvergelijking} in 1 dimensie is gegeven door
\[
\QQ{u}{t}=c^2\QQ{u}{x}
\]
Als beginvoorwaarde geldt $u(x,0)=\varphi(x)$ en
$\partial u(x,0)/\partial t=\Psi(x)$. Dan is de algemene oplossing gegeven
door:
\[
u(x,t)=\frac{1}{2}\left[\varphi(x+ct)+\varphi(x-ct)\right]+\frac{1}{2c}
\int\limits_{x-ct}^{x+ct}\Psi(\xi)d\xi
\]

\subsubsection{De diffusievergelijking}
De {\it diffusievergelijking} is:
\[
\Q{u}{t}=D\nabla^2u
\]
De oplossingen hiervan zijn uit te drukken met de propagatoren $P(x,x',t)$.
Deze hebben de eigenschap dat $P(x,x',0)=\delta(x-x')$. In 1 dimensie geldt:
\[
P(x,x',t)=\frac{1}{2\sqrt{\pi Dt}}\exp\left(\frac{-(x-x')^2}{4Dt}\right)
\]
In 3 dimensies geldt:
\[
P(x,x',t)=\frac{1}{8(\pi Dt)^{3/2}}\exp\left(\frac{-(\vec{x}-\vvec{x}')^2}{4Dt}\right)
\]
Met beginvoorwaarde $u(x,0)=f(x)$ is de oplossing:
\[
u(x,t)=\int\limits_{\cal G}f(x')P(x,x',t)dx'
\]
De oplossing van de vergelijking
\[
\Q{u}{t}-D\QQ{u}{x}=g(x,t)
\]
is gegeven door
\[
u(x,t)=\int dt' \int dx'g(x',t')P(x,x',t-t')
\]

\subsubsection{De vergelijking van Helmholtz}
De vergelijking van Helmholtz verkrijgt men door substitutie van
$u(\vec{x},t)=v(\vec{x})\exp(i\omega t)$ in de golfvergelijking. Dit geeft
voor $v$:
\[
\nabla^2v(\vec{x},\omega)+k^2v(\vec{x},\omega)=0
\]
Dit geeft als oplossingen voor $v$:
\begin{enumerate}
\item In cartesische co\"ordinaten: substitutie van
$v=A\exp(i\vec{k}\cdot\vvec{x})$ geeft:
\[
v(\vvec{x})=\int\cdots\int A(k){\rm e}^{i\vec{k}\cdot\vec{x}}dk
\]
met de integralen over $\vvec{k}^2=k^2$.
\item In poolco\"ordinaten:
\[
v(r,\varphi)=\sum_{m=0}^\infty(A_mJ_m(kr)+B_mN_m(kr)){\rm e}^{im\varphi}
\]
\item In bolco\"ordinaten:
\[
v(r,\theta,\varphi)=\sum_{l=0}^\infty\sum_{m=-l}^l[A_{lm}J_{l+\frac{1}{2}}(kr)+B_{lm}J_{-l-\frac{1}{2}}(kr)]\frac{Y(\theta,\varphi)}{\sqrt{r}}
\]
\end{enumerate}

\subsection{Potentiaaltheorie en de stelling van Green}
Onderwerp van de potentiaaltheorie vormen de {\it Poisson vergelijking}
$\nabla^2u=-f(\vvec{x})$ met $f$ een gegeven functie, en de {\it Laplace
vergelijking} $\nabla^2u=0$. De oplossingen hiervan zijn vaak te interpreteren
als een potentiaal. De oplossingen van de Laplace vergelijking worden
{\it harmonische functies} genoemd.
\npar
Wanneer een vectorveld $\vec{v}$ gegeven is door $\vec{v}={\rm grad}\varphi$
volgt:
\[
\int\limits_a^b(\vec{v},\vvec{t})ds=\varphi(\vvec{b})-\varphi(\vvec{a})
\]
In dit geval bestaan er functies $\varphi$ en $\vec{w}$ zodanig dat
$\vec{v}={\rm grad}\varphi+{\rm rot}\vec{w}$.
\npar
De {\it veldlijnen} van het veld $\vec{v}(\vec{x})$ volgen uit:
\[
\dot{\vvec{x}}(t)=\lambda\vec{v}(\vvec{x})
\]
De {\it eerste identiteit van Green} is:
\[
\mathop{\iiint}\limits_{\cal\!\!\! G}[u\nabla^2v+(\nabla u,\nabla v)]d^3V=\mathop{\oiint}\limits_{\cal\!\!\!S}u\Q{v}{n}d^2A
\]
De {\it tweede identiteit van Green} is:
\[
\mathop{\iiint}\limits_{\cal\!\!\! G}[u\nabla^2v-v\nabla^2u]d^3V=\mathop{\oiint}\limits_{\cal\!\!\!S}\left(u\Q{v}{n}-v\Q{u}{n}\right)d^2A
\]
Een harmonische functie die 0 is op de rand van een gebied is 0 binnen dat
gebied. Een harmonische functie waarvan de normaalafgeleide 0 is op de rand van
een gebied is constant binnen dat gebied.
\npar
Het {\it Dirichlet probleem} is:
\[
\nabla^2u(\vvec{x})=-f(\vvec{x})~~,~~\vec{x}\in R~~,~~u(\vvec{x})=g(\vvec{x})~~\mbox{voor alle}~~\vec{x}\in S.
\]
De oplossing hiervan is eenduidig.
\npar
Het {\it Neumann probleem} is:
\[
\nabla^2u(\vvec{x})=-f(\vvec{x})~~,~~\vec{x}\in R~~,~~\Q{u(\vvec{x})}{n}=h(\vvec{x})~~\mbox{voor alle}~~\vec{x}\in S.
\]
De oplossing van het Neumann probleem is eenduidig op een constante na. De
oplossing bestaat als:
\[
-\mathop{\iiint}\limits_{\!\!\!R}f(\vvec{x})d^3V=\mathop{\oiint}\limits_{\!\!\!S}h(\vvec{x})d^2A
\]
Een {\it fundamentele oplossing} van de Laplace vergelijking voldoet aan:
\[
\nabla^2u(\vvec{x})=-\delta(\vvec{x})
\]
De oplossing in 2 dimensies hiervan is in poolco\"ordinaten:
\[
u(r)=\frac{\ln(r)}{2\pi}
\]
De oplossing in 3 dimensies is in bolco\"ordinaten:
\[
u(r)=\frac{1}{4\pi r}
\]
De vergelijking $\nabla^2v=-\delta(\vec{x}-\vvec{\xi})$ heeft als oplossing
\[
v(\vvec{x})=\frac{1}{4\pi|\vec{x}-\vvec{\xi}|}
\]
Na substitutie hiervan in de 2e identiteit van Green en toepassing van de
zeefeigenschap van de $\delta$ functie volgt de 3e identiteit van Green:
\[
u(\vvec{\xi})=-\frac{1}{4\pi}\mathop{\iiint}\limits_{\!\!\!R}\frac{\nabla^2u}{r}d^3V+
\frac{1}{4\pi}\mathop{\oiint}\limits_{\!\!\!S}\left[\frac{1}{r}\Q{u}{n}-u\Q{}{n}\left(\frac{1}{r}\right)\right]d^2A
\]
De {\it Greense functie} $G(\vec{x},\vvec{\xi})$ is gedefinieerd door:
$\nabla^2G=-\delta(\vec{x}-\vvec{\xi})$, en op rand $S$ is $G(\vec{x},\vvec{\xi})=0$.
Dan is $G$ te schrijven als:
\[
G(\vec{x},\vvec{\xi})=\frac{1}{4\pi|\vec{x}-\vvec{\xi}|}+g(\vec{x},\vvec{\xi})
\]
Dan is $g(\vec{x},\vvec{\xi})$ een oplossing van het Dirichlet probleem. De
oplossing van de Poisson vergelijking $\nabla^2u=-f(\vvec{x})$ als op rand $S$
geldt: $u(\vvec{x})=g(\vvec{x})$, is dan:
\[
u(\vvec{\xi})=\mathop{\iiint}\limits_{\!\!\!R}G(\vec{x},\vvec{\xi})f(\vvec{x})d^3V-
\mathop{\oiint}\limits_{\!\!\!S}g(\vvec{x})\Q{G(\vec{x},\vvec{\xi})}{n}d^2A
\]


\chapter{Lineaire algebra en analyse}
\typeout{Lineaire algebra en analyse}
\section{Vectorruimten}
$\cal G$ is een groep voor de operatie $\otimes$ als:
\begin{enumerate}
\item $\forall a,b\in{\cal G}\Rightarrow a\otimes b\in\cal G$: een groep is
      {\it gesloten}.
\item $(a\otimes b)\otimes c = a\otimes (b\otimes c)$: een groep is
      {\it associatief}.
\item $\exists e\in{\cal G}$ zodanig dat $a\otimes e=e\otimes a=a$: er bestaat
      een {\it eenheidselement}.
\item $\forall a\in{\cal G}\exists \overline{a}\in{\cal G}$ zodanig dat $a\otimes\overline{a}=e$:
      elk element heeft een {\it inverse}.
\end{enumerate}
Als\\
\hspace*{4.5mm}5. $a\otimes b=b\otimes a$
\npar
is de groep Abels of commutatief.
Vectorruimten vormen een Abelse groep voor de optelling en vermenigvuldiging:
$1\cdot\vec{a}=\vec{a}$, $\lambda(\mu\vec{a})=(\lambda\mu)\vec{a}$,
$(\lambda+\mu)(\vec{a}+\vec{b})=\lambda\vec{a}+\lambda\vec{b}+\mu\vec{a}+\mu\vec{b}$.
\npar
$W$ is een {\it lineaire deelruimte} als $\forall \vec{w}_1,\vec{w}_2\in W$
geldt: $\lambda\vec{w}_1+\mu\vec{w}_2\in W$.
\npar
$W$ is een {\it invariante deelruimte} van $V$ voor de afbeelding $A$ als
$\forall \vec{w}\in W$ geldt: $A\vec{w}\in W$.

\section{Basis}
Voor een orthogonale basis geldt: $(\vec{e}_i,\vec{e}_j)=c\delta_{ij}$. Voor
een orthonormale basis geldt: $(\vec{e}_i,\vec{e}_j)=\delta_{ij}$.
\npar
De set vectoren $\{\vec{a}_n\}$ is onafhankelijk als:
\[
\sum\limits_i\lambda_i\vec{a}_i=0~~\Leftrightarrow~~\forall_i\lambda_i=0
\]
De set $\{\vec{a}_n\}$ is een basis als ze 1. onafhankelijk zijn, en 2.
$V=<\vec{a}_1,\vec{a_2},...>=\sum\lambda_i\vec{a}_i$.

\section{Matrices}
\subsection{Basisbewerkingen}
Voor de matrixvermenigvuldiging van matrices $A=a_{ij}$ en $B=b_{kl}$ geldt:
met $^r$ de rijindex en $^k$ de kolomindex:
\[
A^{r_1k_1}\cdot B^{r_2k_2}=C^{r_1k_2}~~,~~(AB)_{ij}=\sum_ka_{ik}b_{kj}
\]
met $^r$ het aantal rijen en $^k$ het aantal kolommen.
\npar
De {\it getransponeerde} van $A$ is gedefinieerd door: $a_{ij}^T=a_{ji}$.
Hiervoor geldt dat $(AB)^T=B^TA^T$, en $(A^T)^{-1}=(A^{-1})^T$. Voor de
{\it inverse matrix} geldt: $(A\cdot B)^{-1}=B^{-1}\cdot A^{-1}$. De inverse
matrix $A^{-1}$ heeft de eigenschap dat $A\cdot A^{-1}=\II$ en kan gevonden
worden door vegen: $(A_{ij}|\II)\sim(\II|A_{ij}^{-1})$.
\npar
De inverse van een $2\times2$ matrix is:
\[
\left(\begin{array}{cc}a&b\\ c&d\end{array}\right)^{-1}=\frac{1}{ad-bc}
\left(\begin{array}{cc}d&-b\\ -c&a\end{array}\right)
\]
\npar
De {\it determinant functie} $D=\det(A)$ is gedefinieerd door:
\[
\det(A)=D(\vec{a}_{*1},\vec{a}_{*2},...,\vec{a}_{*n})
\]
Voor de determinant $\det(A)$ van een matrix $A$ geldt:
$\det(AB)=\det(A)\cdot\det(B)$. Een $2\times2$ matrix heeft als determinant:
\[
\det\left(\begin{array}{cc}a&b\\ c&d \end{array}\right)=ad-cb
\]
De afgeleide van een matrix is een matrix met de afgeleiden van de co\"effici\"enten:
\[
\frac{dA}{dt}=\frac{da_{ij}}{dt}~~~\mbox{en}~~~\frac{dAB}{dt}=B\frac{dA}{dt}+A\frac{dB}{dt}
\]
De afgeleide van de determinant is gegeven door:
\[
\frac{d\det(A)}{dt}=D(\frac{d\vec{a}_1}{dt},...,\vec{a}_n)+
D(\vec{a}_1,\frac{d\vec{a}_2}{dt},...,\vec{a}_n)+...+D(\vec{a}_1,...,\frac{d\vec{a}_n}{dt})
\]
Wanneer men de rijen van een matrix als vectoren beschouwd is de {\it rijenrang}
van deze matrix het aantal onafhankelijke vectoren in deze set. Idem voor de
{\it kolommenrang}. Voor elke matrix is de rijenrang gelijk aan de kolommenrang.
\npar
Laat $\tilde{A}:\tilde{V}\rightarrow\tilde{V}$ de complexe uitbreiding zijn van
de re\"ele lineaire afbeelding $A:V\rightarrow V$ in een eindig dimensionale
$V$. Dan hebben $A$ en $\tilde{A}$ dezelfde karakteristieke vergelijking.
\npar
Als $A_{ij}\in\RR$ en als $\vec{v}_1+i\vec{v_2}$ een eigenvector van $A$ is
bij eigenwaarde $\lambda=\lambda_1+i\lambda_2$, dan geldt:
\begin{enumerate}
\item $A\vec{v}_1=\lambda_1\vec{v}_1-\lambda_2\vec{v}_2$ en $A\vec{v}_2=\lambda_2\vec{v}_1+\lambda_1\vec{v}_2$.
\item $\vec{v}^{~*}=\vec{v}_1-i\vec{v}_2$ een eigenwaarde bij $\lambda^*=\lambda_1-i\lambda_2$.
\item Het lineaire opspansel $<\vec{v}_1,\vec{v}_2>$ is een invariante deelruimte van $A$.
\end{enumerate}
Als $\vec{k}_n$ de kolommen van $A$ zijn is de beeldruimte van $A$ gegeven door:
\[
R(A)=<A\vec{e}_1,...,A\vec{e}_n>=<\vec{k}_1,...,\vec{k}_n>
\]
Als de kolommen $\vec{k}_n$ van een $n\times m$ matrix $A$ onafhankelijk zijn,
dan is de nulruimte ${\cal N}(A)=\{\vvec{0}\}$.

\subsection{Matrixvergelijkingen}
We gaan uit van de vergelijking
\[
A\cdot\vec{x}=\vec{b}
\]
met $\vec{b}\neq\vec{0}$. Als $\det(A)=0$ is de enige oplossing hiervan
$\vec{0}$. Als $\det(A)\neq0$ is er precies \'e\'en oplossing $\neq\vec{0}$.
\npar
De vergelijking
\[
A\cdot\vec{x}=\vec{0}
\]
heeft precies \'e\'en oplossing $\neq\vec{0}$ als $\det(A)=0$, en als
$\det(A)\neq0$ is de oplossing $\vec{0}$.
\npar
De regel van Cramer voor de oplossingen van stelsels vergelijkingen luidt:
als de vergelijking te schrijven is als
\[
A\cdot\vec{x}=\vec{b}\equiv \vec{a}_1x_1+...+\vec{a}_nx_n=\vec{b}
\]
dan is $x_j$ gegeven door:
\[
x_j=\frac{D(\vec{a}_1,...,\vec{a}_{j-1},\vec{b},\vec{a}_{j+1},...,\vec{a}_n)}{\det(A)}
\]

\section{Lineaire afbeeldingen}
Een afbeelding $A$ is lineair als geldt:
$A(\lambda\vec{x}+\beta\vvec{y})=\lambda A\vec{x}+\beta A\vec{y}$.
\npar
Enkele lineaire afbeeldingen zijn:
\begin{center}
\begin{tabular}{||p{7cm}|p{6cm}||}
\hline
\bf Soort afbeelding&\bf Formule\\
\hline
\hline
Projectie op de lijn $<\vec{a}>$              &$P(\vvec{x})=(\vec{a},\vvec{x})\vec{a}/(\vec{a},\vvec{a})$\\
Projectie op het vlak $(\vec{a},\vvec{x})=0$ &$Q(\vvec{x})=\vec{x}-P(\vvec{x})$\\
Spiegeling in de lijn $<\vec{a}>$             &$S(\vvec{x})=2P(\vvec{x})-\vec{x}$\\
Spiegeling in het vlak $(\vec{a},\vvec{x})=0$&$T(\vvec{x})=2Q(\vvec{x})-\vec{x}=\vec{x}-2P(\vvec{x})$\\
\hline
\end{tabular}
\end{center}
Voor een projectie geldt: $\vec{x}-P_W(\vvec{x})\perp P_W(\vvec{x})$ en
$P_W(\vvec{x})\in W$.
\npar
Als voor een afbeelding $A$ geldt: $(A\vec{x},\vvec{y})=(\vec{x},A\vvec{y})=(A\vec{x},A\vvec{y})$
dan is $A$ een projectie.
\npar
Zij $A:W\rightarrow W$ een lineaire afbeelding; we defini\"eren:
\begin{itemize}
\item Als $S$ een deelverzameling van $V$ is: $A(S):=\{A\vec{x}\in W|\vec{x}\in S\}$
\item Als $T$ een deelverzameling van $W$ is: $A^\leftarrow(T):=\{\vec{x}\in V|A(\vvec{x})\in T\}$
\end{itemize}
Dan is $A(S)$ een lineaire deelruimte van $W$ en de {\it inverse afbeelding}
$A^\leftarrow(T)$ een lineaire deelruimte van $V$. Hieruit volgt dat $A(V)$ de
{\it beeldruimte} van $A$ is, notatie: ${\cal R}(A)$. $A^\leftarrow(\vvec{0})=E_0$
is een lineaire deelruimte in $V$, de {\it nulruimte} van $A$, notatie:
${\cal N}(A)$.  Er geldt dat:
\[
{\rm dim}({\cal N}(A))+{\rm dim}({\cal R}(A))={\rm dim}(V)
\]

\section{Vlak en lijn}
De vergelijking van een lijn door de punten $\vec{a}$ en $\vec{b}$ is:
\[
\vec{x}=\vec{a}+\lambda(\vec{b}-\vvec{a})=\vec{a}+\lambda\vec{r}
\]
De vergelijking van een vlak is:
\[
\vec{x}=\vec{a}+\lambda(\vec{b}-\vvec{a})+\mu(\vec{c}-\vvec{a})=\vec{a}+\lambda\vec{r}_1+\mu\vec{r}_2
\]
Indien dit een vlak in $\RR^3$ is, is de {\it normaal} op dit vlak gegeven
door:
\[
\vec{n}_V=\frac{\vec{r}_1\times\vec{r}_2}{|\vec{r}_1\times\vec{r}_2|}
\]
Men kan een lijn ook voorstellen door de punten die voldoen aan de vergelijking
lijn $\ell$: $(\vec{a},\vec{x})+b=0$ en vlak V: $(\vec{a},\vec{x})+k=0$.
De normaal op V is dan: $\vec{a}/|\vec{a}|$.
\npar
De afstand $d$ tussen 2 punten $\vec{p}$ en $\vec{q}$ is gegeven door
$d(\vec{p},\vvec{q})=\|\vec{p}-\vvec{q}\|$.
\npar
In $\RR^2$ geldt:
De afstand van een punt $\vec{p}$ tot de lijn $(\vec{a},\vvec{x})+b=0$ is:
\[
d(\vec{p},\ell)=\frac{|(\vec{a},\vvec{p})+b|}{|\vec{a}|}
\]
Analoog in $\RR^3$:
De afstand van een punt $\vec{p}$ tot het vlak $(\vec{a},\vvec{x})+k=0$ is:
\[
d(\vec{p},V)=\frac{|(\vec{a},\vvec{p})+k|}{|\vec{a}|}
\]
Dit is te veralgemeniseren voor $\RR^n$ en $\CC^n$ (stelling van Hesse).

\section{Co\"ordinatentransformaties}
De lineaire afbeelding $A$ van $\KK^n\rightarrow\KK^m$ is gegeven door
($\KK=\RR$ of $\CC$):
\[
\vec{y}=A^{m\times n}\vec{x}
\]
waarin een kolom van $A$ het beeld is van een basisvector in het origineel.
\npar
De matrix $A_\alpha^\beta$ voert een vector die gegeven is t.o.v. basis
$\alpha$ over in een vector t.o.v. basis $\beta$. Ze is gegeven door:
\[
A_\alpha^\beta=\left(\beta(A\vec{a}_1),...,\beta(A\vec{a}_n)\right)
\]
waarin $\beta(\vvec{x})$ de representatie van de vector $\vec{x}$
t.o.v.\ basis $\beta$ is.
\npar
De {\it overgangsmatrix} $S_\alpha^\beta$ voert vectoren van stelsel $\alpha$
naar co\"ordinatenstelsel $\beta$ over:
\[
S_\alpha^\beta:=\II_\alpha^\beta=\left(\beta(\vec{a}_1),...,\beta(\vec{a}_n)\right)
\]
Er geldt: $S_\alpha^\beta\cdot S_\beta^\alpha=\II$
\npar
De matrix van een afbeelding $A$ is dan gegeven door:
\[
A_\alpha^\beta=\left(A_\alpha^\beta\vec{e}_1,...,A_\alpha^\beta\vec{e}_n\right)
\]
Voor de transformatie van matrixoperatoren naar een ander co\"ordinatenstelsel
geldt: $A_\alpha^\delta=S_\lambda^\delta A_\beta^\lambda S_\alpha^\beta$,
$A_\alpha^\alpha=S_\beta^\alpha A_\beta^\beta S_\alpha^\beta$ en
$(AB)_\alpha^\lambda=A_\beta^\lambda B_\alpha^\beta$.
\npar
Verder is $A_\alpha^\beta=S_\alpha^\beta A_\alpha^\alpha$,
$A_\beta^\alpha=A_\alpha^\alpha S_\beta^\alpha$. Een vector wordt
getransformeerd via $X_\alpha=S_\alpha^\beta X_\beta$.

\section{Eigenwaarden}
De {\it eigenwaarde vergelijking}
\[
A\vec{x}=\lambda\vec{x}
\]
met {\it eigenwaarden} $\lambda$ is op te lossen met
$(A-\lambda\II)=\vec{0}\Rightarrow\det(A-\lambda\II)=0$. Uit deze
karakteristieke vergelijking volgen de eigenwaarden. Er geldt dat
$\det(A)=\prod\limits_i\lambda_i$ en
${\rm Tr}(A)=\sum\limits_ia_{ii}=\sum\limits_i\lambda_i$.
\npar
De eigenwaarden $\lambda_i$ zijn basisonafhankelijk. Indien $S$ de overgangsmatrix
is naar een basis van eigenvectoren: $S=(E_{\lambda_1},...,E_{\lambda_n})$, is
de matrix van $A$ in deze basis gegeven door:
\[
\Lambda=S^{-1}AS={\rm diag}(\lambda_1,...,\lambda_n)
\]
Als 0 een eigenwaarde is van $A$ dan is $E_0(A)={\cal N}(A)$.
\npar
Als $\lambda$ een eigenwaarde is van $A$ geldt: $A^n\vec{x}=\lambda^n\vec{x}$.

\section{Soorten afbeeldingen}
\subsubsection{Isometrische afbeeldingen}
Een afbeelding is een {\it isometrie} als geldt: $\|A\vec{x}\|=\|\vec{x}\|$.
Hieruit volgt dat de eigenwaarden van een isometrie gegeven zijn door
$\lambda=\exp(i\varphi)\Rightarrow|\lambda|=1$. Tevens geldt dan:
$(A\vec{x},A\vvec{y})=(\vec{x},\vvec{y})$.
\npar
Als $W$ een invariante deelruimte van de isometrie $A$ is met dim$(A)<\infty$
dan is ook $W^\perp$ een invariante deelruimte.

\subsubsection{Orthogonale afbeeldingen}
Een afbeelding $A$ is {\it orthogonaal} als $A$ een isometrie is {\it en}
de inverse $A^\leftarrow$ bestaat. Voor een orthogonale afbeelding $O$ geldt
$O^TO=\II$, dus geldt: $O^T=O^{-1}$. Als $A$ en $B$ orthogonaal zijn, dan is
$AB$ en $A^{-1}$ ook orthogonaal.
\npar
Laat $A:V\rightarrow V$ orthogonaal zijn met dim$(V)<\infty$. Dan is $A$:
\npar
{\bf Direct orthogonaal} als $\det(A)=+1$. $A$ stelt een rotatie voor.
In $\RR^2$ wordt een rotatie om hoek $\varphi$ gegeven door:
\[
R=
\left(\begin{array}{cc}
\cos(\varphi)&-\sin(\varphi)\\
\sin(\varphi)&\cos(\varphi)
\end{array}\right)
\]
De rotatiehoek $\varphi$ is dus bepaald door Tr$(A)=2\cos(\varphi)$ met
$0\leq\varphi\leq\pi$. Laat $\lambda_1$ en $\lambda_2$ de wortels van de
karakteristieke vergelijking zijn, dan geldt tevens:
$\Re(\lambda_1)=\Re(\lambda_2)=\cos(\varphi)$, en $\lambda_1=\exp(i\varphi)$,
$\lambda_2=\exp(-i\varphi)$.
\npar
In $\RR^3$ geldt: $\lambda_1=1$, $\lambda_2=\lambda_3^*=\exp(i\varphi)$. Een
rotatie om $E_{\lambda_1}$ is gegeven door de matrix
\[
\left(\begin{array}{ccc}
1&0&0\\
0&\cos(\varphi)&-\sin(\varphi)\\
0&\sin(\varphi)&\cos(\varphi)
\end{array}\right)
\]
{\bf Gespiegeld orthogonaal} als $\det(A)=-1$. Vectoren uit $E_{-1}$ worden door
$A$ gespiegeld t.o.v. de invariante deelruimte $E^\perp_{-1}$. In $\RR^2$ wordt
een spiegeling in $<(\cos(\half\varphi),\sin(\half\varphi))>$ gegeven door:
\[
S=
\left(\begin{array}{cc}
\cos(\varphi)&\sin(\varphi)\\
\sin(\varphi)&-\cos(\varphi)
\end{array}\right)
\]
In $\RR^3$ zijn gespiegeld orthogonale afbeeldingen draaispiegelingen: dit is
een rotatie om as $<\vec{a}_1>$ om hoek $\varphi$ en spiegelvlak
$<\vec{a}_1>^\perp$. De matrix van deze afbeelding is gegeven door:
\[
\left(\begin{array}{ccc}
-1&0&0\\
0&\cos(\varphi)&-\sin(\varphi)\\
0&\sin(\varphi)&\cos(\varphi)
\end{array}\right)
\]
In $\RR^3$ geldt voor alle orthogonale afbeeldingen $O$ dat
$O(\vvec{x})\times O(\vvec{y})=O(\vec{x}\times\vvec{y})$.
\npar
$\RR^n$ $(n<\infty)$ kan men voor een orthogonale afbeelding ontbinden in
onderling loodrechte invariante deelruimten met dimensie 1 of 2.

\subsubsection{Unitaire afbeeldingen}
Laat $V$ een complexe inproductruimte zijn, dan is een lineaire afbeelding $U$
{\it unitair} als $U$ een isometrie is {\it en} de inverse afbeelding
$A^\leftarrow$ bestaat. Een $n\times n$ matrix heet unitair als $U^HU=\II$.
Er geldt dat $|\det(U)|=1$.
Elke isometrie in een eindig dimensionale complexe vectorruimte is unitair.
\npar
{\bf Stelling}: voor een $n\times n$ matrix $A$ zijn de volgende uitspraken
equivalent:
\begin{enumerate}
\item $A$ is unitair,
\item De kolommen van $A$ vormen een orthonormaal stelsel,
\item De rijen van $A$ vormen een orthonormaal stelsel.
\end{enumerate}

\subsubsection{Symmetrische afbeeldingen}
Een afbeelding $A$ op $\RR^n$ is {\it symmetrisch} als
$(A\vec{x},\vvec{y})=(\vec{x},A\vvec{y})$. Een matrix $A\in\MM^{n\times n}$
is symmetrisch als $A=A^T$. Een lineaire operator is slechts symmetrisch als
zijn matrix t.o.v. een willekeurige orthogonale basis symmetrisch is. Voor
een symmetrische afbeelding zijn alle eigenwaarden $\in\RR$. De eigenvectoren
bij verschillende eigenwaarden zijn onderling loodrecht.
Als $A$ symmetrisch is, dan is $A^T=A=A^H$ op een orthogonale basis.
\npar
Voor iedere matrix $B\in\MM^{m\times n}$ is $B^TB$ symmetrisch.

\subsubsection{Hermitische afbeeldingen}
Een afbeelding $H:V\rightarrow V$ met $V=\CC^n$ is {\it Hermitisch} als
$(H\vec{x},\vvec{y})=(\vec{x},H\vvec{y})$. De {\it Hermitisch geconjugeerde}
afbeelding $A^H$ van $A$ is: $[a_{ij}]^H=[a_{ji}^*]$. Dit wordt ook wel
genoteerd als volgt: $A^H=A^\dagger$. Het inproduct van twee vectoren $\vec{x}$
en $\vec{y}$ is nu als volgt te schrijven: $(\vec{x},\vvec{y})=\vec{x}^H\vec{y}$.
\npar
Als de afbeeldingen $A$ en $B$ Hermitisch zijn is hun product $AB$ Hermitisch
als: $[A,B]=AB-BA=0$. $[A,B]$ heet de {\it commutator} van $A$ en $B$.
\npar
De eigenwaarden van een Hermitische afbeelding zijn $\in\RR$.
\npar
Aan een Hermitische operator $L$ is een matrixrepresentatie te koppelen.
Ten opzichte van een basis $\vec{e}_i$ is deze gegeven door
$L_{mn}=(\vec{e}_m,L\vec{e}_n)$.

\subsubsection{Normale afbeeldingen}
In een complexe vectorruimte $V$ bestaat er bij iedere lineaire afbeelding $A$
precies \'e\'en lineaire afbeelding $B$ zodanig dat $(A\vec{x},\vvec{y})=(\vec{x},B\vvec{y})$.
Deze $B$ heet de {\it geadjungeerde afbeelding} van $A$. Notatie: $B=A^*$. Er
geldt: $(CD)^*=D^*C^*$. Als $A$ een unitaire afbeelding is, dan is $A^*=A^{-1}$,
als $A$ Hermitisch is dan is $A^*=A$.
\npar
{\bf Definitie}: in een complexe vectorruimte $V$ is de lineaire afbeelding
$A$ {\it normaal} als $A^*A=AA^*$. Dit is slechts het geval als voor de matrix
van $S$ t.o.v. een orthonormale basis geldt: $A^\dagger A=AA^\dagger$.
\npar
Als $A$ normaal is, dan geldt:
\begin{enumerate}
\item Voor alle vectoren $\vec{x}\in V$ en een normale afbeelding $A$ geldt:
\[
(A\vec{x},A\vvec{y})=(A^*A\vec{x},\vvec{y})=(AA^*\vec{x},\vvec{y})=(A^*\vec{x},A^*\vvec{y})
\]
\item $\vec{x}$ is een eigenvector van $A$ dan en slechts dan als $\vec{x}$ een
      eigenvector van $A^*$ is.
\item Eigenvectoren van $A$ die bij verschillende eigenwaarden horen staan
      onderling loodrecht.
\item Als $E_\lambda$ een eigenruimte is van $A$ dan is het orthoplement
      $E_\lambda^\perp$ een invariante deelruimte van $A$.
\end{enumerate}
Stel $\beta_i$ zijn de verschillende wortels van de karakteristieke vergelijking van
$A$ met multipliciteiten $n_i$. Dan is de dimensie van elke eigenruimte $V_i$
gelijk aan $n_i$. Deze eigenruimten staan onderling loodrecht en iedere vector
$\vec{x}\in V$ kan op precies \'e\'en manier geschreven worden als
\[
\vec{x}=\sum_i\vec{x}_i~~~\mbox{met}~~~\vec{x}_i\in V_i
\]
Men kan ook schrijven: $\vec{x}_i=P_i\vec{x}$ met $P_i$ een projectie op $V_i$.
Dit leidt tot de {\it spectraalstelling}: laat $A$ een normale afbeelding in
een complexe vectorruimte $V$ zijn met dim$(V)=n$. Dan geldt:
\begin{enumerate}
\item Er bestaan projectieafbeeldingen $P_i$, $1\leq i\leq p$, met de
      eigenschappen:
      \begin{itemize}
      \item $P_i\cdot P_j=0$ voor $i\neq j$,
      \item $P_1+...+P_p=\II$,
      \item ${\rm dim}P_1(V)+...+{\rm dim}P_p(V)=n$
      \end{itemize}
      en complexe getallen $\alpha_1,...,\alpha_p$ zodanig dat
      $A=\alpha_1P_1+...+\alpha_pP_p$.
\item Als $A$ unitair is dan geldt $|\alpha_i|=1~\forall i$.
\item Als $A$ Hermitisch is dan is $\alpha_i\in\RR~\forall i$.
\end{enumerate}

\subsubsection{Volledige stelsels commuterende Hermitische afbeeldingen}
Beschouw $m$ Hermitische lineaire afbeeldingen $A_i$ in een $n$ dimensionale
complexe inproductruimte $V$. Stel deze commuteren onderling.
\npar
{\bf Lemma}: als $E_\lambda$ de eigenruimte bij eigenwaarde $\lambda$ is van
$A_1$, dan is $E_\lambda$ een invariante deelruimte van alle afbeeldingen
$A_i$. D.w.z. als $\vec{x}\in E_\lambda$, dan $A_i\vec{x}\in E_\lambda$.
\npar
{\bf Stelling}. Beschouw $m$ commuterende Hermitische matrices $A_i$. Dan
bestaat er een unitaire matrix $U$ zodanig dat alle matrices $U^\dagger A_iU$
diagonaal zijn. De kolommen van $U$ zijn de gemeenschappelijke eigenvectoren
van alle matrices $A_j$.
\npar
Als van een Hermitische lineaire afbeelding in een $n$ dimensionale complexe
vectorruimte alle $n$ eigenwaarden van elkaar verschillen dan ligt een
genormeerde eigenvector op een fasefactor $\exp(i\alpha)$ na vast.
\npar
{\it Definitie}: een commuterend stelsel Hermitische afbeeldingen heet
{\it volledig} als bij elk tweetal gemeenschappelijke eigenvectoren $\vec{v}_i,\vec{v}_j$
er tenminste een afbeelding $A_k$ is zo dat $\vec{v}_i$ en $\vec{v}_j$ eigenvectoren
met verschillende eigenwaarden van $A_k$ zijn.
\npar
Meestal wordt een commuterend stelsel zo klein mogelijk genomen. In de
quantummechanica spreekt men ook van commuterende observabelen. Het aantal
dat nodig is is gelijk aan het aantal quantumgetallen die nodig zijn om een
toestand te karakteriseren.

\section{Homogene co\"ordinaten}
Wanneer men zowel rotaties als translaties in \'e\'en matrixafbeelding wil
verwerken worden homogene co\"ordinaten gebruikt. Hierbij wordt een extra
co\"ordinaat ingevoerd om de niet-lineariteiten te beschrijven. Homogene
co\"ordinaten worden uit cartesische verkregen door:
\[
\left(\begin{array}{c}x\\ y\\ z\end{array}\right)_{\rm cart}=
\left(\begin{array}{c}wx\\ wy\\ wz\\ w\end{array}\right)_{\rm hom}=
\left(\begin{array}{c}X\\ Y\\ Z\\ w\end{array}\right)_{\rm hom}
\]
dus $x=X/w$, $y=Y/w$ en $z=Z/w$. Afbeeldingen in homogene co\"ordinaten
worden beschreven door de volgende matrices:
\begin{enumerate}
\item Translatie over de vector $(X_0, Y_0, Z_0, w_0)$:
\[
T=\left(\begin{array}{cccc}
w_0&0&0&X_0\\
0&w_0&0&Y_0\\
0&0&w_0&Z_0\\
0&0&0&w_0
\end{array}\right)
\]
\item Rotaties om de $x,y,z$ assen over hoeken resp. $\alpha,\beta,\gamma$:
\[
R_x(\alpha)=\left(\begin{array}{cccc}
1&0&0&0\\
0&\cos\alpha&-\sin\alpha&0\\
0&\sin\alpha&\cos\alpha&0\\
0&0&0&1
\end{array}\right)~~~~
R_y(\beta)=\left(\begin{array}{cccc}
\cos\beta&0&\sin\beta&0\\
0&1&0&0\\
-\sin\beta&0&\cos\beta&0\\
0&0&0&1
\end{array}\right)~~~~
\]
\[
R_z(\gamma)=\left(\begin{array}{cccc}
\cos\gamma&-\sin\gamma&0&0\\
\sin\gamma&\cos\gamma&0&0\\
0&0&1&0\\
0&0&0&1
\end{array}\right)
\]
\item Een perspectivistische projectie op beeldvlak $z=c$ met het
      projectiecentrum in de oorsprong. Deze afbeelding is niet omkeerbaar.
\[
P(z=c)=\left(\begin{array}{cccc}
1&0&0&0\\
0&1&0&0\\
0&0&1&0\\
0&0&1/c&0
\end{array}\right)
\]
\end{enumerate}

\section{Inproductruimten}
Op complexe vectorruimten wordt een complex inproduct als volgt gedefinieerd:
\begin{enumerate}
\item $(\vec{a},\vvec{b})=\overline{(\vec{b},\vvec{a})}$,
\item $(\vec{a},\beta_1\vec{b}_1+\beta_2\vvec{b}_2)=\beta_1(\vec{a},\vvec{b}_1)+\beta_2(\vec{a},\vvec{b}_2)$
      voor alle $\vec{a},\vec{b}_1,\vec{b}_2\in V$ en $\beta_1,\beta_2\in\CC$.
\item $(\vec{a},\vvec{a})\geq0$ voor alle $\vec{a}\in V$,
      $(\vec{a},\vvec{a})=0$ dan en slechts dan als $\vec{a}=\vec{0}$.
\end{enumerate}
Wegens (1) is $(\vec{a},\vvec{a})\in\RR$. De {\it inproductruimte} $\CC^n$ is
de complexe vectorruimte waarop een complex inproduct gedefinieerd is door:
\[
(\vec{a},\vvec{b})=\sum_{i=1}^na_i^*b_i
\]
Voor functieruimten geldt:
\[
(f,g)=\int\limits_a^bf^*(t)g(t)dt
\]
Voor iedere $\vec{a}$ is de lengte $\|\vvec{a}\|$ gedefinieerd door:
$\|\vvec{a}\|=\sqrt{(\vec{a},\vvec{a})}$. Er geldt:
$\|\vvec{a}\|-\|\vvec{b}\|\leq\|\vec{a}+\vvec{b}\|\leq\|\vvec{a}\|+\|\vvec{b}\|$,
en met $\varphi$ de hoek tussen $\vec{a}$ en $\vec{b}$ geldt:
$(\vec{a},\vvec{b})=\|\vvec{a}\|\cdot\|\vvec{b}\|\cos(\varphi)$.
\npar
Laat $\{\vec{a}_1,...,\vec{a}_n\}$ een stelsel vectoren zijn in een inproductruimte
$V$. Dan is de {\it Grammatrix G} van dit stelsel: $G_{ij}=(\vec{a}_i,\vec{a}_j)$.
Het stelsel vectoren is onafhankelijk dan en slechts dan als $\det(G)=0$.
\npar
Een stelsel is {\it orthonormaal} als $(\vec{a}_i,\vec{a}_j)=\delta_{ij}$.
Als $\vec{e}_1,\vec{e}_2,...$ een orthonormale rij is in een oneindig
dimensionale vectorruimte geldt de ongelijkheid van Bessel:
\[
\|\vvec{x}\|^2\geq\sum_{i=1}^\infty|(\vec{e}_i,\vvec{x})|^2
\]
Het gelijkteken geldt dan en slechts dan als
$\lim\limits_{n\rightarrow\infty}\|\vec{x}_n-\vvec{x}\|=0$.
\npar
De inproductruimte $\ell^2$ is gedefinieerd in $\CC^\infty$ door:
\[
\ell^2=\left\{\vec{a}=(a_1,a_2,...)~|~\sum_{n=1}^\infty|a_n|^2<\infty\right\}
\]
Als een ruimte $\ell^2$ is en tevens voldoet aan:
$\lim\limits_{n\rightarrow\infty}|a_{n+1}-a_n|=0$,
dan is het een {\it Hilbertruimte}.

\section{De Laplace transformatie}
De klasse LT bestaat uit functies die voldoen aan:
\begin{enumerate}
\item Op ieder interval $[0,A]$, $A>0$ zijn niet meer dan eindig veel
      discontinu\"{\i}teiten en in iedere discontinu\"{\i}teit bestaan de
      boven- en onder limiet,
\item $\exists t_0\in[0,\infty>$ en een $a,M\in\RR$ z.d.d. voor $t\geq t_0$
      geldt: $|f(t)|\exp(-at)<M$.
\end{enumerate}
Dan bestaat voor $f$ de Laplace getransformeerde.
\npar
De Laplace transformatie is een generalisatie van de Fouriertransformatie.
Indien $s\in\CC$ en $t\geq0$ is de Laplace getransformeerde
${\cal L}(f(t))=F(s)$ van een functie $f(t)$ gegeven door:
\[
F(s)=\int\limits_0^\infty f(t){\rm e}^{-st}dt
\]
De Laplace getransformeerde van een afgeleide functie is gegeven door:
\[
{\cal L}\left(f^{(n)}(t)\right)=-f^{(n-1)}(0)-sf^{(n-2)}(0)-...-s^{n-1}f(0)+s^nF(s)
\]
De operator $\cal L$ heeft de volgende eigenschappen:
\begin{enumerate}
\item Gelijkvormigheid: Als $a>0$ dan
\[
{\cal L}\left(f(at)\right)=\frac{1}{a}F\left(\frac{s}{a}\right)
\]
\item Demping: ${\cal L}\left({\rm e}^{-at}f(t)\right)=F(s+a)$
\item Verschuiving: Zij $a>0$ en $g$ gedefinieerd door $g(t)=f(t-a)$ als
$t>a$ en $g(t)=0$ voor $t\leq a$ dan geldt:
${\cal L}\left(g(t)\right)={\rm e}^{-sa}{\cal L}(f(t))$.
\end{enumerate}
Als $s\in\RR$ geldt dat $\Re(\lambda f)={\cal L}(\Re(f))$ en
$\Im(\lambda f)={\cal L}(\Im(f))$.
\npar
Voor enkele veel voorkomende functies geldt:
\begin{center}
\begin{tabular}{||c||c||}
\hline
$f(t)=$&$F(s)={\cal L}(f(t))=$\\
\hline
\hline
$\displaystyle\frac{t^n}{n!}{\rm e}^{at}$&$(s-a)^{-n-1}$\rule{0pt}{15pt}\\
${\rm e}^{at}\cos(\omega t)$&$\displaystyle\frac{s-a}{(s-a)^2+\omega^2}$\rule{0pt}{15pt}\\
${\rm e}^{at}\sin(\omega t)$&$\displaystyle\frac{\omega}{(s-a)^2+\omega^2}$\rule{0pt}{15pt}\\
$\delta(t-a)$&$\exp(-as)$\rule{0pt}{13pt}\\
\hline
\end{tabular}
\end{center}

\section{De convolutie}
De convolutie integraal is gedefinieerd door:
\[
(f*g)(t)=\int\limits_0^tf(u)g(t-u)du
\]
De convolutie heeft de volgende eigenschappen:
\begin{enumerate}
\item $f*g\in$LT
\item ${\cal L}(f*g)={\cal L}(f)\cdot{\cal L}(g)$
\item Distributie: $f*(g+h)=f*g+f*h$
\item Commutatie: $f*g=g*f$
\item Homogeniteit: $f*(\lambda g)=\lambda f*g$
\end{enumerate}
Als ${\cal L}(f)=F_1\cdot F_2$, dan is $f(t)=f_1*f_2$.

\section{Stelsels lineaire differentiaalvergelijkingen}
We gaan uit van de vergelijking $\dot{\vec{x}}=A\vec{x}$. Stel $\vec{x}=\vec{v}\exp(\lambda t)$,
dan volgt: $A\vec{v}=\lambda\vec{v}$. In het $2\times2$ geval geldt:
\begin{enumerate}
\item $\lambda_1=\lambda_2$: dan is $\vec{x}(t)=\sum\vec{v}_i\exp(\lambda_it)$.
\item $\lambda_1\neq\lambda_2$: dan is $\vec{x}(t)=(\vec{u}t+\vec{v})\exp(\lambda t)$.
\end{enumerate}
Stel dat $\lambda=\alpha+i\beta$ een eigenwaarde is bij eigenvector $\vec{v}$,
dan is $\lambda^*$ ook een eigenwaarde bij eigenvector $\vvec{v}^*$. Ontbind
$\vec{v}=\vec{u}+i\vec{w}$, dan zijn de re\"ele oplossingen
\[
c_1[\vec{u}\cos(\beta t)-\vec{w}\sin(\beta t)]{\rm e}^{\alpha t}+c_2[\vec{v}\cos(\beta t)+\vec{u}\sin(\beta t)]{\rm e}^{\alpha t}
\]
\npar
Voor de vergelijking $\ddot{\vec{x}}=A\vec{x}$ zijn twee oplossingsstrategie\"en:
\begin{enumerate}
\item Stel $\vec{x}=\vec{v}\exp(\lambda t)\Rightarrow\det(A-\lambda^2\II)=0$.
\item Voer in: $\dot{x}=u$ en $\dot{y}=v$, dat geeft $\ddot{x}=\dot{u}$ en
      $\ddot{y}=\dot{v}$. Zo wordt een $n$-dimensionaal stelsel van tweede orde
      omgezet in een $2n$-dimensionaal stelsel van de eerste orde.
\end{enumerate}

\section{Kwadrieken}
\subsection{Kwadrieken in $\RR^2$}
De algemene vergelijking voor een kwadriek is:
$\vec{x}^TA\vec{x}+2\vec{x}^TP+S=0$. Hierin is $A$ een symmetrische matrix.
Als $\Lambda=S^{-1}AS={\rm diag}(\lambda_1,...,\lambda_n)$ geldt:
$\vec{u}^T\Lambda\vec{u}+2\vec{u}^TP+S=0$, zodat alle kruistermen 0 zijn.
Om dezelfde ori\"entatie te behouden als het stelsel $(x,y,z)$ heeft moet men
$\vec{u}=(u,v,w)$ zodanig kiezen dat det$(S)=+1$.
\npar
Uitgaande van de vergelijking
\[
ax^2+2bxy+cy^2+dx+ey+f=0
\]
geldt dat $|A|=ac-b^2$. Een ellips heeft $|A|>0$, een parabool $|A|=0$ en
een hyperbool $|A|<0$. In poolco\"ordinaten is dit te schrijven als:
\[
r=\frac{ep}{1-e\cos(\theta)}
\]
Een ellips heeft $e<1$, een parabool $e=1$ en een hyperbool $e>1$.

\subsection{Tweedegraads oppervlakken in $\RR^3$}
Rang 3:
\[
p\frac{x^2}{a^2}+q\frac{y^2}{b^2}+r\frac{z^2}{c^2}=d
\]
\begin{itemize}
\item Ellipso\"{\i}de: $p=q=r=d=1$, $a,b,c$ zijn de lengtes van de halve assen.
\item Eenbladige hyperbolo\"{\i}de: $p=q=d=1$, $r=-1$.
\item Tweebladige hyperbolo\"{\i}de: $r=d=1$, $p=q=-1$.
\item Kegel: $p=q=1$, $r=-1$, $d=0$.
\end{itemize}
Rang 2:
\[
p\frac{x^2}{a^2}+q\frac{y^2}{b^2}+r\frac{z}{c^2}=d
\]
\begin{itemize}
\item Elliptische parabolo\"{\i}de: $p=q=1$, $r=-1$, $d=0$.
\item Hyperbolische parabolo\"{\i}de: $p=r=-1$, $q=1$, $d=0$.
\item Elliptische cylinder: $p=q=-1$, $r=d=0$.
\item Hyperbolische cylinder: $p=d=1$, $q=-1$, $r=0$.
\item Vlakkenpaar: $p=1$, $q=-1$, $d=0$.
\end{itemize}
Rang 1:
\[
py^2+qx=d
\]
\begin{itemize}
\item Parabolische cylinder: $p,q>0$.
\item Evenwijdig vlakkenpaar: $d>0$, $q=0$, $p\neq 0$.
\item Dubbelvlak: $p\neq 0$, $q=d=0$.
\end{itemize}


\chapter{Complexe functietheorie}
\typeout{Complexe functietheorie}
\section{Functies van complexe variabelen}
Complexe functietheorie beschouwt complexe functies van een complexe variabele.
Enkele definities:
\npar
$f$ is {\it analytisch} op $\cal G$ als $f$ continu en differentieerbaar is
op $\cal G$.
\npar
Een {\it Jordankromme} is een kromme die gesloten en enkelvoudig is.
\npar
Als K een boog is in $\CC$ met parametervoorstelling $z=\phi(t)=x(t)+iy(t)$,
$a\leq t\leq b$, dan is de lengte $L$ van K gelijk aan:
\[
L=\int\limits_a^b \sqrt{\left(\frac{dx}{dt}\right)^2+\left(\frac{dy}{dt}\right)^2}dt=
\int\limits_a^b\left|\frac{dz}{dt}\right|dt=\int\limits_a^b|\phi'(t)|dt
\]
De afgeleide van $f$ in het punt $z=a$ is:
\[
f'(a)=\lim_{z\rightarrow a}\frac{f(z)-f(a)}{z-a}
\]
Als $f(z)=u(x,y)+iv(x,y)$ is de afgeleide:
\[
f'(z)=\Q{u}{x}+i\Q{v}{x}=-i\Q{u}{y}+\Q{v}{y}
\]
Gelijkstellen van beide uitkomsten levert de vergelijkingen van Cauchy-Riemann:
\[
\Q{u}{x}=\Q{v}{y}~~~,~~~\Q{u}{y}=-\Q{v}{x}
\]
Deze vergelijkingen impliceren dat dan $\nabla^2u=\nabla^2v=0$.
$f$ is analytisch als $u$ en $v$ voldoen aan deze vergelijkingen.

\section{Integratie}
\subsection{Hoofdstelling}
Zij $K$ een boog beschreven door $z=\phi(t)$ op $a\leq t\leq b$ en $f(z)$
is continu op $K$. Dan is de integraal van $f$ over $K$:
\[
\int\limits_Kf(z)dz=\int\limits_a^bf(\phi(t))\dot{\phi}(t)dt
\stackrel{f\mbox{ continu}}{=}F(b)-F(a)
\]
{\bf Lemma}: zij $K$ de cirkel met middelpunt $a$ en straal $r$, in positieve
richting doorlopen. Dan geldt voor gehele $m$:
\[
\frac{1}{2\pi i}\oint\limits_K \frac{dz}{(z-a)^m}=\left\{
\begin{array}{l}
0~~\mbox{als}~~m\neq1\\
1~~\mbox{als}~~m=1
\end{array}\right.
\]
{\bf Stelling}: als $L$ de lengte van kromme $K$ is, en zij $|f(z)|\leq M$ voor
$z\in K$, dan geldt als de integraal bestaat
\[
\left|\int\limits_K f(z)dz\right|\leq ML
\]
{\bf Stelling}: zij $f$ continu op een gebied $G$ en $p$ een vast punt van
$G$. Laat $F(z)=\int_p^zf(\xi)d\xi$ voor alle $z\in G$ alleen afhangen van
$z$ en niet van de integratieweg. Dan is $F(z)$ analytisch in $G$ met $F'(z)=f(z)$.
\npar
Dit leidt tot de twee equivalente formuleringen van de {\it hoofdstelling der
complexe integratie}: zij de functie $f$ analytisch in een gebied $G$. Laat
$K$ en $K'$ twee krommen zijn met dezelfde begin- en eindpunten, die door
continue vervorming binnen $G$ in elkaar zijn over te voeren. Laat $B$ een
Jordankromme zijn. Dan geldt
\[
\int\limits_Kf(z)dz=\int\limits_{K'}f(z)dz\Leftrightarrow\oint\limits_Bf(z)dz=0
\]
Door de hoofdstelling toe te passen op ${\rm e}^{iz}/z$ kan men afleiden dat
\[
\int\limits_0^\infty\frac{\sin(x)}{x}dx=\frac{\pi}{2}
\]

\subsection{Residu}
Een punt $a\in\CC$ is een {\it regulier punt} van een functie $f(z)$ als $f$
analytisch is in $a$. Anders is $a$ een {\it singulier punt} van $f(z)$. Het
{\it residu} van $f$ in $a$ wordt gedefinieerd door
\[
\mathop{\rm Res}\limits_{z=a}f(z)=\frac{1}{2\pi i}\oint\limits_Kf(z)dz
\]
waarin $K$ een Jordankromme is die $a$ in positieve richting omsluit. In
reguliere punten is het residu 0, in singuliere punten kan het zowel 0 als
$\neq0$ zijn. De residustelling van Cauchy luidt: zij $f$ analytisch binnen en
op een Jordankromme $K$ met uitzondering van een eindig aantal singuliere
punten $a_i$ binnen $K$. Dan geldt als $K$ in positieve richting doorlopen
wordt:
\[
\frac{1}{2\pi i}\oint\limits_Kf(z)dz=\sum_{k=1}^n\mathop{\rm Res}\limits_{z=a_k}f(z)
\]
{\bf Lemma}: zij de functie $f$ analytisch in $a$, dan geldt:
\[
\mathop{\rm Res}\limits_{z=a}\frac{f(z)}{z-a}=f(a)
\]
Dit leidt tot de integraalstelling van Cauchy: als $f$ analytisch is op de in
positieve richting doorlopen Jordankromme $K$, dan geldt
\[
\frac{1}{2\pi i}\oint\limits_K\frac{f(z)}{z-a}dz=\left\{\begin{array}{l}
f(a)~~\mbox{als}~~a~~\mbox{binnen}~~K\\
0~~\mbox{als}~~a~~\mbox{buiten}~~K
\end{array}\right.
\]
{\bf Stelling}: zij $K$ een kromme (hoeft niet gesloten te zijn) en zij
$\phi(\xi)$ continu op $K$. Dan is de functie
\[
f(z)=\int\limits_K\frac{\phi(\xi)d\xi}{\xi-z}
\]
analytisch met $n$-de afgeleide
\[
f^{(n)}(z)=n!\int\limits_K\frac{\phi(\xi)d\xi}{(\xi-z)^{(n+1)}}
\]
{\bf Stelling}: zij $K$ een kromme en $G$ een gebied. Zij $\phi(\xi,z)$
gedefinieerd voor $\xi\in K$, $z\in G$, met de volgende eigenschappen:
\begin{enumerate}
\item $\phi(\xi,z)$ is begrensd, d.w.z. $|\phi(\xi,z)|\leq M$ voor $\xi\in K$, $z\in G$,
\item Voor vaste $\xi\in K$ is $\phi(\xi,z)$ een analytische functie van $z$ in $G$,
\item Voor vaste $z\in G$ zijn $\phi(\xi,z)$ en $\partial\phi(\xi,z)/\partial z$
      continue functies van $\xi$ op $K$.
\end{enumerate}
Dan is de functie
\[
f(z)=\int\limits_K\phi(\xi,z)d\xi
\]
analytisch met afgeleide
\[
f'(z)=\int\limits_K\Q{\phi(\xi,z)}{z}d\xi
\]
{\bf Ongelijkheid van Cauchy}: zij de functie $f(z)$ analytisch binnen en op de
cirkel $C:|z-a|=R$, en zij $|f(z)|\leq M$ voor $z\in C$. Dan geldt
\[
\left|f^{(n)}(a)\right|\leq\frac{Mn!}{R^n}
\]

\section{Analytische functies gedefinieerd door reeksen}
De reeks $\sum f_n(z)$ heet puntsgewijs convergent op een gebied $G$ met som
$F(z)$ als
\[
\forall_{\varepsilon>0}\forall_{z\in G}\exists_{N_0\in\RR}\forall_{n>n_0}
\left[~\left|f(z)-\sum_{n=1}^Nf_n(z)\right|<\varepsilon\right]
\]
De reeks heet {\it uniform convergent} als
\[
\forall_{\varepsilon>0}\exists_{N_0\in\RR}\forall_{n>n_0}\exists_{z\in G}
\left[~\left|f(z)-\sum_{n=1}^Nf_n(z)\right|<\varepsilon\right]
\]
Uniforme convergentie impliceert puntsgewijze convergentie, omgekeerd hoeft
dat niet zo te zijn.
\npar
{\bf Stelling}: laat de machtreeks $\sum\limits_{n=0}^\infty a_nz^n$ de
convergentiestraal $R$ hebben. $R$ is de afstand tot de eerste niet-ophefbare
singulariteit.
\begin{itemize}
\item Als $\displaystyle\lim_{n\rightarrow\infty}\sqrt[n]{|a_n|}=L$ bestaat, dan is $R=1/L$.
\item Als $\displaystyle\lim_{n\rightarrow\infty}|a_{n+1}|/|a_n|=L$ bestaat, dan is $R=1/L$.
\end{itemize}
Als geen van beide limieten bestaat is $R$ te bepalen met de formule van
Cauchy-Hadamard:
\[
\frac{1}{R}=\lim_{n\rightarrow\infty}{\rm sup}\sqrt[n]{|a_n|}
\]

\section{Laurent reeksen}
{\bf Stelling van Taylor}: zij $f$ analytisch in een gebied $G$ en laat punt
$a\in G$ afstand $r$ tot de rand van $G$ hebben. Dan is $f(z)$ te ontwikkelen
in de Taylorreeks rond $a$:
\[
f(z)=\sum_{n=0}^\infty c_n(z-a)^n~~~\mbox{met}~~~c_n=\frac{f^{(n)}(a)}{n!}
\]
geldig voor $|z-a|<r$. De convergentiestraal van de Taylorreeks is $\geq r$.
Indien $f$ een $k$-voudig nulpunt heeft in $a$ is $c_1,...,c_{k-1}=0$, $c_k\neq0$.
\npar
{\bf Stelling van Laurent}: zij $f$ analytisch in het ringgebied
$G:r<|z-a|<R$. Dan is $f(z)$ te ontwikkelen in een Laurentreeks met
middelpunt $a$:
\[
f(z)=\sum_{n=-\infty}^\infty c_n(z-a)^n~~~\mbox{met}~~~
c_n=\frac{1}{2\pi i}\oint\limits_K\frac{f(w)dw}{(w-a)^{n+1}}~~,~~n\in\ZZ
\]
geldig voor $r<|z-a|<R$ en $K$ een willekeurige Jordankromme in $G$ die het
punt $a$ in positieve richting omsluit.
\npar
Het {\it hoofddeel} van een Laurentreeks is: $\sum\limits_{n=1}^\infty c_{-n}(z-a)^{-n}$.
Hiermee zijn singuliere punten te classificeren. Er zijn 3 gevallen:
\begin{enumerate}
\item Het hoofddeel ontbreekt. Dan is $a$ een ophefbare singulariteit.
      Definieer $f(a)=c_0$ en de reeksontwikkeling geldt ook voor $|z-a|<R$
      en $f$ is analytisch in $a$.
\item Het hoofddeel bevat eindig veel termen. Dan bestaat er een $k\in\NN$ z.d.d.
      $\lim\limits_{z\rightarrow a}(z-a)^kf(z)=c_{-k}\neq0$. De functie
      $g(z)=(z-a)^kf(z)$ heeft dan een ophefbare singulariteit in $a$. Men
      zegt dat $f$ een $k$-voudige pool heeft in $z=a$.
\item Het hoofddeel bevat oneindig veel termen. Dan is $a$ een essentieel
      singulier punt van $f$, bv. $\exp(1/z)$ voor $z=0$.
\end{enumerate}
Als $f$ en $g$ analytisch zijn, $f(a)\neq0$, $g(a)=0$, $g'(a)\neq0$ dan heeft
$f(z)/g(z)$ een enkelvoudige pool in $z=a$ met
\[
\mathop{\rm Res}\limits_{z=a}\frac{f(z)}{g(z)}=\frac{f(a)}{g'(a)}
\]

\section{De stelling van Jordan}
Residuen worden vaak gebruikt bij het berekenen van bepaalde integralen. Met de
notaties $C_\rho^+=\{z||z|=\rho,\Im(z)\geq0\}$ en $C_\rho^-=\{z||z|=\rho,\Im(z)\leq0\}$
en $M^+(\rho,f)=\mathop{\rm max}\limits_{z\in C_\rho^+}|f(z)|$,
$M^-(\rho,f)=\mathop{\rm max}\limits_{z\in C_\rho^-}|f(z)|$. We nemen aan dat
$f(z)$ analytisch is voor $\Im(z)>0$ met eventuele uitzondering van een eindig
aantal singuliere punten die niet op de re\"ele as liggen,
$\lim\limits_{\rho\rightarrow\infty}\rho M^+(\rho,f)=0$ en de integraal bestaat,
dan is
\[
\int\limits_{-\infty}^\infty f(x)dx=2\pi i\sum{\rm Res}f(z)~~~\mbox{in}~~~\Im(z)>0
\]
Vervang $M^+$ door $M^-$ in bovenstaande eisen en er volgt:
\[
\int\limits_{-\infty}^\infty f(x)dx=-2\pi i\sum{\rm Res}f(z)~~~\mbox{in}~~~\Im(z)<0
\]
Het {\it Lemma van Jordan}: zij $f$ continu voor $|z|\geq R$, $\Im(z)\geq0$ en
$\lim\limits_{\rho\rightarrow\infty}M^+(\rho,f)=0$. Dan geldt voor $\alpha>0$
\[
\lim_{\rho\rightarrow\infty}\int\limits_{C_\rho^+}f(z){\rm e}^{i\alpha z}dz=0
\]
Zij $f$ continu voor $|z|\geq R$, $\Im(z)\leq0$ en
$\lim\limits_{\rho\rightarrow\infty}M^-(\rho,f)=0$. Dan geldt voor $\alpha<0$
\[
\lim_{\rho\rightarrow\infty}\int\limits_{C_\rho^-}f(z){\rm e}^{i\alpha z}dz=0
\]
Laat $z=a$ een enkelvoudige pool zijn van $f(z)$ en zij $C_\delta$ de halve
cirkel $|z-a|=\delta,0\leq{\rm arg}(z-a)\leq\pi$, doorlopen van $a+\delta$ naar
$a-\delta$. Dan is
\[
\lim_{\delta\downarrow0}\frac{1}{2\pi i}\int\limits_{C_\delta}f(z)dz=\half\mathop{\rm Res}\limits_{z=a}f(z)
\]


\chapter{Tensorrekening}
\typeout{Tensorrekening}
\section{Vectoren en covectoren}
We noteren een eindig dimensionale vectorruimte met $\cal V, W$. De
vectorruimte van lineaire afbeeldingen van $\cal V$ naar $\cal W$ wordt
aangegeven met $\cal L(V,W)$. Beschouw ${\cal L(V,}\RR):={\cal V}^*$. We
noemen $\cal V^*$ de {\it duale ruimte} van $\cal V$. We kunnen nu
{\it vectoren} in $\cal V$  met basis $\vec{c}$ en {\it covectoren} in 
$\cal V^*$ met basis $\hat{\vec{c}}$ defini\"eeren. Eigenschappen van beide
zijn:
\begin{enumerate}
\item Vectoren: $\vec{x}=x^i\vec{c}_i$ met de basisvectoren $\vec{c}_i$:
      \[
      \vec{c}_i=\Q{}{x^i}
      \]
      Overgang van stelsel $i$ naar $i'$ wordt gegeven door:
      \[
      \vec{c}_{i'}=A_{i'}^i\vec{c}_i=\partial_i\in{\cal V}~~,~~x^{i'}=A_i^{i'}x^i
      \]
\item Covectoren: $\hat{\vec{x}}=x_i\hat{\vec{c}}^{~i}$ met de basisvectoren $\hat{\vec{c}}^{~i}$
      \[
      \hat{\vec{c}}^{~i}=dx^i
      \]
      Overgang van stelsel $i$ naar $i'$ wordt gegeven door:
      \[
      \hat{\vec{c}}^{~i'}=A_i^{i'}\hat{\vec{c}}^{~i}\in{\cal V}^*~~,~~\vec{x}_{i'}=A_{i'}^i\vec{x}_i
      \]
\end{enumerate}
Hierbij is de {\it Einsteinconventie} gebruikt:
\[
a^ib_i:=\sum_ia^ib_i
\]
De co\"ordinatentransformatie wordt gegeven door:
\[
A_{i'}^i=\Q{x^i}{x^{i'}}~~,~~A_i^{i'}=\Q{x^{i'}}{x^i}
\]
Hieruit volgt dat $A_k^i\cdot A_l^k=\delta_l^i$ en $A_{i'}^i=(A_i^{i'})^{-1}$.
\npar
De co\"ordinatentransformaties zijn in differentiaalnotatie gegeven door:
\[
dx^i=\Q{x^i}{x^{i'}}dx^{i'}~~~\mbox{en}~~~\Q{}{x^{i'}}=\Q{x^i}{x^{i'}}\Q{}{x^i}
\]
In het algemene geval geldt voor de transformatie van een tensor $T$:
\[
T_{s_1...s_m}^{q_1...q_n}=\left|\Q{\vec{x}}{\vec{u}}\right|^\ell
\Q{u^{q_1}}{x^{p_1}}\cdots\Q{u^{q_n}}{x^{p_n}}\cdot\Q{x^{r_1}}{u^{s_1}}\cdots\Q{x^{r_m}}{u^{s_m}}T_{r_1...r_m}^{p_1...p_n}
\]
bij een {\it absolute tensor} is $\ell=0$.

\section{Tensoralgebra}
Er geldt:
\[
a_{ij}(x_i+y_i)\equiv a_{ij}x_i+a_{ij}y_i,~~~\mbox{maar:}~~a_{ij}(x_i+y_j)\not\equiv a_{ij}x_i+a_{ij}y_j
\]
en
\[
(a_{ij}+a_{ji})x_ix_j\equiv2a_{ij}x_ix_j,~~~\mbox{maar:}~~(a_{ij}+a_{ji})x_iy_j\not\equiv 2a_{ij}x_iy_j
\]
en $(a_{ij}-a_{ji})x_ix_j\equiv0$.
\npar
De som en verschil van 2 tensoren is een tensor van gelijke rang: $A_q^p\pm
B_q^p$. Bij het uitwendig tensorproduct is de rang van het resultaat de som
van de rangen van beide tensoren: $A_q^{pr}\cdot B_s^m=C_{qs}^{prm}$. Bij de
{\it contractie} worden 2 indices gelijk gesteld en erover gesommeerd. Stel
we nemen $r=s$ bij tensor $A_{qs}^{mpr}$, dit geeft: $\sum\limits_r
A_{qr}^{mpr}=B_q^{mp}$. Het {\it inproduct} van twee tensoren is gedefinieerd
als het uitproduct waarna een contractie over een index volgt.

\section{Inwendig product}
{\bf Definitie}: de bilineaire afbeelding $B:{\cal V}\times{\cal V}^*\rightarrow\RR$,
$B(\vec{x},\hat{\vec{y}}\,)=\hat{\vec{y}}(\vvec{x})$ wordt aangegeven met
$<\vec{x},\hat{\vec{y}}\,>$.
Voor deze {\it paringsoperator} $<\cdot,\cdot>=\delta$ geldt:
\[
\hat{\vec{y}}(\vec{x})=<\vec{x},\hat{\vec{y}}>=y_ix^i~~~,~~~<\hat{\vec{c}^{~i}},\vec{c}_j>=\delta_j^i
\]
Zij $G:{\cal V}\rightarrow{\cal V}^*$ een lineaire bijectie. Defini\"eer de
bilineaire vormen
\begin{eqnarray*}
g:{\cal V\times V}\rightarrow\RR&~~~&g(\vec{x},\vvec{y})=<\vec{x},G\vvec{y}>\\
h:{\cal V^*\times V^*}\rightarrow\RR&~~~&h(\hat{\vec{x}},\hat{\vec{y}}\,)=<G^{-1}\hat{\vec{x}},\hat{\vec{y}}\,>
\end{eqnarray*}
Beide zijn niet-gedegenereerd. Er geldt: $h(G\vec{x},G\vvec{y})=<\vec{x},G\vec{y}>=g(\vec{x},\vvec{y})$.
Als we $\cal V$ en $\cal V^*$ identificeren m.b.v. $G$ dan geeft $g$ (of $h$)
een inwendig product op $\cal V$.
\npar
Het inproduct $(,)_\Lambda$ op $\Lambda^k(\cal V)$ wordt gedefinieerd door:
\[
(\Phi,\Psi)_\Lambda=\frac{1}{k!}(\Phi,\Psi)_{T^0_k(\cal V)}
\]
Het inproduct van 2 vectoren is dan gegeven door:
\[
(\vec{x},\vvec{y})=x^iy^i<\vec{c}_i,G\vec{c}_j>=g_{ij}x^ix^j
\]
De matrix $g_{ij}$ van $G$ is gegeven door
\[
g_{ij}\hat{\vec{c}}^{~j}=G\vec{c}_i
\]
De matrix $g^{ij}$ van $G^{-1}$ is gegeven door:
\[
g^{kl}\vec{c}_l=G^{-1}\hat{\vec{c}}^{~k}
\]
Voor deze {\it metrische tensor} $g_{ij}$ geldt: $g_{ij}g^{jk}=\delta_i^k$.
Deze tensor kan indices omhoog of omlaag halen:
\[
x_j=g_{ij}x^i~~~,~~~x^i=g^{ij}x_j
\]
en $du^i=\hat{\vec{c}}^{~i}=g^{ij}\vec{c}_j$.
\npar

\section{Tensorproduct}
{\bf Definitie}: laat $\cal U$ en $\cal V$ twee eindig dimensionale vectorruimten
zijn met dimensies $m$ resp. $n$. Zij $\cal U^*\times V^*$ het cartesisch
product van $\cal U$ en $\cal V$. Een functie $t:{\cal U^*\times V^*}\rightarrow\RR$;
$(\hat{\vec{u}};\hat{\vec{v}}\,)\mapsto t(\hat{\vec{u}};\hat{\vec{v}}\,)=t^{\alpha\beta}u_\alpha u_\beta\in\RR$
heet een tensor als $t$ lineair is in $\hat{\vec{u}}$ en $\hat{\vec{v}}$.
De tensoren $t$ vormen een vectorruimte genoteerd met $\cal U\otimes V$.
De elementen $T\in\cal V\otimes V$ heten contravariante 2-tensoren:
$T=T^{ij}\vec{c}_i\otimes\vec{c}_j=T^{ij}\partial_i\otimes\partial_j$. De
elementen $T\in\cal V^*\otimes V^*$ heten covariante 2-tensoren:
$T=T_{ij}\hat{\vec{c}}^{~i}\otimes\hat{\vec{c}}^{~j}=T_{ij}dx^i\otimes dx^j$.
De elementen $T\in\cal V^*\otimes V$ heten gemengde 2 tensoren:
$T=T_i^{.j}\hat{\vec{c}}^{~i}\otimes\vec{c}_j=T_i^{.j}dx^i\otimes\partial_j$,
en analoog voor $T\in\cal V\otimes V^*$.
\npar
De getallen, gegeven door
\[
t^{\alpha\beta}=t(\hat{\vec{c}}^{~\alpha},\hat{\vec{c}}^{~\beta}\,)
\]
met $1\leq\alpha\leq m$ en $1\leq\beta\leq n$ vormen de componenten of
kentallen van $t$.
\npar
Neem $\vec{x}\in\cal U$ en $\vec{y}\in\cal V$. Dan is de functie
$\vec{x}\otimes\vec{y}$, gedefinieerd door
\[
(\vec{x}\otimes\vec{y})(\hat{\vec{u}},\hat{\vec{v}})=<\vec{x},\hat{\vec{u}}>_U<\vec{y},\hat{\vec{v}}>_V
\]
een tensor. De kentallen volgen uit: $(\vec{u}\otimes\vvec{v})_{ij}=u_iv^j$.
Het tensorproduct van 2 tensoren is gegeven door:
\begin{eqnarray*}
{2\choose0}~\mbox{vorm:}~&&(\vec{v}\otimes\vec{w})(\hat{\vec{p}},\hat{\vec{q}})=v^ip_iw^kq_k=T^{ik}p_iq_k\\
{0\choose2}~\mbox{vorm:}~&&(\hat{\vec{p}}\otimes\hat{\vec{q}})(\vec{v},\vec{w})=p_iv^iq_kw^k=T_{ik}v^iw^k\\
{1\choose1}~\mbox{vorm:}~&&(\vec{v}\otimes\hat{\vec{p}})(\hat{\vec{q}},\vec{w})=v^iq_ip_kw^k=T_k^iq_iw^k
\end{eqnarray*}

\section{Symmetrische - en antisymmetrische tensoren}
Een tensor $t\in{\cal V\otimes V}$ heet symmetrisch resp.\ antisymmetrisch als
$\forall\hat{\vec{x}},\hat{\vec{y}}\in{\cal V^*}$ geldt:
$t(\hat{\vec{x}},\hat{\vec{y}}\,)=t(\hat{\vec{y}},\hat{\vec{x}}\,)$ resp.
$t(\hat{\vec{x}},\hat{\vec{y}}\,)=-t(\hat{\vec{y}},\hat{\vec{x}}\,)$.
\npar
Een tensor $t\in{\cal V^*\otimes V^*}$ heet symmetrisch resp. antisymmetrisch
als $\forall\vec{x},\vec{y}\in{\cal V}$ geldt:
$t(\vec{x},\vvec{y})=t(\vec{y},\vvec{x})$ resp.
$t(\vec{x},\vvec{y})=-t(\vec{y},\vvec{x})$. In $\cal V\otimes W$ worden de
lineaire afbeeldingen $\cal S$ en $\cal A$ gedefinieerd door:
\begin{eqnarray*}
{\cal S}t(\hat{\vec{x}},\hat{\vec{y}}\,)&=&\half(t(\hat{\vec{x}},\hat{\vec{y}})+t(\hat{\vec{y}},\hat{\vec{x}}\,))\\
{\cal A}t(\hat{\vec{x}},\hat{\vec{y}}\,)&=&\half(t(\hat{\vec{x}},\hat{\vec{y}})-t(\hat{\vec{y}},\hat{\vec{x}}\,))
\end{eqnarray*}
In $\cal V^*\otimes V^*$ analoog. Als $t$ symmetrisch resp. antisymmetrisch is,
dan is ${\cal S}t=t$ resp. ${\cal A}t=t$.
\npar
De tensoren $\vec{e}_i\vee\vec{e}_j=\vec{e}_i\vec{e}_j=2{\cal S}(\vec{e}_i\otimes\vec{e}_j)$,
met $1\leq i\leq j\leq n$ vormen een basis in $\cal S(V\otimes V)$ met dimensie
$\half n(n+1)$.
\npar
De tensoren $\vec{e}_i\wedge\vec{e}_j=2{\cal A}(\vec{e}_i\otimes\vec{e}_j)$,
met $1\leq i\leq j\leq n$ vormen een basis in $\cal A(V\otimes V)$ met dimensie
$\half n(n-1)$.
\npar
De volledig antisymmetrische tensor $\varepsilon$ is gegeven door:
$\varepsilon_{ijk}\varepsilon_{klm}=\delta_{il}\delta_{jm}-\delta_{im}\delta_{jl}$.
\npar
De permutatie-operatoren $e_{pqr}$ zijn gedefinieerd door:
$e_{123}=e_{231}=e_{312}=1$, $e_{213}=e_{132}=e_{321}=-1$, voor alle andere
combinaties is $e_{pqr}=0$. Er is een samenhang met de $\varepsilon$ tensor:
$\varepsilon_{pqr}=g^{-1/2}e_{pqr}$ en $\varepsilon^{pqr}=g^{1/2}e^{pqr}$.

\section{Uitwendig product}
Zij $\alpha\in\Lambda^k(\cal V)$ en $\beta\in\Lambda^l(\cal V)$. Dan wordt
$\alpha\wedge\beta\in\Lambda^{k+l}(\cal V)$ gedefinieerd door:
\[
\alpha\wedge\beta=\frac{(k+l)!}{k!l!}{\cal A}(\alpha\otimes\beta)
\]
Als $\alpha$ en $\beta\in\Lambda^1(\cal V)={\cal V}^*$, dan geldt:
$\alpha\wedge\beta=\alpha\otimes\beta-\beta\otimes\alpha$
\npar
Het uitproduct is te schrijven als: $(\vec{a}\times\vec{b})_i=\varepsilon_{ijk}a^jb^k$,
$\vec{a}\times\vec{b}=G^{-1}\cdot*(G\vec{a}\wedge G\vvec{b})$.
\npar
Neem $\vec{a},\vec{b},\vec{c},\vec{d}\in\RR^4$. Dan is
$(dt\wedge dz)(\vec{a},\vvec{b})=a_0b_4-b_0a_4$ de geori\"enteerde oppervlakte
van de projectie op het $tz$-vlak van het parallellogram opgespannen door
$\vec{a}$ en $\vec{b}$.
\npar
Verder is
\[
(dt\wedge dy\wedge dz)(\vec{a},\vec{b},\vec{c})=\det\left|\begin{array}{ccc}
a_0&b_0&c_0\\ a_2&b_2&c_2\\ a_4&b_4&c_4 \end{array}\right|
\]
de geori\"enteerde 3-dimensionale inhoud van de projectie op het $tyz$-vlak
van het parallellepipedum opgespannen door $\vec{a}$, $\vec{b}$ en $\vec{c}$.
\npar
$(dt\wedge dx\wedge dy\wedge dz)(\vec{a},\vec{b},\vec{c},\vec{d})=\det(\vec{a},\vec{b},\vec{c},\vec{d})$
is de 4-dimensionale inhoud van het hyperparallellepipedum opgespannen door
$\vec{a}$, $\vec{b}$, $\vec{c}$ en $\vec{d}$.

\section{De Hodge afbeelding}
Omdat ${n\choose k}={n\choose{n-k}}$ voor $1\leq k\leq n$ hebben
$\Lambda^k(\cal V)$ en $\Lambda^{n-k}(\cal V)$ dezelfde dimensie.
Dim$(\Lambda^n({\cal V}))=1$. De keuze van een basis betekent de keuze van een
geori\"enteerde inhoudsmaat, een volume $\mu$, in $\cal V$. We kunnen $\mu$
zodanig ijken dat voor een orthonormale basis $\vec{e}_i$ geldt:
$\mu(\vec{e}_i)=1$. Deze basis heet dan per definitie positief geori\"enteerd
als $\mu=\hat{\vec{e}}^{~1}\wedge \hat{\vec{e}}^{~2}\wedge...\wedge \hat{\vec{e}}^{~n}=1$.
\npar
Vanwege de gelijke dimensies kan men zich afvragen of er een bijectie tussen
beide ruimten bestaat. Als $\cal V$ geen extra structuur heeft is dit niet
het geval, echter, als $\cal V$ voorzien is van een inproduct en daarbij
behorende volumevorm $\mu$, dan bestaat zo'n afbeelding wel en wordt de
{\it Hodge-ster-afbeelding} genoemd en genoteerd met $*$. Er geldt dat
\[
\forall_{w\in\Lambda^k({\cal V})}\exists_{*w\in\Lambda^{k-n}({\cal V})}\forall_{\theta\in\Lambda^k({\cal V})}~~
\theta\wedge*w=(\theta,w)_\lambda\mu
\]
Voor een orthonormale basis in $\RR^3$ geldt: het volume: $\mu=dx\wedge dy\wedge dz$,
$*dx\wedge dy\wedge dz=1$, $*dx=dy\wedge dz$, $*dz=dx\wedge dy$, $*dy=-dx\wedge dz$,
$*(dx\wedge dy)=dz$, $*(dy\wedge dz)=dx$, $*(dx\wedge dz)=-dy$.
\npar
Voor een Minkowski basis in $\RR^4$ geldt: $\mu=dt\wedge dx\wedge dy\wedge dz$,
$G=dt\otimes dt-dx\otimes dx-dy\otimes dy-dz\otimes dz$, en
$*dt\wedge dx\wedge dy\wedge dz=1$ en $*1=dt\wedge dx\wedge dy\wedge dz$. Verder
$*dt=dx\wedge dy\wedge dz$ en $*dx=dt\wedge dy\wedge dz$.

\section{Differentiaaloperaties}
\subsection{De richtingsafgeleide}
De {\it richtingsafgeleide} in het punt $\vec{a}$ is gegeven door:
\[
{\cal L}_{\vec{a}}f=<\vec{a},df>=a^i\Q{f}{x^i}
\]

\subsection{De Lie-afgeleide}
De {\it Lie-afgeleide} is gegeven door:
\[
({\cal L}_{\vec{v}}\vec{w})^j=w^i\partial_iv^j-v^i\partial_iw^j
\]

\subsection{Christoffelsymbolen}
Aan ieder kromlijnig co\"ordinatenstelsel $u^i$ voegen we een stelsel van $n^3$
functies $\Gamma^i_{jk}$ van $\vec{u}$ toe gedefinieerd door
\[
\frac{\partial^2\vec{x}}{\partial u^i\partial u^k}=\Gamma_{jk}^i\Q{\vec{x}}{u^i}
\]
Dit zijn {\it Christoffelsymbolen van de tweede soort}. Christoffelsymbolen
zijn geen tensoren. De Christoffelsymbolen van de tweede soort zijn gegeven
door:
\[
\left\{\begin{array}{@{}c@{}}i\\ jk \end{array}\right\}:=\Gamma^i_{jk}=
\left\langle\frac{\partial^2\vec{x}}{\partial u^k\partial u^j},dx^i\right\rangle
\]
Er geldt dat $\Gamma^i_{jk}=\Gamma^i_{kj}$. Hun transformatie naar een ander
co\"ordinatenstelsel is gegeven door:
\[
\Gamma_{j'k'}^{i'}=A_{i'}^iA_{j'}^jA_{k'}^k\Gamma^i_{jk}+A_i^{i'}(\partial_{j'}A_{k'}^i)
\]
Als de geaccentueerde co\"ordinaten cartesisch zijn is de eerste term hierin
gelijk aan 0.
\npar
Er is een relatie tussen Christoffelsymbolen en de metriek:
\[
\Gamma_{jk}^i=\half g^{ir}(\partial_j g_{kr}+\partial_k g_{rj}-\partial_r g_{jk})
\]
en $\Gamma^\alpha_{\beta\alpha}=\partial_\beta(\ln(\sqrt{|g|}))$.
\npar
Omlaaghalen van een index geeft de {\it Christoffelsymbolen van de eerste soort}:
$\Gamma^i_{jk}=g^{il}\Gamma_{jkl}$.

\subsection{De covariante afgeleide}
De {\it covariante afgeleide} $\nabla_j$ van een vector, covector en van
tensoren van rang 2 is gegeven door:
\begin{eqnarray*}
\nabla_ja^i                  &=&\partial_ja^i+\Gamma^i_{jk}a^k\\
\nabla_ja_i                  &=&\partial_ja_i-\Gamma^k_{ij}a_k\\
\nabla_\gamma a^\alpha_\beta &=&\partial_\gamma a^\alpha_\beta -\Gamma^\varepsilon_{\gamma\beta} a^\alpha_\varepsilon+\Gamma^\alpha_{\gamma\varepsilon}a_\beta^\varepsilon\\
\nabla_\gamma a_{\alpha\beta}&=&\partial_\gamma a_{\alpha\beta}-\Gamma^\varepsilon_{\gamma\alpha}a_{\varepsilon\beta}-\Gamma^\varepsilon_{\gamma\beta}a_{\alpha\varepsilon}\\
\nabla_\gamma a^{\alpha\beta}&=&\partial_\gamma a^{\alpha\beta}+\Gamma^\alpha_{\gamma\varepsilon}a^{\varepsilon\beta}+\Gamma^\beta_{\gamma\varepsilon}a^{\alpha\varepsilon}
\end{eqnarray*}
De stelling van Ricci luidt dat
\[
\nabla_\gamma g_{\alpha\beta}=\nabla_\gamma g^{\alpha\beta}=0
\]

\section{Differentiaaloperatoren}
\subsubsection{De Gradi\"ent}
is gegeven door:
\[
{\rm grad}(f)=G^{-1}df=g^{ki}\Q{f}{x^i}\Q{}{x^k}
\]

\subsubsection{De divergentie}
is gegeven door:
\[
{\rm div}(a^i)=\nabla_ia^i=\frac{1}{\sqrt{g}}\partial_k(\sqrt{g}\,a^k)
\]

\subsubsection{De rotatie}
is gegeven door:
\[
{\rm rot}(a)=G^{-1}\cdot*\cdot d\cdot G\vec{a}=-\varepsilon^{pqr}\nabla_qa_p=\nabla_qa_p-\nabla_pa_q
\]

\subsubsection{De Laplaciaan}
is gegeven door:
\[
\Delta(f)={\rm div~grad}(f)=*d*df=\nabla_ig^{ij}\partial_jf=g^{ij}\nabla_i\nabla_jf=
\frac{1}{\sqrt{g}}\frac{\partial}{\partial x^i}\left(\sqrt{g}\,g^{ij}\frac{\partial f}{\partial x^j}\right)
\]

\section{Differentiaalmeetkunde}
\subsection{Ruimtekrommen}
We beperken ons tot $\RR^3$ met een vaste orthonormale basis. Een punt wordt
voorgesteld door $\vec{x}=(x^1,x^2,x^3)$. Een ruimtekromme is een verzameling
punten die voldoen aan $\vec{x}=\vec{x}(t)$. De booglengte van een ruimtekromme
is gegeven door:
\[
s(t)=\int\limits_{t_0}^t\sqrt{\left(\frac{dx}{d\tau}\right)^2+\left(\frac{dy}{d\tau}\right)^2+\left(\frac{dz}{d\tau}\right)^2}d\tau
\]
De afgeleide van $s$ naar $t$ is de lengte van de vector $d\vec{x}/dt$:
\[
\left(\frac{ds}{dt}\right)^2=\left(\frac{d\vec{x}}{dt},\frac{d\vec{x}}{dt}\right)
\]
Het {\it osculatievlak} in een punt $P$ van een ruimtekromme is de limietstand
van het vlak door de raaklijn in punt $P$ en een punt $Q$ wanneer $Q$ langs de
ruimtekromme nadert tot $P$. Het osculatievlak is evenwijdig aan $\dot{\vec{x}}(s)$.
De voorstelling van het osculatievlak is, als $\ddot{\vec{x}}\neq0$:
\[
\vec{y}=\vec{x}+\lambda\dot{\vec{x}}+\mu\ddot{\vec{x}}~~~\mbox{dus}~~~
\det(\vec{y}-\vec{x},\dot{\vec{x}},\ddot{\vec{x}}\,)=0
\]
In een buigpunt geldt, als $\dddot{\vec{x}}\neq0$:
\[
\vec{y}=\vec{x}+\lambda\dot{\vec{x}}+\mu\dddot{\vec{x}}
\]
De {\it raaklijn} heeft eenheidsvector $\vec{\ell}=\dot{\vec{x}}$, de
{\it hoofdnormaal} eenheidsvector $\vec{n}=\ddot{\vec{x}}$ en de
{\it binormaal} $\vec{b}=\dot{\vec{x}}\times\ddot{\vec{x}}$. De hoofdnormaal
ligt dus in het osculatievlak, de binormaal staat daar loodrecht op.
\npar
Zij $P$ een punt en $Q$ een naburig punt van een ruimtekromme $\vec{x}(s)$. Zij
$\Delta\varphi$ de hoek tussen de raaklijnen in $P$ en $Q$ en zij $\Delta\psi$
de hoek tussen de osculatievlakken (binormalen) in $P$ en $Q$. Dan zijn de
{\it kromming} $\rho$ en de {\it torsie} $\tau$ in $P$ gedefinieerd door:
\[
\rho^2=\left(\frac{d\varphi}{ds}\right)^2=\lim_{\Delta s\rightarrow0}\left(\frac{\Delta\varphi}{\Delta s}\right)^2~~~,~~~
\tau^2=\left(\frac{d\psi}{ds}\right)^2
\]
en $\rho>0$. Voor vlakke krommen is $\rho$ de traditionele kromming en is
$\tau=0$. Er geldt:
\[
\rho^2=(\vec{\ell},\vec{\ell})=(\ddot{\vec{x}},\ddot{\vec{x}}\,)~~~\mbox{en}~~~
\tau^2=(\dot{\vec{b}},\dot{\vec{b}})
\]
De formules van Frenet drukken de afgeleiden uit als lineaire combinaties van
deze vectoren:
\[
\dot{\vec{\ell}}=\rho\vec{n}~~,~~\dot{\vec{n}}=-\rho\vec{\ell}+\tau\vec{b}~~,~~
\dot{\vec{b}}=-\tau\vec{n}
\]
Hieruit volgt dat $\det(\dot{\vec{x}},\ddot{\vec{x}},\dddot{\vec{x}}\,)=\rho^2\tau$.
\npar
Enkele krommen en hun eigenschappen zijn:
\begin{center}
\begin{tabular}{||l@{\hspace*{1cm}}l||}
\hline
Schroeflijn      &$\tau/\rho=$constant\\
Cirkelschroeflijn&$\tau=$constant, $\rho=$constant\\
Vlakke krommen   &$\tau=0$\\
Cirkels          &$\rho=$constant, $\tau=0$\\
Rechten          &$\rho=\tau=0$\\
\hline
\end{tabular}
\end{center}

\subsection{Oppervlakken in $\RR^3$}
Een oppervlak in $\RR^3$ is de verzameling eindpunten der vectoren $\vec{x}=\vec{x}(u,v)$,
dus $x^h=x^h(u^\alpha)$. Op het oppervlak liggen 2 stelsels krommen, nl. met
$u=$constant en met $v=$constant.
\npar
Het raakvlak in een punt $P$ aan het oppervlak heeft als basis:
\[
\vec{c}_1=\partial_1\vec{x}~~~\mbox{en}~~~\vec{c}_2=\partial_2\vec{x}
\]

\subsection{De eerste fundamentaaltensor}
Zij $P$ een punt van het oppervlak $\vec{x}=\vec{x}(u^\alpha)$. Van twee krommen
door $P$, aan te geven door $u^\alpha=u^\alpha(t)$, $u^\alpha=v^\alpha(\tau)$,
zijn de raakvectoren in $P$
\[
\frac{d\vec{x}}{dt}=\frac{du^\alpha}{dt}\partial_\alpha\vec{x}~~~,~~~
\frac{d\vec{x}}{d\tau}=\frac{dv^\beta}{d\tau}\partial_\beta\vec{x}
\]
De {\it eerste fundamentaaltensor} van het oppervlak in $P$ is het inproduct
van deze raakvectoren:
\[
\left(\frac{d\vec{x}}{dt},\frac{d\vec{x}}{d\tau}\right)=
(\vec{c}_\alpha,\vec{c}_\beta)\frac{du^\alpha}{dt}\frac{dv^\beta}{d\tau}
\]
De covariante componenten t.o.v. de basis $\vec{c}_\alpha=\partial_\alpha\vec{x}$
zijn:
\[
g_{\alpha\beta}=(\vec{c}_\alpha,\vec{c}_\beta)
\]
Voor de hoek $\phi$ tussen de parameterkrommen in $P$: $u=t,v=$constant en
$u=$constant, $v=\tau$ geldt:
\[
\cos(\phi)=\frac{g_{12}}{\sqrt{g_{11}g_{22}}}
\]
Voor de booglengte $s$ vanuit $P$ langs de kromme $u^\alpha(t)$ geldt:
\[
ds^2=g_{\alpha\beta}du^\alpha du^\beta
\]
Deze uitdrukking heet het {\it lijnelement}.

\subsection{De tweede fundamentaaltensor}
De 4 afgeleiden van de raakvectoren $\partial_\alpha\partial_\beta\vec{x}=\partial_\alpha\vec{c}_\beta$
zijn elk lineair afhankelijk van de vectoren $\vec{c}_1$, $\vec{c}_2$ en $\vec{N}$
met $\vec{N}$ loodrecht op $\vec{c}_1$ en $\vec{c}_2$. Dit wordt geschreven als:
\[
\partial_\alpha\vec{c}_\beta=\Gamma^\gamma_{\alpha\beta}\vec{c}_\gamma+h_{\alpha\beta}\vec{N}
\]
Hieruit volgt:
\[
\Gamma^\gamma_{\alpha\beta}=(\vec{c}^{~\gamma},\partial_\alpha\vec{c}_\beta)~~~,~~~
h_{\alpha\beta}=(\vec{N},\partial_\alpha\vec{c}_\beta)=\frac{1}{\sqrt{\det|g|}}\det(\vec{c}_1,\vec{c}_2,\partial_\alpha\vec{c}_\beta)
\]

\subsection{Geodetische kromming}
Op het oppervlak $\vec{x}(u^\alpha)$ wordt een kromme gegeven door:
$u^\alpha=u^\alpha(s)$, dan $\vec{x}=\vec{x}(u^\alpha(s))$ met $s$ de booglengte
van de kromme. De lengte van $\ddot{\vec{x}}$ is de kromming $\rho$ van de
kromme in $P$. De projectie van $\ddot{\vec{x}}$ op het raakvlak is een
vector met componenten
\[
p^\gamma=\ddot{u}^\gamma+\Gamma^\gamma_{\alpha\beta}\dot{u}^\alpha\dot{u}^\beta
\]
waarvan de lengte de {\it geodetische kromming} heet van de kromme in $p$ en
blijft hetzelfde als het oppervlak met behoud van het lijnelement wordt
verbogen. De projectie van $\ddot{\vec{x}}$ op $\vec{N}$ heeft als lengte
\[
p=h_{\alpha\beta}\dot{u}^\alpha\dot{u}^\beta
\]
die de {\it normale kromming} van de kromme in $P$ heet. De {\it stelling van
Meusnier} luidt dat verschillende krommen op het oppervlak met dezelfde
raakvector in $P$ gelijke normale kromming hebben.
\npar
Een {\it geodetische lijn} van een oppervlak is een kromme op het oppervlak
waarvoor in elk punt de hoofdnormaal van de kromme samenvalt met de normaal op
het oppervlak. Voor een geodetische lijn moet in elk punt $p^\gamma=0$ dus
\[
\frac{d^2u^\gamma}{ds^2}+\Gamma^\gamma_{\alpha\beta}\frac{du^\alpha}{ds}\frac{du^\beta}{ds}=0
\]
De covariante afgeleide $\nabla/dt$ in $P$ van een vectorveld van een oppervlak
langs een kromme is de projectie op het raakvlak in $P$ van de gewone afgeleide
in $P$.
\npar
Voor twee vectorvelden $\vec{v}(t)$ en $\vec{w}(t)$ langs dezelfde kromme van
het oppervlak volgt de regel van Leibniz:
\[
\frac{d(\vec{v},\vvec{w})}{dt}=\left(\vec{v},\frac{\nabla\vec{w}}{dt}\right)+\left(\vec{w},\frac{\nabla\vec{v}}{dt}\right)
\]
Langs een kromme geldt:
\[
\frac{\nabla}{dt}(v^\alpha\vec{c}_\alpha)=\left(\frac{dv^\gamma}{dt}+\Gamma^\gamma_{\alpha\beta}\frac{du^\alpha}{dt}v^\beta\right)\vec{c}_\gamma
\]

\section{Riemannse meetkunde}
De {\it Riemann tensor} $R$ is gedefinieerd door:
\[
R^\mu_{\nu\alpha\beta}T^\nu=\nabla_\alpha\nabla_\beta T^\mu-\nabla_\beta\nabla_\alpha T^\mu
\]
Dit is een $1\choose 3$ tensor met $n^2(n^2-1)/12$ onafhankelijke componenten
die niet identiek gelijk aan 0 zijn. Deze tensor is een maat voor de kromming
van de beschouwde ruimte. Als ze 0 is is de ruimte een vlak manifold.
Ze heeft de volgende symmetrie-eigenschappen:
\[
R_{\alpha\beta\mu\nu}=R_{\mu\nu\alpha\beta}=-R_{\beta\alpha\mu\nu}=-R_{\alpha\beta\nu\mu}
\]
Er geldt:
\[
[\nabla_\alpha,\nabla_\beta]T_\nu^\mu=R_{\sigma\alpha\beta}^\mu T_\nu^\sigma+R_{\nu\alpha\beta}^\sigma T_\sigma^\mu
\]
De Riemann tensor hangt samen met de Christoffelsymbolen via
\[
R^\alpha_{\beta\mu\nu}=\partial_\mu\Gamma^\alpha_{\beta\nu}-\partial_\nu\Gamma^\alpha_{\beta\mu}+\Gamma^\alpha_{\sigma\mu}\Gamma^\sigma_{\beta\nu}-\Gamma^\alpha_{\sigma\nu}\Gamma^\sigma_{\beta\mu}
\]
In een ruimte en co\"ordinatenstelsel waar de Christoffelsymbolen 0 zijn gaat
dit over in
\[
R^\alpha_{\beta\mu\nu}=\half g^{\alpha\sigma}(\partial_\beta\partial_\mu g_{\sigma\nu}-\partial_\beta\partial_\nu g_{\sigma\mu}+\partial_\sigma\partial_\nu g_{\beta\mu}-\partial_\sigma\partial_\mu g_{\beta\nu})
\]
De {\it Bianchi identiteiten} zijn: $\nabla_\lambda R_{\alpha\beta\mu\nu}+\nabla_\nu R_{\alpha\beta\lambda\mu} +\nabla_\mu R_{\alpha\beta\nu\lambda}=0$.
\npar
De {\it Ricci tensor} verkrijgt men door contractie: $R_{\alpha\beta}\equiv R_{\alpha\mu\beta}^\mu$
en is symmetrisch in haar indices: $R_{\alpha\beta}=R_{\beta\alpha}$. De
{\it Einstein tensor} $G$ is gedefinieerd door: $G^{\alpha\beta}\equiv R^{\alpha\beta}-\half g^{\alpha\beta}$.
Hiervoor geldt: $\nabla_\beta G^{\alpha\beta}=0$. De Ricci scalar is
$R=g^{\alpha\beta}R_{\alpha\beta}$.


\chapter{Numerieke wiskunde}
\typeout{Numerieke wiskunde}
\label{chap:num}
\section{Fouten}
Als een op te lossen probleem een aantal parameters bevat die niet exact bekend
zijn zal de oplossing een fout bevatten. De samenhang tussen fouten in gegevens
en oplossing is uit te drukken in het {\it conditiegetal} $c$. Als het probleem
gegeven is door $x=\phi(a)$ geldt in eerste orde voor een fout $\delta a$ in
$a$:
\[
\frac{\delta x}{x}=\frac{a\phi'(a)}{\phi(a)}\cdot\frac{\delta a}{a}
\]
Het getal $c(a)=|a\phi'(a)|/|\phi(a)|$. Als het probleem goed geconditioneerd
is dan is $c\ll1$.

\section{Floating point rekenen}
De floating point representatie hangt af van 4 natuurlijke getallen:
\begin{enumerate}
\item Het {\it grondtal} of basis van het getalstelsel $\beta$,
\item De mantisselengte $t$,
\item De lengte van de exponent $q$,
\item Het teken $s$,
\end{enumerate}
De representatie van machinegetallen is dan: \fbox{rd$(x)=s\cdot m\cdot\beta^e$}
waarin mantisse $m$ een getal is met $t$ $\beta$-tallige cijfers dat voldoet
aan $1/\beta\leq|m|<1$, en $e$ een getal is met $q$ $\beta$-tallige cijfers
met $|e|\leq\beta^q-1$. Hieraan wordt toegevoegd het getal 0 met bv. $m=e=0$.
Het grootste machinegetal is
\[
a_{\rm max}=(1-\beta^{-t})\beta^{\beta^q-1}
\]
en het kleinste positieve machinegetal is
\[
a_{\rm min}=\beta^{-\beta^q}
\]
De afstand tussen twee opeenvolgende machinegetallen in het interval
$[\beta^{p-1},\beta^p]$ is $\beta^{p-t}$. Als $x$ re\"eel is en het
dichtstbijzijnde machinegetal is ${\rm rd}(x)$, dan geldt:
\begin{eqnarray*}
{\rm rd}(x)=x(1+\varepsilon) &~~~\mbox{met}~~~&|\varepsilon|\leq\half\beta^{1-t}\\
x={\rm rd}(x)(1+\varepsilon')&~~~\mbox{met}~~~&|\varepsilon'|\leq\half\beta^{1-t}
\end{eqnarray*}
Het getal $\eta:=\half\beta^{1-t}$ heet de machine-nauwkeurigheid waarvoor geldt:
\[
\varepsilon,\varepsilon'\leq\eta\left|\frac{x-{\rm rd}(x)}{x}\right|\leq\eta
\]
Een veelgebruikt 32 bits float formaat is: 1 bit voor $s$, 8 voor de exponent
en $23$ voor de mantisse. Het grondtal hierin is 2.

\section{Stelsels vergelijkingen}
We willen de matrixvergelijking $A\vec{x}=\vec{b}$ oplossen voor een
niet-singuliere $A$, hetgeen equivalent is aan het vinden van de inverse matrix
$A^{-1}$. Oplossing van een $n\times n$ matrix via de regel van Cramer vereist
te veel vermenigvuldigingen $f(n)$ met $n!\leq f(n)\leq (e-1)n!$, dus zijn
andere methoden wenselijk.

\subsection{Driehoeksstelsels}
Beschouw de vergelijking $U\vec{x}=\vec{c}$ waarin $U$ een rechtsboven matrix
is, d.w.z. $U_{ij}=0$ voor alle $j<i$, en alle $U_{ii}\neq0$. Dan is:
\begin{eqnarray*}
x_n    &=&c_n/U_{nn}\\
x_{n-1}&=&(c_{n-1}-U_{n-1,n}x_n)/U_{n-1,n-1}\\
\vdots & &\vdots\\
x_1    &=&(c_1-\sum_{j=2}^nU_{1j}x_j)/U_{11}
\end{eqnarray*}
In code:
\begin{verbatim}
for (k = n; k > 0; k--)
{
  S = c[k];
  for (j = k + 1; j < n; j++)
  {
    S -= U[k][j] * x[j];
  }
  x[k] = S / U[k][k];
}
\end{verbatim}
Dit algoritme vereist $\frac{1}{2}n(n+1)$ floating point berekeningen.

\subsection{De eliminatiemethode van Gauss}
Beschouw een algemeen stelsel $A\vec{x}=\vec{b}$. Dit is via de eliminatiemethode
van Gauss tot een driehoeksvorm te brengen door de eerste vergelijking
vermenigvuldigd met $A_{i1}/A_{11}$ van alle anderen af te trekken; nu is de
eerste kolom, op $A_{11}$ na, allemaal 0. Daarna wordt de 2e vergelijking zodanig
van de anderen afgetrokken dat alle elementen onder $A_{22}$ 0 zijn, etc.
In code:
\begin{verbatim}
for (k = 1; k <= n; k++)
{
  for (j = k; j <= n; j++) U[k][j] = A[k][j];
  c[k] = b[k];

  for (i = k + 1; i <= n; i++)
  {
    L = A[i][k] / U[k][k];
    for (j = k + 1; j <= n; j++)
    {
      A[i][j] -= L * U[k][j];
    }
    b[i] -= L * c[k];
  }
}
\end{verbatim}
Dit algoritme vereist $\frac{1}{3}n(n^2-1)$ floating point vermenigvuldigingen
en delingen voor bewerking van de co\"effici\"entenmatrix en $\frac{1}{2}n(n-1)$
vermenigvuldigingen voor de bewerking van de rechterleden, waarna ook het
driehoeksstelsel moet worden opgelost met $\frac{1}{2}n(n+1)$ berekeningen.

\subsection{Pivot strategie}
Als de hoekelementen $A_{11}, A^{(1)}_{22},...$ niet alle $\neq0$ zijn zullen
er vergelijkingen verwisseld moeten worden om Gauss eliminatie te laten werken
($A^{(n)}$ is het element na de $n$-de iteratieslag).
Een methode is: als $A^{(k-1)}_{kk}=0$, dan zoek een element $A^{(k-1)}_{pk}$
met $p>k$ dat $\neq0$ is en verwissel de $p$-de en $k$-de vergelijking. Deze
strategie mislukt alleen indien het stelsel singulier is en dus geen oplossing
heeft.

\section{Nulpunten van functies}
\subsection{Successieve substitutie}
Het doel is het oplossen van de vergelijking $F(x)=0$, dus het vinden van het
nulpunt $\alpha$, met $F(\alpha)=0$.
\npar
Veel oplossingsmethoden komen neer op het volgende:
\begin{enumerate}
\item Schrijf de vergelijking om in de vorm $x=f(x)$ z.d.d. een oplossing
      hiervan ook een oplossing van $F(x)=0$ is. Bovendien mag $f(x)$ in de
      buurt van $\alpha$ niet te sterk van $x$ afhangen.
\item Neem een beginschatting $x_0$ voor $\alpha$, en bepaal de rij $x_n$
      met $x_n=f(x_{n-1})$ en hoop dat $\lim\limits_{n\rightarrow\infty}x_n=\alpha$.
\end{enumerate}
Een voorbeeld: kies
\[
f(x)=\beta-\varepsilon\frac{h(x)}{g(x)}=x-\frac{F(x)}{G(x)}
\]
dan is te verwachten dat de rij $x_n$ met
\begin{eqnarray*}
x_0&=&\beta\\
x_n&=&x_{n-1}-\varepsilon\frac{h(x_{n-1})}{g(x_{n-1})}
\end{eqnarray*}
naar $\alpha$ convergeert.

\subsection{De lokale convergentiestelling}
Zij $\alpha$ een oplossing van $x=f(x)$ en zij $x_n=f(x_{n-1})$ bij gegeven
$x_0$. Zij $f'(x)$ continu in een omgeving van $\alpha$. Zij $f'(\alpha)=A$
met $|A|<1$. Dan is er een $\delta>0$ z.d.d. voor iedere
$x_0$ met $|x_0-\alpha|\leq\delta$ geldt:
\begin{enumerate}
\item $\lim\limits_{n\rightarrow\infty}n_n=\alpha$,
\item Als voor zekere $k$ geldt: $x_k=\alpha$, dan geldt voor iedere $n\geq k$
      dat $x_n=\alpha$. Als $x_n\neq\alpha$ voor alle $n$ dan geldt
\[
\lim_{n\rightarrow\infty}\frac{\alpha-x_n}{\alpha-x_{n-1}}=A~~~,~~~
\lim_{n\rightarrow\infty}\frac{x_n-x_{n-1}}{x_{n-1}-x_{n-2}}=A~~~,~~~
\lim_{n\rightarrow\infty}\frac{\alpha-x_n}{x_n-x_{n-1}}=\frac{A}{1-A}
\]
\end{enumerate}
De grootheid $A$ wordt de {\it asymptotische convergentiefactor} genoemd, de
grootheid $B=-{}^{10}\log|A|$ de {\it asymptotische convergentiesnelheid}.

\subsection{Extrapolatie volgens Aitken}
We stellen
\[
A=\lim_{n\rightarrow\infty}\frac{x_n-x_{n-1}}{x_{n-1}-x_{n-2}}
\]
$A$ convergeert naar $f'(a)$. Dan zal de rij
\[
\alpha_n=x_n+\frac{A_n}{1-A_n}(x_n-x_{n-1})
\]
convergeren naar $\alpha$.

\subsection{De iteratiemethode van Newton}
Het omvormen van $F(x)=0$ in $x=f(x)$ kan op meerdere manieren. Een
essenti\"ele voorwaarde daarbij is dat in een omgeving van een wortel
$\alpha$ geldt dat $|f'(x)|<1$, en hoe kleiner $f'(x)$ is, hoe sneller
de reeks convergeert. Een algemene manier om $f(x)$ te construeren is
\[
f(x)=x-\Phi(x)F(x)
\]
met $\Phi(x)\neq0$ in een omgeving van $\alpha$. Als men kiest:
\[
\Phi(x)=\frac{1}{F'(x)}
\]
dan geeft dit de methode van Newton. De iteratieformule wordt dan:
\[
x_n=x_{n-1}-\frac{F(x_{n-1})}{F'(x_{n-1})}
\]
Enkele eigenschappen:
\begin{itemize}
\item Dit resultaat is ook m.b.v. Taylorreeksen af te leiden.
\item De globale convergentie is vaak moeilijk te onderzoeken.
\item Als $x_n$ ver van $\alpha$ af ligt kan de convergentie soms erg traag zijn.
\item De aanname $F'(\alpha)\neq0$ betekent dat $\alpha$ een enkelvoudig
      nulpunt is.
\end{itemize}
Voor $F(x)=x^k-a$ wordt de reeks:
\[
x_n=\frac{1}{k}\left((k-1)x_{n-1}+\frac{a}{x_{n-1}^{k-1}}\right)
\]
Dit is een bekende methode om wortels te berekenen.
\npar
De volgende code vindt het nulpunt van een functie via de methode van Newton.
Het nulpunt ligt binnen het interval {\tt [x1, x2]}. De waarde wordt
aangepast totdat de nauwkeurigheid beter is dan {\tt$\pm$eps}. De functie
{\tt funcd} is een routine die zowel de functie en de eerste afgeleide in
punt {\tt x} teruggeeft en pointers daarnaartoe invult.
\begin{verbatim}
float SolveNewton(void (*funcd)(float, float*, float*), float x1, float x2, float eps)
{
  int   j, max_iter = 25;
  float df, dx, f, nulpunt;

  nulpunt = 0.5 * (x1 + x2);
  for (j = 1; j <= max_iter; j++)
  {
    (*funcd)(nulpunt, &f, &df);
    dx  = f/df;
    nulpunt = -dx;
    if ( (x1 - nulpunt)*(nulpunt - x2) < 0.0 )
    {
      perror("Jumped out of brackets in SolveNewton.");
      exit(1);
    }
    if ( fabs(dx) < eps ) return nulpunt; /* Convergence */
  }
  perror("Maximum number of iterations exceeded in SolveNewton.");
  exit(1);
  return 0.0;
}
\end{verbatim}

\subsection{De secant methode}
Dit is, i.t.t. de vorige methoden, een tweetraps methode. Als men twee
benaderingen $x_n$ en $x_{n-1}$ voor een nulpunt heeft dan kan men als
volgende benadering nemen het snijpunt
\[
x_{n+1}=x_n-F(x_n)\frac{x_n-x_{n-1}}{F(x_n)-F(x_{n-1})}
\]
Er is sprake van interpolatie als $F(x_n)$ en $F(x_{n-1})$ een verschillend
teken hebben, anders van extrapolatie.

\section{Polynoom interpolatie}
Een basis voor polynomen van graad $n$ is gegeven door de
{\it interpolatiepolynomen van Lagrange}:
\[
L_j(x)=\prod_{{l=0}\atop{l\neq j}}^n\frac{x-x_l}{x_j-x_l}
\]
Er geldt dat:
\begin{enumerate}
\item Iedere $L_j(x)$ heeft graad $n$,
\item $L_j(x_i)=\delta_{ij}$ voor $i,j=0,1,...,n$,
\item Ieder polynoom $p(x)$ is op eenduidige wijze te schrijven als
\[
p(x)=\sum_{j=0}^n c_jL_j(x)~~~\mbox{met}~~~c_j=p(x_j)
\]
\end{enumerate}
Dit is geen geschikte wijze om de waarde van een polynoom in een gegeven punt
$x=a$ te berekenen. Hiervoor is het Horner algoritme beter geschikt: de
waarde $s=\sum_k c_kx^k$ in $x=a$ is als volgt te berekenen:
\begin{verbatim}
float GetPolyValue(float c[], int n)
{
  int i; float s = c[n];
  for (i = n - 1; i >= 0; i--)
  {
    s = s * a + c[i];
  }
  return s;
}
\end{verbatim}
dan heeft {\tt s} na afloop de waarde $p(a)$.

\section{Bepaalde integralen}
Vrijwel alle numerieke integratie methoden zijn gebaseerd op een formule van
de vorm:
\[
\int\limits_a^bf(x)dx=\sum_{i=0}^nc_if(x_i)+R(f)
\]
met $n$, $c_i$ en $x_i$ onafhankelijk van $f(x)$ en $R(f)$ de restfout die
voor alle gebruikelijke methoden de vorm $R(f)=Cf^{(q)}(\xi)$ heeft met
$\xi\in(a,b)$ en $q\geq n+1$. Vaak worden de punten $x_i$ equidistant gekozen.
Enkele gebruikelijke formules zijn:
\begin{itemize}
\item De trapeziumregel: $n=1$, $x_0=a$, $x_1=b$, $h=b-a$:
\[
\int\limits_a^bf(x)dx=\frac{h}{2}[f(x_0)+f(x_1)]-\frac{h^3}{12}f''(\xi)
\]
\item Regel van Simpson: $n=2$, $x_0=a$, $x_1=\half(a+b)$, $x_2=b$, $h=\half(b-a)$:
\[
\int\limits_a^bf(x)dx=\frac{h}{3}[f(x_0)+4f(x_1)+f(x_2)]-\frac{h^5}{90}f^{(4)}(\xi)
\]
\item De midpoint regel: $n=0$, $x_0=\half(a+b)$, $h=b-a$:
\[
\int\limits_a^bf(x)dx=hf(x_0)+\frac{h^3}{24}f''(\xi)
\]
\end{itemize}
Als $f$ sterk varieert binnen het interval zal men dit interval gewoonlijk
opsplitsen en de integratieformules op de deelintervallen toepassen.
\npar
Als men zowel de co\"effici\"enten $c_j$ als de punten $x_j$ in een
integratieformule zodanig wil bepalen dat de integratieformule exact geldt
voor polynomen van een zo hoog mogelijke graad, dan krijgt men een
integratieformule van Gauss. Twee voorbeelden hiervan zijn:
\begin{enumerate}
\item De tweepuntsformule:
\[
\int\limits_{-h}^hf(x)dx=h\left[f\left(\frac{-h}{\sqrt{3}}\right)+f\left(\frac{h}{\sqrt{3}}\right)\right]+\frac{h^5}{135}f^{(4)}(\xi)
\]
\item De driepuntsformule:
\[
\int\limits_{-h}^hf(x)dx=\frac{h}{9}\left[5f\left(-h\sqrt{\mbox{$\frac{3}{5}$}}\right)+8f(0)+5f\left(h\sqrt{\mbox{$\frac{3}{5}$}}\right)\right]+\frac{h^7}{15750}f^{(6)}(\xi)
\]
\end{enumerate}

\section{Differenti\"eren}
Voor de numerieke berekening van $f'(x)$ staan verschillende
differentiatieformules ter beschikking:
\begin{itemize}
\item Voorwaartse differentiatie:
\[
f'(x)=\frac{f(x+h)-f(x)}{h}-\half hf''(\xi)
\]
\item Achterwaartse differentiatie:
\[
f'(x)=\frac{f(x)-f(x-h)}{h}+\half hf''(\xi)
\]
\item Centrale differentiatie:
\[
f'(x)=\frac{f(x+h)-f(x-h)}{2h}-\frac{h^2}{6}f'''(\xi)
\]
\item Als meer functiewaarden gebruikt worden is de benadering beter:
\[
f'(x)=\frac{-f(x+2h)+8f(x+h)-8f(x-h)+f(x-2h)}{12h}+\frac{h^4}{30}f^{(5)}(\xi)
\]
\end{itemize}
Ook voor hogere afgeleiden bestaan er formules:
\[
f''(x)=\frac{-f(x+2h)+16f(x+h)-30f(x)+16f(x-h)-f(x-2h)}{12h^2}+\frac{h^4}{90}f^{(6)}(\xi)
\]

\section{Differentiaalvergelijkingen}
We gaan uit van de eerste orde DV $y'(x)=f(x,y)$ voor $x>x_0$ en beginvoorwaarde
$y(x_0)=x_0$. Stel dat we benaderingen $z_1,z_2,...,z_n$ voor $y(x_1)$,
$y(x_2)$,..., $y(x_n)$ gevonden hebben. Dan kunnen we ter bepaling van
$z_{n+1}$ als benadering voor $y(x_{n+1})$ enkele formules afleiden:
\begin{itemize}
\item Euler (eenstaps, expliciet):
\[
z_{n+1}=z_n+hf(x_n,z_n)+\frac{h^2}{2}y''(\xi)
\]
\item Midpointregel (tweestaps, expliciet):
\[
z_{n+1}=z_{n-1}+2hf(x_n,z_n)+\frac{h^3}{3}y'''(\xi)
\]
\item Trapeziumregel (eenstaps, impliciet):
\[
z_{n+1}=z_n+\half h(f(x_n,z_n)+f(x_{n+1},z_{n+1}))-\frac{h^3}{12}y'''(\xi)
\]
\end{itemize}
Runge-Kutta methoden zijn een belangrijke klasse van eenstaps methoden. Ze
worden gemotiveerd doordat de oplossing $y(x)$ voldoet aan:
\[
y_{n+1}=y_n+hf(\xi_n,y(\xi_n))~~~\mbox{met}~~~\xi_n\in(x_n,x_{n+1})
\]
Omdat $\xi_n$ onbekend is worden er enkele ``metingen'' gedaan van de
incrementfunctie $k=hf(x,y)$ in goed gekozen punten nabij de oplossingskromme
en vervolgens neemt men voor $z_{n+1}-z_n$ een gewogen gemiddelde van de
meetwaarden. Zo geldt voor \'e\'en van de mogelijke 3e orde Runge-Kutta methoden:
\begin{eqnarray*}
k_1&=&hf(x_n,z_n)\\
k_2&=&hf(x_n+\half h,z_n+\half k_1)\\
k_3&=&hf(x_n+\mbox{$\frac{3}{4}$}h,z_n+\mbox{$\frac{3}{4}$}k_2)\\
z_{n+1}&=&z_n+\mbox{$\frac{1}{9}$}(2k_1+3k_2+4k_3)
\end{eqnarray*}
en de klassieke 4e orde methode is:
\begin{eqnarray*}
k_1&=&hf(x_n,z_n)\\
k_2&=&hf(x_n+\half h,z_n+\half k_1)\\
k_3&=&hf(x_n+\half h,z_n+\half k_2)\\
k_4&=&hf(x_n+h,z_n+k_3)\\
z_{n+1}&=&z_n+\mbox{$\frac{1}{6}$}(k_1+2k_2+2k_3+k_4)
\end{eqnarray*}
Vaak wordt de nauwkeurigheid nog vergroot door de stapgrootte per stap aan
te passen aan de geschatte afrondingsfout. Bij 4e orde Runge-Kutta is
stapverdubbeling de meest voor de hand liggende methode.

\section{De fast Fourier transformatie}
We kunnen de Fouriertransformatie van een functie benaderen als we een aantal
discrete punten hebben. Stel we hebben $N$ opeenvolgende samples $h_k=h(t_k)$,
$t_k=k\Delta$, $k=0,1,2,...,N-1$. De discrete Fourier transformatie is dan
gegeven door:
\[
H_n=\sum_{k=0}^{N-1}h_k{\rm e}^{2\pi ikn/N}
\]
en de inverse Fouriertransformatie door
\[
h_k=\frac{1}{N}\sum_{n=0}^{N-1}H_n{\rm e}^{-2\pi ikn/N}
\]
Deze operatie is van orde $N^2$. Dit kan sneller, orde $N\cdot^2\log(N)$,
met de Fast Fourier transformatie. Hiertoe ziet men in dat een discrete
Fouriertransformatie met lengte $N$ recursief opgedeeld kan worden in de som
van 2 discrete transformaties met lengte $N/2$, \'e\'en gevormd van de even
punten, de ander van de oneven.
\npar
Dit is als volgt te implementeren.
Het array {\tt data[1..2*nn]} bevat op de oneven plaatsen het re\"ele en
op de even plaatsen de imaginaire delen van de invoerwaarden: {\tt data[1]}
is het re\"ele deel en {\tt data[2]} het imaginaire deel van $f_0$, etc. De
volgende routine vervangt de waarden in {\tt data} door hun discreet Fourier
getransformeerden als {\tt isign} $=1$, en door hun invers getransformeerde
elementen als {\tt isign} $=-1$. {\tt nn} moet een macht van 2 zijn.
\begin{verbatim}
#include <math.h>
#define SWAP(a,b) tempr=(a);(a)=(b);(b)=tempr

void FourierTransform(float data[], unsigned long nn, int isign)
{
  unsigned long n, mmax, m, j, istep, i;
  double        wtemp, wr, wpr, wpi, wi, theta;
  float         tempr, tempi;

  n = nn << 1;
  j = 1;
  for (i = 1; i < n; i += 2)
  {
    if ( j > i )
    {
      SWAP(data[j], data[i]);
      SWAP(data[j+1], data[i+1]);
    }
    m = n >> 1;
    while ( m >= 2 && j > m )
    {
      j -= m;
      m >>= 1;
    }
    j += m;
  }
  mmax = 2;
  while ( n > mmax ) /* Buitenste loop, wordt log2(nn) maal uitgevoerd */
  {
    istep = mmax << 1;
    theta = isign * (6.28318530717959/mmax);
    wtemp = sin(0.5 * theta);
    wpr   = -2.0 * wtemp * wtemp;
    wpi   = sin(theta);
    wr    = 1.0;
    wi    = 0.0;
    for (m = 1; m < mmax; m += 2)
    {
      for (i = m; i <= n; i += istep) /* Danielson-Lanczos formule */
      {
        j          = i + mmax;
        tempr      = wr * data[j]   - wi * data[j+1];
        tempi      = wr * data[j+1] + wi * data[j];
        data[j]    = data[i]   - tempr;
        data[j+1]  = data[i+1] - tempi;
        data[i]   += tempr;
        data[i+1] += tempi;
      }
      wr = (wtemp = wr) * wpr - wi * wpi + wr;
      wi = wi * wpr + wtemp * wpi + wi;
    }
    mmax=istep;
  }
}
\end{verbatim}

\end{document}

